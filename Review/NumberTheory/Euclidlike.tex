\section{类欧几里得算法}
已知$k_1,k_2,a,b,c,n$,求
\begin{displaymath}
    F(k_1,k_2,a,b,c,n)=\sum_{x=0}^n{x^{k_1}
    {\left \lfloor\frac{ax+b}{c}\right \rfloor}^{k_2}}
\end{displaymath}

其中$k_1,k_2$较小。

以下除法操作均指向下取整除法,$\lambda$为易求得的常数。讨论多种情况的处理方法:
\begin{itemize}
    \item 若$an+b<c \wedge k_2\neq 0$,直接返回0。
    \item 若$k_2=0 \vee a=0$,答案为$\lambda \displaystyle \sum_{x=0}^n{x^{k_1}}$。
    利用~\ref{psum}所述方法求解。
    \item $a\geq c \vee b\geq c$:

    令$\displaystyle F(k_1,k_2,a,b,c,n)=\\\sum_{x=0}^n{x^{k_1}
    ((a/c)x+(b/c)+((a \bmod c) \cdot x+(b\bmod c))/c)^{k_2}}$,
    对指数为$k_2$的部分进行多项式展开,可以拆出类似$\sum{\lambda F}$的形式,
    递归求解。

    \item 否则$a<c \wedge b<c \wedge k_2 \neq 0$,
    使用类似求期望的技巧,枚举指数为$k_2$的幂并差分,记$m=(an+b)/c$,
    有\begin{eqnarray*}
        F(k_1,k_2,a,b,c,n)&=&\sum_{i=0}^{m-1}{\left(
            ((i+1)^{k_2}-i^{k_2})\sum_{x=0}^n{
                x^{k_1}[(ax+b)/c>i]
            }
        \right)}\\
        &=&\sum_{i=0}^{m-1}{\left(
            ((i+1)^{k_2}-i^{k_2})\sum_{x=0}^n{
                x^{k_1}[x>(ic+c-b-1)/a]
            }
        \right)}\\
        &=&\sum_{i=0}^{m-1}{((i+1)^{k_2}-i^{k_2})}
        \cdot \sum_{x=0}^n{x^{k_1}}\\
        & &-\sum_{i=0}^{m-1}{\left(
            ((i+1)^{k_2}-i^{k_2})\sum_{x=0}^{(ic+c-b-1)/a}{
                x^{k_1}
            }
        \right)}
    \end{eqnarray*}
    前一部分可以插值求出,后一部分把多项式$\displaystyle \sum_{i=0}^n{i^k}=
    \sum_{i=0}^{k+1}{c_i\cdot n^i}$
    展开,化为$\sum{\lambda F}$的形式递归求解。
\end{itemize}

实现时可以令函数返回$g(n,a,b,c)=R^{k_1\cdot k_2}$存储所有需要的值,避免重复计算。
递归层数为$O(\lg n)$。

模板(LOJ\#138):
\lstinputlisting{Source/Templates/Euclidlike.cpp}

上述内容参考了fjzzq2002的博客\footnote{
    类欧几里得算法
    \url{https://www.cnblogs.com/zzqsblog/p/8904010.html}
}。
