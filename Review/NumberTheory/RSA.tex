\section{RSA算法}
\subsection{原理}
\begin{theorem}[素数定理]\label{PT}
    $\lim_{n\rightarrow\infty}\frac{\pi(n)}{n/\ln n}=1$
\end{theorem}
RSA的安全性基于以下事实:寻找大素数很容易(根据定理~\ref{PT},素数密度还是挺大的),
但把一个数分解为两个质数之积却很难。

RSA算法的基本步骤如下:
\begin{enumerate}
    \item 随机选取两个大素数$p,q$,使得$p\neq q$,令$n=pq$;
    \item 选取一个与$\varphi(n)=(p-1)(q-1)$复制的小奇数$e$,
    计算出$e$的乘法逆元$d$;
    \item 将$P(e,n)$公开,作为{\bfseries RSA公钥};\\
          将$S(d,n)$保密,作为{\bfseries RSA私钥}。
\end{enumerate}

对于消息$M$,公钥持有者可进行运算:$P(M)=M^e \bmod n$;
私钥持有者可进行运算:$S(M)=M^d mod n$。
对于用公/私钥加密$M$得到的密文$C$,只有使用私/公钥才能得到$M$。
由于$\bmod n$的缘故,消息$M$的域为$Z_n$。

下面证明RSA算法的正确性,即证明:
\begin{displaymath}
    P(S(M))=S(P(M))=M^{ed}\equiv M \pmod{n}
\end{displaymath}

因为$e,d$是模$\varphi(n)$意义下的乘法逆元,所以有$ed=1+k(p-1)(q-1)$。

\begin{itemize}
    \item 若$M\not\equiv 0 \pmod{p}$,则有
    \begin{eqnarray*}
        M^{ed}&\equiv& M^{1+k(p-1)(q-1)} \pmod{p}\\
        &\equiv& M\cdot (M^{p-1})^{k(q-1)} \pmod{p}\\
        &\equiv& M\cdot 1^{k(q-1)} \pmod{p}\\
        &\equiv& M \pmod{p}
    \end{eqnarray*}
    \item 若$M\equiv 0 \pmod{p}$,上述等式仍成立。
\end{itemize}

同样地,对于$q$有$M^{ed}\equiv M \pmod{q}$。根据引理~\ref{EEL1},有
$M^{ed}\equiv M \pmod{n}$,证毕。

\subsection{应用}
\subsubsection{消息加密}
发送方使用接收方的公钥$P$把消息$M$加密得到密文$C$,将密文$C$发送给
接收方。接收方使用自己的私钥$P$解密得到消息$M$。
\paragraph{快速无公钥加密系统}
若消息过长,则仅用$P$加密对称加密算法的随机密钥$K$,同时用密钥$K$加密
$M$得到密文$C$,把$(P(K),C)$发送给接收方。接收方使用$P$解密得到$K$,
再用$K$对$C$解密即可。
\subsubsection{数字签名}
发送方使用自己的私钥$S$把消息$M$签署得到签名$C$,将消息$M$与签名$C$
发送给接收方。接收方使用发送方的公钥$P$解密得到消息$M$,验证消息是否正确。
\paragraph{快速数字签名}
同理把对称加密算法的密钥改为快速散列函数的值即可。
\paragraph{证书链}
以一个可信根为起点,大家都知道这个根的公钥。下一级可以将自己的公钥和被上一级
签署后的公钥作为签名证书,由此验证证书链上每一级的正确性,从而证明链尾端消息的
正确性。
以上内容参考了算法导论\cite{ITA3}第31.7节。
