\section{欧拉定理}
\subsection{费马小定理}
\index{F!Fermat's Little Theorem}
\begin{theorem}[Fermat's Little Theorem]\label{FLT}
	~\\
	$\forall p~is~a~prime~number,a\in \mathbb{Z},a^p \equiv a \pmod{p}$
\end{theorem}

该定理是定理~\ref{ET}的特殊化,不证。

利用定理~\ref{FLT}可知:

\begin{inference}
	$a^{-1} \equiv a^{p-2} \pmod{p}$
\end{inference}

使用快速幂即可在$O(lgp)$的复杂度下求某个数的逆元。

\subsection{线性推逆元}

如果需要获得$a\in [1,p)$内模$p$的逆元,复杂度为$O(plgp)$的逐个快速幂
并不是最优方法。

首先有$1^{-1}\equiv 1 \pmod{p}$。

令$p=qa+r$,其中$q=[\frac{p}{a}],r=p \bmod a$。

再把$p \equiv 0 \pmod{q}$中的$p$用$qa+r$代替,两边同时乘上$(ar)^{-1}$,
移项得$a^{-1}\equiv -qr^{-1} \pmod{q}$,即
$a^{-1}\equiv -[\frac{p}{a}](p \bmod a) \pmod{p}$。

代码如下:
\begin{lstlisting}[title=inv]
inv[1]=1;
for(int i=2;i<=n;++i)
    inv[i]=asInt64(mod-mod/i)*inv[mod%i]%mod;
\end{lstlisting}

以上内容参考了Miskcoo的博客\footnote{[数论]线性求所有逆元的方法 – Miskcoo's Space
	\url{http://blog.miskcoo.com/2014/09/linear-find-all-invert}}

\subsection{欧拉定理}
\index{E!Euler's Theorem}
\begin{theorem}[Euler's Theorem]\label{ET}
	~\\
	对于任意互质正整数对$(a,n)$,有$a^\phi(n) \equiv 1 \pmod{n}$
\end{theorem}
证明:

\index{L!Lagrange's Theorem}
\begin{theorem}[Lagrange's Theorem]\label{LT}
	若$(S,\oplus)$是一个有限群,$(S',\oplus)$是$(S,\oplus)$的子群,则
	$|S'|$是$|S|$的约数。
\end{theorem}

令$S=\{x|x\in Z^+ \land x~is~coprime~to~n\}$,则有限群$(S,\cdot_n)$的
阶为$\phi(n)$。

对于任意一个与$n$互质的正整数$a$,$a$的幂模$n$的值$a,a^2,\ldots,a^k$
构成了一个子群,其中$a^k\equiv 1 \pmod{n}$。

根据定理~\ref{LT},有$k|\phi(n)$,令$M=\phi(n)/k$,有
$a^\phi(n)=a^{kM}=(a^k)^M=1^M\equiv 1 \pmod{n}$。

上述证明源自Wikipedia-EN\footnote{Euler's theorem - Wikipedia
	\url{https://en.wikipedia.org/wiki/Euler's_theorem}}。
\subsection{扩展欧拉定理}
\begin{theorem}
	$a^x\equiv a^{x \bmod \phi(m)+\phi(m)} \pmod{m}$
\end{theorem}

\begin{lemma}
	\begin{displaymath}
		\left\{
		\begin{array}{l}
			x\equiv y \pmod{m_1} \\
			x\equiv y \pmod{m_2}
		\end{array}
		\right.
		\Rightarrow x\equiv y \pmod{lcm(m_1,m_2)}
	\end{displaymath}
\end{lemma}

证明:


以上证明源自后缀自动机·张的文章\footnote{微小的欧拉定理EXT证明
	\url{https://zhuanlan.zhihu.com/p/24902174}}。
