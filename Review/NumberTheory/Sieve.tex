\section{低于线性时间复杂度筛法}
积性函数前缀和算法的思维难度:杜教筛>min\_筛>Powerful~Number。
\subsection{杜教筛}
杜教筛主要用于计算大数据规模积性函数求和。
\subsubsection{约数函数前缀和}
求$\displaystyle \sum_{i=1}^n{\sigma(i)},n\leq 10^{12}$。
\begin{eqnarray*}
    \sum_{i=1}^n{\sigma(i)}&=&\sum_{i=1}^n{\sum_{d|i}d}\\
    &=&\sum_{d=1}^n{d[\frac{n}{d}]}
\end{eqnarray*}
由于$[\frac{n}{d}]$存在许多连续相同的值,使用整除分块法可做到$O(\sqrt{n})$。
\subsubsection{欧拉函数前缀和}
求$\displaystyle \sum_{i=1}^n{\varphi(i)},n\leq 10^{11}$。
由定理~\ref{PhiT}可得
$\displaystyle \varphi(n)=n-\sum_{d|n,d<n}{\varphi(d)}$。
\begin{eqnarray*}
    ans(n)&=&\sum_{i=1}^n{\varphi(i)}\\
    &=&\sum_{i=1}^n{\left(i-\sum_{d|i,d<i}{\varphi(d)}\right)}\\
    &=&\frac{n(n+1)}{2}-\sum_{i=2}^{n}{\sum_{d|i,d<i}{\varphi(d)}}\\
    &=&\frac{n(n+1)}{2}-\sum_{\frac{i}{d}=2}^n
    {\sum_{d=1}^{[\frac{n}{\frac{i}{d}}]}{\varphi(d)}}\\
    &=&\frac{n(n+1)}{2}-\sum_{t=2}^n
    {\sum_{d=1}^{[\frac{n}{t}]}{\varphi(d)}}\\
    &=&\frac{n(n+1)}{2}-\sum_{t=2}^n{ans([\frac{n}{t}])}
\end{eqnarray*}
同理使用分块+递归询问区间和来计算答案。为了降低复杂度,应该先线性筛预处理前一部分值。
当预处理$k=n^\frac{2}{3}$时可以取到复杂度$T(n)=O(n^\frac{2}{3})$。
\subsubsection{莫比乌斯函数前缀和}
求$\displaystyle \sum_{i=1}^n{\mu(i)},n\leq 10^{11}$。
由定理~\ref{MobiusT}可得
$\displaystyle \mu(n)=[n=1]-\sum_{d|n,d<n}{\mu(d)}$。
\begin{eqnarray*}
    ans(n)&=&\sum_{i=1}^n{\mu(i)}\\
    &=&\sum_{i=1}^n{\left([i=1]-\sum_{d|i,d<i}{\mu(d)}\right)}\\
    &=&1-\sum_{i=1}^n{\sum_{d|i,d<i}{\mu(d)}}\\
    &=&1-\sum_{t=2}^n{ans([\frac{n}{t}])}
\end{eqnarray*}
\subsubsection{其它函数前缀和}
主要思路是使用狄利克雷卷积构造出一个简单的前缀和函数,且用于卷积的另一个函数也容易计算。

令$\displaystyle A(n)=\sum_{i=1}^n\frac{i}{(n,i)}$,求
$\displaystyle \sum_{i=1}^n{A(n)},n\leq 10^{9}$。

先化简$A(n)$:
\begin{eqnarray*}
    A(n)&=&\sum_{i=1}^n\frac{i}{(n,i)}\\
    &=&\sum_{d|n}{\sum_{i=1}^n{[(n,i)=d]\cdot\frac{i}{d}}}\\
    &=&\sum_{d|n}{\sum_{\frac{i}{d}=1}^{\frac{n}{d}}
    {[(\frac{n}{d},\frac{i}{d})=1]\cdot\frac{i}{d}}}\\
    &=&\frac{1}{2}\left(1+\sum_{d|n}{d\cdot\varphi(d)}\right)
\end{eqnarray*}

那么答案即为$\displaystyle \frac{1}{2}\left(n+\sum_{t=1}^n
    {\sum_{d=1}^{[\frac{n}{t}]}{d\cdot\varphi(d)}}\right)$。

考虑计算$\displaystyle \sum_{d=1}^n{d\cdot\varphi(d)}$的值:

易知$(id\cdot\varphi)*id=id^2$,因为\begin{displaymath}
    \sum_{d|n}d\cdot\varphi(d)\cdot\frac{n}{d}=
    n\cdot\sum_{d|n}\varphi(d)=n^2
\end{displaymath}

所以有\begin{eqnarray*}
    \frac{n(n+1)(2n+1)}{6}&=&\sum_{i=1}^n{(id\cdot\varphi)*id}\\
    &=&\sum_{t=1}^n{t\cdot\sum_{d=1}^{[\frac{n}{t}]}{d\cdot\varphi(d)}}
\end{eqnarray*}

\subsubsection{总结}
欲求积性函数$f(x)$的前缀和,构造$h=f*g$,其中$h(x)$和$g(x)$都是积性函数,且易求得
$h(x)$与$g(x)$的前缀和。

考虑$h(x)$前缀和的表达式(记大写形式为前缀和,即$F(n)=\displaystyle \sum_{i=1}^n{f(i)}$):
\begin{displaymath}
    H(n)=\sum_{i=1}^n{\sum_{d|i}{f(\frac{i}{d})g(d)}}=
    \sum_{d=1}^n{g(d)\sum_{i=1}^{\lfloor\frac{n}{d}\rfloor}{f(i)}}=
    \sum_{d=1}^n{g(d)F(\lfloor\frac{n}{d}\rfloor)}
\end{displaymath}

提出$F(n)$,即$F(n)=H(n)-\displaystyle \sum_{d=2}^n{g(d)F(\frac{n}{d})}$。

在递归时可以预处理一部分前缀和,同时使用HashTable缓存计算结果。
\subsubsection{时间复杂度分析}
使用整除分块法需要计算前$i$项前缀和,与前$\lfloor\frac{n}{i} \rfloor$项前缀和,
其中$i\leq \sqrt{n}$。后一部分比前一部分复杂度更高。考虑使用积分近似,有
\begin{displaymath}
    \int_0^{\sqrt{n}}{\sqrt{\frac{n}{x}} \ud x}=O(n^{3/4})
\end{displaymath}

预处理一部分前缀和可以有效降低算法时间复杂度:记预处理前$k$个前缀和,$k\geq \sqrt{n}$。

那么时间复杂度为
\begin{displaymath}
    O(k)+\int_0^{\frac{n}{k}}{\sqrt{\frac{n}{x}}\ud x}=O(k)+O(\frac{n}{\sqrt{k}})
\end{displaymath}

平衡两边的复杂度,解得$k=n^{1/3}$。

以上例题来自skywalkert的博客\footnote{浅谈一类积性函数的前缀和\\
    \url{https://blog.csdn.net/skywalkert/article/details/50500009}},
总结部分参考了国家集训队2016论文集任之洲的论文《积性函数求和的几种方法》。

\subsection{min\_25筛}
这里求和的积性函数$F$满足$F(p)$是一个关于$p$的低阶多项式且能够快速求出$F(p^k)$。
据说min\_25筛踩爆洲阁筛,那我就不学洲阁筛了。在此附上洲阁筛教程\footnote{
    洲阁筛学习 | \_\_debug's Home\\
    \url{http://debug18.com/posts/calculate-the-sum-of-multiplicative-function/}
}。

\subsubsection{预处理}
首先考虑求$\displaystyle \sum_{p\leq n}{F(p)}$。

记$g(n,j)$为满足$x$为$n$以内素数,或者$x$的最小质因子$>p_j$的$F(x)$之和,
所求值即为$g(n,|P|)$。考虑$g(n,j)$如何从$g(n,j-1)$转移。易知最小质因子为
$p_j$的合数为$p_j^2$,若其$>n$,则$g(n,j)$与$g(n,j-1)$都只求素数的积性函
数值之和,所以$g(n,j)=g(n,j-1)$。若$p_j^2\leq n$,则转移时会损失掉一些
$F(x)$,这些$x$的最小质因子为$p_j$。考虑提出$x$的$p_j$,满足$\frac{x}{p_j}$
的最小质因子$\geq p_j$,计算$\frac{x}{p_j}$的积性函数和,发现$g(\frac{n}{p_j},j-1)$
包括了它们,又因为$\frac{n}{p_j}\geq p_j > p_{j-1}$,$g(\frac{n}{p_j},j-1)$还有
$\displaystyle \sum_{p<p_j}F(p)$,需要扣除。{\bfseries 由于积性函数$F$的特殊性,
把不同次数的项拆开算,单项为完全积性函数,乘上$F(p_j)$即为需要减去的值。}

{\bfseries 若拆开后某一项系数不为1,这一项就不是完全积性。算这一项的贡献时先去除系数,
最后整体乘以系数。}

综上,有\begin{displaymath}
    g(n,j)=\left\{\begin{array}{lr}
        g(n,j-1)        & p_j^2>n   \\
        g(n,j-1)-F(p_j)(g(\frac{n}{p_j},j-1)-\displaystyle \sum_{p<p_j}{F(p)}) & p_j^2\leq n \\
    \end{array}\right.
\end{displaymath}

预处理素数时只需要筛$\sqrt{n}$内的素数,边界$g(n,0)$是所有数按照素数的计算方式计算
的值之和。由于最后只需要$g(n,|P|)$,非质数的贡献会被筛掉。

实质上$g(n,j)$就是埃氏筛法筛完$p_j$后未被筛的合数以及素数的积性函数值之和。

接下来尝试求出所有的$g(x,|P|),x=\lfloor \frac{n}{i}\rfloor$。
这里有一个存储上的trick:由于$\lfloor \frac{n}{i}\rfloor$有连续重复项,
最多$2\sqrt{n}$个,对于$x=\lfloor \frac{n}{i}\rfloor>\sqrt{n}$,把它
映射到$\lfloor \frac{n}{x}\rfloor$上存储,这样保证了空间复杂度为$O(\sqrt{n})$。

由于最后只要$g(x,|P|)$,$g$数组只要开1维滚动更新。

伪代码如下:
\begin{lstlisting}
int g[2][sqsiz],q[2*sqsiz];
int& getG(int x) {
    if(x<=sqr) return g[0][x];
    return g[1][n/x];
}
void calcG() {
    int m=0,i=1;
    while(i<=n) {
        int val=n/i;
        q[++m]=val;
        getG(val)=sumf(val);
        i=n/val+1;
    }
    for(int i=1;i<=psiz;++i) {
        int cp=p[i],cp2=cp*cp;
        for(int j=1;j<=m && cp2<=q[j];++j) {
            int k=q[j],&val=getG(k);
            val=sub(val,mul(f(cp),getG(k/cp)-sumpf[i-1]));
        }
    }
}
\end{lstlisting}

计算$G$时始终不考虑1,在求和时才加入。
\subsubsection{求和}
记$S(n,j)$为$n$以内最小质因子$\geq p_j$的积性函数值和,所求答案即为$S(n,1)+f(1)$。

把$S(n,j)$分为素数和合数求解:
\begin{itemize}
    \item 对于素数部分,$g(n,|P|)$代表了素数积性函数值和,再扣去不满足
    最小质因子要求的素数,最终贡献为$g(n,|P|)-\displaystyle \sum_{p<p_j}F(p)$。
    \item 对于合数部分,枚举其最小质因子$p_k$及其幂次$c$,单独贡献为\\
    $F(p_k^c)S(\frac{n}{p_k^c},k+1)+F(p_k^{c+1})$。注意此处的$F\cdot S$直接利用
    了积性函数的定义,因为$S$部分无$p_k$因子。由于$S$不处理$j=k$的部分,需要另外加上
    $F(p_k^{c+1})$。
\end{itemize}

递归的边界条件为$n\leq 1 \vee n<p_j$,无需记忆化。

时间复杂度为$O(\frac{n^\frac{3}{4}}{\lg n})$,空间复杂度为$O(\sqrt{n})$。

模板(LOJ\#6053. 简单的函数):
\lstinputlisting{Source/Templates/min_25.cpp}

上述内容参考了小蒟蒻yyb\footnote{
    min\_25筛
    \url{https://www.cnblogs.com/cjyyb/p/9185093.html}
}和租酥雨\footnote{
    Min\_25 筛
    \url{https://www.cnblogs.com/zhoushuyu/p/9187319.html}
}的博客。

Min\_25筛似乎也被称作``通用筛法''、``扩展埃拉托斯特尼筛法'',严格证明参见
zbh2047的文章\footnote{关于一种积性函数前缀和的通用筛法的时间复杂度证明

    \url{https://www.cnblogs.com/zbh2047/p/8552551.html}

    \url{https://zhuanlan.zhihu.com/p/33544708}}。
\subsection{Powerful~Number}
定义Powerful~Number为所有质因子的指数都$\geq 2$的数,那么每个Powerful~Number都可以
被表示为$a^2b^3$的形式(若指数为奇数则分配一个立方给$b$,其余分给$a$)。
{\bfseries 注意1也是Powerful~Number。}

\begin{theorem}
    $n$以内的Powerful~Number个数为$O(\sqrt{n})$。
\end{theorem}

证明:枚举$a$,将$b$的个数累积,可得式子
\begin{displaymath}
    \sum_{i=1}^{\lfloor\sqrt{n}\rfloor}{\lfloor\sqrt[3]{\frac{n}{i^2}}\rfloor}
\end{displaymath}

使用积分近似求出其上界为$\int_1^{\sqrt{n}+1}{\sqrt[3]{\frac{n}{(x-1)^2}} \ud x}=O(\sqrt{n})$。
根据Wikipedia-EN\footnote{Powerful~number
    \url{https://en.wikipedia.org/wiki/Powerful\_number}}的描述,其上界常数为
    $\frac{\zeta(\frac{3}{2})}{\zeta(3)}\approx 2.173$。

对于某个复杂的积性函数$f(x)$,若$f(p^e)$易于计算且存在一个简单(易于求$g(p^e)$与前缀和)
的积性函数$g(x)$,满足对于所有素数$p$,有$f(p)=g(p)$,称函数$g$拟合了函数$f$。

设$h=f/g$,这里的除法是狄利克雷除法,等价于$h=f*g^{-1}$。由于狄利克雷逆$g^{-1}$是积性
函数,狄利克雷卷积$h$也是积性函数。那么对于所有素数$p$,有$f(p)=h(1)g(p)+h(p)g(1)$,
由于$h(1)=1,f(p)=g(p)$,可得$h(p)=0$。由于$h(x)$是积性函数,$h(x)$可能非0当且仅当$x$
是Powerful~Number。

现在要求$\displaystyle Ans=\sum_{i=1}^n{f(n)}$,由于$f=h*g$,有
$\displaystyle Ans=\sum_{ab\leq n}{h(a)g(b)}$。由上文的推导可知$h(x)$仅在
Powerful~Number处有贡献,且Powerful~Number的个数是$O(\sqrt{n})$的,可以
$O(\sqrt{n})$DFS暴力枚举质因子组合得到$a$。记$n$以内的Powerful~Number组成的集合为$S$,
原式变为$\displaystyle Ans=\sum_{a\in S}{h(a)\sum_{b=1}^{\lfloor \frac{n}{a} \rfloor}{g(b)}}$。
易求$g(x)$的前缀和,问题在于如何快速推得$h(a)$的值。

由于$h(x)$是积性函数且$x$是Powerful~Number,在DFS时仅需计算$h(p^e),e>1$的值。
使用$f(p^e)$展开式,快速求得$f(p^e)$与$g(p^e)$,再根据历史信息$h(p^{e'}),e'<e$,
可以快速得到$h(p^e)$。如果$g(x)$是完全积性函数,可以对$f(p^e)$展开式平移得到$f(p^{e+1})$
的展开式。由于$e$很小,$h(p^e)$的求值不是瓶颈。预处理可以节省DFS时的重复计算。

{\bfseries 十分有效的优化:DFS递归时会遇到大量的0次项,这些不必要的递归会导致实际运行缓慢。
可以DFS钦定一些质因子必选,使得每层DFS都对最终的$a$有贡献,杜绝爆栈。}

具体实现参考Project~Euler~484:
\lstinputlisting{Source/Source/'Number Theory'/PE484.cpp}

{\bfseries 记得要计算$a=1$时的贡献!!!}

上述内容参考了fjzzq2002的博客\footnote{
利用powerful~number求积性函数前缀和
    \url{https://www.cnblogs.com/zzqsblog/p/9904271.html}
}。

Min\_25使用Powerful~Number得到新的做法,参见
Sum~of~Multiplicative~Function~on~Powerful~Numbers
\footnote{\url{https://min-25.hatenablog.com/entry/2018/11/11/172228}}。
\subsection{素数k次幂前缀和}
可以使用Min\_25筛的前半部分在$O(\frac{n^{3/4}}{\lg n})$内解决,但这不是最优的。

一般使用Meissel Lehmer方法在$O\left(\left(\frac{n}{\lg n}\right)^{2/3}\right)$
内解决,留坑待补。
\index{*TODO!Meissel Lehmer}
\subsection{约数个数函数前缀和}
使用整除分块可以在$O(\sqrt{n})$内解决,但还有更优算法。

考虑变化后的式子$S(n)=\displaystyle \sum_{i=1}^n{\lfloor\frac{n}{i}\rfloor}$,
可以将其理解为在$[1,n]$内在双曲线$xy=n$与$x$轴内的整点数。使用Stern-Brocot Tree可以
在$O(n^{1/3}\lg n)$内解决,留坑待补。

\index{*TODO!约数个数函数前缀和}
