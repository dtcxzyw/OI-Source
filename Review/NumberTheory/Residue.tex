\section{二次剩余与三次剩余}
\subsection{勒让德符号}
\index{L!Legendre Symbol}
定义勒让德符号:
\begin{displaymath}
	\Legendre{a}{p}=
	\left\{\begin{array}{lr}
		0  & a\equiv 0 \pmod{p}                  \\
		1  & \exists x,x^2\equiv a \pmod{p}      \\
		-1 & \not \exists x,x^2\equiv a \pmod{p}
	\end{array}\right.
\end{displaymath}
勒让德符号是完全积性函数,即
\begin{displaymath}
	\Legendre{ab}{p}=\Legendre{a}{p}\Legendre{b}{p}
\end{displaymath}
\subsubsection{与斐波那契数列的关系}
\begin{theorem}
	若$p$为素数,则
	\begin{displaymath}
		F_{p-\Legendre{p}{5}}\equiv 0 \pmod{p}
	\end{displaymath}
	\begin{displaymath}
		F_p\equiv \Legendre{p}{5} \pmod{p}
	\end{displaymath}
\end{theorem}
该定理用于求超大斐波那契数取模。
\subsubsection{二次互反律}
\index{Q!Quadratic Reciprocity Law}
\begin{theorem}
	若$p,q$为不同的奇素数,则
	\begin{displaymath}
		\Legendre{p}{q}\Legendre{q}{p}=(-1)^\frac{(p-1)(q-1)}{4}
	\end{displaymath}
\end{theorem}
此外有两个补充结论:
\begin{theorem}
    \begin{displaymath}
        \Legendre{-1}{p}=(-1)^\frac{p-1}{2}=\left\{\begin{array}{lr}
            1 & \textrm{if~} p\equiv 1\pmod{4}\\
            -1 & \textrm{if~} p\equiv 3\pmod{4}\\
        \end{array}\right.
    \end{displaymath}
\end{theorem}
\begin{theorem}
    \begin{displaymath}
        \Legendre{2}{p}=(-1)^\frac{p^2-1}{8}=\left\{\begin{array}{lr}
            1 & \textrm{if~} p\equiv 1,7\pmod{4}\\
            -1 & \textrm{if~} p\equiv 3,5\pmod{4}\\
        \end{array}\right.
    \end{displaymath}
\end{theorem}
\subsection{二次剩余}
\index{Q!Quadratic Residue}
求解二次剩余即求解下列同余方程:
\begin{displaymath}
	x^2\equiv a \pmod{p}
\end{displaymath}
\subsubsection{欧拉判别准则}
\index{E!Euler's Criterion}
\begin{theorem}[Euler's Criterion]
    若$p$为奇素数且$p\nmid a$,则
    \begin{displaymath}
        \Legendre{a}{p}\equiv a^\frac{p-1}{2}\pmod{p}
    \end{displaymath}
\end{theorem}
\subsubsection{模奇素数}
这里使用ACdreamer介绍的Cipolla随机化算法。
\index{C!Cipolla's Algorithm}
\begin{theorem}
    设$b$满足$\omega=b^2-a$不是模$p$的二次剩余,则
    $x\equiv (b+\sqrt{\omega})^\frac{p-1}{2}\pmod{p}$是
    方程$x^2\equiv a\pmod{p}$的解。
\end{theorem}
证明:

\subsubsection{模奇素数幂}
\subsubsection{模2的幂}
\subsubsection{模合数}

上述内容参考了Miskcoo\footnote{
	[数论]二次剩余及计算方法
	\url{http://blog.miskcoo.com/2014/08/quadratic-residue}
}和ACdreamer\footnote{
    二次同余方程的解
    \url{https://blog.csdn.net/acdreamers/article/details/10182281}
}的博客与百度百科(勒让德符号,二次互反律与欧拉判别准则)。
\subsection{三次剩余}
