\section{二次剩余及其扩展}
\subsection{勒让德符号}
\index{L!Legendre Symbol}
定义勒让德符号:
\begin{displaymath}
	\Legendre{a}{p}=
	\left\{\begin{array}{lr}
		0  & a\equiv 0 \pmod{p}                  \\
		1  & \exists x,x^2\equiv a \pmod{p}      \\
		-1 & \not \exists x,x^2\equiv a \pmod{p}
	\end{array}\right.
\end{displaymath}
勒让德符号是完全积性函数,即
\begin{displaymath}
	\Legendre{ab}{p}=\Legendre{a}{p}\Legendre{b}{p}
\end{displaymath}
\subsubsection{与斐波那契数列的关系}
\begin{theorem}
	若$p$为素数,则
	\begin{displaymath}
		F_{p-\Legendre{p}{5}}\equiv 0 \pmod{p}
	\end{displaymath}
	\begin{displaymath}
		F_p\equiv \Legendre{p}{5} \pmod{p}
	\end{displaymath}
\end{theorem}
该定理用于求超大斐波那契数取模。
\subsubsection{二次互反律}
\index{Q!Quadratic Reciprocity Law}
\begin{theorem}
	若$p,q$为不同的奇素数,则
	\begin{displaymath}
		\Legendre{p}{q}\Legendre{q}{p}=(-1)^\frac{(p-1)(q-1)}{4}
	\end{displaymath}
\end{theorem}
此外有两个补充结论:
\begin{theorem}
    \begin{displaymath}
        \Legendre{-1}{p}=(-1)^\frac{p-1}{2}=\left\{\begin{array}{lr}
            1 & \textrm{if~} p\equiv 1\pmod{4}\\
            -1 & \textrm{if~} p\equiv 3\pmod{4}\\
        \end{array}\right.
    \end{displaymath}
\end{theorem}
\begin{theorem}
    \begin{displaymath}
        \Legendre{2}{p}=(-1)^\frac{p^2-1}{8}=\left\{\begin{array}{lr}
            1 & \textrm{if~} p\equiv 1,7\pmod{4}\\
            -1 & \textrm{if~} p\equiv 3,5\pmod{4}\\
        \end{array}\right.
    \end{displaymath}
\end{theorem}
\subsection{二次剩余}
\index{Q!Quadratic Residue}
求解二次剩余即求解下列同余方程:
\begin{displaymath}
	x^2\equiv a \pmod{p}
\end{displaymath}
\subsubsection{欧拉判别准则}
\index{E!Euler's Criterion}
\begin{theorem}[Euler's Criterion]
    若$p$为奇素数且$p\nmid a$,则
    \begin{displaymath}
        \Legendre{a}{p}\equiv a^\frac{p-1}{2}\pmod{p}
    \end{displaymath}
\end{theorem}
\subsubsection{模奇素数}
这里使用ACdreamer介绍的Cipolla随机化算法。
\index{C!Cipolla's Algorithm}
\begin{theorem}
    设$b$满足$\omega=b^2-a$不是模$p$的二次剩余,则
    $x\equiv (b+\sqrt{\omega})^\frac{p+1}{2}\pmod{p}$是
    方程$x^2\equiv a\pmod{p}$的解。
\end{theorem}
证明:

由推论~\ref{LucasI}可得:
\begin{displaymath}
    (b+\sqrt{\omega})^p\equiv b+\omega^\frac{p-1}{2}\sqrt{\omega}
    \equiv b-\sqrt{\omega}
\end{displaymath}

那么有
\begin{eqnarray*}
    (b+\sqrt{\omega})^{p+1}&\equiv&(b+\sqrt{\omega})^p(b+\sqrt{\omega})\\
    &\equiv&(b-\sqrt{\omega})(b+\sqrt{\omega})\\
    &=&b^2-\omega\\
    &=&a \pmod{p}
\end{eqnarray*}

$b$可以随机选取,因为有一半左右的$\omega$不是模$p$的二次剩余。
然后把$\sqrt{\omega}$当做虚数单位做复数快速幂,实数部分就是答案。
该结果的相反数也是一个解。
\subsubsection{模奇素数幂}
即$p$为奇素数且$(a,p)=1$,求解下列方程:
\begin{displaymath}
	x^2\equiv a \pmod{p^n}
\end{displaymath}
若$(a,p)\neq 1$,令$a=A\cdot p^k$。这里仅考虑$k<n$的非平凡情况。

若$2\nmid k$则$a$不是模$p^n$的二次剩余,假设$a$是模$p^n$的二次剩余,
则存在$r,s\nmid p$满足$(p^r\cdot s)^2\equiv A\cdot p^k\pmod{p^n}$,
那么有$p^k\equiv p^{2r}\pmod{p^n}$,该式与假设矛盾。

若$2\mid k$则将原方程化为$x_0^2\equiv A \pmod{p^{n-k}}$,这里满足开头的假设$(a,p)=1$,
用下面的方法求解$x_0$,答案即为$x_0\cdot p^\frac{k}{2}$。

首先求出$x^2\equiv a \pmod{p}$的解$r$,有$r^2-a=kp$。
进一步推得$(r^2-a)^n\equiv 0 \pmod{p^n}$,将$r^2-a$分解为
$(r+\sqrt{a})(r-\sqrt{a})$,两边快速幂得$(r\pm\sqrt{a})^n=u\pm v\sqrt{a}$
(这两个因式是共轭复数,所以结果也共轭)。由此可得$(r^2-a)^n=u^2-v^2a\equiv 0\pmod{p}$,
因此最终答案为$\frac{u}{v}$。这里的$u,v$均与$p$互质,可以扩欧或求出$\varphi(p^n)$后
快速幂。
\subsubsection{模2的幂}
即求解下列方程:
\begin{displaymath}
	x^2\equiv a \pmod{2^n}
\end{displaymath}
类似地,仅讨论$(a,2)=1$的情况。

\begin{itemize}
    \item 若$n=1$,当且仅当$a=1$时有解$1$;
    \item 若$n=2$,当且仅当$a=1$时有解$\mp 1$;
    \item 若$n>2$,当且仅当$a\equiv 1\pmod{8}$时有四个解。
    当$n=3$时解为$\mp 1,\mp 5$,这里只保留一个解$x=1$用来递推。
    因为若存在$x_0$满足该方程,则$2^n-x_0$,$2^{n-1}-x_0$,$2^{n-1}+x_0$
    也是该方程的解。用递推式$x_k=\frac{A -x_{k-1}^2}{2} \bmod 2^k$求出解。
\end{itemize}
具体证明留坑待补。

\subsubsection{一般解法}
将模数分解为质数幂之积后CRT求解,注意有多个解。

上述内容参考了Miskcoo\footnote{
	[数论]二次剩余及计算方法
	\url{http://blog.miskcoo.com/2014/08/quadratic-residue}
}、ACdreamer\footnote{
    二次同余方程的解
    \url{https://blog.csdn.net/acdreamers/article/details/10182281}
}和Quack\_quack\footnote{
    二次同余方程模合数的一般解法
    \url{https://blog.csdn.net/Quack\_quack/article/details/50189111}
}
的博客与百度百科(勒让德符号,二次互反律与欧拉判别准则)。
\subsection{三次剩余}
首先特判$a=0$和$p\leq 3$的情况。其它处理方法同上文,下面仅讨论$p$为素数的情况。

若$p\equiv 2 \pmod{3}$则方程有唯一解$x\equiv a^\frac{2p-1}{3} \pmod{p}$。
否则当$\Legendre{-3}{p}=1$时才有解,记$\epsilon=\frac{\sqrt{-3}-1}{2}$为
三次单位根,有$\epsilon^3\equiv 1\pmod{p}$。

类似于二次剩余,有三分之一左右的数是三次剩余,且若找到一个解$x$,$x\epsilon,x\epsilon^2$
也是该方程的解。

类比Cipolla随机化算法,随机出一个值$b$满足$b^3-a$是三次非剩余(
当且仅当$a^\frac{p-1}{3}\equiv 1\pmod{p}$时为三次剩余),令单位根
$\omega=\sqrt[3]{b^3-a}$,把域内所有数表示为$\omega^0,\omega^1,\omega^2$
的线性组合,然后答案即为$(b-\omega)^\frac{p^2+p+1}{3}$。

证明:

\begin{lemma}
    $\omega^{p-1}=(t^3-a)^\frac{p-1}{3}=\epsilon \pmod{p}$
\end{lemma}

\begin{lemma}
    $\epsilon(1+\epsilon)=(\epsilon-\epsilon^3)/(1-\epsilon)= -1 \pmod{p}$
\end{lemma}

\begin{eqnarray*}
    (b-\omega)^{p^2+p+1}&=&(b-\omega)(b-\omega)^{p^2}(b-\omega)^p\\
    &\equiv&(b-\omega)(b^{p^2}-\omega^{p^2})(b^p-\omega^p)\\
    &\equiv&(b-\omega)(b^2-b\omega\epsilon-b\omega\epsilon^2+\omega^2)\\
    &=&(b-\omega)(b^2+b\omega+\omega^2)\\
    &=&b^3-\omega^3\\
    &=&a \pmod{p}
\end{eqnarray*}

上述内容参考了skywalkert\footnote{
  寻找模质数意义下的二次剩余与三次剩余
  \url{https://blog.csdn.net/skywalkert/article/details/52591343}
}
与hzq84621\footnote{
    二次剩余和三次剩余相关
    \url{http://hzq84621.is-programmer.com/posts/208648.html}
}的博客。
\subsection{高次剩余}
高次剩余仍然可以使用上文的随机化算法解决,即寻找
$b$满足$(b^n-a)^\frac{p-1}{n}\not \equiv 1$,记$\omega=\sqrt[n]{b^n-a}$,
把数域扩充为$\omega^0,\omega^1,\cdots,\omega^{n-1}$的线性组合,答案为
$(b-\omega)^\frac{\displaystyle \sum_{i=0}^{n-1}{p^i}}{n}$。
然后解可乘以单位$n$次根的幂以生成整个解集,单位$n$次根可以使用求原根得到。
期望时间$n^{n-1}\lg p$。

这里介绍另一种方法:若模$p$有原根$g$,令$x=g^t$,有$g^{nt}\equiv a\pmod{p}$,
变形得$(g^n)^t\equiv a \pmod{p}$,转换为离散对数问题求解,具体方法参见~\ref{BSGS}
节所述内容。

上述内容参考了静听风吟。\footnote{
    数学:二次剩余与n次剩余
    \url{https://www.cnblogs.com/aininot260/p/9709283.html}
}的博客。
\index{*TODO!剩余系列证明与代码}
