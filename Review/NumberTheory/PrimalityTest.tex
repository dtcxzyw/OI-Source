\section{素性测试}
\subsection{Miller Rabin随机性素性测试}
\index{M!Miller–Rabin Primality Test}
\subsubsection{朴素算法}
根据~\ref{FLTS}中所述的费马定理,若要测试$p$是否为素数,
选取基数$a\in [2,p)$,检查$a^{p-1}\equiv 1 \pmod{p}$是否对
所有的$a$均成立。
\subsubsection{Miller-Rabin}
考虑每次随机选取多个$a$,当有$k$个$a$满足时,出错的概率最多为$2^{-k}$。证明详见
算法导论\cite{ITA3}第31.8节定理31.39。
Miller-Rabin沿用了朴素算法的思路,并且使用$witness$测试来代替朴素检查算法以尽可能避免把
{\bfseries Carmichael}数当做素数。

$witness(x,base)$返回$true$当$base$可以证明$x$是合数。
\begin{lstlisting}[title=witness]
bool witness(int x, int base) {
    int end = x - 1;
    int c = countTZ(end);
    int t = powm(base, end >> c, x);
    while(c--) {
        int ct = mulm(t, t, x);
        if(t != 1 && t != x - 1 && ct == 1)
            return true;//case 1
        t = ct;
    }
    return t != 1;//case 2
}
\end{lstlisting}
接下来证明正确性:
\begin{itemize}
	\item 如果从case~1处返回$true$,则说明找到了模$x$意义
            下$1$的一个非平凡平方根。

          \begin{theorem}\label{WITNESST}
              如果p是一个奇素数且$e\geq 1$,则方程

              \begin{equation}\label{sqrm}
                   x^2 \equiv 1 \pmod{p^e}
              \end{equation}
              只有两个解$x=\pm 1$。
	      \end{theorem}

          证明:方程~\ref{sqrm}等价于$p^e|(x-1)(x+1)$。因为$p>2$,所以
          $p|(x-1)$与$p|(x+1)$仅有一个成立(否则$p|((x+1)-(x-1))=2$),
          两个解为$x=\pm 1$。

	      \begin{inference}
		      如果模$n$意义下存在$1$的非平凡平方根,则$n$为合数。
	      \end{inference}

	      证明:定理~\ref{WITNESST}的逆否命题也成立,所以$n$不可能为奇素数的幂,且对于
	      $n=1,2$均不存在非平凡平方根,因此$n$必为合数。

	      根据该推论可得case~1有证据证明$x$为合数。
	\item 如果从case~2处返回$true$,则说明$x$不满足费马定理。
\end{itemize}
以上内容参考了算法导论\cite{ITA3}第31.8节。
\subsubsection{实现细节}
\begin{itemize}
    \item 在MillerRabin之前可先用前几个素数筛掉大部分的合数。
    \item 如果数据范围在4759123141内,即在$uint32\_t$范围内,
    只用2,7,61为基数判断。
    \item 如果数据范围在$10^{16}$内,使用2,3,7,61,24251作为基数,
    唯一的强伪素数为46856248255981。
\end{itemize}

以上内容参考了Matrix67的博客\footnote{
	数论部分第一节:素数与素性测试 | Matrix67: The Aha Moments
	\url{http://www.matrix67.com/blog/archives/234}}。
\subsection{Baillie–PSW素性测试}
\index{L!Baillie–PSW Primality Test}
这个方法在WC2019朱震霆的讲课课件《\sout{简单}数论算法》中被提及。它在$2^64$次方内的
结果完全正确,适用于一般情况。\sout{现在仍然没有发现确定的合数能够通过这个测试,不过这样
的数确实是存在的。}试验表明它比MillerRabin的效率更低,代码更长。

该算法结合了基于2的强Fermat测试(即Miller-Rabin的子过程witness)与强Lucas测试。
虽然这两个测试的伪素数十分多,但是这两个伪素数集合的交的大小(集合大小仍然是正无穷,
这里的大小可以理解为分布密度)要小得多。\sout{unsigned long long内靠谱就行。}

\subsubsection{强Lucas测试}
\index{S!Strong Lucas Probable\\ Prime Test}
\paragraph{雅可比符号}
\index{J!Jacobi Symbol}
雅可比符号是勒让德符号(\ref{Legendre})的推广。它不再要求$p$是奇素数,而仅要求$p$是
大于1的奇数。为了防止误解$p$,这里的$p$使用$n$代替。

雅可比符号具有以下性质:
\begin{eqnarray}
    a\equiv b\pmod{n}&\Rightarrow& \Jacobi{a}{n}=\Jacobi{b}{n}\label{JSR1}\\
    \Jacobi{a}{n}&=&\left\{
    \begin{array}{lr}
        0&(a,n)\neq 1\\
        \pm 1&(a,n)=1
    \end{array}
    \right.\label{JSR2}\\
    \Jacobi{a}{bc}&=&\Jacobi{a}{b}\Jacobi{a}{c}\label{JSR3}\\
\end{eqnarray}

此外,二次互反律,完全积性函数,以及$a=2$时的勒让德符号的性质对于雅可比符号仍然成立。

根据性质~\ref{JSR3}可以得出若$\displaystyle n=\prod{p_i^{e_i}}$,
则$\displaystyle \Jacobi{a}{n}=\prod{\Legendre{a}{p_i}^{e_i}}$。

雅可比符号的计算可在$O(\lg a\lg n)$的时间内解决,记过程为$jacobi(a,b)$:

\begin{enumerate}
    \item 根据性质~\ref{JSR1}可以等价调用$jacobi(a\%b,b)$。
    \item 根据完全积性函数与$a=2$时的性质把$a$的因子2消去。
    \item 若$b=1$,根据完全积性函数的性质,返回1。
    \item 若$(a,b)\neq 1$,根据性质~\ref{JSR2},返回0。
    \item 否则$(a,b)=1$,使用二次互反律调用子过程$jacobi(b,a)$。
\end{enumerate}

这个过程与欧几里得算法很像。

\paragraph{Lucas序列生成}
\index{L!Lucas Sequence}
Lucas序列有参数$P,Q,D$,可由数列$(U,V)$组合而成,在此只单独研究这两个数列。

它们满足以下递推关系:

\begin{eqnarray*}
    U_0&=&0\\
    V_0&=&2\\
    U_{2k}&=&U_kV_k\\
    V_{2k}&=&V_k^2-2Q^k\\
    U_{2k+1}&=&(PU_{2k}+V_{2k})/2\\
    V_{2k+1}&=&(DU_{2k}+PV_{2k})/2
\end{eqnarray*}

除法操作中如果分子为奇数则再加上模数???

由递推关系可以从高位到低位构造出$(U_n,V_n)$。

\paragraph{判定}
若$U_{n-\Jacobi{D}{n}}\not \equiv 0\pmod{n}$,则$n$必为合数。
对于$\Jacobi{D}{n}=-1$,该条件改为$U_{n+1}\not \equiv 0\pmod{n}$。

此外可以检查$V_{n+1}\not \equiv 2Q \pmod{n}$,满足任意一个条件就判定它为合数,
这个几乎不增加计算代价的操作提高了判定合数的概率。
\subsubsection{实现}

算法步骤如下:
\begin{enumerate}
    \item 预筛:使用小素数试除可以筛去绝大多数合数(然而素数判定板子题生成的合数
    基本上是大质数之积)。
    \item 执行基于2的强Fermat测试。也可以使用其它的基,不过2已经经过了大量测试。
    \item 从数列A157142\footnote{参见\url{http://oeis.org/A157142}}
    ($1,-3,5,-7,9,-11,13,-15\cdots$)的第三项开始,找到第一个整数$D$满足
    $\Jacobi{D}{n}=-1$。选择这个数列的原因是易于生成且平均测试次数约为
    3.147755149。注意如果$n$是一个完全平方数,那么找不到满足条件的$D$,可以使用
    二分法或牛顿法快速开平方根预先判断。
    \item 以参数$D,P=1,Q=(1-D)/4$执行强Lucas测试。
\end{enumerate}

模板代码如下:
\lstinputlisting{Source/Templates/BailliePSW.cpp}

正确性证明留坑待补。
\index{*TODO!Baillie-PSW素性测试正确性证明}

上述内容参考了Wikipedia-EN\footnote{
    Jacobi Symbol
    \url{https://en.wikipedia.org/wiki/Jacobi\_symbol}

    Baillie–PSW primality test
    \url{https://en.wikipedia.org/wiki/Baillie-PSW\_primality\_test}

    Lucas pseudoprime
    \url{https://en.wikipedia.org/wiki/Lucas\_pseudoprime}

    Lucas sequence
    \url{https://en.wikipedia.org/wiki/Lucas\_sequence}
}。英文论文参见Lucas Pseudoprimes\cite{BPSW}
(\url{http://mpqs.free.fr/LucasPseudoprimes.pdf})
