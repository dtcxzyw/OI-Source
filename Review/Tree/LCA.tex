\section{最小公共祖先}
\subsection{倍增法}
预处理:DFS时计算每个节点的深度和$2^k$级祖先。

查询:首先将较深节点跳到同一高度,若原节点在一条链上,则
较浅的点为LCA,算法结束。否则按$k$从大到小尽量跳,保持不
跳到同一祖先。最后这两个节点的父亲就是原节点的LCA。

\begin{lstlisting}
int d[size],p[size][20];
void DFS(int u) {
    for(int i=1;i<20;++i)
        p[u][i]=p[u][i-1][i-1];
    for(int i=last[u];i;i=E[i].next) {
        int v=E[i].to;
        if(p[u][0]!=v) {
            p[v][0]=u;
            d[v]=d[u]+1;
            DFS(v);
        }
    }
}
int lca(int u,int v){
    if(d[u]<d[v])std::swap(u,v);
    int delta=d[u]-d[v];
    for(int i=0;i<20;++i)
        if(delta&(1<<i))
            u=p[u][i];
    if(u==v)
        return u;
    for(int i=19;i>=0;--i)
        if(p[u][i]!=p[v][i])
            u=p[u][i],v=p[v][i];
    return p[u][0];
}
\end{lstlisting}
预处理$O(n\lg n)$,查询$O(\lg n)$。
\subsection{树链剖分}
树链剖分后,如果在同一条链上则返回较浅者,否则令链头较深的节点向上跳。

\begin{lstlisting}
int lca(int u,int v) {
    while(top[u]!=top[v]) {
        if(d[top[u]]>d[top[v]])u=p[top[u]];
        else v=p[top[v]];
    }
    return d[u]<d[v]?u:v;
}
\end{lstlisting}

两趟DFS预处理$O(n)$。
由于树链剖分后重链不超过$\lg n$条,所以查询也是$O(\lg n)$的,常数比倍增法小。
\subsection{欧拉序+ST表}
考虑DFS序,显然两个来自节点$i$的不同子树的点的LCA为节点$i$,那么可以在
DFS完一棵子树后加入节点$i$的深度作为隔板,按访问时间戳查询ST表即可。为了
应对在一条链上的情况,同时也为了给节点$i$一个时间戳,在遍历节点$i$之初就插入一个隔板。
\begin{lstlisting}
int L[size],d[size],A[20][size*2],siz=0;
void DFS(int u,int p) {
    A[0][++siz]=d[u];
    L[siz]=u;
    for(int i=last[u];i;i=E[i].next) {
        int v=E[i].to;
        if(v!=p) {
            d[v]=d[u]+1;
            DFS(v,u);
            A[0][++siz]=d[u];
        }
    }
}
void buildST() {
    for(int i=1;i<20;++i) {
        int end=siz-(1<<i)+1,off=1<<(i-1);
        for(int j=1;j<=end;++j)
            st[i][j]=std::min(st[i-1][j],st[i-1][j+off]);
    }
}
int query(int l,int r) {
    int siz=r-l+1;
    int p=ilg2(siz);
    return std::min(A[p][l],A[p][r-(1<<p)+1]);
}
int lca(int u,int v) {
    u=L[u],v=L[v];
    if(u<v)std::swap(u,v);
    return query(u,v);
}
\end{lstlisting}

预处理$O(n\lg n)$,查询$O(1)$。
\subsection{Tarjan}
当查询离线时,可使用Tarjan算法。

\lstinputlisting[title=Tarjan]{Tree/Tarjan.cpp}

原理与欧拉序+ST表法类似,当节点分别在两棵不同的子树时,若另一节点已处理完毕,
他的祖先肯定是LCA(因为LCA处还没有遍历完,未合并到更高的祖先上去)。

\paragraph{优化} 可以把find改为用队列实现,迭代更新父亲。
