\section{可持久化数据结构}
可持久化数据结构的核心思想就是\emph{Copy On Write}(写时复制),当一个对象将
被改变时,简单地复制其整体,未修改的部分仍引用原对象的数据,达到节省拷贝时间与
空间的目的。

可持久化数据结构有主席树(可持久化线段树),可持久化可并堆,可持久化Trie,
可持久化数组,可持久化并查集,可持久化平衡树等。
\subsection{主席树}
用主席树做的经典模型有:
\begin{itemize}
    \item 差分
    \item 对于每一个节点为左节点,维护其右边节点为右节点时的答案
    \item 将某一维离散化后不断插入新数据进行预处理以回答在线询问
\end{itemize}
\subsection{可持久化Trie}
若遇到求区间xor最大值之类的问题,使用可持久化Trie。
\subsection{可持久化数组}
可持久化数组有两种实现:
\begin{itemize}
    \item 块状数组
    \item 主席树
\end{itemize}
可持久化并查集可使用可持久化数组实现。
\subsection{优化}
\subsubsection{标记永久化}
将对整个区间的操作记录在管理此区间的节点,标记不下传,统计时参与计算。
此法节约了$push$的时间且对可持久化友好。
\subsubsection{克隆开关}
若已知按照原方法有一个节点不再被任何时间的数据结构引用时,直接在该节点上修改即可
(当然也可以gc,比较麻烦)。
因此在操作前可以设置一个$enableClone$开关,若为$false$则直接返回原节点即可。
代码如下:
\begin{lstlisting}[title=cloneA]
bool enableClone=true;
int cloneNode(int src) {
    if(enableClone) {
        int id=allocNode();
        T[id]=T[src];
        return id;
    }
    return src;
}
\end{lstlisting}
对于可持久化并查集,若使用路径压缩,则不好判断是否$clone$,在每个节点上记录其被
创建时的时间戳,与当前版本时间戳比较即可。
代码如下:
\begin{lstlisting}[title=cloneB]
int timeStamp=0;
int cloneNode(int src) {
    if(T[src].ts!=timeStamp) {
        int id=allocNode();
        T[id]=T[src];
        T[id].ts=timeStamp;
        return id;
    }
    return src;
}
\end{lstlisting}
此法节约了复制节点时的时间与空间。
