\section{DLX舞蹈链}
\index{D!Dancing Links X}
DLX用来求解精确覆盖问题。

\paragraph{精确覆盖问题} 给定一个01矩阵,求使得每一列恰好有1个1的行集合。
\subsection{X算法}
X算法使用递归+回溯搜索可行解。

算法步骤如下:
\begin{enumerate}
	\item 从矩阵中选取一行;
	\item 将该行和该行所有1对应的列以及与该行冲突的行从矩阵中删除得到一个新矩阵。
	\item 若该矩阵为空矩阵,则跳到步骤4;否则递归求解新矩阵的精确覆盖,若返回false则
	      返回步骤1选取下一行;
	\item 若选取的行全部为1,则返回true,否则返回false。
\end{enumerate}
\subsection{DLX}
递归+回溯使得存储与维护矩阵既麻烦又费时。Donald E.Knuth使用双向链表
来维护矩阵,这个数据结构被称为Dancing Links。它利用了双向链表删除与恢复的方便性。

对于矩阵内的每一个1(此种矩阵一般为稀疏矩阵),维护其上下左右元素标号和自身坐标。
每个元素既是所属行的链表元素,又是所属列的链表元素。每个列的链表还有链头$C_i$(即0行元素),
这些链头又与总链头$head$串在一起,以便检查覆盖情况。

算法步骤如下:

记标示列链表链头$C$为将元素$C$所在列元素以及这些元素所在行元素删除,回标$C$为其逆操作。
\begin{enumerate}
	\item 检查$head.right$是否为自身,若是则覆盖完毕,输出答案栈内所有元素,返回true;
	\item 记$C=head.right$,标示$C$,枚举$C$所在链表内的行$D$:
	      \begin{enumerate}
		      \item 标示元素$D$所在链表行元素对应列链表链头。
		      \item 将其压入答案栈中;
		      \item 递归求解,若返回true则退出,否则逆序回标,枚举下一行。
	      \end{enumerate}
	\item 回标$C$,返回false。
\end{enumerate}

{\bfseries 为了提高搜索效率可以维护每列1的个数,每次选取1个数最少的列遍历。}

板子:
\lstinputlisting{Source/Templates/DLX.cpp}

上述内容参考了万仓一黍的博客\footnote{
	跳跃的舞者,舞蹈链(Dancing Links)算法——求解精确覆盖问题
	\url{http://www.cnblogs.com/grenet/p/3145800.html}
}。
