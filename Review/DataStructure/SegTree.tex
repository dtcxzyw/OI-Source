\section{线段树}
\subsection{技巧}
\subsubsection{全局最优值剪枝}
可以使用全局变量维护自己当前遍历到的最优值,若父节点维护的信息表明管辖范围内不可能
出现更优值,则直接返回减少递归深度。(在kd-tree中比较有效)
\subsubsection{局部最优值剪枝}
例如维护区间max值时,顺便维护区间min值,modify时发现操作无效直接退出。
\subsubsection{不使用子树信息合并更新}

\subsubsection{标记永久化}
直接将对整个区间的操作存到标记中而不下放,统计时加回去,减少空间占用与。
\subsection{zkw线段树}
\index{Z!zkw's Segment Tree}
留坑待填。
\index{*TODO!zkw线段树}
\subsection{势能分析线段树}
对于某些无法打标记的区间操作(例如区间元素开根号),若该操作对某个元素施加有限次操作就会使
区间元素不再变化,同样可以使用线段树。每次区间操作暴力修改,合并时维护下次操作是否可以被跳过。

更多应用需要SegmentTreeBeats,留坑待填。
\index{*TODO!Segment Tree Beats}
\index{S!Segment Tree Beats}
\subsection{线段树分治}
对于可离线线段树套其它数据结构的树套树问题,有时会因为空间不够导致无法使用树套树,并且
树套树代码量大,不易调试。

既然这类问题可离线,可以考虑将所有询问同时在线段树上移动。对询问进行分治,每次处理分治区间
时,先建出查询数据结构,然后对完全包含这段区间的询问进行查询并更新答案,再清空数据结构,
将其它询问分治到左右子区间中求解。使用分治可以保证任意时刻只有单个区间的查询数据结构。

\subsubsection{例题 [FJOI2015]火星商店问题}
典型的线段树套可持久化Trie问题,使用线段树分治毫无空间压力,时间复杂度与原做法相同。

参考代码:
\lstinputlisting{Source/Source/SegmentTree/P4585.cpp}
\subsection{线段树优化建图}
对于一个点到一个或多个连续区间内的点有连边且区间内的边权相等,考虑使用线段树优化建图。
即使用线段树的上层节点代表管辖区间内的所有节点,建树时上层节点向左右儿子连权值为0的边
({\bfseries 注意上层节点区分入点与出点,与儿子连有向边}),建图时类似modify操作连边,边数由
$O(mn)$降为$O(m\lg n)$。上层节点起到了``捆线带''的作用。

例题:CF786B Legacy

线段树优化建图+裸最短路。

\lstinputlisting{Source/Source/SegmentTree/CF786B.cpp}
\subsection{猫树}
当无修改,区间查询次数多,区间信息仅支持结合律与快速合并时(不可减,不可重叠),
猫树可以以更多预处理时间与空间的代价使单次查询复杂度变为$O(1)$。

这种情况下一般使用线段树解决,但是线段树查询时需要合并$O(\lg n)$次信息,考虑如何将合并
次数降到$O(1)$。以下不讨论$l=r$的平凡情况。由于区间端点$l,r$在线段树上对应了某个节点
的mid,可以考虑预处理所有节点的mid到管辖范围中每个点的信息(左右方向在$m$处一开一闭),
查询时快速找到某个mid满足$l,r$在它的管辖区间内且分别在它的两侧,利用预处理的$[l,mid)$与
$[mid,r]$的信息,$O(1)$合并回答询问。

接下来考虑如何寻找mid对应的线段树节点id。事实上这个节点就是$l,r$对应节点的LCA。
{\bfseries 猫树使用的是zkw线段树的节点标号方式,}$l,r$的节点编号的LCP就是mid的
节点编号。使用位运算$id_l>>(1+\lg2(id_l\oplus id_r))$可以实现$O(1)$求LCP。

参考代码(LOJ\#6057. 「HNOI2016」序列 数据加强版):
\lstinputlisting{Source/Source/SegmentTree/LOJ6057.cpp}

\subsubsection{扩展}
虽然必须``无修改'',但是猫树结合二进制分组可以支持``向后插入'',可以用来优化DP。

对于树上链信息询问,如果使用DFS序+树链剖分会带来$O(\lg n)$的复杂度。考虑点分治,
预处理重心到子树内每个点的链信息。由于区间信息合并不可重叠,要分别预处理包括重心与不包括重心
的信息。查询链对应的重心时在点分树上查询LCA。为了得到$O(1)$的查询复杂度,需要$O(1)$LCA
与$O(1)$hashTable求节点到重心的信息对应位置。

参考代码(LOJ\#2013. 「SCOI2016」幸运数字):

这题使用点分治+猫树有点杀鸡用牛刀的感觉。因为本题可离线。

\lstinputlisting{Source/Source/'Point-Based Partition'/LOJ2013.cpp}

上述内容参考了immortalCO的博客\footnote{
    一种高效处理无修改区间或树上询问的数据结构(附代码)
    \url{http://immortalco.blog.uoj.ac/blog/2102}
}。
