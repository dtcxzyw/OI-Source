\section{单/多源最短路}
\subsection{SPFA}
最坏时间复杂度$O(VE)$。
\paragraph{严重警告}
是SPFA让我在NOI2018Day1中滚粗的。

SPFA优化的思路基本上是使队列尽可能接近优先队列。
\subsubsection{SLF优化}
插入队列时,若当前节点比队首距离更短则插入队首,否则插入队尾。
\subsubsection{LLL优化}
每次选取一个距离小于队列距离平均值的节点松弛。
\subsubsection{堆优化}
沿着LLL优化的思路:使用距离尽可能小的节点来松弛。使用堆就可以维护了。
\subsubsection{SLF带容错}
令边权和为$W$,若当前节点比队首距离多$\sqrt{W}$才插入队尾。在边权和
小的时候表现不错。
\subsubsection{mcfx优化}
在第$[2,\sqrt{V}]$次访问当前节点时插入队首,否则插入队尾。网格图
上表现优秀。搭配SLF带容错优化效果更佳。
\subsubsection{SLF+Swap}
插入队尾后若队首大于队尾则交换首尾。

更多优化与Hack参见\url{https://www.zhihu.com/question/292283275}。
\index{*TODO!卡SPFA}
\subsection{Dijkstra}
使用二叉堆的最坏时间复杂度$O((V+E)\lg V)$,在稀疏图中表现良好。
稠密图可以使用$O(V^2)$暴力。
\paragraph{强烈安利}
大型考试还是使用Dijkstra算法好了。

注意Dijkstra中不方便的堆修改可以改为加入新点(标号+松弛距离)的形式,
出堆时与数组中记录的值比较,匹配才进行松弛。
\subsection{Johnson算法}
留坑待补。
\index{*TODO!Johnson算法}
