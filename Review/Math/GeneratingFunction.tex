\section{生成函数}\label{GF}
\index{G!Generating Function}
\CJKsout{生成函数为形式幂级数,假装其收敛。}
\subsection{普通型生成函数}
若数列$<f_0,f_1,\cdots>$,若形式幂级数
\begin{displaymath}
	F(x)=\sum_{i=0}^\infty{f_ix^i}
\end{displaymath}
则称$F(x)$为数列${f_n}$的普通型生成函数(OGF,Ordinary Generating Function)。

OGF之积对应数列的卷积,不考虑元素之间的顺序,适用于组合计算。

常见OGF:

\begin{tabular}{|l|l|}
	\hline
	数列                                   & OGF                                          \\
	\hline
	$<1,1,1,\cdots>$                       & $\frac{1}{1-x}$                              \\
	\hline
	$<0,1,2,3,\cdots>$                       & $\frac{x}{(1-x)^2}$(将$\frac{1}{1-x}$微分后平移) \\
	\hline
	$<1.-1.1,-1,\cdots>$                   & $\frac{1}{1+x}$($(1+x)\frac{1}{1+x}=1$)    \\
	\hline
	$<1,c,c^2,\cdots>$                     & $\frac{1}{1-cx}$                             \\
	\hline
	$<0,1,\frac{1}{2},\frac{1}{3},\cdots>$ & $\ln\frac{1}{1-x}$                           \\
	\hline
	$<0,1,1,2,3,5,\cdots>$ & $\frac{x}{1-x-x^2}$\\
	\hline
\end{tabular}
\subsection{指数型生成函数}
若数列$<f_0,f_1,\cdots>$,若形式幂级数
\begin{displaymath}
	F(x)=\sum_{i=0}^\infty{f_i\cdot\frac{x^i}{i!}}
\end{displaymath}
则称$F(x)$为数列${f_n}$的指数型生成函数(Exponential Generating Function)。

EGF之积对应数列卷积乘以组合数,适合计算排列。注意EGF在计算时存储的是$\frac{a_i}{i!}$,
初始化EGF时要将系数除以$i!$。

常见EGF:

以下生成函数对应的数列可由泰勒展开推导。

\begin{tabular}{|l|l|}
	\hline
	数列               & EGF      \\
	\hline
	$<1,1,1,\cdots>$   & $e^x$    \\
	\hline
	$<0,1,2,3,\cdots>$ & $xe^x$   \\
	\hline
	$<1,c,c^2,\cdots>$ & $e^{cx}$ \\
	\hline
\end{tabular}

\subsection{贝尔数}
贝尔数$B_n$是大小为$n$的集合的划分方案数。

贝尔数的前几项为:$1, 1, 2, 5, 15, 52, 203, 877,\cdots$(参见OEIS-A000110\\
\url{https://oeis.org/A000110})

$B_n$满足等式:
\begin{eqnarray*}
	B_0=B_1&=&1\\
	B_{n+1}&=&\sum_{i=0}^n{\binomial{n}{i}B_i}\\
	B_n&=&\sum_{i=1}^n{\stirlingA{n}{i}}
\end{eqnarray*}

\subsubsection{因子分解应用}
若整数$\displaystyle n=\sum_{i=1}^k{p_i}$,
则整数$n$可以分解为$B_k$种因子的乘积。
\subsubsection{贝尔数计算}
\paragraph{DFS}若要得到不同方案的状态表示,限制下阶段DFS的集合大小以避免重复计算。
\paragraph{贝尔三角形}
与推杨辉三角形类似,递推方法如下:
\begin{eqnarray*}
	A[0][1]&=&1\\
	A[n][1]&=&A[n-1][n]\\
	A[n][m]&=&A[n][m-1]+A[n-1][m-1]\\
	B_n&=&A[n][0]
\end{eqnarray*}
使用滚动数组优化空间。
\paragraph{生成函数法}
利用Bell数的生成函数$e^{e^x-1}$,计算多项式exp。
\subsection{自然数幂求和}

记$B_n$为第$i$个伯努利数,其中$B_0=1$。
首先使用公式
\begin{displaymath}
	\sum_{i=1}^n{n^k}=\frac{1}{k+1}\sum_{i=1}^{k+1}{\binomial{k+1}{i}B_{k+1-i}(n+1)^i}
\end{displaymath}
只要预处理出逆元和伯努利数,就可以$O(k)$求出答案。

至于伯努利数,可以使用
\begin{displaymath}
	\sum_{i=0}^n{\binomial{n+1}{i}B_i}=0
	\Leftrightarrow
	B_n=-\frac{1}{n+1}\sum_{i=0}^{n-1}{\binomial{n+1}{i}B_i}
\end{displaymath}
$O(k^2)$预处理。

当$k$较大时,使用其对应的生成函数展开
\begin{eqnarray*}
    F(x)&=&\sum_{i=0}^\infty{B_i\cdot\frac{x^i}{i!}}\\
    &=&\frac{x}{e^x-1}\\
	&=&\frac{1}{\displaystyle \sum_{i=0}^\infty{\frac{x^i}{(i+1)!}}}
\end{eqnarray*}
$O(k\lg k)$多项式求逆解决。

以上内容参考了\_rqy\footnote{生成函数简介 | \_rqy's Blog
	\url{https://rqy.moe/Algorithms/generating-function/}
}。
、acdreamers\footnote{伯努利数与自然数幂和
	\url{https://blog.csdn.net/acdreamers/article/details/38929067}
}
和Miskcoo\footnote{多项式求逆元 – Miskcoo's Space
	\url{http://blog.miskcoo.com/2015/05/polynomial-inverse}
}
的博客。
\subsection{生成函数处理背包问题}
生成函数法用来计算大规模背包的方案数。
\subsubsection{多重背包}
每个物品的生成函数为$\displaystyle \sum_{k=0}^{c_i}{x^{kw_i}}$,化简为
$\frac{1-x^{(c_i+1)w_i}}{1-x^{w_i}}$。由于乘积指数项的大小不可控,考虑将
乘积变为取对数求和。设答案为的生成函数为$A(x)$,那么有
\begin{displaymath}
\ln A(x)=\sum_{i=1}^n{\ln(1-x^{(c_i+1)w_i})-\ln (1-x^{w_i})}
\end{displaymath}

使用泰勒展开$\displaystyle \ln(1-x^k)=-\sum_{i=1}^\infty{\frac{x^{ki}}{i}}$得到
$ln A(x)$后多项式exp即为答案。注意$\ln (1-x^k)$要合并同类项,否则一堆$1-x$可以轻易将其
卡掉。合并同类项后根据调和级数可知构造多项式的复杂度为$O(n\lg n)$,$n$为背包容量。
\subsubsection{01背包}
每个物品的生成函数为$1+x^{w_i}$,总方案的生成函数为这些函数之积。不过由于指数上界
$\sum{w_i}$不可控,只能转化为多重背包解决。
\subsubsection{完全背包}
背包的容量已知,单个物品最多装入的个数是有限的,因此完全背包可转化为多重背包。

该讨论源自AntiLeaf的题目「Antileaf's Round」咱们去烧菜吧\footnote{
	AntiLeaf's Round 题解
	\url{https://loj.ac/article/304}
}。
\subsection{递推式与生成函数之间的转换}
递推式满足$\displaystyle f_n=\sum_{i=1}^k{c_if_{n-i}}$,记生成函数为$F$。
\subsubsection{递推式$\rightarrow$生成函数}
考虑生成函数的$x^n$的系数,它的系数减去其它系数的线性加权和等于0。为了做整个函数的加减,
使用$c_ix_i$进行平移,最后有$F*(1-\displaystyle \sum_{i=1}^k{c_ix^i})=$余项的
形式,容易解出$F$。
\subsubsection{生成函数$\rightarrow$递推式}
由正推的过程发现分母部分为递推式的系数,而分子部分指示了递推式的初始项。

注意如果推出了式子后为一元二次方程的形式,不一定使用求根公式,牛顿迭代法直接求解
可以避免系数$a$无逆的情况。
\subsection{快速求指定数集的幂和}
该需求来自LuoguP4705 玩游戏。

给定数集$a_1,a_2,\cdots,a_n$,求$k=1,2,\cdots,t$,
$f(k)=\displaystyle \sum_{i=1}^n{a_i^k}$的值。

考虑令$F(x)$为数列$f$的生成函数,由于对于单个的$a_i$,它的幂的生成函数为
$\frac{1}{1-a_ix}$,而$F(x)$就是它的总和,即
$\displaystyle \sum_{i=1}^n{\frac{1}{1-a_ix}}$,但这个式子仍然不好算。
注意到$(\ln (1-a_ix))'=\frac{-a_i}{1-a_ix}$,令
$G(x)=\displaystyle \sum_{i=1}^n{\frac{-a_i}{1-a_ix}}$,有$F(x)=-xG(x)+n$。
将对数函数带回$G(x)$并化简得
$\displaystyle G(x)=\left(\ln\left(\prod_{i=1}^n{(1-a_ix)}\right)\right)'$。
使用分治NTT很容易计算出乘积。
\subsection{概率生成函数}
例题:CTSC2006歌唱王国

有时需要求到达目标状态的期望步数,但例题无法使用高斯消元通过(KMP的转移不好快速处理)。

此时可以大胆地设一个生成函数$F(x)=\displaystyle \sum_{i=0}^\infty{Pr[step=i]x^i}$,
第$i$项表示结束步长为$i$的概率,那么我们要求的期望就是$F'(1)$。同时根据概率的性质,有
$F(1)=1$。

再设一个生成函数$G(x)$,第$i$项表示未结束步长为$i$的概率。考虑从未结束状态转移一步,
可知$F(x)+G(x)=xG(x)+1$。

然后再构造一个等式,强制其结束的暴力方法就是直接加入模式串,其生成函数就是
$G(x)\cdot (\frac{1}{n})^m$。这种方法可能导致在已结束的母串后继续加字符,这种情况
可以从$F(x)$转移。考虑转移的条件,发现模式串的末尾$i$个字符必须与开头$i$个字符匹配,
再接上$m-i$个字符就能等价于强制结束的情况。记末尾$i$个字符与开头$i$个字符的匹配情况
为$a_i$,其生成函数为$\displaystyle \sum_{i=1}^m{a_i\cdot F(x)\cdot
 (\frac{1}{n})^{m-i}}$。$i\neq 0$是因为此类状态是从还未结束的状态开始转移的。

继续推导式子,对$F(x)+G(x)=xG(x)+1$求导,可得$F'(x)+G'(x)=xG'(x)+G(x)$,带入$x=1$
得$F'(1)=G(1)$。继续带入$x=1$求出$G(1)=\displaystyle \sum_{i=1}^m{a_iF(1)n^i}$。
又因为$F(1)=1$,所以可以$O(n)$求解$F'(1)$(Hash/KMP均可)。

启示:生成函数有时只是推算式子的工具,并不需要实际卷积。

该内容参考了水滴的题解\footnote{
	P4548 [CTSC2006]歌唱王国 题解\\
	\url{https://www.luogu.org/problemnew/solution/P4548}
}
