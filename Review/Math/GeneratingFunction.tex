\section{生成函数}
\index{G!Generating Function}
{\bfseries 生成函数为形式幂级数,不考虑其是否发散或收敛。}
\subsection{普通型生成函数}
若数列$<f_0,f_1,\cdots>$,若形式幂级数
\begin{displaymath}
    F(x)=\sum_{i=0}^\infty{f_ix^i}
\end{displaymath}
则称$F(x)$为数列${f_n}$的普通型生成函数(OGF,Ordinary Generating Function)。

OGF之积对应数列的卷积,适用于组合计算。

常见OGF:
\begin{tabular}{|l|l|}
    \hline
    数列&OGF\\
    \hline
    $<1,1,1,\cdots>$&$\frac{1}{1-x}$\\
    \hline
    $<1,2,3,\cdots>$&$\frac{x}{(1-x)^2}$(将$\frac{1}{1-x}$微分)\\
    \hline
    $<1.-1.1,-1,\cdots>$&$\frac{1}{1+x}$($(1+x)\frac{1}{1+x}=1$)\\
    \hline
\end{tabular}
\subsection{指数型生成函数}
若数列$<f_0,f_1,\cdots>$,若形式幂级数
\begin{displaymath}
    F(x)=\sum_{i=0}^\infty{f_i\cdot\frac{x^i}{i!}}
\end{displaymath}
则称$F(x)$为数列${f_n}$的指数型生成函数(Exponential Generating Function)。

EGF之积对应数列卷积乘以组合数,适合计算排列。

\subsection{贝尔数}
\subsection{自然数幂求和}
以上内容参考了\_rqy的博客\footnote{生成函数简介 | \_rqy's Blog\\
    \url{https://rqy.moe/\%E7\%AE\%97\%E6\%B3\%95/generating-function/}
}。
