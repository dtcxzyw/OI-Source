\section{线性时间复杂度整点插值}
给定$k$次多项式$P(x)$在$P(0),P(1),\cdots,P(k)$上的值,
求$P(n),n \in \mathbb{N}$。

考虑$\binomial{x}{i}$是一个关于$x$的$i$次多项式,
所以可以使用$\binomial{x}{0},\binomial{x}{1},\cdots,\binomial{x}{k}$
进行线性组合(因为它们的次数两两不同,线性无关),即
\begin{equation}\label{LIP1}
    P(x)=\sum_{i=0}^k{\binomial{x}{i}p(i)}
\end{equation}
当$x\leq k$时,$P(i)$可以表示为
\begin{equation*}
    P(x)=\sum_{i=0}^x{\binomial{x}{i}p(i)},x\leq k
\end{equation*}
然后套推论~\ref{BII}得:
\begin{equation}\label{LIP3}
    p(x)=\sum_{i=0}^x{(-1)^{x-i}\binomial{x}{i}P(i)},x\leq k
\end{equation}
将~\ref{LIP3}代回~\ref{LIP1}得:
\begin{eqnarray*}
    P(x)&=&\sum_{i=0}^k{\binomial{x}{i}\sum_{j=0}^i{
        (-1)^{i-j}\binomial{i}{j}P(j)}}\\
    &=&\sum_{j=0}^k{P(j)\sum_{i=j}^k{(-1)^{i-j}\binomial{x}{i}
    \binomial{i}{j}}} \textrm{~(提出$P(j)$)}\\
    &=&\sum_{j=0}^k{P(j)\sum_{i=0}^{k-j}{(-1)^i\binomial{x}{i+j}
    \binomial{i+j}{j}}}\\
    &=&\sum_{j=0}^k{\binomial{x}{j}P(j)\sum_{i=0}^{k-j}{
        (-1)^i\binomial{x-j}{i}}}\\
    &=&\sum_{j=0}^k{\binomial{x}{j}P(j)
        \binomial{k-x}{k-j}}\textrm{~(根据定理~\ref{BSum})}\\
    &=&\sum_{j=0}^k{(-1)^{k-j}\binomial{x}{j}
        \binomial{x-j-1}{k-j}P(j)}\textrm{~(根据引理~\ref{BSL})}
\end{eqnarray*}
\begin{displaymath}
    \textrm{其中~} \binomial{x}{j}\binomial{x-j-1}{k-j}=
    \frac{x!}{(x-j)(x-k-1)!}\cdot\frac{1}{j!(k-j)!}
\end{displaymath}

对组合数部分划分后,左边部分的$\frac{x!}{(x-k-1)!}$是$k+1$个值之积,$x-j$是其中的值,
$O(k)$预处理前缀积与后缀积,$O(1)$合并。
右边部分使用线性推逆元$O(k)$完成。

以上内容参考了Miskcoo的博客\footnote{
特殊多项式在整点上的线性插值方法 – Miskcoo's Space\\
\url{http://blog.miskcoo.com/2015/08/special-polynomial-linear-interpolation}
}。
