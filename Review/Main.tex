\documentclass[10pt,AutoFakeBold,AutoFakeSlant]{book}
%\usepackage{titletoc}
%\usepackage{titlesec}
%\usepackage{ctexcap}
\usepackage{xeCJK}
\setCJKmainfont{SimSun}
\usepackage[english]{babel}
\usepackage[T1]{fontenc}
\usepackage{listings}
\usepackage{xcolor}
\lstset{
    columns=fixed,
    frame=none,
    backgroundcolor=\color[RGB]{255,255,255},
    keywordstyle=\color[RGB]{40,40,255},
    commentstyle=\it\color[RGB]{0,96,96},
    stringstyle=\rmfamily\slshape\color[RGB]{128,0,0},
    showstringspaces=false,
    escapeinside=``,
    language=c++,
    morekeywords={alignas,continute,friend,register,true,alignof,decltype,goto,
    reinterpret_cast,try,asm,defult,if,return,typedef,auto,delete,inline,short,
    typeid,bool,do,int,signed,typename,break,double,long,sizeof,union,case,
    dynamic_cast,mutable,static,unsigned,catch,else,namespace,static_assert,using,
    char,enum,new,static_cast,virtual,char16_t,char32_t,explict,noexcept,struct,
    void,export,nullptr,switch,volatile,class,extern,operator,template,wchar_t,
    const,false,private,this,while,constexpr,float,protected,thread_local,
    const_cast,for,public,throw}
}
\usepackage{amssymb}
\usepackage{amsmath}
\usepackage[hyperfootnotes=true]{hyperref}
\usepackage{tabularx}
\usepackage{url}
\usepackage{makeidx}
\usepackage[numbib,numindex,chapter]{tocbibind}
\usepackage{lastpage}
\usepackage{array}
\makeindex
\pagestyle{headings}
\begin{document}
\newtheorem{theorem}{定理}[chapter]
\newtheorem{lemma}[theorem]{引理}
\newtheorem{property}{性质}[chapter]
\newtheorem{inference}[theorem]{推论}
\title{OI知识点复习笔记}
\author{dtcxzyw}
\frontmatter
\maketitle
\chapter{前言}
本人写复习笔记的目的有两个:
\begin{itemize}
	\item 系统地复习知识点并挖掘一些有用的性质。
	\item 学会使用\LaTeX{}。
\end{itemize}
当前共\pageref{LastPage}页,约\input{../latex/charcnt}字。
%\renewcommand\contentsname{目录}
\tableofcontents
\mainmatter
\chapter{博弈论}
\section{SG函数}
一切Impartial Combinatorial Games都等价于Nim游戏,可以使用SG函数解决。

该类游戏拥有如下特征:

\begin{itemize}

	\item

\end{itemize}

\section{Nim系列游戏}

\subsection{Nim游戏}

普通Nim游戏的定义:
有两个玩家轮流从许多堆中移除对象。在每个回合中,玩家选择一个非空的堆,可以移除任何数量
的对象,但至少移除一个对象。无法操作的玩家为败者。

此类游戏可看做是Bash游戏的特殊化。

\begin{Theorem}
	$SG_{Nim}(x)=x$
\end{Theorem}

证明略。

\subsection{Bash游戏}

Bash游戏与普通Nim游戏的区别是增加了每次最多移除k个对象的限制。

\begin{Theorem}
	$SG_{Bash}(x)=x~mod~(k+1)$
\end{Theorem}

证明略。

\subsection{NimK游戏}

NimK游戏与普通Nim游戏的区别是每次可以从不超过k个堆中移除任意数目对象。

\begin{Theorem}\label{NimK}
	将每堆对象的数目拆位,若每位上1的个数mod(k+1)均为0,则必败,反之必胜。
\end{Theorem}

记忆:普通Nim游戏可理解为mod 2的情况。

算法正确性证明:



定理~\ref{NimK}得证。

\subsection{Anti Nim}

不能操作的玩家胜利。

\begin{Theorem}\label{AntiNim}
	先手必胜当且仅当满足以下条件之一:
	\begin{enumerate}
		\item $SG(x)=0$ 且所有堆的对象数都为1
		\item $SG(x)\not=0$ 且至少有一堆对象数大于1
	\end{enumerate}

\end{Theorem}

证明:
定义对象数为1的叫A堆,大于1的叫B堆。

\begin{enumerate}
	\item 若所有堆均为A堆,则奇数堆先手必败,反之必胜。
	\item 若B堆数等于1,显然$SG(x)\not=0$,则可根据堆的总数确定取该堆的数目,
	      使下一状态为情况1的奇数堆,所以先手必胜。
	\item 若B堆数大于1,则
	      \begin{enumerate}
		      \item 若$SG(x)=0$,则必须留下超过2个B堆并使$SG(x')\not=0$,否则会使
		            对方进入情况2的必胜态。
		      \item 若$SG(x)\not=0$,则根据Nim游戏的理论(必胜态->必败态),存在一种方法转移至情况3的子情况1。
	      \end{enumerate}
	      若玩家处于情况3的子情况2中,则可以在有限次回合内使对方无法转移至子情况2,
	      因此该状态为必胜态。
\end{enumerate}

定理~\ref{AntiNim}得证。

\subsection{阶梯博弈(Staircase Nim)}


\subsubsection{例题}

Luogu P3480 [POI2009]KAM-Pebbles\footnote{\url{https://www.luogu.org/problemnew/show/P3480}}

对于这题可将原条件通过差分转换为阶梯博弈模型($A_i \geq A_i-1 \Leftrightarrow
	A_i-A_i-1 \geq 0$)。

\lstinputlisting[title=Luogu P3480]{Source/'Game Theory'/3480.cpp}

出题灵感:Anti BashK游戏

以上内容参考了forezxl\footnote{anti-Nim游戏(反Nim游戏)简介
	\url{https://blog.csdn.net/a1799342217/article/details/78274410}}和
hehedad\footnote{关于nimk类型博弈的详细理解与解释
	\url{https://blog.csdn.net/chenshibo17/article/details/79783523}}的博客。

\section{Alpha-Beta剪枝}

\section{本章注记}
更多博弈论模型参见~博弈游戏的各种经典模型(备忘) - Randolph87 - 博客园
 \url{http://www.cnblogs.com/Randolph87/p/5804798.html},待补充。
\index{TODO!补充博弈论经典模型}

\chapter{网络流}
\minitoc
\section{二分图}
\index{B!Bipartite Graph}
\subsection{二分图判定}
\begin{property}
	二分图中不存在奇环。
\end{property}

如果存在奇环,则必有一条边的端点属于同一集合。所以可以使用DFS染色来判定二分图,
遇到矛盾则退出。

\lstinputlisting[title=BGJudge.cpp]{NetworkFlows/BGJudge.cpp}

\subsection{二分图最大匹配}

\subsubsection{匈牙利算法}
\index{H!Hungarian Algorithm}

匈牙利算法的主要步骤就是遍历左集合的每一个顶点,使得其尽可能找到一个匹配。
要为该顶点找到一个匹配,首先遍历边,如果右顶点已经有匹配,则递归尝试让该
匹配点重新找一个匹配,如果右顶点无匹配或者更换匹配成功,则这条边是一个匹配。

原则:有机会上,没机会创造机会也要上。
\footnote{Dark\_Scope 趣写算法系列之--匈牙利算法
	\url{https://blog.csdn.net/dark\_scope/article/details/8880547}}

感性的算法的正确性证明:每次递归时匹配数只增不减,且递归有权修改整个连通块
的着色情况。(似乎并没有什么说服力)。

匈牙利算法的时间复杂度为$O(VE)$,每次尝试匹配的复杂度为$O(E)$。

\index{*TODO!匈牙利算法标准描述与正确性证明}

\subsubsection{Greedy Matching}
可以先遍历一次图,贪心地连边,以减少尝试拆开匹配边的次数。
在图很大的时候有加速效果。

该方法参考了江任捷的演算法筆記\footnote{
    演算法筆記 - Matching
    \url{http://www.csie.ntnu.edu.tw/\~u91029/Matching.html\#4}
}。
\subsubsection{Hopcroft–Karp Algorithm}
\index{H!Hopcroft–Karp Algorithm}
暂时先坑着\CJKsout{为什么不写Dinic呢}。
\index{*TODO!Hopcroft–Karp算法}

\subsubsection{例题}

Luogu P1129 [ZJOI2007]矩阵游戏
\footnote{\url{https://www.luogu.org/problemnew/show/P1129}}

首先用二分图最大匹配找到n个不同行且不同列的黑格子(置换矩阵P),然后就可以操作得到
目标矩阵(单位矩阵I)了。

\lstinputlisting[title=Luogu P1129]{Source/Unclassified/Done/1129.cpp}

\subsection{二分图最大权匹配 Kuhn-Munkras Algorithm}
\index{K!Kuhn-Munkras Algorithm}
\CJKsout{先用费用流做吧,暂时先坑着。}
\subsubsection{起步}
维护每个左/右顶点的权值(称为顶标),所有节点的顶标和为答案上界。
令每个左顶点的顶标为出边边权最大值,右顶点顶标为0。

对每个顶点运行匈牙利算法,若左右顶点顶标之和等于边权,则考虑连边;
若无法为当前点找到匹配,则将访问到的左顶点顶标-1,右顶点顶标+1,
等价于使答案上界-1(DFS访问树中的叶子必为左顶点),重新为该点寻找匹配。
把任意二分图当做完全二分图(不存在的边权值为0),迭代必定会结束。

这种做法能够保证在找到最大匹配的情况下使权值和最大。
\subsubsection{优化1}
可以发现在左-1右+1后,原先边权等于左右顶点顶标之和的边仍然被经过,
一个简单的思路是一次性突破``瓶颈'',即令下次增广时终点位置处的某条边从
不可连边变为可连边,每次DFS增广时维护(顶标和-边权)的最小值$d$,
若匹配失败则左$-d$右$+d$。

这才是复杂度比较靠谱的算法($O(n^3)$)。
\subsubsection{优化2}
在匹配每个点时,初始化所有右顶点的松弛函数$slack$为$\infty$,然后
DFS时$slack$维护(顶标和-边权)的最小值。若匹配失败则令$d$为未访问右
顶点的$slack$函数最小值,左$-d$右$+d$,同时未访问节点的$slack-=d$。

该优化的复杂度不变,但实测该方法比优化1的效率更高(3x)。
\subsubsection{优化3}
考虑记录其增广时的路径,然后将递归算法转换为非递归算法。
\begin{lstlisting}
int w[size][size],lh[size],rh[size],pair[size],
    pre[size],slack[size];
bool flag[size];
void aug(int s) {
    reset(flag);
    reset(pre);
    reset(slack,0x3f);
    pair[0]=s;
    int u=0;
    do {
        int v=pair[u],minh=inf,nxt;
        flag[u]=true;
        // 再次DFS后新访问到了点u和它的匹配点
        // 为点v找新匹配点
        for(int i=1;i<=n;++i)
            if(!flag[i]){
                int delta=lh[v]+rh[i]-w[v][i];
                if(delta<slack[i])
                    slack[i]=delta,pre[i]=u;
                    //点i的匹配点有可能置换为u的匹配点,
                    //以腾出u的匹配点的空位
                if(minh>slack[i])
                    minh=slack[i],nxt=i;//点i下次将被访问
            }
        //松弛
        for(int i=0;i<=n;++i)
            if(flag[i])lh[pair[i]]-=minh,rh[i]+=minh;
            else slack[i]-=minh;
        u=nxt;
    } while(pair[u]);//直到找到未匹配点为止
    //置换匹配
    while(u) {
        int p=pre[u];
        pair[u]=pair[p];
        u=p;
    }
}
int KM(int n) {
    for(int i=1;i<=n;++i) {
        int maxh=0;
        for(int j=1;j<=n;++j)
            maxh=std::max(maxh,w[i][j]);
        lh[i]=maxh;
    }
    reset(rh);
    reset(pair);
    for(int i=1;i<=n;++i)
        aug(i);
    int res=0;
    for(int i=1;i<=n;++i)
        res+=w[pair[i]][i];
    return res;
}
\end{lstlisting}
实测该方法比优化2的效率更高(2x)。
\index{*TODO!解释KM算法优化的合理性}
\subsection{二分图常见模型}
\subsubsection{最小点覆盖}
\index{K!König's theorem}
\begin{theorem}[König's Theorem]
	最小点覆盖数=最大匹配数。
\end{theorem}

使用反证法证明:如果有一条边两端顶点都不在最大匹配上,那么这条边可以进入最大匹配
成为一个更大的匹配边集,所以与最大匹配的假设矛盾。

\subsubsection{最大独立集}

\begin{theorem}
	最大独立集大小=顶点数-最小点覆盖数=顶点数-最大匹配数
\end{theorem}

证明:容易发现去掉二分图中的最小点覆盖可得到一个独立集(若其不是独立集,则说明存在一条
边未被覆盖,与点覆盖的定义矛盾)。尝试以此独立集为基础扩展,可以发现若要使点覆盖
中的某个点变为独立集的点,由最小点覆盖数=最大匹配数可知,最小点覆盖的每个点都与$\geq 1$
的边相连,因此必须使不少于1个原独立集的点被删除。所以无论如何修改,最多得到与之大小
相等的独立集。

\subsubsection{DAG最小路径覆盖}

\paragraph{最小不相交路径覆盖}

将顶点拆成左右两点,若存在边$u\rightarrow v$则连边$Lu\rightarrow Rv$,求二分图最大匹配。

\begin{theorem}
	最小路径覆盖数=顶点数-二分图最大匹配数。
\end{theorem}

证明:二分图中每增加一个匹配,就意味着减少一条路径。

\paragraph{最小可相交路径覆盖}

先用Floyd求出传递闭包,转化为最小不相交路径覆盖问题。
因为如果要从a走到b,直接连边可以避开中间点的流量限制。

以上内容参考了罗茜\footnote{二分图详解及总结
	\url{https://www.cnblogs.com/alihenaixiao/p/4695298.html}},
justPassBy\footnote{有向无环图(DAG)的最小路径覆盖
	\url{https://www.cnblogs.com/justPassBy/p/5369930.html}}和
不可不戒\footnote{二分图:最大独立集\&最大匹配\&最小顶点覆盖
	\url{https://blog.csdn.net/lezg\_bkbj/article/details/9872189}}
的博客。
\subsection{Hall定理}
\index{H!Hall's Marriage Theorem}
Hall定理用于判断二分图是否存在完美匹配。
\begin{theorem}\label{Hall}
    二分图$G=\{V1,V2,E\},|V1|\leq|V2|$存在完美匹配当且仅当$V1$中任意$k$个顶点
    至少与$V2$中任意$k$个顶点相连。
\end{theorem}
\paragraph{证明}
充分性:假设二分图$G$不存在完美匹配,记$G$的最大匹配为$M$,$V1$上至少有一点
$u$不在$M$上。由条件可知点$u$有一条不在$M$上的边,记对面的点为$v$。若点$v$不在
$M$上,则与$M$为最大匹配矛盾;否则尝试使用匈牙利算法寻找增广路,记涉及到的$V1$的子集
为$S$,则右边至少有$|S|$个节点与其相连,因而存在增广路,与$M$为最大匹配矛盾。

必要性:由于二分图$G$有完美匹配,$V1$的$k$个顶点至少与各自的匹配相连。

还有一个比较有用的推论:
\begin{inference}
   对于二分图$G=\{V1,V2,E\},|V1|\leq|V2|$,若存在整数$t$,满足$V1$中
   任意节点的度数$\geq t$,$V2$中任意节点的度数$\leq t$,则$G$存在完美匹配。
\end{inference}

\paragraph{例题}
[POI2009]LYZ-Ice Skates

由定理~\ref{Hall}可以考虑枚举所有集合,但复杂度无法接受,考虑排掉一些显然不优的集合。
选出的集合可以分为3类:
\begin{itemize}
    \item 脚的大小连续;
    \item 脚的大小不连续但是鞋号区间连续,把中间未被选中的脚的大小选中,但是鞋号区间不变,
    可以有更充分的证据证明不存在完美匹配;
    \item 脚的大小不连续且鞋号区间不连续,这个集合可以根据鞋号区间的连续性分为
    前两种集合,每个集合是独立的子问题。
\end{itemize}
因此只需考虑脚的大小连续的集合。

记脚的大小为$i$的人数有$a_i$个,根据定理有$\displaystyle \sum_{i=l}^r{a_i}
\leq (r+d-l+1)*k$。让右端为常数,得$\displaystyle \sum_{i=l}^r{(a_i-k)}\leq d*k$,
可用线段树维护最大子段和。

代码:
\lstinputlisting{Source/Templates/Hall.cpp}

上述内容参考了Feynman1999的博客\footnote{
    Hall定理(二分图匹配问题,Hungary算法基础)
    \url{https://blog.csdn.net/feynman1999/article/details/76037603}
}。

\section{最大流}
Dinic与ISAP属于Ford-Fulkerson方法中的SAP(Shortest Augment Path)系。
而HLPP属于Push–Relabel算法。
\subsection{Dinic算法}
\index{D!Dinic}
个人比较喜欢使用Dinic算法\sout{(因为我只会这个)}。

Dinic的计算流程如下:
\begin{enumerate}
	\item BFS建分层图,若找不到增广路则退出;
	\item DFS在分层图上找增广路并修改流量,重复步骤1。
\end{enumerate}

时间复杂度证明:

\begin{enumerate}
	\item \begin{lemma}
		Dinic每次BFS后的阻塞流层数是递增的(即$d[t]$递增)。
	\end{lemma}
	\item 每次BFS的时间复杂度为$O(E)$。
	\item 每次DFS的时间复杂度为$O(VE)$。
\end{enumerate}

因此算法的时间复杂度为$O(V^2E)$。

在容量均为1的图上,Dinic的时间复杂度为$O(min \{ V^\frac{2}{3},E^\frac{1}{2} \} E)$,
证明:

留坑待填,参见\cite{NFTGC}。

做二分图最大匹配时Dinic跑得飞快,时间复杂度$O(\sqrt V E)$,证明:

留坑待填,参见\cite{DSNA}。

\index{*TODO!特殊图下Dinic的时间复杂度证明}

时间复杂度证明源自Wikipedia-EN\footnote{
	Dinic's algorithm - Wikipedia
	\url{https://en.wikipedia.org/wiki/Dinic\%27s\_algorithm}}以及
	permui的博客\footnote{ 最大流算法-ISAP - permui
		\url{https://www.cnblogs.com/owenyu/p/6852664.html}}
\subsubsection{优化}
\begin{itemize}
	\item 当前弧优化:每次从未遍历的边开始遍历,减少重复计算(就算前面的边没满,
	      下一次还可以增广)。
	\item 记录无法增广的点(将其深度设为-1),避免重复计算。
	\item (玄学,未测试)BFS找到一条增广路就退出,无法解释。
	\item 若图为分层图,在Dinic之前贪心预流(依旧玄学,未测试):
	      \begin{enumerate}
		      \item 从$s$开始逐层递推,计算能够流出节点$i$的流量$out[i]$;
		      \item 从$t$开始逐层倒推,计算每条边的实际流量。
	      \end{enumerate}
	      代码:

	      \lstinputlisting[title=PreFlow]{NetworkFlows/PreFlow.cpp}

	      该方法源自沐阳的博客。
	      \footnote{ZOJ-2364 Data Transmission 分层图阻塞流 Dinic+贪心预流 - 沐阳
		      \url{https://www.cnblogs.com/Lyush/p/3204099.html}}
\end{itemize}

\subsubsection{板子}

常规优化:
\lstinputlisting[title=DinicA]{Source/Templates/DinicA.cpp}

玄学优化(注意在随机数据下表现可能更差):

\begin{itemize}
	\item 伸缩操作:首先按照边的容量从大到小排序,然后按照
	$cap>=2^k,2^(k-1),\cdots,2^0$加边,每加一组边跑一次Dinic。
	时间复杂度$O(VE\lg C)$。
	\item 延迟加反向边:建图时仍然加正反向边,但是第一次Dinic
	时避开反向边,第二次Dinic时才考虑反向边。
	\item 不退流跑,一次性退流:BFS失败时才退流,若退流后仍然失败才退出迭代。
\end{itemize}

这些优化参见kczno1的博客\footnote{
	论如何用dinic ac 最大流 加强版
	\url{http://kczno1.blog.uoj.ac/blog/3375}}。

参考代码:

常规优化+伸缩操作+延迟加反向边(实践中还是这个比较好用):
\lstinputlisting[title=DinicB]{Source/Templates/DinicB.cpp}

kczno1的最新做法-不退流跑,一次性退流:
\lstinputlisting[title=DinicC]{Source/Templates/DinicC.cpp}

\subsubsection{当Dinic遇上LCT}

留坑待补。
\index{*TODO!Dinic with LCT}

\subsection{ISAP算法}
\index{I!Improved Shortest Augment Path}

Dinic每次BFS计算分层图的过程为找最短增广路的过程。每次BFS
重新计算层次编号$d$似乎有些浪费,因此ISAP在Dinic的基础上用
DFS直接修改层次编号的方式来优化算法。ISAP的时间复杂度仍然为$O(V^2E)$。
记数组$d[u]$为残存网络中点$u$到汇点的最短距离,为了编码方便让$d[T]=1$。

算法步骤如下:
\begin{itemize}
	\item 迭代DFS增广,若找不到满足$d[u]=d[v]+1$的可增广边则说明此时的最短路标号
	已经过时,为了让点$u$可增广,令$d[u]=min\{d[v]\}+1$。
	\item 若$d[S]>|V|$则说明已不存在简单增广路径,退出迭代。
\end{itemize}

\subsubsection{优化}
\begin{itemize}
	\item 若数组$d$被初始化为0,则DFS需要$O(n^2)$的时间来初始化
	数组$d$。可以在增广前从汇点开始BFS$O(n+m)$预处理数组$d$。
	\item gap优化:维护每种层次编号的数量$gap[d]$,若$gap[d]=0$则说明
	出现了断层,不存在新的增广路。此时简单地令$d[S]=n+1$结束算法。
	\item 类似Dinic可以使用当前弧优化,{\bfseries 但在层次标号被修改后要重置链头}。
	\item 层次标号的修改是连续的,每次增广完后$++d[u]$。
	\item 流量用完后直接退出。
\end{itemize}

板子(代码似乎比DinicA还短而且跑得比DinicB还快):
\lstinputlisting[title=ISAP]{Source/Templates/ISAP.cpp}

{\bfseries 注意$mf=0$时直接返回不要更新层次标号。}

ISAP算法参考了permui的博客\footnote{ 最大流算法-ISAP - permui
\url{https://www.cnblogs.com/owenyu/p/6852664.html}}。

\subsection{HLPP算法}
\index{H!Highest-label push–relabel\\ algorithm}

\sout{算法导论\cite{ITA3}~26.4节讲的推送-重贴标签算法是$O(V^3)$的。。。}

HLPP算法使用``推送-重贴标签''算法,其时间复杂度为$O(V^2\sqrt{E})$。虽然时间复杂度
比Dinic优,但由于HLPP算法上界较紧,在实践中往往跑不过Dinic(加了优化后表现还行)。

\subsubsection{推送-重贴标签算法}

以水流类比网络流,每条边都是一根有流量限制的水管,允许每个点暂时存储一些多余的水,
称为超额流。特别地,源汇点可以长期存储无限多的水。其它点需要伺机将自身的超额流推送
出去,这里给每个节点再引入一个``高度''参数,规定流量只能往低处走。固定源点的高度为$V$。
当某个节点高于源点时,它的超额流将退回给源点。{\bfseries 注意高度可以达到$2V-1$}

该算法由两个基本操作组成:
\begin{itemize}
	\item ``推送'':一个节点把自己的超额流推送给高度比自己低1的节点(源点无高度差限制)。
	\item ``重贴标签'':当一个节点无法推送完超额流时,将自身高度加到
	连边有残存流量的最低邻接点的高度+1。
\end{itemize}

首先令S的出边满流,然后维护超额流节点队列,每次取出节点对其进行推送或重贴标签操作。
直至不存在超额流节点。时间复杂度$O(V^2E)$。

\subsubsection{前置重贴标签算法}

每次重贴标签时将节点移至队首,可将时间复杂度优化至$O(V^3)$。

参见算法导论\cite{ITA3}~第26.5节。

\subsubsection{HLPP实现与优化}

使用优先队列以高度为关键字维护超额流节点,每次选取最高标号的节点进行``推送-重贴标签''。

优化:
\begin{itemize}
	\item gap优化:当一个点被重贴标签后,若没有其他点拥有其原来的高度,
	高于此高度的点就无法把流量推送到汇点。将这些点的高度全部设为$V+1$使其流量
	流回源点。
	\item 高度预计算(我因此而TLE多次):将$d$初始化为每个点到汇点的最短路径长。
	{\bfseries 注意源点的高度固定为$V$。}
	\item 使用桶维护优先队列:注意到高度值的范围不大,使用桶来维护较为快速。
\end{itemize}

板子:

优先队列版:
\lstinputlisting[title=HLPPA]{Source/Templates/HLPPA.cpp}

桶版(参考PM250的代码\footnote{
	R13845988 评测详情
	\url{https://www.luogu.org/record/show?rid=13845988}
},自己不会用vector然后就用set代替了,常数大好多):
\lstinputlisting[title=HLPPB]{Source/Templates/HLPPB.cpp}

HLPP算法参考了Mr\_Spade的博客\footnote{
	网络最大流——最高标号预流推进
	\url{https://www.cnblogs.com/Mr-Spade/p/9636935.html}
}。

\subsection{最大流与最小割}

\index{M!Max-flow min-cut theorem}
\begin{theorem}[Max-flow min-cut theorem]\label{MFMCT}
	最大流=最小割。
\end{theorem}

证明:
\begin{itemize}
	\item
	\begin{lemma}\label{MCA}
		最大流$\leq$最小割
	\end{lemma}
	由于流量被割边所限制,所以最大流$\leq$任意割,所以最大流$\leq$最小割。
	\item
	\begin{lemma}\label{MCB}
		最大流$\geq$最小割
	\end{lemma}
	证明:跑完最大流后残量网络内$s$与$t$不连通,所以得到了一个割,
	即最大流$\geq$最小割。
\end{itemize}

结合引理~\ref{MCA}与~\ref{MCB}可得最大流=最小割。
\subsection{无向图最小割}
\subsubsection{Stoer-Wagner Algorithm}
\index{S!Stoer-Wagner Algorithm}
若需要求全局最小割,使用Stoer-Wagner Algorithm。

算法步骤如下:
\begin{enumerate}
	\item 任意指定一个节点作为初始点集;
	\item 查询到点集内的点边权和的最大的点集外的点;
	\item 合并最后加入的两个节点$s,t$并更新最小割;
	\item 重复第一步直至整个图被合并。
\end{enumerate}
具体做法见代码。边权可用优先队列维护,时间复杂度$O(|V||E|\lg |E|)$。

模板(SP12056 FZ10B - Nubulsa Expo):
\lstinputlisting{Source/Templates/Stoer-Wagner.cpp}

这题$|V|$比较小所以可以用邻接矩阵存图,$O(|V|^3)$解决。
\lstinputlisting{Source/Templates/Stoer-WagnerV3.cpp}

不知为何两种方法在SPOJ上都TLE了。
\index{*TODO!证明无向图最小割算法的正确性并修改模板}
上述内容参考了Oyking的博客\footnote{
	全局最小割StoerWagner算法详解
	\url{https://www.cnblogs.com/oyking/p/7339153.html}
}。
\subsubsection{流量构造法}
若指定源汇点,连边时给正反向边的残余流量都初始化为割边代价,然后跑Dinic。

\section{费用流}
\index{M!MCMF}
从普通的EK算法扩展,既然每次增加的流量是一样的,那么我们就选择费用最小(大)
的增广路径,从而保证在得到最大流的前提下费用最小(大)。

求最短路时使用SPFA,若没有负权边尽量使用Dijkstra。

一般的建图思路是通过流量限制来保证方案合法,然后设计边的费用引导至最优代价。

\subsection{使用Dijkstra实现费用流}\label{DijMCMF}
其实即使有负权边,也是可以使用Dijkstra来求费用流的\CJKsout{(但是仍然需要SPFA)}。

核心思想是对原图适当地修改变为不带负权边的图。首先在迭代外用SPFA求从源点到每个点的最短路,
记距离为$h[u]$,满足三角不等式$h[u]+w[u][v]\geq h[v]$。将该式变形得$w[u][v]+h[u]-h[v]
\geq 0$,令左式为边的新权值,就可以在迭代中使用Dijkstra求最短路了,记实际最短距离为
$md[i]$,计算得到的最短距离为$dis[i]$。容易发现最短路径边权和中的$h[]$抵消后,可以得到
$dis[i]=md[i]+h[S]-h[i]$,其中$h[S]=0$,那么实际距离比计算距离多$h[i]$。由此可得
本次迭代产生的费用贡献为$(dis[T]+h[T])*minf$,并且需要在当前迭代结束前更新最短距离
$h'[u]=h[u]+dis[u]$。

上述内容参考了Mogician的博客\footnote{
    最大流与Dijkstra做费用流 - Mogician's blog - 洛谷博客
    \url{https://www.luogu.org/blog/Mogician/Network-Flow-Guide}}。
\subsection{多路增广费用流}
一般使用该方法作为费用流模板。

与普通费用流的差别如下:
\begin{itemize}
    \item 使用vector存边对缓存友好
    \item SPFA使用SLF带容错优化
    \item SPFA从T开始找增广路
    \item DFS多路增广从S沿着最短路跑,可以使用当前弧优化,注意不要走环
\end{itemize}

参考代码:
\lstinputlisting{Source/Templates/MCMF.cpp}

该方法来自Melacau的博客(翻fjsdfzoj时发现的)\footnote{
    【模板】板子的集合\\
    \url{https://www.cnblogs.com/Melacau/p/ban.html}
}。

\section{带上下界网络流}
\section{常见网络流/最小割模型}
\subsection{平面图转对偶图}

平面图与对偶图的定义:
\begin{itemize}
	\item 平面图(Planar Graph):在平面上画出来可以使边与边只在顶点上相交的图。
	      \index{P!Planar Graph}
	\item 对偶图(Dual Graph):将平面图的每条边两边的区域连边而成的新平面图。
	      \index{D!Dual Graph}
\end{itemize}

记平面图$G$的对偶图为$G^*$,平面集合为$P_G$。

对偶图$G^*$有两个性质:
\begin{itemize}
	\item
	      \begin{character}
		      $G^*$中的环对应$G$中的一个割。
	      \end{character}
	\item
	      \begin{character}
		      $|P_G|=|V_{G^*}|,|E_G|=|E_{G^*}|$
	      \end{character}
\end{itemize}

实际应用时,首先连接$(s,t)$使得外部平面被分为两个平面,以获得源汇点$s',t'$(同时连
到一个点上并没有什么用),然后按照定义建图即可(注意不要加入边$s',t'$)。

那么$s,t$的最小割=$s'->t'$的最短路(即拆点前的最小环),时间复杂度降低不少。

\subsubsection{例题}

Luogu P4001 [BJOI2006]狼抓兔子\footnote{【P4001】[BJOI2006]狼抓兔子 - 洛谷
\url{https://www.luogu.org/problemnew/show/P4001}}

根据定理~\ref{MFMCT}转换为求最大流,将右上角当做起点,右下角当做终点,然后使用上述
方法连边即可。

\lstinputlisting[title=Luogu P4001]{Source/Unclassified/Done/4001.cpp}

\subsection{最大权闭合子图}

$S$向非负权点连容量为权值的边,负权点向$T$连容量为权值相反数的边,如果选择点$u$必须
选择点$v$,就从$u$向$v$连容量为$\infty$的边。

答案=正权值之和-最小割。

简单理解:如果割去正权点的权值,则说明舍弃该正权点,权值从答案中扣除;如果割去负权点
的权值,则说明选择之前的正权点并从答案扣除该负权值。

严格的正确性证明待补充。\index{TODO!最大权闭合子图算法的正确性}

\subsubsection{板子}

Luogu P4174 [NOI2006]最大获利\footnote{【P4174】[NOI2006]最大获利 - 洛谷
\url{https://www.luogu.org/problemnew/show/P4174}}

\lstinputlisting[title=Luogu P4174]{Source/Source/'Network Flows'/4174.cpp}

\subsubsection{输出方案}

\begin{theorem}
    Dinic最后一次增广时可访问到的点就是最终方案。
\end{theorem}

简单理解:最后一次增广后BFS必然找不到增广路,此时割掉的边无法继续增广,对应的点无法被
访问到,剩余的点就是最终方案了。

上述内容参考了appgle\footnote{网络流算法基本模型 - appgle
	\url{https://www.cnblogs.com/hyl2000/p/6618519.html}},
MaxMercer\footnote{关于平面图到对偶图的转化 \\
	\url{https://blog.csdn.net/MaxMercer/article/details/77976666}}和
Cold\_Chair\footnote{网络流——最大权闭合子图 \\
	\url{https://blog.csdn.net/Cold\_Chair/article/details/52841351}}
的博客。

\section{最小割树}
\index{G!Gomory–Hu Tree}
\subsection{构造}
考虑单次求最小割的过程,最小割将顶点集合一分为二,设求$u-v$的最小割$cut(u,v)$,
顶点被分割为集合$U,V$。
\begin{lemma}
	$\forall x\in U,y\in V,cut(x,y)\leq cut(u,v)$
\end{lemma}
\paragraph{证明}
若存在$x,y$使得$cut(x,y)>cut(u,v)$,则$cut(u,v)$无法把$x,y$分开,也就意味着无法把
$u,v$分开,$cut(u,v)$不是割。

然后对每个点集再次选择两个点求最小割,将其切分为两个集合,直到所有集合都只有一个点为止。
每次求完最小割后给$u-v$连一条权重为$cut(u,v)$的边,这样做最后能得到一棵树。
\begin{theorem}
	$u-v$在树上的链上最小边权等于$cut(u,v)$。
\end{theorem}
\subsection{询问}
建出树后可以使用倍增法或线段树+树链剖分$O(\lg n)$查询点对答案。

板子:
\lstinputlisting{Source/Templates/GHT.cpp}
上述内容参考了UranusITS的博客\footnote{
	[学习笔记]最小割树(Gomory-Hu Tree)
	\url{http://www.cnblogs.com/coder-Uranus/p/9771919.html}
}。

\subsection{Gusfield算法}
\index{G!Gusfield Algorithm}

\subsection{应用}
\subsubsection{k小割}
\subsubsection{最小割计数}

\section{技巧总结}
\subsection{最大流}
\begin{itemize}
    \item 若一个点只能被经过有限次,将其拆为入点和出点,入点到出点连流量为
    经过次数限制的边。
    \item 树形最大流可以贪心解决。
\end{itemize}
\subsection{最小割}
\begin{itemize}
    \item 最大化收益可以理解为已经拿到所有收益,最小化损失。然后将其转化为最小割解决。
    \item 使用$+\infty$边描述依赖关系,可以保证这条边不出现在最小割中。
    \item 用$S,T$与点的连边来表示点的权。
\end{itemize}
\subsection{费用流}
\begin{itemize}
    \item 要求费用最小且边数最小:类比进制的思想,实际费用乘以一个大于总边数的因子,再加上
    1作为该边边权。
    \item 若已知走一条边之前必定已经走完了另外几条边,则考虑动态加边。
    \item 对于层数较少,结构简单的图,考虑使用其它数据结构贪心模拟费用流。
    \item (待验证)判断一条边是否一定被选:在残量网络上跑SPFA,若距离差不等于边权则必选。
    \item 餐巾计划问题:
\end{itemize}

上述内容参考了胡伯涛的2007年国家集训队论文《最小割模型在信息学竞赛中的应用》
\cite{MCIOI}和租酥雨的博客\footnote{
    网络流总结\\
    \url{https://www.cnblogs.com/zhoushuyu/p/8137534.html}
}。


\chapter{数据结构}
\section{树状数组}
\subsection{标号管辖范围}

\subsection{lowbit函数原理}

lowbit函数定义为:$lowbit(i)=i\&-i$。

由于$i$始终为正,所以$-i$的补码表示是$i$的位取反再加1。$i$末尾的0对应取反后的1,
再加1后就会变成$1000$的形式,1的位置就是$i$末尾1的位置,而$i$与$-i$之前的位均不同,
所以位与后为0,因此$i\&-i$仅保留末尾1的位。


\section{线段树}
\subsection{技巧}
\subsubsection{全局最优值剪枝}
可以使用全局变量维护自己当前遍历到的最优值,若父节点维护的信息表明管辖范围内不可能
出现更优值,则直接返回减少递归深度。(在kd-tree中比较有效)
\subsubsection{标记永久化}
直接将对整个区间的操作存到标记中而不下放,统计时加回去,减少常数。
\subsection{zkw线段树}
\index{Z!zkw's Segment Tree}
留坑待填。
\index{*TODO!zkw线段树}
\subsection{势能分析线段树}
对于某种无法打标记的区间操作(例如区间开根号),若该操作对某个元素施加少数次该操作就会使其
趋于稳定或区间内的值相等,同样可以使用线段树。每次区间操作暴力修改,合并时维护下次操作是否
可以跳过/缩点。

更多应用需要SegmentTreeBeats,留坑待填。
\index{*TODO!Segment Tree Beats}
\index{S!Segment Tree Beats}
\subsection{线段树分治}
留坑待填。
\index{*TODO!线段树分治}

\section{划分树}
划分树是一种类似于线段树但很少使用的数据结构,用来求解区间第k大问题,可用
主席树代替,权当了解。
\subsection{构建}
和线段树类似,将每一段区间在下一层分为两个子区间,即以这段区间的中位数将区
间内的数划分为左右两部分,并且同一边的数之间相对次数不变。为了支持查询,还
需要记录区间内每一个数及之前的数有多少个划分到左区间中。
\subsubsection{注意事项}
\begin{itemize}
    \item 由于每一层都只有n个数,所以只要每一层开n个数的数组即可,记$A[d]$为
    第$d$层的划分情况,$cnt[d][i]$为第$d$层的第$i$个数及同区间之前的数有多
    少个数进入了左子树。
    \item 中位数可预先对数组排序获得,记排序后数组为$B$。
    \item 注意有多个中位数的情况(否则左区间的数将覆盖到右区间的存储区域上)。
\end{itemize}
\subsection{查询}

\begin{enumerate}
    \item 当到达叶子节点时直接返回$A[d][i]$或$B[i]$。
    \item 利用$cnt$数组差分可以查询到$[l,r]$之间有多少数进入了左子树,
    决定往哪棵子树走。
    \item 根据$cnt$数组重新计算目标范围在下一层的区间,递归查询。
\end{enumerate}

\subsection{板子}

SP3946 MKTHNUM - K-th Number

\lstinputlisting[title=PartitionedTree.cpp]
{DataStructure/PartitionedTree.cpp}

以上内容参考了hchlqlz\footnote{划分树讲解 - hchlqlz
\url{https://www.cnblogs.com/hchlqlz-oj-mrj/p/5744308.html}}的博客。

\section{平衡二叉树}
\subsection{FHQTreap}\label{FHQTreap}
\index{F!FHQTreap}
FHQTreap是一种比较好写的平衡二叉树,虽然效率不太高
(不如子节~\ref{splay}\\的splay),但其易于理解,不需要旋转,
且对可持久化友好。
FHQTreap的使用基于两个基本操作:split与merge。

对于二叉搜索树的操作,由于FHQTreap本来就能够满足二叉搜索树的定义
,因此操作方法相同(在对效率要求不高的情况下灵活地使用split和merge
还可以减少代码量)。

对于序列操作,使用子节~\ref{split}所述的splitKth即可完成区间提取。

\subsubsection{split}\label{split}

split函数的作用是将一棵树按照权值或位置划分成两棵子树。

\begin{itemize}
\item
按位置划分:$split(rt,k,x,y)$表示将以$rt$为根的树分为以$x$为根的
左子树和以$y$为根的右子树,其中左子树内的节点是原树的前$k$个,需要维护
每个节点的子树大小$siz$。

代码如下:
\begin{lstlisting}[title=splitKth]
    void split(int u,int k,int& x,int& y) {
        if(u) {
            push(u);
            int lsiz=T[T[u].ls].siz;
            //`决定当节点u为第k个时被分到哪棵子树`
            if(k<=lsiz) {
                y=u;
                split(T[u].ls,k,x,T[u].ls);
            }
            else{
                x=u;
                split(T[u].rs,k-lsiz-1,T[u].rs,y);
            }
            update(u);
        }
        else x=y=0;
    }
\end{lstlisting}
\item
按权值划分:$split(rt,k,x,y)$表示将以$rt$为根的树分为以$x$为根的
左子树和以$y$为根的右子树,其中左子树内的节点值均小于等于$k$。

代码如下:
\begin{lstlisting}[title=splitKey]
    void split(int u,int k,int& x,int& y) {
        if(u) {
            push(u);
            //`=决定当T[u].val==k时被分到哪棵子树`
            if(T[u].val<=k) {

                x=u;
                split(T[u].rs,k,T[u].rs,y);
            }
            else{
                y=u;
                split(T[u].ls,k,x,T[u].ls);
            }
            update(u);
        }
        else x=y=0;
    }
\end{lstlisting}
\end{itemize}

根据树的实际意义(二叉搜索树还是序列)以及实际需要来决定使用哪种split。

\subsubsection{merge}

merge将两棵树按照左右顺序(中序遍历)合并。
和treap一样,merge使用随机权重来保持树的平衡。

代码如下:

\begin{lstlisting}[title=merge]
    int merge(int u,int v) {
        if(u && v) {
            if(T[u].pri<T[v].pri) {
                push(u);
                T[u].rs=megre(T[u].rs,v);
                update(u);
                return u;
            }
            else{
                push(v);
                T[v].ls=merge(u,T[v].ls);
                update(v);
                return v;
            }
        }
        return u|v;
    }
\end{lstlisting}

\subsubsection{指示权重的伪随机数生成器}\label{WRG}

显然FHQTreap也是一个Treap,所以需要一个表现良好的伪随机数生成器
来指示该节点的权。

最易于实现的伪随机数生成算法就是线性同余法(LCG)了。
\index{L!Linear Congruential\\ Generator}
C++11中<random>的$std::linear\_congruential\_engine$给出
了两组预置的参数:
\begin{itemize}
    \item $minstd\_rand0:(a=16807, c=0, m=2147483647)$

    Discovered in 1969 by Lewis, Goodman and Miller, adopted as
    "Minimal standard" in 1988 by Park and Miller.
    \item $minstd\_rand:(a=48271, c=0, m=2147483647)$

    Newer "Minimum standard", recommended by Park, Miller, and Stockmeyer in 1993.

\end{itemize}

通常选择第2个即$a=48271$,代码如下:

\begin{lstlisting}[title=minstd\_rand]
int getRand() {
    static int seed = 347;
    return seed = seed * 48271LL % 2147483647;
}
\end{lstlisting}

现在来试试说明它的优越性:

\begin{itemize}
    \item 根据节~\ref{PrimitiveRoot}所述,如果$g$是模数$P$的一个原根,
    则$g$的幂模$P$可以取到$[1,P-1]$内的每一个数,且循环周期长度为$P-1$。
    由于2147483647是梅森素数,所以它一定存在原根。
    下列程序可证明48271是2147483647的一个原根:
    \lstinputlisting[title=RandomTestA.cpp]{DataStructure/RandomTestA.cpp}
    \item 48271可以较早地使int溢出,从而避免出现$a$过小而导致``锯齿波''。
    下列程序可证明48271可以满足OI考试的需要:
    程序输出minc=7884 maxc=8515 except=8192 s2=88.492,可见数据还是蛮均匀的。
\end{itemize}

参见cppreference\footnote{
    \url{https://en.cppreference.com/w/cpp/numeric/random/
    linear\_congruential\_engine}}与
    Wikipedia-EN\footnote{
    \url{https://en.wikipedia.org/wiki/Linear\_congruential\_generator}}。

如果需要更均匀的随机数,可以使用如下方案(质量从低到高):
\begin{enumerate}
    \item 使用比LCG更好的梅森旋转算法
    \footnote{std::mersenne\_twister\_engine - cppreference.com
    \url{https://en.cppreference.com/w/cpp/numeric/random/
    mersenne\_twister\_engine}};
    \item Intel指令集内置RDRAND;
    \item 使用由一些机构提供的真随机数生成器SDK,如\url{https://www.random.org/};
    \item 在需要蒙特卡洛采样的场合使用低差异序列如Halton,Sobol等。
\end{enumerate}

\subsection{splay}\label{splay}
\index{S!splay}

由于Treap做LCT复杂度多一个log(而且我还看不懂),所以还是学一下好了。

splay主要由$rotate$和$splay$函数组成:

\subsubsection{rotate}

$rotate(u)$表示将节点$u$旋转到$u$的父亲上。



在实践中可使用$connect(u,p,c)$把节点$u$挂到节点$p$的位置$c$下,$getPos(u)$获得
节点$u$相对于父亲的位置。

代码如下:

\begin{lstlisting}[title=rotate]
int getPos(int u) {
    return u == T[T[u].p].c[1];
}
void connect(int u, int p, int c) {
    T[u].p = p;
    T[p].c[c] = u;
}
void rotate(int u) {
    int ku = getPos(u);
    int p = T[u].p;
    int kp = getPos(p);
    int pp = T[p].p;
    int t = T[u].c[ku ^ 1];
    T[u].p = pp;
    if (!isRoot(p))
        connect(u, pp, kp);
    connect(t, p, ku);
    connect(p, u, ku ^ 1);
    update(p);
    update(u);
}
\end{lstlisting}

\subsubsection{splay}

\subsubsection{具体应用}

对于二叉搜索树的操作,同子节~\ref{FHQTreap}相同,直接用二叉搜索树的
操作即可。

对于序列操作,可使用splay来提取区间:

\begin{enumerate}
    \item
\end{enumerate}

上述内容参考了自为风月马前卒的博客\footnote{splay详解(一) - 自为风月马前卒
\url{http://www.cnblogs.com/zwfymqz/p/7896036.html}}。

\section{动态树}
\subsection{常规操作}
\subsection{技巧}
\subsubsection{DSU代替连通性检测}

\section{并查集}\label{DSU}
\index{D!Disjoint Set Union}
\subsection{路径压缩}
路径压缩的原理很简单,即把找到的最新的祖先存储下来,于是该节点的深度被缩为1。
注意路径压缩会使树的形状改变,若维护的数据与树的形状有关则只能使用
LCT或本节~\ref{RankMerge}所述的按秩合并。设$find$次数为$f$,时间复杂度
$O(n+f\cdot (1+\log_{2+\frac{f}{n}}n))$。
\subsection{按秩合并}\label{RankMerge}
对于每个节点维护秩,代表该节点高度的上界。合并时按照启发式策略将较小秩的连通块
并到较大的连通块。设总操作数为$m$,时间复杂度为$O(m+n\lg n)$。
\subsection{复杂度证明}
留坑待填,参见算法导论\cite{ITA3}中的21.3与21.4节。
\index{*TODO!并查集复杂度证明}
\subsection{并查集的分裂}
若要将集合中的某个点从原集合剥离,且不需要可持久化(即不查询历史信息),则可以
考虑``金蝉脱壳'',即保留原节点不动,但消除其对集合的影响,另建新点代表该点。
具体步骤为为每个点维护一个$id$,指向当前实际所指向的点,分裂时先消除原$id$指向的点
对原集合的影响,再创建新的点并更新$id$。

\subsubsection{例题}

UVA11987 Almost Union-Find \footnote{
    【UVA11987】Almost Union-Find - 洛谷
    \url{https://www.luogu.org/problemnew/show/UVA11987}}

步骤同上所述,{\bfseries 注意只有同时使用路径压缩和按秩合并才能达到
$O(m\alpha(n))$的复杂度}。

代码如下:
\lstinputlisting[title=UVA11987]{DataStructure/UVA11987.cpp}

\subsection{并查集重构树}
类似于Kruskal重构树,当两个连通块相连时建一个新的父亲节点并连边,
可以发现两个节点第一次被连接的时间点就是它们的LCA建立的时间点。

\paragraph{例题~BZOJ 3712: [PA2014]Fiolki}

可以发现两种物质在同一瓶内当它们在并查集重构树上的LCA建立时。
以LCA深度为第一关键字,反应优先级为第二关键字对反应进行排序。
按照排序后的顺序模拟反应。

代码:
\lstinputlisting{Source/Source/DSU/BZOJ3712.cpp}

该内容参考了小蒟蒻yyb的博客\footnote{
    【BZOJ3712】Fiolki(并查集重构树)
    \url{https://www.cnblogs.com/cjyyb/p/9368629.html}
}。

\section{K-D Tree}
K-D Tree是一棵二叉树,每一层按照某个轴将本空间内的所有点分为较为均匀的两部分,
该节点保存划分的中点。查询时依靠不断剪枝来提高查询速度。
\subsection{构树}
具体步骤如下:
\begin{enumerate}
	\item 对于当前子空间,选取一个轴来划分(使用$std::nth\_element$)出中点;
	\item 将中点存储在当前节点上;
	\item 递归建左右子树;
	\item 更新子树信息。
\end{enumerate}
$std::nth\_element$的复杂度为$O(n)$,因此构树的复杂度为$O(nlgn)$。
\subsection{插入}
\subsubsection{离线标记}
构树时将所有点加入,记录每个点的id,然后加入点时打标记一路更新即可,
不过这样做影响了查询的复杂度。
\subsubsection{替罪羊树}
与二叉搜索树的插入相同,注意需要确定每一个节点的划分轴。当二叉树不平衡时会影响
查询复杂度,采用替罪羊树的策略,维护每棵子树的size,如果
$max(siz_l,siz_r)>=siz_u \cdot fac$则暴力重构子树(注意每次插入只要找到最高
的不平衡子树重构即可),一般$fac$取0.75。
\subsection{删除}
删除节点后的处理方法与插入相同。
注意被删除的节点可以gc。

\begin{lstlisting}[title=gc]
    std::vector<int> pool;
    int newNode() {
        static int cnt=0;
        int id;
        if(pool.size()) {
            id=pool.back();
            pool.pop_hack();
        }
        else id=++cnt;
        return id;
    }
    void freeNode(int u) {
        pool.push_back(u);
    }
\end{lstlisting}

\subsection{查询}
\begin{enumerate}
	\item 如果整棵子树均不满足要求,就直接返回;
	\item 如果整棵子树均满足要求且可以不需要继续递归,就记录答案(或者打标记)后返回;
	\item 计算当前节点;
	\item 递归左右子树。
\end{enumerate}
在随机数据下,查询的时间复杂度是$O(lgn)$,在构造数据下复杂度约
是$O(n^\frac{d-1}{d})$。
证明待补充。
\index{TODO!K-D Tree查询复杂度证明}
\subsection{估值}
下列为一些常见估值函数:

由于每个方向上是的独立的,对每个方向贪心后加起来即可。
\subsubsection{曼哈顿距离最小}
$w=\sum_{i=1}^d{max(mind_i-p_i,0)+max(p_i-maxd_i,0)}$
当$p_i$在区域内时估值为0,在一边时估值为到最近一边的值(另一边由于符号问题值为0)。
\subsubsection{曼哈顿距离最大}
$w=\sum_{i=1}^d{max(abs(mind_i-p_i),abs(maxd_i-p_i))}$
选择距离最大的一边。
\subsubsection{欧几里得距离最小}
$w=\sum_{i=1}^d{max(mind_i-p_i,p_i-maxd_i,0)^2}$
\subsubsection{欧几里得距离最大}
$w=\sum_{i=1}^d{max((mind_i-p_i)^2,(p_i-maxd_i)^2)}$
\subsection{技巧}
\subsubsection{全局最优值剪枝}
如果通过该节点维护的子树信息可以确定子树内不存在更优解,搜索该子树
已经没有意义了。还可以搭配另一个优化:先求两棵子树的估价函数值,
选择最优的先进入(更有可能获得最优值然后减少在另一棵子树上的计算)。
\subsubsection{预处理降维}
如果插入与查询离线,则可以对某一维排序,边插入边查询,降低kd-Tree查询复杂度。

以上内容参考了n+e的课件\emph{K-D Tree 在信息学竞赛中的应用}\cite{kdTree}。

\section{堆}
\subsection{左偏树}
\index{L!Leftist Tree}
左偏树(Leftist Tree)也是一种二叉堆,核心操作是$merge$函数,
它可以以$O(\lg n)$合并两棵左偏树。

定义外节点为没有左子树或右子树的节点。对于左偏树的每一个节点,维护其到子树外节
点的最近距离,其中外节点的$dist=0$,$null$的$dist=-1$(其实没必要太严格,
差不多平衡就够了)。

左偏树具有左偏性质:
\begin{property}\label{LTC}
    $dist_l \geq dist_r$
\end{property}

由此定义可得到一个推论:

\begin{inference}
    $dist_u=min(dist_l,dist_r)+1=dist_r+1$
\end{inference}

考虑一棵距离为$k$的左偏树的最小节点数,得到以下定理:

\begin{theorem}
    一棵距离为$k$的左偏树为满二叉树时节点数最少,有$2^{k+1}-1$个节点。
\end{theorem}

由此得到推论~\ref{LTI}:

\begin{inference}\label{LTI}
    一棵节点数为$n$的左偏树,距离最大为$[lg(n+1)-1]$。
\end{inference}

先给出引理~\ref{LTL}:

\begin{lemma}\label{LTL}
    左偏树的最右链恰好有一个外节点。
\end{lemma}

证明:由于左偏树是一棵树,最右链至少有一个外节点;若存在两个及以上的外节点,则
对于某个非深度最深的点,必有右子树(否则链就断了),却没有左子树(由外节点定义可知),
与性质~\ref{LTC}矛盾。

由推论~\ref{LTI}与引理~\ref{LTL}可得如下定理:

\begin{theorem}\label{LTT}
    一棵由$n$个节点组成的左偏树最右链最多有$[lg(n+1)]$个节点。
\end{theorem}

以上证明参考了阿波罗2003的博客\footnote{
    浅谈左偏树 - 阿波罗2003
    \url{https://www.cnblogs.com/yyf0309/p/LeftistTree.html}
}。

我原来的简单理解:对于左偏树中的每一个节点,维护其子树高度。每次$merge$时先往右子树塞,
若右子树的深度比左子树的深度更大,就把左子树换过来塞。以此保证树的高度尽可能小。

$merge(u,v)$的操作如下:

\begin{enumerate}
    \item 如果$u$或$v$有一个为$null$则返回另一个节点;
    \item 若$v$应该在$u$的上一层则$swap(u,v)$;
    \item 递归将节点$v$的子树与$u$的右子树合并;
    \item 若$dist_l<dist_r$则$swap$左右子树;
    \item 更新节点$u$的距离$dist_u=dist_r+1$;
    \item 返回该树的根$u$。
\end{enumerate}

根据定理~\ref{LTT}可得$merge$的复杂度为$O(lgn)$。

代码如下(以大根堆为例):

\begin{lstlisting}[title=merge]
int merge(int u, int v) {
    if (u && v) {
        if (T[u].val < T[v].val)
            std::swap(u, v);
        T[u].r = merge(T[u].r, v);
        if (T[T[u].l].dis < T[T[u].r].dis)
            std::swap(T[u].l, T[u].r);
        T[u].dis = T[T[u].r].dis + 1;
    }
    return u | v;
}
\end{lstlisting}

\subsubsection{修改}
\begin{itemize}
    \item 插入节点时,新建一个只有插入元素的堆,然后合并两个堆。
    \item 删除节点时,合并堆顶左右儿子表示的子堆。
\end{itemize}

\subsection{斜堆}
斜堆的操作与左偏树差不多,它们的区别是斜堆不维护到外节点的最近距离,
而是在每一次$merge$时简单地$swap$左右子树。
\subsection{可删堆}\label{MultiSet}
一个简单的方法是使用$std::multiset$,但是其常数很大;
更保险的做法是使用两个优先队列(已加入/已删除)来完成操作:
\begin{itemize}
	\item 加入时将元素加入``已加入堆'';
	\item 删除时将元素加入``已删除堆'';
	\item 取堆顶时,若两堆堆顶相等则弹出,直到两堆堆顶不相等,返回``已加入堆''
	      的堆顶。
\end{itemize}

\section{可持久化数据结构}
可持久化数据结构的核心思想就是\emph{Copy On Write}(写时复制),当一个对象将
被改变时,简单地复制其整体,未修改的部分仍引用原对象的数据,达到节省拷贝时间与
空间的目的。

可持久化数据结构有主席树(可持久化线段树),可持久化可并堆,可持久化Trie,
可持久化数组,可持久化并查集,可持久化平衡树等。
\subsection{主席树}
用主席树做的经典模型有:
\begin{itemize}
    \item 差分
    \item 对于每一个节点为左节点,维护其右边节点为右节点时的答案
    \item 将某一维离散化后不断插入新数据进行预处理以回答在线询问
\end{itemize}
\subsection{可持久化Trie}
若遇到求区间xor最大值之类的问题,使用可持久化Trie。
\subsection{可持久化数组}
可持久化数组有两种实现:
\begin{itemize}
    \item 块状数组
    \item 主席树
\end{itemize}
可持久化并查集可使用可持久化数组实现。
\subsection{优化}
\subsubsection{标记永久化}
将对整个区间的操作记录在管理此区间的节点,标记不下传,统计时参与计算。
此法节约了$push$的时间且对可持久化友好。
\subsubsection{克隆开关}
若已知按照原方法有一个节点不再被任何时间的数据结构引用时,直接在该节点上修改即可
(当然也可以gc,比较麻烦)。
因此在操作前可以设置一个$enableClone$开关,若为$false$则直接返回原节点即可。
代码如下:
\begin{lstlisting}[title=cloneA]
bool enableClone=true;
int cloneNode(int src) {
    if(enableClone) {
        int id=allocNode();
        T[id]=T[src];
        return id;
    }
    return src;
}
\end{lstlisting}
对于可持久化并查集,若使用路径压缩,则不好判断是否$clone$,在每个节点上记录其被
创建时的时间戳,与当前版本时间戳比较即可。
代码如下:
\begin{lstlisting}[title=cloneB]
int timeStamp=0;
int cloneNode(int src) {
    if(T[src].ts!=timeStamp) {
        int id=allocNode();
        T[id]=T[src];
        T[id].ts=timeStamp;
        return id;
    }
    return src;
}
\end{lstlisting}
此法节约了复制节点时的时间与空间。

\section{DLX舞蹈链}
\index{D!Dancing Links X}
DLX用来求解精确覆盖问题。

\paragraph{精确覆盖问题} 给定一个01矩阵,求使得每一列恰好有1个1的行集合。
\subsection{X算法}
X算法使用递归+回溯搜索可行解。

算法步骤如下:
\begin{enumerate}
	\item 从矩阵中选取一行;
	\item 将该行和该行所有1对应的列以及与该行冲突的行从矩阵中删除得到一个新矩阵。
	\item 若该矩阵为空矩阵,则跳到步骤4;否则递归求解新矩阵的精确覆盖,若返回false则
	      返回步骤1选取下一行;
	\item 若选取的行全部为1,则返回true,否则返回false。
\end{enumerate}
\subsection{DLX}
递归+回溯使得存储与维护矩阵既麻烦又费时。Donald E.Knuth使用双向链表
来维护矩阵,这个数据结构被称为Dancing Links。它利用了双向链表删除与恢复的方便性。

对于矩阵内的每一个1(此种矩阵一般为稀疏矩阵),维护其上下左右元素标号和自身坐标。
每个元素既是所属行的链表元素,又是所属列的链表元素。每个列的链表还有链头$C_i$(即0行元素),
这些链头又与总链头$head$串在一起,以便检查覆盖情况。

算法步骤如下:

记标示列链表链头$C$为将元素$C$所在列元素以及这些元素所在行元素删除,回标$C$为其逆操作。
\begin{enumerate}
	\item 检查$head.right$是否为自身,若是则覆盖完毕,输出答案栈内所有元素,返回true;
	\item 记$C=head.right$,标示$C$,枚举$C$所在链表内的行$D$:
	      \begin{enumerate}
		      \item 标示元素$D$所在链表行元素对应列链表链头。
		      \item 将其压入答案栈中;
		      \item 递归求解,若返回true则退出,否则逆序回标,枚举下一行。
	      \end{enumerate}
	\item 回标$C$,返回false。
\end{enumerate}

{\bfseries 为了提高搜索效率可以维护每列1的个数,每次选取1个数最少的列遍历。}

板子:
\lstinputlisting{Source/Templates/DLX.cpp}

上述内容参考了万仓一黍的博客\footnote{
	跳跃的舞者,舞蹈链(Dancing Links)算法——求解精确覆盖问题
	\url{http://www.cnblogs.com/grenet/p/3145800.html}
}。


\chapter{数论}
\subsection{辗转相除法GCD}
\subsubsection{裴蜀定理}
\index{B!Bézout's Theorem}
\begin{theorem}[Bézout's Theorem]\label{BT}
    对于任意$a,b\in \mathbb{Z}$,关于$x,y$的线性不定方程(裴蜀方程)
    $ax+by=c$有无穷多整数解$(x,y)$当且仅当$(a,b)|c$。特别地,
    一定存在$(x,y)$使得$ax+by=(a,b)$成立。
\end{theorem}

由此可得推论:

\begin{inference}
    $a,b$互质的充要条件是存在整数$(x,y)$使得$ax+by=1$。
\end{inference}

接下来证明一定存在$(x,y)$使得$ax+by=(a,b)$成立:

设$s$是$a$和$b$线性组合集中的最小正元素,对于某个整数组$(x,y)$有$ax+by=s$,
令$q=[a/s],r=a mod s=a-q(ax+by)=a(1-qx)+b(-qy)$,所以$r$也是一个线性组合。
因为$s$是线性组合集中的最小正元素,且$0\leq r \le s$,所以$r=0$,可得$s|a$。
同理$s|b$,因此$s$是$a,b$的公约数,可得$(a,b) \geq s$。因为$(a,b)|ax+by$
且$s>0$,所以$(a,b) \leq s$。结合$(a,b) \geq s$与$(a,b) \leq s$可得
$s=(a,b)$。

然后证明对于任意$a,b\in \mathbb{Z}$,$ax+by=c$有整数解
$(x,y) \Leftrightarrow (a,b)|c$:

充分性:

必要性:

至于无穷多整数解嘛。。。拿最小公倍数调一调初始解$(x,y)$即可。

证明参考了霜刃未曾试的博客\footnote{关于裴蜀定理的一些证明\\
\url{https://blog.csdn.net/discreeter/article/details/69833579}}与
算法导论\cite{ITA3}第31.1节定理31.2的证明。
\subsubsection{exgcd}
由定理~\ref{BT}可知一定存在整数解$(x,y)$满足$ax+by=(a,b)$,如何构造
出一组解呢?

$exgcd$(扩展欧几里得算法)可求出一组特殊的整数解。

$exgcd$构造出的解特殊性在于

\subsubsection{位运算gcd}
原理

位扫描优化

如果某个数末尾有多个0,则可以直接使用右移k位代替不断右移1位。
下面是统计末尾0的个数k的方法:

\begin{itemize}
    \item GCC自带了对位扫描指令的封装,即$\_\_builtin\_$系列函数,
    直接使用$\_\_builtin\_clz$函数即可。
    \item
\end{itemize}

实现

\section{欧拉定理}
\subsection{费马小定理}\label{FLTS}
\index{F!Fermat's Little Theorem}
\begin{theorem}[Fermat's Little Theorem]\label{FLT}
	~\\
	$\forall p \textrm{ is a prime number},a\in \mathbb{Z},a^p \equiv a \pmod{p}$
\end{theorem}

该定理是定理~\ref{ET}的特殊化,不证。

由定理~\ref{FLT}可得:

\begin{inference}
	$a^{-1} \equiv a^{p-2} \pmod{p}$
\end{inference}

可使用快速幂在$O(lgp)$的复杂度下求某个数的逆元。

\subsection{线性推逆元}

如果需要获得$a\in [1,p)$内模$p$的逆元,复杂度为$O(plgp)$逐个快速幂的方法
并不是最优的。

首先有$1^{-1}\equiv 1 \pmod{p}$。

令$p=qa+r$,其中$q=[\frac{p}{a}],r=p \bmod a$。

再把$p \equiv 0 \pmod{q}$中的$p$用$qa+r$代替,两边同时乘上$(ar)^{-1}$,
移项得$a^{-1}\equiv -qr^{-1} \pmod{q}$,即
$a^{-1}\equiv -[\frac{p}{a}](p \bmod a)^{-1} \pmod{p}$。

代码如下:
\begin{lstlisting}[title=inv]
inv[1]=1;
for(int i=2;i<=n;++i)
    inv[i]=asInt64(mod-mod/i)*inv[mod%i]%mod;
\end{lstlisting}

以上内容参考了Miskcoo的博客\footnote{[数论]线性求所有逆元的方法 – Miskcoo's Space\\
	\url{http://blog.miskcoo.com/2014/09/linear-find-all-invert}}

Update:还有更一般的做法,可以在$O(n+\lg p)$内推出任意$n$个非0数的逆元:

首先计算出前$i$个数的前缀积$M_i$,然后快速幂计算$M_n^{-1}$,最后从后往前倒推计算每个数的
逆元。比如要计算$A_i$的逆元,倒推维护$X=M_n^{-1}\prod_{j=i+1}^n{A_j}$,那么
$A_i^{-1}=M_{i-1}X$。

该方法源自WAAutoMaton的博客\CJKsout{(知识都是在乱翻他人博客中学到的)}\footnote{
	[loj ???] 乘法逆元2 题解
	\url{https://wa-am.com/2019/03/08/loj-乘法逆元2-题解}
}。
\subsection{欧拉定理}
\index{E!Euler's Theorem}
\begin{theorem}[Euler's Theorem]\label{ET}
	~\\
	对于任意互质正整数对$(a,n)$,有$a^{\varphi(n)} \equiv 1 \pmod{n}$
\end{theorem}
证明:

令$S=\{[x]_n\in Z_n|(a,n)=1\}$(由与$n$互质的模$n$剩余类组成的集合),
它与$\cdot_n$构成整数模$n$乘法群,$(S,\cdot_n)$的阶为$\varphi(n)$。

接着有两种证明思路:
\begin{itemize}
	\item 对于任意一个与$n$互质的正整数$a$,$a$的幂模$n$的值$a,a^2,\cdots,a^k$
	      构成了一个子群,其中$a^k\equiv 1 \pmod{n}$。

	      根据定理~\ref{LT},有$k|\varphi(n)$,令$M=\varphi(n)/k$,有
	      $a^{\varphi(n)}=a^{kM}=(a^k)^M\equiv 1^M\equiv 1 \pmod{n}$。
	\item 根据定义得对于$[x]_n\in S$和$S$中的所有元素$[a_1]_n,[a_2]_n,\cdots,
		[a_{\varphi(n)}]_n$,$[x]_n \cdot_n [a_i]_n$\\组成的集合仍然是$S$,
		因此有$x^{\varphi(n)}[a_1]_n[a_2]_n\cdots[a_{\varphi(n)}]_n=
		(x[a_1]_n)(x[a_2]_n)\cdots\\(x[a_{\varphi(n)}]_n)\equiv[a_1]_n
		[a_2]_n\cdots[a_{\varphi(n)}]_n\pmod{n}$,两边消去可得
		$x^{\varphi(n)}\equiv 1\pmod{n}$。
\end{itemize}

上述证明源自Wikipedia-EN\footnote{Euler's theorem - Wikipedia
	\url{https://en.wikipedia.org/wiki/Euler's\_theorem}}和Eden Harder
的博客\footnote{RSA 加密周边 - Eden Harder
	\url{http://edenharder.is-programmer.com/posts/43247.html}}。
\subsection{扩展欧拉定理}
\begin{theorem}\label{exEuler}
	$\forall a\in \mathbb{Z},x,m\in \mathbb{Z^+},x\geq \varphi(m)
		,a^x\equiv a^{x \bmod \varphi(m)+\varphi(m)} \pmod{m}$
\end{theorem}

\begin{lemma}\label{EEL1}
	\begin{displaymath}
		\left\{
		\begin{array}{l}
			x\equiv y \pmod{m_1} \\
			x\equiv y \pmod{m_2}
		\end{array}
		\right.
		\Rightarrow x\equiv y \pmod{lcm(m_1,m_2)}
	\end{displaymath}
\end{lemma}

证明:
\begin{displaymath}
	\left\{
	\begin{array}{l}
		x\equiv y \pmod{m_1} \\
		x\equiv y \pmod{m_2}
	\end{array}
	\right.
	\Rightarrow
	\left\{
	\begin{array}{l}
		x+c_1m_1=y \\
		x+c_2m_2=y
	\end{array}
	\right.
\end{displaymath}
\begin{displaymath}
	\Rightarrow
	c_1m_1=c_2m_2=k\cdot lcm(m_1,m_2)
	\Rightarrow
\end{displaymath}
\begin{displaymath}
	x \equiv y \pmod{lcm(m_1,m_2)}
\end{displaymath}

\begin{inference}\label{EEL1I}
	当$a,b$互质时,$x\equiv y \pmod{ab}$
\end{inference}

\begin{inference}
	\begin{displaymath}
		\left\{
		\begin{array}{l}
			x\equiv y \pmod{m_1} \\
			\cdots               \\
			x\equiv y \pmod{m_n}
		\end{array}
		\right.
		\Rightarrow x\equiv y \pmod{lcm(m_1,\cdots,m_n)}
	\end{displaymath}
\end{inference}

\begin{lemma}\label{EEL2}
	\begin{displaymath}
		\forall p\textrm{ is a prime number},q\in \mathbb{Z^+},q>1,
		\varphi(p^q)\geq q.
	\end{displaymath}
\end{lemma}

证明:首先有$\varphi(p^q)=(p-1)p^{q-1}$,当$p$固定时,$q$取2使得$\varphi(p^q)-q$
最小,但该值仍非负。当且仅当$p=2,q=2$时,$\varphi(p^q)=q$。

接下来证明定理~\ref{exEuler}:

首先证明当$m$为素数$p$的幂$(m=p^q)$时成立:
\begin{itemize}
	\item 若$gcd(a,p)=1$,则$gcd(a,p^q)=1$,根据欧拉定理可证在该情况下成立;
	\item 若$gcd(a,p)=p$,由适用范围可知$x\geq q$,由引理~\ref{EEL2}可知
	      $x \bmod \varphi(p^q) + \varphi(p^q) \geq q$,因此
	      $a^x\equiv 0 \equiv a^{x \bmod \varphi(p^q)+\varphi(p^q)} \pmod{p^q}$
\end{itemize}

对于任意$m$,可根据算术基本定理将其分解为素数幂之积。因为$\varphi(p^q)|\varphi(m)$,所以有
$a^x\equiv a^{x \bmod \varphi(p^q)+\varphi(p^q)}
	\equiv a^{x \bmod \varphi(m)+\varphi(m)} \pmod{p^q}$。
根据引理~\ref{EEL1}及其推论合并这些式子可证明该定理。

以上证明源自后缀自动机·张的文章\footnote{微小的欧拉定理EXT证明
	\url{https://zhuanlan.zhihu.com/p/24902174}}。

{Warning:扩展欧拉定理在模意义下矩阵幂的应用中,有时正确,但是已经出现被Hack的例子。
尽可能使用矩阵乘法以外的递推方式。}
\index{*TODO!扩展欧拉定理在矩阵幂中的应用}

\section{Miller Rabin素性测试}
\section{Pollard Rho启发式因子分解}
\section{RSA算法}
\subsection{原理}
\begin{theorem}[素数定理]\label{PT}
    $\lim_{n\rightarrow\infty}\frac{\pi(n)}{n/\ln n}=1$
\end{theorem}
RSA的安全性基于以下事实:寻找大素数很容易(根据定理~\ref{PT},素数密度还是挺大的),
但把一个数分解为两个质数之积却很难。

RSA算法的基本步骤如下:
\begin{enumerate}
    \item 随机选取两个大素数$p,q$,使得$p\neq q$,令$n=pq$;
    \item 选取一个与$\varphi(n)=(p-1)(q-1)$复制的小奇数$e$,
    计算出$e$的乘法逆元$d$;
    \item 将$P(e,n)$公开,作为{\bfseries RSA公钥};\\
          将$S(d,n)$保密,作为{\bfseries RSA私钥}。
\end{enumerate}

对于消息$M$,公钥持有者可进行运算:$P(M)=M^e \bmod n$;
私钥持有者可进行运算:$S(M)=M^d mod n$。
对于用公/私钥加密$M$得到的密文$C$,只有使用私/公钥才能得到$M$。
由于$\bmod n$的缘故,消息$M$的域为$Z_n$。

下面证明RSA算法的正确性,即证明:
\begin{displaymath}
    P(S(M))=S(P(M))=M^{ed}\equiv M \pmod{n}
\end{displaymath}

因为$e,d$是模$\varphi(n)$意义下的乘法逆元,所以有$ed=1+k(p-1)(q-1)$。

\begin{itemize}
    \item 若$M\not\equiv 0 \pmod{p}$,则有
    \begin{eqnarray*}
        M^{ed}&\equiv& M^{1+k(p-1)(q-1)} \pmod{p}\\
        &\equiv& M\cdot (M^{p-1})^{k(q-1)} \pmod{p}\\
        &\equiv& M\cdot 1^{k(q-1)} \pmod{p}\\
        &\equiv& M \pmod{p}
    \end{eqnarray*}
    \item 若$M\equiv 0 \pmod{p}$,上述等式仍成立。
\end{itemize}

同样地,对于$q$有$M^{ed}\equiv M \pmod{q}$。根据引理~\ref{EEL1},有
$M^{ed}\equiv M \pmod{n}$,证毕。

\subsection{应用}
\subsubsection{消息加密}
发送方使用接收方的公钥$P$把消息$M$加密得到密文$C$,将密文$C$发送给
接收方。接收方使用自己的私钥$P$解密得到消息$M$。
\paragraph{快速无公钥加密系统}
若消息过长,则仅用$P$加密对称加密算法的随机密钥$K$,同时用密钥$K$加密
$M$得到密文$C$,把$(P(K),C)$发送给接收方。接收方使用$P$解密得到$K$,
再用$K$对$C$解密即可。
\subsubsection{数字签名}
发送方使用自己的私钥$S$把消息$M$签署得到签名$C$,将消息$M$与签名$C$
发送给接收方。接收方使用发送方的公钥$P$解密得到消息$M$,验证消息是否正确。
\paragraph{快速数字签名}
同理把对称加密算法的密钥改为快速散列函数的值即可。
\paragraph{证书链}
以一个可信根为起点,大家都知道这个根的公钥。下一级可以将自己的公钥和被上一级
签署后的公钥作为签名证书,由此验证证书链上每一级的正确性,从而证明链尾端消息的
正确性。
以上内容参考了算法导论\cite{ITA3}第31.7节。

\section{中国剩余定理CRT}
\subsection{CRT}
\index{C!Chinese Remainder Theorem}
\begin{theorem}[Chinese Remainder Theorem]
	对于模线性方程组:
	\begin{displaymath}
		\left\{\begin{array}{l}
			x \equiv a_1 \pmod{n_1} \\
			x \equiv a_2 \pmod{n_2} \\
			\cdots                  \\
			x \equiv a_k \pmod{n_k}
		\end{array}\right.
	\end{displaymath}\\
	其中$n_1,n_2,\cdots,n_k$两两互质,令$\displaystyle N=\prod_{i=1}^k{n_i}$,
	该模线性方程组在$[0,N)$内有唯一解。
\end{theorem}
如何求解该线性方程组呢?和拉格朗日插值法的思路相同,对于每一个方程都给最终的解
贡献一个$x_i$,满足
\begin{displaymath}
	x_i \bmod n_j =
	\left\{\begin{array}{ll}
		0   & \textrm{if $i\neq j$} \\
		a_i & \textrm{if $i=j$}
	\end{array}\right.
\end{displaymath}
答案即为$\displaystyle \sum_{i=1}^n{x_i} \bmod N$。
考虑$i\neq j$时$x_i$应该整除$n_j$,因此$x_i$应该有系数$M=N/n_i$;当$i=j$时,
$x_i$应该有系数$a_i$,为了抵消$M$带来的影响,再乘上$M$模$n_i$的乘法逆元即可(
由于$n$两两互质,$M$与$n_i$也互质,根据定理~\ref{ET},保证其乘法逆元存在)。
\subsection{ExCRT}
当$n$不满足两两互质的条件时,可能会找不到其乘法逆元。
所以我们采用另一种思路求解方程:每次选择两个方程将其合并,直到只剩一个方程为止。

考虑两个方程组成的方程组:
\begin{displaymath}
	\left\{\begin{array}{l}
		x \equiv a_1 \pmod{n_1} \\
		x \equiv a_2 \pmod{n_2} \\
	\end{array}\right.
\end{displaymath}
等价于
\begin{eqnarray}
	x0=a_1+k_1n_1\label{CRTE}\\
	x0=a_2+k_2n_2
\end{eqnarray}
移项得$k_1n_1-k_2n_2=a_2-a_1$,可以使用
$exgcd$求出$c_1n_1+c_2n_2=gcd(n_1,n_2)$的各项参数。根据定理~\ref{BT},
若$gcd(n_1,n_2)\nmid(a_2-a_1)$则该方程组无解。等比例缩放方程求出$k1$,
带入方程~\ref{CRTE}反推出$x0$,得到新的模线性方程$x \equiv x0
	\pmod{lcm(n_1,n_2)}$。

\section{积性函数与线性筛}
\subsection{定义}
\index{A!Arithmetic Function}
\paragraph{数论函数(Arithmetic Function)}
若函数$f:\mathbb{Z^+}\rightarrow\mathbb{C}$,则称函数$f$为数论函数。

\index{M!Multiplicative Function}
\paragraph{积性函数(Multiplicative Function)}
若函数$f$为数论函数,且$f(1)=1$,对于任意互质的正整数$a,b$都有$f(ab)=f(a)f(b)$,
则称函数$f$为积性函数。

\index{C!Completely Multiplicative\\ Function}
\paragraph{完全积性函数(Completely Multiplicative Function)}
若函数$f$为积性函数且对于任意正整数$a,b$都有$f(ab)=f(a)f(b)$,
则称函数$f$为完全积性函数。
\begin{property}\label{MFC}
	若$f$为积性函数,对于正整数$\displaystyle n=\prod_{i=1}^m{{p_i}^{c_i}}$,有
	$\displaystyle f(n)=\prod_{i=1}^m{f({p_i}^{c_i})}$
\end{property}
\begin{property}
	若$f$为完全积性函数,对于正整数$\displaystyle n=\prod_{i=1}^m{{p_i}^{c_i}}$,
	有$\displaystyle f(n)=\prod_{i=1}^m{f(p_i)^{c_i}}$
\end{property}
\subsection{常见积性函数}
\subsubsection{积性函数}
\begin{itemize}
	\item 除数函数$\displaystyle \sigma_k(n)=\sum_{d|n}{d^k}$,\\
	      根据性质~\ref{MFC}可得
	      $\displaystyle \sigma_k(n)=\prod_{i=1}^m{\sum_{j=0}^{c_i}{p_i^{jk}}}$
	\item 约数个数函数$\tau(n)=\sigma_0(n)$
	\item 约数和函数$\sigma(n)=\sigma_1(n)$
	\item \index{E!Euler Totient Function}
	      欧拉函数(Euler Totient Function)
	      $\displaystyle \varphi(n)=\sum_{i=1}^n{[(n,i)=1]}=n\prod_{p|n}{(1-\frac{1}{p})}$,
	      且有$\displaystyle \sum_{i=1}^n{[(n,i)=1]*i}=\frac{n\varphi(n)+[n=1]}{2}$
	\item \index{M!Möbius function}
	      莫比乌斯函数定义为:
	      \begin{displaymath}
		      \mu(d)=
		      \left\{
		      \begin{array}{ll}
			      1      & \textrm{if $d=1$}                                \\
			      (-1)^k & \textrm{if $\displaystyle d=\prod_{i=1}^k{p_i}$} \\
			      0      & \textrm{otherwise}
		      \end{array}
		      \right.
	      \end{displaymath}

	      简单来说就是如果存在平方因子则$\mu(n)$为0,否则$\mu(n)=(-1)^\textrm{质因子数}$。
\end{itemize}
\begin{theorem}\label{MobiusT}
	\begin{displaymath}
		[n=1]=\sum_{d|n}{\mu(d)}
	\end{displaymath}
\end{theorem}
证明:当$n=1$时,该等式成立。
对于$n>1$的情况,将$n$分解为$\displaystyle \prod_{i=1}^m{{p_i}^{c_i}}$,令
$\displaystyle X=\prod_{i=1}^m{p_i}$,
仅考虑$\mu(d)\neq 0$的部分,$\mu(d)$有贡献当且仅当$d|X$,因此$d$可表示为一个长度为$m$
的01向量。
由排列组合知识可知选取奇数个1的向量方案数等于选取偶数个1的向量方案数,即正负贡献抵消。
\begin{theorem}\label{PhiT}
	\begin{displaymath}
		n=\sum_{d|n}{\varphi(d)}
	\end{displaymath}
\end{theorem}
证明:将$n$个分数$\frac{1}{n},\frac{2}{n},\cdots,\frac{n}{n}$化为最简分数,
$\varphi(x)$即表示分母为$x$的最简分数个数。
\begin{theorem}\label{SigmaT}
	\begin{eqnarray*}
		\sum_{i=1}^n{\tau(i)}&=&\sum_{i=1}^n{[\frac{n}{i}]}\\
		\sum_{i=1}^n{\sigma(i)}&=&\sum_{i=1}^n{i\cdot[\frac{n}{i}]}
	\end{eqnarray*}
\end{theorem}
证明:枚举因子$i$,$n$以内有$[\frac{n}{i}]$个因子。
\subsubsection{完全积性函数}
\begin{itemize}
	\item \index{U!Unit Function}
	      元函数(Unit Function)~$\epsilon(n)=[n=1]$
	\item \index{C!Constant Function}
	      恒等函数(Constant Function)~$1(n)=1$
	\item 单位函数$id(n)=n$
	\item 幂函数$id^k(n)=n^k$
\end{itemize}
以上内容参考了skywalkert的博客\footnote{浅谈一类积性函数的前缀和\\
	\url{https://blog.csdn.net/skywalkert/article/details/50500009}}与
Wikipedia-EN\footnote{Arithmetic function - Wikipedia
	\url{https://en.wikipedia.org/wiki/Arithmetic\_function}}。
\subsection{线性筛}
主要思路是每次拿当前的数和已经筛出的素数构造成新的合数并将其筛去。

代码如下:
\begin{lstlisting}[title=Euler]
int prime[size/log(size)],pcnt=0;
bool flag[size];
void pre(int n) {
    for(int i=1;i<=n,++i) {
        if(!flag[i])
            prime[++pcnt]=i;
        for(int j=1;j<=pcnt && prime[j]*i<=n;++j) {
            flag[prime[j]*i]=true;
            if(i%prime[j]==0)
                break;//case 1
        }
    }
}
\end{lstlisting}
注意到case 1中的优化,它保证了每个合数最多被筛1次,从而使时间复杂度变为$O(n)$,
并且增加了一个性质:合数只被其最小质因子筛去。
接下来证明该优化的正确性:当$i\bmod p_j=0$时,有$i=kp_j$,
若要用$p_{j+x}$筛去后面的合数$ip_{j+x}=kp_jp_{j+x}$,可知该合数未来将被合数$kp_{j+x}$与素数
$p_j$筛去,直接跳出不会影响结果,且保证合数被最小质因子筛除,便于质因数分解。
\subsection{积性函数筛}
\subsubsection{欧拉函数}
\begin{itemize}
	\item $\varphi(1)=1$;
	\item 若$i$为素数,则$\varphi(i)=i-1$;
	\item 若$i \bmod p_j=0$,则说明$ip_j$存在至少两个因子$p_j$,因此
	      $\varphi(ip_j)=\varphi(i)p_j$;
	\item 若$i \bmod p_j\neq 0$,则根据积性函数性质可得
	      $\varphi(ip_j)=\varphi(i)(p_j-1)$。
\end{itemize}
\subsubsection{莫比乌斯函数}
\begin{itemize}
	\item $\mu(1)=1$;
	\item 若$i$为素数,则$\mu(i)=-1$;
	\item 若$i \bmod p_j=0$,则说明$ip_j$存在至少两个因子$p_j$,因此
	      $\mu(ip_j)=0$。注意若数组已清零则不赋值;
	\item 若$i \bmod p_j\neq 0$,则根据积性函数性质可得
	      $\mu(ip_j)=-\mu(i)$。
\end{itemize}
\subsubsection{约数个数}
记数组$A_i$为$i$中最小质因子的次数。
\begin{itemize}
	\item $\tau(1)=1,A_1=0$;
	\item 若$i$为素数,则$\tau(i)=2,A_i=1$;
	\item 若$i \bmod p_j=0$,则说明$ip_j$存在至少两个因子$p_j$,因此
	      $\tau(ip_j)=\tau(i)\cdot\frac{A_i+2}{A_i+1}$且$A_{ip_j}=A_i+1$;
	\item 若$i \bmod p_j\neq 0$,则根据积性函数性质可得
	      $\tau(ip_j)=2\tau(i)$且$A_{ip_j}=1$。
\end{itemize}
\subsubsection{约数和}
由性质~\ref{MFC}可得
\begin{displaymath}
	\sigma(n)=\prod_{i=1}^m{\sum_{j=0}^{c_i}{p_i^j}}
\end{displaymath}
记数组$low_i$为$i$中最小质因子的幂,$sum_i$为$i$中最小质因子的贡献。
\begin{itemize}
	\item $\sigma(1)=1,low_1=1,sum_1=1$;
	\item 若$i$为素数,则$\sigma(i)=i+1,low_i=i,sum_i=i+1$;
	\item 若$i \bmod p_j=0$,则说明$ip_j$存在至少两个因子$p_j$,因此
	      $\sigma(ip_j)=\sigma(i)\cdot\frac{sum_{ip_j}}{sum_i}$且
	      $low_{ip_j}=low_i*p_j,sum_{ip_j}=sum_i+low_{ip_j}$;
	\item 若$i \bmod p_j\neq 0$,则根据积性函数性质可得
	      $\sigma(ip_j)=(p_j+1)\sigma(i)$且
	      $low_{ip_j}=p_j,sum_{ip_j}=p_j+1$。
\end{itemize}
\subsubsection{普通积性函数}
同约数和的思想,记数组$sum_i$为$i$中最小质因子的贡献。
要求能够快速推出$f({p_i}^{c_i})$的值。
\begin{itemize}
	\item $f(1)=1,sum_1=???$;
	\item 若$i$为素数,则$f(i)=sum_i=???$;
	\item 若$i \bmod p_j=0$,则说明$ip_j$存在至少两个因子$p_j$,因此
	      $f(ip_j)=f(i)\cdot\frac{sum_{ip_j}}{sum_i}$;
	\item 若$i \bmod p_j\neq 0$,则根据积性函数性质可得
	      $f(ip_j)=f(p_j)f(i)$。
\end{itemize}
以上内容参考了租酥雨的博客\footnote{积性函数与线性筛 - 租酥雨
	\url{https://www.cnblogs.com/zhoushuyu/p/8275530.html}}。
\subsection{因子分解}
通过在每次筛除时记录其最小质因子,可以于$O(\lg n)$复杂度内分解因子。

\section{狄利克雷卷积,狄利克雷逆与莫比乌斯反演}
\subsection{狄利克雷卷积}
\index{D!Dirichlet Convolution}
对于数论函数$f,g$,定义狄利克雷卷积
\begin{displaymath}
	(f*g)(n)=\sum_{d|n}{f(d)g(\frac{n}{d})}=\sum_{ab=n}{f(a)g(b)}
\end{displaymath}
由积性函数集合与狄利克雷卷积组成的群的乘法单位元为元函数$\epsilon$。

狄利克雷卷积有如下性质:
\begin{eqnarray*}
	\textrm{结合律} & (f*g)*h=f*(g*h)\\
	\textrm{分配律} & f*(g+h)=f*g+f*h\\
	\textrm{交换律} & f*g=g*f;\\
	\textrm{单位元} & f*\epsilon=\epsilon*f=f。
\end{eqnarray*}
\subsection{狄利克雷逆}
\index{D!Dirichlet Inverse}
已知数论函数$f$,求$g=f^{-1}$,满足$f*g=\epsilon$。
\begin{itemize}
	\item 当$n=1$时,有$(f*g)(1)=f(1)g(1)=\epsilon(1)=1$,
	      解得$g(1)=\frac{1}{f(1)}$。
	\item 当$n>1$时,
	      有$\displaystyle (f*g)(n)=\sum_{ab=n}{f(a)g(b)}=\epsilon(n)=0$,
	      解得$\displaystyle g(n)=\frac{-1}{f(1)}
		      \sum_{d|n,d<n}{f(\frac{n}{d})g(d)}$。
\end{itemize}
\subsubsection{狄利克雷逆性质}
\begin{property}
	积性函数的狄利克雷逆仍然是积性函数。
\end{property}
\begin{property}
	若数论函数$f,g$为积性函数,则$(f*g)^{-1}=f^{-1}*g^{-1}$。
\end{property}
\begin{property}\label{CMFP}
	积性函数$f$为完全积性函数当且仅当$f^{-1}(n)=\mu(n)f(n)$。
\end{property}
证明:\begin{eqnarray*}
	(f*f^{-1})(n)&=&\sum_{ab=n}{f(a)f^{-1}(b)}\\
	&=&\sum_{ab=n}(f(a)\mu(b)f(b))\\
	&=&f(n)\sum_{d|n}{\mu(d)}\\
	&=&f(n)\epsilon(n)\\
	&=&\epsilon(n)
\end{eqnarray*}
\subsubsection{常见数论函数及其狄利克雷逆}
\begin{itemize}
	\item $1*\mu=\epsilon$\\
	      参见定理~\ref{MobiusT}的证明。
	\item $id^\alpha*(\mu\cdot id^\alpha)=\epsilon$\\
	      根据性质~\ref{CMFP}可证明。
	\item $\displaystyle \varphi*(\sum_{d|n}{\mu(d)d})=\epsilon$\\
	      由定理~\ref{PhiT}可得$id=\varphi*1$,两边同时乘上$\mu$
	      可得$id*\mu=\varphi$,所以$\varphi^{-1}=id^{-1}*\mu^{-1}=id^{-1}*1$。
	\item $\sigma_\alpha*(\sum_{d|n}{\mu(d)\mu(\frac{n}{d})d^\alpha})=\epsilon$

	      $\sigma_\alpha=id^\alpha*1$可推出
	      $(\sigma_\alpha)^{-1}=(id^\alpha)^{-1}*\mu$
\end{itemize}
以上内容参考了Wikipedia-EN\footnote{Dirichlet convolution - Wikipedia\\
	\url{https://en.wikipedia.org/wiki/Dirichlet\_convolution}}。
\subsection{莫比乌斯反演}
\index{M!Möbius Inversion}
\begin{theorem}
	对于数论函数$f,g$,满足$\displaystyle g(n)=\sum_{d|n}f(d)$,则有
	\begin{displaymath}
		f(n)=\sum_{d|n}\mu(d)g(\frac{n}{d})
	\end{displaymath}
\end{theorem}
莫比乌斯反演可表示为若$g=f*1$则$f=\mu*g$。
证明:将$g=f*1$两边同时乘上$\mu$即可。
证明源自Wikipedia-EN\footnote{Möbius inversion formula - Wikipedia\\
	\url{https://en.wikipedia.org/wiki/Mobius_inversion}}。
\subsection{常见技巧}
\begin{itemize}
	\item
	      对于数论函数$g,f$,
	      \begin{displaymath}
		      g(n)=\sum_{n|d}{f(d)}\Rightarrow
		      f(n)=\sum_{n|d}{\mu(d)g(\frac{d}{n})}
	      \end{displaymath}
	\item
	      若$\displaystyle n=\prod_{i=1}^m{{p_i}^{c_i}},g(n)=\sum_{d|n}{f(d)}$
	      且$f$为积性函数,将$g$看做$f*1$可知$g$也是积性函数,则$g(n)=\prod_{i=1}^m
		      {\sum_{j=0}^{c_i}{f(p_i^j)}}$。
	\item 交换内外求和顺序。
	\item 枚举倍数,最大公约数等有共性的值并换元。
	\item 在化简前缀和函数时可能会遇到如下式子:
		\begin{eqnarray*}
			ans(n)&=&\sum_{i=1}^n{f(i)}\\
			&=&A(n)+B(n)\sum_{i=1}^n{\sum_{d|i}{f(d)}}\\
			&=&A(n)+B(n)\sum_{\frac{i}{d}=1}^n{\sum_{d=1}^{[\frac{n}{\frac{i}{d}}]}{f(d)}}\\
			&=&A(n)+B(n)\sum_{t=1}^n{\sum_{d=1}^{[\frac{n}{t}]}{f(d)}}\\
			&=&A(n)+B(n)\sum_{t=1}^n{ans([\frac{n}{t}])}
	      \end{eqnarray*}
		线性筛预处理一部分前缀和(一般预处理规模为$n^{2/3}$,最终时间复杂度
		$O(n^{2/3})$,大规模前缀和使用根号分块法递归计算。

		注意这里可以使用存储Trick来Cache计算结果(多次询问使用map或时间戳数组
		清零,下面只讨论单次询问的情况)。设预处理了前$k$个前缀和,其中$k\geq \sqrt{n}$。
		那么$[\frac{n}{t}]>k$的值不超过$\sqrt{n}$个,并且$t$不同对应的值也不同。所以
		可以以$t$为下标把计算结果存入另一个数组中。
	\item 同时除以最大公约数使其互质,然后套用$\varphi$。
	\item $\displaystyle [gcd(i,j)=1]=\sum_{k|gcd(i,j)}\mu(k)=
		      \sum_{k|i,k|j}\mu(k)$
	\item $\displaystyle \sum_{i=1}^n{i}=
		      \sum_{i=1}^n{\sum_{d|i}\varphi(d)}=
		      \sum_{d=1}^n{\varphi(d)\cdot[\frac{n}{d}]}$
	\item $(id\cdot\varphi)*id=id^2$
\end{itemize}
更多技巧待补充。

\section{低于线性时间复杂度筛法}
积性函数前缀和算法的思维难度:杜教筛>min\_筛>Powerful~Number。
\subsection{杜教筛}
杜教筛主要用于计算大数据规模积性函数求和。
\subsubsection{约数函数前缀和}
求$\displaystyle \sum_{i=1}^n{\sigma(i)},n\leq 10^{12}$。
\begin{eqnarray*}
    \sum_{i=1}^n{\sigma(i)}&=&\sum_{i=1}^n{\sum_{d|i}d}\\
    &=&\sum_{d=1}^n{d[\frac{n}{d}]}
\end{eqnarray*}
由于$[\frac{n}{d}]$存在许多连续相同的值,使用整除分块法可做到$O(\sqrt{n})$。
\subsubsection{欧拉函数前缀和}
求$\displaystyle \sum_{i=1}^n{\varphi(i)},n\leq 10^{11}$。
由定理~\ref{PhiT}可得
$\displaystyle \varphi(n)=n-\sum_{d|n,d<n}{\varphi(d)}$。
\begin{eqnarray*}
    ans(n)&=&\sum_{i=1}^n{\varphi(i)}\\
    &=&\sum_{i=1}^n{\left(i-\sum_{d|i,d<i}{\varphi(d)}\right)}\\
    &=&\frac{n(n+1)}{2}-\sum_{i=2}^{n}{\sum_{d|i,d<i}{\varphi(d)}}\\
    &=&\frac{n(n+1)}{2}-\sum_{\frac{i}{d}=2}^n
    {\sum_{d=1}^{[\frac{n}{\frac{i}{d}}]}{\varphi(d)}}\\
    &=&\frac{n(n+1)}{2}-\sum_{t=2}^n
    {\sum_{d=1}^{[\frac{n}{t}]}{\varphi(d)}}\\
    &=&\frac{n(n+1)}{2}-\sum_{t=2}^n{ans([\frac{n}{t}])}
\end{eqnarray*}
同理使用分块+递归询问区间和来计算答案。为了降低复杂度,应该先线性筛预处理前一部分值。
当预处理$k=n^\frac{2}{3}$时可以取到复杂度$T(n)=O(n^\frac{2}{3})$。
\subsubsection{莫比乌斯函数前缀和}
求$\displaystyle \sum_{i=1}^n{\mu(i)},n\leq 10^{11}$。
由定理~\ref{MobiusT}可得
$\displaystyle \mu(n)=[n=1]-\sum_{d|n,d<n}{\mu(d)}$。
\begin{eqnarray*}
    ans(n)&=&\sum_{i=1}^n{\mu(i)}\\
    &=&\sum_{i=1}^n{\left([i=1]-\sum_{d|i,d<i}{\mu(d)}\right)}\\
    &=&1-\sum_{i=1}^n{\sum_{d|i,d<i}{\mu(d)}}\\
    &=&1-\sum_{t=2}^n{ans([\frac{n}{t}])}
\end{eqnarray*}
\subsubsection{其它函数前缀和}
主要思路是使用狄利克雷卷积构造出一个简单的前缀和函数,且用于卷积的另一个函数也容易计算。

令$\displaystyle A(n)=\sum_{i=1}^n\frac{i}{(n,i)}$,求
$\displaystyle \sum_{i=1}^n{A(n)},n\leq 10^{9}$。

先化简$A(n)$:
\begin{eqnarray*}
    A(n)&=&\sum_{i=1}^n\frac{i}{(n,i)}\\
    &=&\sum_{d|n}{\sum_{i=1}^n{[(n,i)=d]\cdot\frac{i}{d}}}\\
    &=&\sum_{d|n}{\sum_{\frac{i}{d}=1}^{\frac{n}{d}}
    {[(\frac{n}{d},\frac{i}{d})=1]\cdot\frac{i}{d}}}\\
    &=&\frac{1}{2}\left(1+\sum_{d|n}{d\cdot\varphi(d)}\right)
\end{eqnarray*}

那么答案即为$\displaystyle \frac{1}{2}\left(n+\sum_{t=1}^n
    {\sum_{d=1}^{[\frac{n}{t}]}{d\cdot\varphi(d)}}\right)$。

考虑计算$\displaystyle \sum_{d=1}^n{d\cdot\varphi(d)}$的值:

易知$(id\cdot\varphi)*id=id^2$,因为\begin{displaymath}
    \sum_{d|n}d\cdot\varphi(d)\cdot\frac{n}{d}=
    n\cdot\sum_{d|n}\varphi(d)=n^2
\end{displaymath}

所以有\begin{eqnarray*}
    \frac{n(n+1)(2n+1)}{6}&=&\sum_{i=1}^n{(id\cdot\varphi)*id}\\
    &=&\sum_{t=1}^n{t\cdot\sum_{d=1}^{[\frac{n}{t}]}{d\cdot\varphi(d)}}
\end{eqnarray*}

\subsubsection{总结}
欲求积性函数$f(x)$的前缀和,构造$h=f*g$,其中$h(x)$和$g(x)$都是积性函数,且易求得
$h(x)$与$g(x)$的前缀和。

考虑$h(x)$前缀和的表达式(记大写形式为前缀和,即$F(n)=\displaystyle \sum_{i=1}^n{f(i)}$):
\begin{displaymath}
    H(n)=\sum_{i=1}^n{\sum_{d|i}{f(\frac{i}{d})g(d)}}=
    \sum_{d=1}^n{g(d)\sum_{i=1}^{\lfloor\frac{n}{d}\rfloor}{f(i)}}=
    \sum_{d=1}^n{g(d)F(\lfloor\frac{n}{d}\rfloor)}
\end{displaymath}

提出$F(n)$,即$F(n)=H(n)-\displaystyle \sum_{d=2}^n{g(d)F(\frac{n}{d})}$。

在递归时可以预处理一部分前缀和,同时使用HashTable缓存计算结果。
\subsubsection{时间复杂度分析}
使用整除分块法需要计算前$i$项前缀和,与前$\lfloor\frac{n}{i} \rfloor$项前缀和,
其中$i\leq \sqrt{n}$。后一部分比前一部分复杂度更高。考虑使用积分近似,有
\begin{displaymath}
    \int_0^{\sqrt{n}}{\sqrt{\frac{n}{x}} \ud x}=O(n^{3/4})
\end{displaymath}

预处理一部分前缀和可以有效降低算法时间复杂度:记预处理前$k$个前缀和,$k\geq \sqrt{n}$。

那么时间复杂度为
\begin{displaymath}
    O(k)+\int_0^{\frac{n}{k}}{\sqrt{\frac{n}{x}}\ud x}=O(k)+O(\frac{n}{\sqrt{k}})
\end{displaymath}

平衡两边的复杂度,解得$k=n^{1/3}$。

以上例题来自skywalkert的博客\footnote{浅谈一类积性函数的前缀和\\
    \url{https://blog.csdn.net/skywalkert/article/details/50500009}},
总结部分参考了国家集训队2016论文集任之洲的论文《积性函数求和的几种方法》。

\subsection{min\_25筛}
这里求和的积性函数$F$满足$F(p)$是一个关于$p$的低阶多项式且能够快速求出$F(p^k)$。
据说min\_25筛踩爆洲阁筛,那我就不学洲阁筛了。在此附上洲阁筛教程\footnote{
    洲阁筛学习 | \_\_debug's Home\\
    \url{http://debug18.com/posts/calculate-the-sum-of-multiplicative-function/}
}。

\subsubsection{预处理}
首先考虑求$\displaystyle \sum_{p\leq n}{F(p)}$。

记$g(n,j)$为满足$x$为$n$以内素数,或者$x$的最小质因子$>p_j$的$F(x)$之和,
所求值即为$g(n,|P|)$。考虑$g(n,j)$如何从$g(n,j-1)$转移。易知最小质因子为
$p_j$的合数为$p_j^2$,若其$>n$,则$g(n,j)$与$g(n,j-1)$都只求素数的积性函
数值之和,所以$g(n,j)=g(n,j-1)$。若$p_j^2\leq n$,则转移时会损失掉一些
$F(x)$,这些$x$的最小质因子为$p_j$。考虑提出$x$的$p_j$,满足$\frac{x}{p_j}$
的最小质因子$\geq p_j$,计算$\frac{x}{p_j}$的积性函数和,发现$g(\frac{n}{p_j},j-1)$
包括了它们,又因为$\frac{n}{p_j}\geq p_j > p_{j-1}$,$g(\frac{n}{p_j},j-1)$还有
$\displaystyle \sum_{p<p_j}F(p)$,需要扣除。{\bfseries 由于积性函数$F$的特殊性,
把不同次数的项拆开算,单项为完全积性函数,乘上$F(p_j)$即为需要减去的值。}

{\bfseries 若拆开后某一项系数不为1,这一项就不是完全积性。算这一项的贡献时先去除系数,
最后整体乘以系数。}

综上,有\begin{displaymath}
    g(n,j)=\left\{\begin{array}{lr}
        g(n,j-1)        & p_j^2>n   \\
        g(n,j-1)-F(p_j)(g(\frac{n}{p_j},j-1)-\displaystyle \sum_{p<p_j}{F(p)}) & p_j^2\leq n \\
    \end{array}\right.
\end{displaymath}

预处理素数时只需要筛$\sqrt{n}$内的素数,边界$g(n,0)$是所有数按照素数的计算方式计算
的值之和。由于最后只需要$g(n,|P|)$,非质数的贡献会被筛掉。

实质上$g(n,j)$就是埃氏筛法筛完$p_j$后未被筛的合数以及素数的积性函数值之和。

接下来尝试求出所有的$g(x,|P|),x=\lfloor \frac{n}{i}\rfloor$。
这里有一个存储上的trick:由于$\lfloor \frac{n}{i}\rfloor$有连续重复项,
最多$2\sqrt{n}$个,对于$x=\lfloor \frac{n}{i}\rfloor>\sqrt{n}$,把它
映射到$\lfloor \frac{n}{x}\rfloor$上存储,这样保证了空间复杂度为$O(\sqrt{n})$。

由于最后只要$g(x,|P|)$,$g$数组只要开1维滚动更新。

伪代码如下:
\begin{lstlisting}
int g[2][sqsiz],q[2*sqsiz];
int& getG(int x) {
    if(x<=sqr) return g[0][x];
    return g[1][n/x];
}
void calcG() {
    int m=0,i=1;
    while(i<=n) {
        int val=n/i;
        q[++m]=val;
        getG(val)=sumf(val);
        i=n/val+1;
    }
    for(int i=1;i<=psiz;++i) {
        int cp=p[i],cp2=cp*cp;
        for(int j=1;j<=m && cp2<=q[j];++j) {
            int k=q[j],&val=getG(k);
            val=sub(val,mul(f(cp),getG(k/cp)-sumpf[i-1]));
        }
    }
}
\end{lstlisting}

计算$G$时始终不考虑1,在求和时才加入。
\subsubsection{求和}
记$S(n,j)$为$n$以内最小质因子$\geq p_j$的积性函数值和,所求答案即为$S(n,1)+f(1)$。

把$S(n,j)$分为素数和合数求解:
\begin{itemize}
    \item 对于素数部分,$g(n,|P|)$代表了素数积性函数值和,再扣去不满足
    最小质因子要求的素数,最终贡献为$g(n,|P|)-\displaystyle \sum_{p<p_j}F(p)$。
    \item 对于合数部分,枚举其最小质因子$p_k$及其幂次$c$,单独贡献为\\
    $F(p_k^c)S(\frac{n}{p_k^c},k+1)+F(p_k^{c+1})$。注意此处的$F\cdot S$直接利用
    了积性函数的定义,因为$S$部分无$p_k$因子。由于$S$不处理$j=k$的部分,需要另外加上
    $F(p_k^{c+1})$。
\end{itemize}

递归的边界条件为$n\leq 1 \vee n<p_j$,无需记忆化。

时间复杂度为$O(\frac{n^\frac{3}{4}}{\lg n})$,空间复杂度为$O(\sqrt{n})$。

模板(LOJ\#6053. 简单的函数):
\lstinputlisting{Source/Templates/min_25.cpp}

上述内容参考了小蒟蒻yyb\footnote{
    min\_25筛
    \url{https://www.cnblogs.com/cjyyb/p/9185093.html}
}和租酥雨\footnote{
    Min\_25 筛
    \url{https://www.cnblogs.com/zhoushuyu/p/9187319.html}
}的博客。

Min\_25筛似乎也被称作``通用筛法''、``扩展埃拉托斯特尼筛法'',严格证明参见
zbh2047的文章\footnote{关于一种积性函数前缀和的通用筛法的时间复杂度证明

    \url{https://www.cnblogs.com/zbh2047/p/8552551.html}

    \url{https://zhuanlan.zhihu.com/p/33544708}}。
\subsection{Powerful~Number}
定义Powerful~Number为所有质因子的指数都$\geq 2$的数,那么每个Powerful~Number都可以
被表示为$a^2b^3$的形式(若指数为奇数则分配一个立方给$b$,其余分给$a$)。
{\bfseries 注意1也是Powerful~Number。}

\begin{theorem}
    $n$以内的Powerful~Number个数为$O(\sqrt{n})$。
\end{theorem}

证明:枚举$a$,将$b$的个数累积,可得式子
\begin{displaymath}
    \sum_{i=1}^{\lfloor\sqrt{n}\rfloor}{\lfloor\sqrt[3]{\frac{n}{i^2}}\rfloor}
\end{displaymath}

使用积分近似求出其上界为$\int_1^{\sqrt{n}+1}{\sqrt[3]{\frac{n}{(x-1)^2}} \ud x}=O(\sqrt{n})$。
根据Wikipedia-EN\footnote{Powerful~number
    \url{https://en.wikipedia.org/wiki/Powerful\_number}}的描述,其上界常数为
    $\frac{\zeta(\frac{3}{2})}{\zeta(3)}\approx 2.173$。

对于某个复杂的积性函数$f(x)$,若$f(p^e)$易于计算且存在一个简单(易于求$g(p^e)$与前缀和)
的积性函数$g(x)$,满足对于所有素数$p$,有$f(p)=g(p)$,称函数$g$拟合了函数$f$。

设$h=f/g$,这里的除法是狄利克雷除法,等价于$h=f*g^{-1}$。由于狄利克雷逆$g^{-1}$是积性
函数,狄利克雷卷积$h$也是积性函数。那么对于所有素数$p$,有$f(p)=h(1)g(p)+h(p)g(1)$,
由于$h(1)=1,f(p)=g(p)$,可得$h(p)=0$。由于$h(x)$是积性函数,$h(x)$可能非0当且仅当$x$
是Powerful~Number。

现在要求$\displaystyle Ans=\sum_{i=1}^n{f(n)}$,由于$f=h*g$,有
$\displaystyle Ans=\sum_{ab\leq n}{h(a)g(b)}$。由上文的推导可知$h(x)$仅在
Powerful~Number处有贡献,且Powerful~Number的个数是$O(\sqrt{n})$的,可以
$O(\sqrt{n})$DFS暴力枚举质因子组合得到$a$。记$n$以内的Powerful~Number组成的集合为$S$,
原式变为$\displaystyle Ans=\sum_{a\in S}{h(a)\sum_{b=1}^{\lfloor \frac{n}{a} \rfloor}{g(b)}}$。
易求$g(x)$的前缀和,问题在于如何快速推得$h(a)$的值。

由于$h(x)$是积性函数且$x$是Powerful~Number,在DFS时仅需计算$h(p^e),e>1$的值。
使用$f(p^e)$展开式,快速求得$f(p^e)$与$g(p^e)$,再根据历史信息$h(p^{e'}),e'<e$,
可以快速得到$h(p^e)$。如果$g(x)$是完全积性函数,可以对$f(p^e)$展开式平移得到$f(p^{e+1})$
的展开式。由于$e$很小,$h(p^e)$的求值不是瓶颈。预处理可以节省DFS时的重复计算。

{\bfseries 十分有效的优化:DFS递归时会遇到大量的0次项,这些不必要的递归会导致实际运行缓慢。
可以DFS钦定一些质因子必选,使得每层DFS都对最终的$a$有贡献,杜绝爆栈。}

具体实现参考Project~Euler~484:
\lstinputlisting{Source/Source/'Number Theory'/PE484.cpp}

{\bfseries 记得要计算$a=1$时的贡献!!!}

上述内容参考了fjzzq2002的博客\footnote{
利用powerful~number求积性函数前缀和
    \url{https://www.cnblogs.com/zzqsblog/p/9904271.html}
}。

Min\_25使用Powerful~Number得到新的做法,参见
Sum~of~Multiplicative~Function~on~Powerful~Numbers
\footnote{\url{https://min-25.hatenablog.com/entry/2018/11/11/172228}}。
\subsection{素数k次幂前缀和}
可以使用Min\_25筛的前半部分在$O(\frac{n^{3/4}}{\lg n})$内解决,但这不是最优的。

一般使用Meissel Lehmer方法在$O\left(\left(\frac{n}{\lg n}\right)^{2/3}\right)$
内解决,留坑待补。
\index{*TODO!Meissel Lehmer}
\subsection{约数个数函数前缀和}
使用整除分块可以在$O(\sqrt{n})$内解决,但还有更优算法。

考虑变化后的式子$S(n)=\displaystyle \sum_{i=1}^n{\lfloor\frac{n}{i}\rfloor}$,
可以将其理解为在$[1,n]$内在双曲线$xy=n$与$x$轴内的整点数。使用Stern-Brocot Tree可以
在$O(n^{1/3}\lg n)$内解决,留坑待补。

\index{*TODO!约数个数函数前缀和}

\section{BSGS}
\index{B!Baby Step Giant Step}
BSGS法(Baby Step Giant Step)用来求解类似于$a^x\equiv b\pmod{P}$的方程。
\subsection{BSGS}
普通BSGS仅考虑$P$为素数的情况。

以下为求解最小非负整数解的方法:

首先根据定理~\ref{ET}可知$x$的最小非负整数解小于$\varphi(P)=P-1$。将x表示为
$\sqrt{P}$进制数,分别用$O(\sqrt{P})$的复杂度枚举值,一半存入HashTable,另一半
查询是否有匹配值。注意参数的枚举顺序。

\begin{enumerate}
    \item 若$a$为$P$的倍数,则特判$b$是否为$0$,算法结束;
    \item 令$m=\lceil\sqrt{P}\rceil,x=im-j$,移项得$a^{im}\equiv ba^j\pmod{P}$;
    \item 枚举$ba^j$的值,按$j$从小到大{\bfseries 覆盖}存入HashTable;
    \item 枚举$(a^m)^i$的值,按$i$从小到大在HashTable中查询,存在则返回$im-j$;
    \item 返回无解。
\end{enumerate}

\subsection{ExBSGS}
ExBSGS可解决$a,P$不互质的问题。主要思路是将原方程化为普通BSGS可解决的方程。

记化简后方程为$Aa^{x-B}\equiv b\pmod{P}$,化简步骤如下:
\begin{enumerate}
    \item 将$A,B$初始化为$1,0$;
    \item 令$d=(a,P)$,
    \begin{itemize}
        \item 若$d\mid b$,则提出一个因子$d$,即$A*=a/d,b/=d,P/=d,++B$;
        \item 若$d\nmid b$,则特判$b$是否为$A$,$b=A$则$x=B$,$b\neq A$则无解;
    \end{itemize}
    \item 重复第2步直至$d=1$。
\end{enumerate}
令$x=im-j+B$转化为普通BSGS,{\bfseries 注意在BSGS前要暴力检查$x\in[0,B)$是否可行}。

代码如下:
\lstinputlisting{Source/Templates/exBSGS.cpp}

以上内容参考了ZigZagK的博客\footnote{BSGS及扩展BSGS
\url{https://blog.csdn.net/zzkksunboy/article/details/73162229}}。

\section{原根}\label{PrimitiveRoot}
\index{P!Primitive Root}
\subsection{基本定义与定理}
\subsubsection{数论阶}
设$n>1,(a,n)=1$,记$\delta_n(a)$为使得$a^r\equiv 1 \pmod{n}$
成立的最小正整数$r$,称其为$a$模$n$的阶。

\begin{theorem}
	设$n>1,(a,n)=1$,若$a^x\equiv 1 \pmod{n}$,则有$\delta_n(a)\mid x$。
\end{theorem}

\subsubsection{原根}
若$\delta_n(a)=\varphi(n)$,则称$a$为模$n$的一个原根。

若$a$为模$n$的原根,根据定理~\ref{ET},对于$0\leq i< \varphi(n)$,$a^i\bmod{n}$两两不同。

\begin{theorem}
	如果模$n$有原根,则它一共有$\varphi(\varphi(n))$个原根。
\end{theorem}

\begin{theorem}
	$n=2,4,p^i,2p^i\Leftrightarrow$模$n$有原根,其中$p$为奇素数。
\end{theorem}

\subsection{求模n的原根}

对$\varphi(n)$进行质因数分解,
对于$\displaystyle \varphi(n)=\prod_{i=1}^m{p_i^{c_i}}$,若
恒有$g^\frac{\varphi(n)}{p_i}\not\equiv 1 \pmod{n}$,则$g$为模$n$的原根。

以上内容参考了mosquito\_zm的博客\footnote{原根
	\url{https://blog.csdn.net/mosquito\_zm/article/details/77227570}}。
\subsection{原根的应用}
\begin{itemize}
	\item 在NTT中用于推算主单位根。
	\item 将乘积恒定转换为幂次和恒定后NTT。
	\paragraph{例题} [SDOI2015]序列统计\footnote{【P3321】[SDOI2015]序列统计 - 洛谷
	\url{https://www.luogu.org/problemnew/show/P3321}}

	将$x$映射为$g^i$,使用生成函数推导出多项式幂的形式,最后对答案进行逆映射。

	\lstinputlisting[title=luogu P3321]{Source/Source/'FFT NTT'/3321.cpp}

\end{itemize}

%\section{二次剩余与三次剩余}
\subsection{勒让德符号}
\index{L!Legendre Symbol}
定义勒让德符号:
\begin{displaymath}
	\Legendre{a}{p}=
	\left\{\begin{array}{lr}
		0  & a\equiv 0 \pmod{p}                  \\
		1  & \exists x,x^2\equiv a \pmod{p}      \\
		-1 & \not \exists x,x^2\equiv a \pmod{p}
	\end{array}\right.
\end{displaymath}
勒让德符号是完全积性函数,即
\begin{displaymath}
	\Legendre{ab}{p}=\Legendre{a}{p}\Legendre{b}{p}
\end{displaymath}
\subsubsection{与斐波那契数列的关系}
\begin{theorem}
	若$p$为素数,则
	\begin{displaymath}
		F_{p-\Legendre{p}{5}}\equiv 0 \pmod{p}
	\end{displaymath}
	\begin{displaymath}
		F_p\equiv \Legendre{p}{5} \pmod{p}
	\end{displaymath}
\end{theorem}
该定理用于求超大斐波那契数取模。
\subsubsection{二次互反律}
\index{Q!Quadratic Reciprocity Law}
\begin{theorem}
	若$p,q$为不同的奇素数,则
	\begin{displaymath}
		\Legendre{p}{q}\Legendre{q}{p}=(-1)^\frac{(p-1)(q-1)}{4}
	\end{displaymath}
\end{theorem}
此外有两个补充结论:
\begin{theorem}
    \begin{displaymath}
        \Legendre{-1}{p}=(-1)^\frac{p-1}{2}=\left\{\begin{array}{lr}
            1 & \textrm{if~} p\equiv 1\pmod{4}\\
            -1 & \textrm{if~} p\equiv 3\pmod{4}\\
        \end{array}\right.
    \end{displaymath}
\end{theorem}
\begin{theorem}
    \begin{displaymath}
        \Legendre{2}{p}=(-1)^\frac{p^2-1}{8}=\left\{\begin{array}{lr}
            1 & \textrm{if~} p\equiv 1,7\pmod{4}\\
            -1 & \textrm{if~} p\equiv 3,5\pmod{4}\\
        \end{array}\right.
    \end{displaymath}
\end{theorem}
\subsection{二次剩余}
\index{Q!Quadratic Residue}
求解二次剩余即求解下列同余方程:
\begin{displaymath}
	x^2\equiv a \pmod{p}
\end{displaymath}
\subsubsection{欧拉判别准则}
\index{E!Euler's Criterion}
\begin{theorem}[Euler's Criterion]
    若$p$为奇素数且$p\nmid a$,则
    \begin{displaymath}
        \Legendre{a}{p}\equiv a^\frac{p-1}{2}\pmod{p}
    \end{displaymath}
\end{theorem}
\subsubsection{模奇素数}
这里使用ACdreamer介绍的Cipolla随机化算法。
\index{C!Cipolla's Algorithm}
\begin{theorem}
    设$b$满足$\omega=b^2-a$不是模$p$的二次剩余,则
    $x\equiv (b+\sqrt{\omega})^\frac{p-1}{2}\pmod{p}$是
    方程$x^2\equiv a\pmod{p}$的解。
\end{theorem}
证明:

\subsubsection{模奇素数幂}
\subsubsection{模2的幂}
\subsubsection{模合数}

上述内容参考了Miskcoo\footnote{
	[数论]二次剩余及计算方法
	\url{http://blog.miskcoo.com/2014/08/quadratic-residue}
}和ACdreamer\footnote{
    二次同余方程的解
    \url{https://blog.csdn.net/acdreamers/article/details/10182281}
}的博客与百度百科(勒让德符号,二次互反律与欧拉判别准则)。
\subsection{三次剩余}


\chapter{集合论~群论}
\minitoc
\section{集合论定理}
\begin{theorem}[对称差]
	\begin{displaymath}
		A\oplus B=(A-B)\cup(B-A)=A\cup B - A\cap B
	\end{displaymath}
\end{theorem}
\index{D!De Morgan's Laws}
\begin{theorem}[De Morgan's Laws]\label{DML}
	\begin{eqnarray*}
		\overline{A\cup B}=\overline{A}\cap \overline{B} \\
		\overline{A\cap B}=\overline{A}\cup \overline{B}
	\end{eqnarray*}
\end{theorem}
\index{I!Inclusion–exclusion Principle}
\begin{theorem}[Inclusion–exclusion Principle]\label{IEP}
	\begin{displaymath}
		\left|\bigcup_{i=1}^n{A_i}\right|=
		\sum_{\emptyset \neq J\subseteq \{1,2,\cdots,n\}}{(-1)^{|J|-1}
			\left|\bigcap_{j\in J}{A_j}\right|}
	\end{displaymath}
\end{theorem}

容斥原理用来求集合并的大小,为了求集合交的大小,可以使用补集转换思想,
由定理~\ref{DML}与~\ref{IEP}可得

\begin{theorem}\label{ExDML}
	\begin{displaymath}
		\left|\bigcap_{i=1}^n{A_i}\right|=
		\left|\overline{\bigcup_{i=1}^n{\overline{A_i}}}\right|=
		|U|+\sum_{\emptyset \neq J\subseteq \{1,2,\cdots,n\}}{(-1)^{|J|}
			\left|\bigcap_{j\in J}{\overline{A_j}}\right|}
	\end{displaymath}
\end{theorem}

以上内容参考了Wikipedia-EN\footnote{Inclusion–exclusion principle - Wikipedia\\
	\url{https://en.wikipedia.org/wiki/Inclusion\%E2\%80\%93exclusion\_principle}

	De Morgan's laws - Wikipedia
	\url{https://en.wikipedia.org/wiki/De\_Morgan\%27s\_laws}}以及
国家集训队2013论文集《浅谈容斥原理》。
\subsection{模意义统计方案}
若要求恰好满足$k$个条件的方案数,且这些条件是等价的。考虑使用至少满足$k$个条件的方案数
容斥求出,后者由于限制条件较松,很容易使用NTT等方法得到。

记$g(x)$为至少满足$x$个条件的方案数,那么$g(i),i\geq k$构成的每种方案都对应
$\binomial{i}{k}$种满足$k$个条件的方案。由容斥可得
$ans=\displaystyle \sum_{i=k}^n{(-1)^{i-k}\binomial{i}{k}g(i)}$。

\section{拉格朗日定理}
\index{L!Lagrange's Theorem}
\begin{theorem}[Lagrange's Theorem]\label{LT}
	若$(S,\oplus)$是一个有限群,$(S',\oplus)$是$(S,\oplus)$的子群,则
	$|S'|$是$|S|$的约数。
\end{theorem}
证明留坑待补。
\index{*TODO!拉格朗日定理证明}

\section{置换群}
\index{P!Permutation Groups}
{\bfseries 置换}是从$[1,n]$到$[1,n]$的一一映射。

置换可以分解为多个循环,计算循环相关数据的方法为:枚举每一个节点
\begin{enumerate}
    \item 若该节点已被访问,则跳过;
    \item 顺着该节点对应的目标节点不断跳跃,标记已访问,直至跳跃到已访问点(即出发点)为止。
    \item 这个环就是一个循环。
\end{enumerate}
\begin{theorem}
    若对于一个置换有$n$个循环,长度分别为$l_1,l_2,\cdots,l_n$,
    则该置换的循环节长度为$lcm(l_1,l_2,\cdots,l_n)$。
\end{theorem}
\paragraph{不动点}
若一个状态$S$经由置换$f$置换后的状态与原状态相同,则状态$S$为$f$的不动点。
\paragraph{等价关系}
对于一个置换集合$F$,若状态$S$能经由$F$中的置换变为状态$S'$,则称$S$与$S'$等价。
\paragraph{等价类}
满足等价关系的状态属于同一等价类。

\subsubsection{Burnside引理}
\index{B!Burnside's Lemma}
\begin{lemma}[Burnside's Lemma]
    等价类数目为置换群$G$中所有置换的不动点数目的平均值。
    \begin{displaymath}
        |X/G|=\frac{1}{|G|}\sum_{g\in G}|X^g|
    \end{displaymath}
\end{lemma}
上述定理证明留坑待补。
\index{*TODO!证明Burnside引理}
\subsubsection{Polya定理}
\index{P!Pólya Enumeration Theorem}

\begin{theorem}[Pólya Enumeration Theorem]
    若对每一个节点进行$m$染色,置换$g$有$c(g)$个循环,则染色方案
    等价类数目为$\displaystyle \frac{1}{|G|}\sum_{g\in G}m^{c(g)}$。
\end{theorem}

证明:一个循环内所有的节点颜色相同,不同循环颜色的选择是独立的,每一个循环颜色选择
方案对应一个不动点,根据乘法原理可知$|X^g|=m^{c(g)}$。

以上内容参考了QAQqwe的博客\footnote{Burnside引理与Polya定理
\url{https://blog.csdn.net/liangzhaoyang1/article/details/72639208}}与
Wikipedia-EN\footnote{
    Burnside's lemma - Wikipedia
    \url{https://en.wikipedia.org/wiki/Burnside\%27s\_lemma}

    Pólya enumeration theorem - Wikipedia
    \url{https://en.wikipedia.org/wiki/P\%C3\%B3lya\_enumeration\_theorem}
}。

\subsubsection{常见题型}
题型来自My\_ACM\_Dream的博客\footnote{polya|burnside定理的一些总结\\
\url{https://blog.csdn.net/My\_ACM\_Dream/article/details/45312365}}。

\paragraph{正方形旋转}
n*n正方形染色:
\begin{itemize}
    \item 旋转$0^\circ$,循环节数$n^2$。
	\item 旋转$90^\circ/270^\circ$,若$n$为偶数,循环节数$\frac{n^2}{4}$;
	若$n$为奇数,循环节数$\frac{n^2-1}{4}+1$。
    \item 旋转$180^\circ$,若$n$为偶数,循环节数$\frac{n^2}{2}$;若$n$为奇数,循环
    节数$\frac{n^2-1}{2}+1$。
\end{itemize}
奇偶循环节数不同的原因是因为$n$为奇数时中间的点自成一个循环节。
\paragraph{正方形反射(对称)}
\begin{tabular}{|c|c|c|}
	\hline
			 & 对角反射& 对边中点反射\\
	\hline
	$n$为奇数 & $\frac{n^2-n}{2}+n$& $\frac{n^2-n}{2}+n$ \\
	\hline
	$n$为偶数 & $\frac{n^2-n}{2}+n$& $\frac{n^2}{2}$\\
	\hline
\end{tabular}
\paragraph{环形旋转}
对于一个有$n$个点的环,旋转$i$个点的置换的循环节数为$(n,i)$。

证明:$i$最小乘上$\frac{n}{(n,i)}$才会被$n$整除,所以每一个循环节的长度为
$\frac{n}{(n,i)}$,循环节个数为$(n,i)$。
\paragraph{环形对称翻转}
\begin{itemize}
	\item $n$为奇数:只有n种置换(以一点一边中点为对称轴),循环节数为
	$[\frac{n}{2}]+1$。
	\item $n$为偶数:\begin{itemize}
		\item 边边中点:$\frac{n}{2}$种,循环节数为$\frac{n}{2}$。
		\item 点点:$\frac{n}{2}$种,循环节数为$\frac{n}{2}+1$。
	\end{itemize}
\end{itemize}
\paragraph{正方体旋转}
注意是{\bfseries 棱边}置换。
\begin{itemize}
	\item 自身不变,置换1种,循环节12个,长度1;
	\item 以对面中心为轴,旋转角为$90^\circ,180^\circ,270^\circ$,
	轴有3种选择,共9种置换。
	\begin{itemize}
		\item $90^\circ/270^\circ$:循环节3个,长度4。
		\item $180^\circ$:循环节6个,长度2。
	\end{itemize}
    \item 以对边中点为轴,旋转角为$180^\circ$,有6对边,置换数为6,
    有5个长度为2的循环和2个长度为1的循环。
    \item 以对顶点为轴,旋转角为$120^\circ,240^\circ$,有4对点,置换数为8,
    均有4个长度为3的循环。
\end{itemize}
总置换数24。
\paragraph{$n$较小}
\begin{itemize}
    \item 颜色不限:裸Polya解决。
    \item 颜色限制:裸Burnside解决。
\end{itemize}
\paragraph{环形旋转且$n$较大}
枚举循环节数(即$d=(n,i)$),利用欧拉函数与容斥解决。
\paragraph{有染色限制}
使用dp与矩阵快速幂解决。


\chapter{组合数学}
\section{Catalan数}
\subsection{性质}
\index{C!Catalan Numbers}
Catalan数是组合数学中的常见数列
\footnote{A000108 - OEIS \url{http://oeis.org/A000108}},其前几项为
\begin{displaymath}
	1, 1, 2, 5, 14, 42, 132, 429, 1430, \ldots
\end{displaymath}

Catalan数(记为$C_n$)满足如下关系:
\begin{eqnarray}
	C_0&=&C_1=1\\
	C_{n+1}&=&\sum_{i=0}^n{C_iC_{n-i}}\label{CT2}\\
	&=&\sum_{i=1}^n{C_iC_{n+1-i}}\label{CT3}\\
	C_n&=&\frac{4n-2}{n+1}C_{n-1}\\
	C_n&=&{2n \choose n}-{2n \choose n+1}=\frac{1}{n+1}{2n \choose n}\\
	C_n&=&\prod_{k=2}^n\frac{n+k}{k}
\end{eqnarray}
根据Striling近似公式
\begin{displaymath}
	n!\sim\sqrt{2\pi n}\left(\frac{n}{e}\right)^n
\end{displaymath}
可得
\begin{displaymath}
	C_n\sim\frac{4^n}{\sqrt{\pi} n^\frac{3}{2}}
\end{displaymath}
\subsection{常见应用}
\subsubsection{括号序列,出栈序列,网格行走}
\paragraph{括号序列} 给定$2n$个位置填上左右括号使括号匹配(对于每一个位置之前的
左括号必须不少于右括号)。
\paragraph{出栈序列} 求将$n$个元素入栈一次(限制顺序)并出栈一次(不限制顺序)
的方案数(对于每一次操作都要保证栈不出现下溢,即入栈元素不少于出栈元素)。
\paragraph{网格行走} 在一个$n*n$的网格内从左下角移动到右上角,纵坐标必须不少于
横坐标,求方案数。
\paragraph{分析}
这三个问题是同构的,都满足操作数为$2n$且限制任意时刻操作A的数目不少于操作B的数目。
它们的答案都是$C_n$,以括号序列问题为例,通过等式~\ref{CT2}理解:
将括号序列看做由一个可分割的序列加上一个不可分割的序列(即最外层有一对配对括号)得来,
左边为$n_1+1$对,右边为$n_2$对,满足$n_1+n_2=n-1$,这种方案的贡献为
$C_{n_1}C_{n_2}$。
\subsubsection{二叉树构型计数}
\paragraph{有$n$个节点的二叉树}
通过等式~\ref{CT2}理解:枚举左右子树大小,满足左右子树节点数为$n-1$。
\paragraph{有$n+1$个叶子节点的满二叉树}
通过等式~\ref{CT3}理解:枚举左右子树叶子节点数,满足其和为$n+1$。
\subsubsection{阶梯填充}
用$n$个长方形填充$n*n$的阶梯的方案数为$C_n$。

不严格证明:填充一个以直角顶点与阶梯顶点为对顶点的长方形,使其分为大小为
$n_1*n_1,n_2*n_2$的两个小阶梯,满足$n_1+n_2=n-1$,分别分配$n_1,n_2$
个长方形的份额,就成为子问题了。该分析满足等式\ref{CT2}。
\subsubsection{凸包分割}
将$n+2$个顶点的凸包分为三角形的方案数为$C_n$。

猜想:最终将分为$n$个三角形。

证明留坑待补。
\subsubsection{圆上点连线}
将圆上的$2n$个点两两配对连线,所连$n$条线段不相交的方案数为$C_n$。

证明留坑待补。

\index{*TODO:Catalan应用证明}
上述内容参考了Wikipedia-EN\footnote{Catalan number - Wikipedia
	\url{https://en.wikipedia.org/wiki/Catalan\_number}}。

\section{Stirling数}
\index{S!Stirling Number}
\subsection{第一类Stirling数}
\newcommand{\stirlingA}[2]{\left[#1 \atop #2\right]}
令$\stirlingA{n}{k}$为把$n$个点放到$k$个环内的方案数,有
\begin{eqnarray*}
    \stirlingA{n}{0}&=&0 \textrm{~for all $n\geq 1$}\\
    \stirlingA{n}{n}&=&1 \textrm{~for all $n\geq 0$}\\
    \stirlingA{n}{k}&=&(n-1)\stirlingA{n-1}{k}+\stirlingA{n-1}{k-1}\\
    \sum_{i=0}^n{\stirlingA{n}{i}}&=&n!
\end{eqnarray*}
证明递推式:
\begin{itemize}
    \item 若将当前点丟给之前的环,则可以选择$p-1$个点在其右边,因此贡献
    $(n-1)\stirlingA{n-1}{k}$。
    \item 当前点自成一环,贡献$\stirlingA{n-1}{k-1}$。
\end{itemize}
\subsection{第二类Stirling数}
\newcommand{\stirlingB}[2]{\left\{#1 \atop #2\right\}}
令$\stirlingB{n}{k}$为把$n$个点放到$k$个集合内的方案数,有
\begin{eqnarray}
    \stirlingB{n}{0}&=&0 \textrm{~for all $n\geq 1$}\\
    \stirlingB{n}{n}&=&1 \textrm{~for all $n\geq 0$}\\
    \stirlingB{n}{k}&=&k\stirlingB{n-1}{k}+\stirlingB{n-1}{k-1}\label{SB3}\\
    \stirlingB{n}{k}&=&\frac{1}{k!}\sum_{i=0}^k{(-1)^{k-i}{k \choose i}i^n}
    \label{SB4}
\end{eqnarray}
证明等式~\ref{SB3}:
\begin{itemize}
    \item 若将当前点丟给之前的集合,则可以选择$k$个集合,因此贡献
    $k\stirlingA{n-1}{k}$。
    \item 当前点自成一集合,贡献$\stirlingB{n-1}{k-1}$。
\end{itemize}

证明~\ref{SB4}:
考虑计算将$n$个点放入$k$个带编号集合且无空集的方案数:
\begin{eqnarray*}
    N&=&k^n+\sum_{i=1}^k{(-1)^i{k \choose i}(k-i)^n}
    \textrm{~(根据定理~\ref{ExDML})}\\
    &=&k^n+\sum_{i=0}^{k-1}{(-1)^{k-i}{k \choose i}i^n}\\
    &=&\sum_{i=0}^k{(-1)^{k-i}{k \choose i}i^n}
\end{eqnarray*}
最后变为无标号时除以$k!$即可。

\section{Lucas/ExLucas}
\index{L!Lucas's Theorem}
\subsection{Lucas定理}
\begin{theorem}[Lucas's Theorem]
	对于非负整数$n,m$以及质数$p$,若
	\begin{eqnarray*}
		n&=&\sum_{i=0}^k{n_ip^i}\\
		m&=&\sum_{i=0}^k{m_ip^i}
	\end{eqnarray*}
	则
	\begin{displaymath}
		{n \choose m}\equiv\prod_{i=0}^k{n_i \choose m_i} \pmod{p}
	\end{displaymath}
\end{theorem}

以上内容参考了Wikipedia-EN\footnote{Lucas's theorem - Wikipedia
	\url{https://en.wikipedia.org/wiki/Lucas\%27s\_theorem}}
\subsection{ExLucas}


\chapter{多项式}
\section{快速傅里叶变换FFT}
\subsection{FFT原理}
FFT求多项式卷积的过程为:$\Theta(n\lg n)$求值->$\Theta(n)$点值乘法->
$\Theta(n\lg n)$插值。

$\Theta(n\lg n)$求值/插值的复杂度是在单位复数根上计算得到的。

\subsubsection{单位复数根}

定义{\bfseries $n$次单位复数根}是满足$\omega^n=1$的复数$\omega$,恰好有$n$个,即
$\omega_n^k=e^{2\pi ik/n},k=0,1,\cdots,n-1$。

定义{\bfseries 主$n$次单位根}$\omega_n=e^{2\pi i/n}$。

下面是关于$n$次单位复数根的性质:

\begin{lemma}[消去引理]\label{FFTL1}
	对于任意整数$n\geq 0,k \geq 0,d>0$,
	\begin{displaymath}
		\omega_{dn}^{dk}=\omega_n^k
	\end{displaymath}
\end{lemma}
证明:
\begin{displaymath}
	\omega_{dn}^{dk}=e^{2\pi i dk/dn}=e^{2\pi i k/n}=\omega_n^k
\end{displaymath}

\begin{inference}\label{FFTI2}
	对于任意偶数$n>0$,有
	\begin{displaymath}
		\omega_n^{n/2}=\omega_2=-1
	\end{displaymath}
\end{inference}

\begin{lemma}[折半引理]
	对于偶数$n>0$,$n$个$n$次单位复数根的平方组成的集合为$n/2$个$n/2$
	次单位复数根的集合。
\end{lemma}
证明:根据引理~\ref{FFTL1}可得$(\omega_n^k)^2=(\omega_n^{k+n/2})^2=
	\omega_{n/2}^k$,每个$n/2$次单位复数根恰好被得到2次。

\begin{lemma}[求和引理]
	对于任意整数$n\geq 1$与不能被$n$整除的非负整数$k$,有
	\begin{displaymath}
		\sum_{i=0}^{n-1}{(w_n^k)^i}=0
	\end{displaymath}
\end{lemma}
证明:
\begin{displaymath}
	\sum_{i=0}^{n-1}{(w_n^k)^i}=\frac{(w_n^k)^n-1}{w_n^k-1}=0
\end{displaymath}
$n$不整除$k$保证了分母不为0。

\subsubsection{DFT}
对于次数界为$n$的多项式
\begin{displaymath}
	A(x)=\sum_{i=0}^{n-1}{a_ix^i}
\end{displaymath}
其DFT为
\begin{displaymath}
	DFT_n(a)=(y_0,y_1,\cdots,y_{n-1})=
	(A(\omega_n^0),A(\omega_n^1),\cdots,A(\omega_n^{n-1}))
\end{displaymath}

\subsubsection{FFT}
FFT采用分治策略,假设$n$是2的幂(不足补0),其步骤如下:
\begin{enumerate}
	\item 若次数界为1,则返回$a_0$。
	\item 定义新的次数界为$n/2$多项式
	      \begin{eqnarray*}
		      A^{[0]}(x)&=&a_0+a_2x+\cdots+a_{n-2}x^{n/2-1}\\
		      A^{[1]}(x)&=&a_1+a_3x+\cdots+a_{n-1}x^{n/2-1}
	      \end{eqnarray*}
	      递归计算其在点$(\omega_n^0)^2,(\omega_n^1)^2,\cdots,(\omega_n^{n-1})^2$
	      的值(实际上递归只求了前一半)。
	\item 该多项式满足等式\begin{equation}\label{RFFTE}
		      A(x)=A^{[0]}(x^2)+xA^{[1]}(x^2)
	      \end{equation}
	      可利用递归计算的值合并。
	      对于$k=0,1,\cdots,n/2-1$,
	      \begin{eqnarray*}
		      y_k&=&y_k^{[0]}+\omega_n^ky_k^{[1]}\\
		      y_{k+n/2}&=&y_k^{[0]}-\omega_n^ky_k^{[1]}
	      \end{eqnarray*}
	      正确性证明:
	      \begin{eqnarray*}
		      y_k&=&y_k^{[0]}+\omega_n^ky_k^{[1]}\\
		      &=&A^{[0]}(\omega_{n/2}^k)+\omega_n^kA^{[1]}(\omega_{n/2}^k)\\
		      &=&A^{[0]}(\omega_n^{2k})+\omega_n^kA^{[1]}(\omega_n^{2k})
		      \textrm{~(根据引理~\ref{FFTL1})}\\
		      &=&A(\omega_n^k) \textrm{~(根据式~\ref{RFFTE})}\\
		      y_{k+n/2}&=&y_k^{[0]}-\omega_n^ky_k^{[1]}\\
		      &=&A^{[0]}(\omega_{n/2}^k)+\omega_n^{k+n/2}A^{[1]}(\omega_{n/2}^k)
		      \textrm{~(根据推论~\ref{FFTI2})}\\
		      &=&A^{[0]}(\omega_n^{2k+n})+\omega_n^{k+n/2}A^{[1]}(\omega_n^{2k+n})
		      \textrm{~(根据引理~\ref{FFTL1}与$\omega_n^n=1$)}\\
		      &=&A(\omega_n^{k+n/2}) \textrm{~(根据式~\ref{RFFTE})}\\
	      \end{eqnarray*}
\end{enumerate}
\subsubsection{逆DFT}
\begin{theorem}
	\begin{displaymath}
		DFT_n^{-1}(a)=\frac{1}{n}DFT_n(DFT_n(a))
	\end{displaymath}
\end{theorem}
证明:

以上内容来自算法导论\cite{ITA3}第30章 多项式与快速傅里叶变换。
\subsection{迭代FFT实现}
\subsubsection{单位复数根预处理}
\subsubsection{离线位逆序}
\subsubsection{在线位逆序}
\subsection{实序列DFT}
\subsection{MTT之拆系数FFT}

\section{快速数论变换NTT}
\subsection{NTT原理}
NTT的原理与FFT类似,即找到单位根$x$满足$x^n\equiv 1 \pmod{p}$。
NTT模数$p$需满足$p$为质数且$p=r\cdot 2^k+1$。

根据定理~\ref{FLT}可知若模数$p$为质数则有$x^{p-1}\equiv 1 \pmod{p}$,
所以当$n|(p-1)$时才能进行NTT。

根据~\ref{PrimitiveRoot}节所述,$p$必有原根,设$p$的原根为$g$,则
$g^\frac{p-1}{n}$就是{\bfseries 主$n$次单位根},$n$个单位根即为
$w_n^k=g^{k\cdot \frac{p-1}{n}}$。

其余部分与FFT相同。

\subsection{NTT实现}
NTT仅预处理单位复数根部分不同,以模998244353为例:
\begin{lstlisting}
int tot, root[size], invR[size];
void init(int n) {
    const int g = 3;
    tot = n;
    Int64 base = powm(g, (mod - 1) / n);
    Int64 invBase = powm(base, mod - 2);
    root[0] = invR[0] = 1;
    for (int i = 1; i < n; ++i)
        root[i] = root[i - 1] * base % mod;
    for (int i = 1; i < n; ++i)
        invR[i] = invR[i - 1] * invBase % mod;
}
\end{lstlisting}
\subsection{NTT常见模数}
\begin{itemize}
    \item $469762049=7*2^26+1$。
    \item $998244353=119*2^23+1$。
    \item $1004535809=479*2^21+1$,加起来不爆int。
    \item $2281701377=17*2^27+1$,平方恰好不爆long long。
\end{itemize}
\index{*Constant!NTT模数P=\{469762049,998244353,1004535809,2281701377\},g=3}
它们的原根均为3。
\subsection{MTT之三模数NTT}
选取3个模数,比如\{469762049,998244353,1004535809\},它们的乘积大于卷积
过程中最大的数,分别以这三个数为模数求NTT,最后解同余方程组即可。

但是使用CRT求解会爆long long,因此先合并前两项,得到
\begin{eqnarray*}
    x&\equiv&n_1 \pmod{p_1}\\
    x&\equiv&n_2 \pmod{p_2}
\end{eqnarray*}
设$x=k_1p_1+n_1=k_2p_2+n_2$,由于$k_1<p_2$,我们可以求解
$k_1p_1\equiv n_2-n_1 \pmod{p_2}$得到$k_1$,带入原式求出$x \bmod{p}$的值。

该方法来自AntiLeaf\footnote{COGS2294 释迦 - AntiLeaf
\url{http://www.cnblogs.com/hzoier/p/6441967.html}}

\section{快速沃尔什变换FWT}
\subsection{FWT原理}
FWT主要用来求下列三种卷积:
\begin{eqnarray*}
    z_n&=&\sum_{i\&j=n}{a_ib_j}\\
    z_n&=&\sum_{i|j=n}{a_ib_j}\\
    z_n&=&\sum_{i\land j=n}{a_ib_j}
\end{eqnarray*}
\subsection{FWT实现}

\section{多项式高级操作}
\subsection{牛顿迭代法}
已知函数$G(z)$,求函数$F(z) \bmod{z^n}$满足$G(F(z))\equiv 0 \pmod{z^n}$。

令$n=2^t$,若$n$不为2的幂,补齐后截断即可。

当$t=0$时,简单地令$F(z)$的常数项为0。

若已知$G(F_0(z)) \equiv 0\pmod{z^{2^t}}$,
可以计算$G(F(z))$在$F_0(z)$上的泰勒展开:
\begin{displaymath}
    G(F(z))=\sum_{i=0}^\infty{\frac{G^{(i)}(F_0(z))}{i!}\cdot (F(z)-F_0(z))^i}
\end{displaymath}
由于$F(z)$与$F_0(z)$后$2^t$项均相等,所以它们之差的平方的最小非0项次数$\geq 2^{t+1}$,
因此仅前两项有效,即
\begin{displaymath}
    G(F(z))\equiv G(F_0(z))+G'(F_0(z))(F(z)-F_0(z)) \pmod{z^{2^{t+1}}}
\end{displaymath}
结合$G(F(z))\equiv 0 \pmod{z^{2^{t+1}}}$可得到新的$F_0(z)$:
\begin{displaymath}
    F(z)\equiv F_0(z)-\frac{G(F_0(z))}{G'(F_0(z))} \pmod{z^{2^{t+1}}}
\end{displaymath}
这就是牛顿迭代法。
\subsection{多项式开方}
对于给定的$A(z)$,求$F(z) \pmod z^n$,使得$F^2(z)\equiv A(z)\pmod{z^n}$。

构造方程$F^2(z)-A(z)\equiv 0\pmod{z^n}$,
同理可得
\begin{eqnarray*}
    F(z)&\equiv& F_0(z)-\frac{F_0(z)^2-A(z)}{2F_0(z)} \pmod{z^{2^{t+1}}}\\
    &\equiv& \frac{F_0(z)^2+A(z)}{2F_0(z)} \pmod{z^{2^{t+1}}}
\end{eqnarray*}

注意当$t=0$时可能需要用二次剩余在模意义下开根。

\subsection{多项式求逆}
对于给定的$A(z)$,求$F(z) \pmod z^n$,使得$F(z)\cdot A(z)\equiv 1\pmod{z^n}$。

构造方程$F(z)\cdot A(z)-1\equiv 0\pmod{z^n}$,
同理可得
\begin{eqnarray*}
    F(z)&\equiv& F_0(z)-\frac{F_0(z)\cdot A(z)-1}{A(z)} \pmod{z^{2^{t+1}}}\\
    &\equiv& F_0(z)-(F_0(z)\cdot A(z)-1)\cdot{F_0(z)} \pmod{z^{2^{t+1}}}
    \textrm{~(考虑最小非0项可知可乘$F_0(z)$代替$F(z)$)}\\
    &\equiv& 2F_0(z)-F_0^2(z)\cdot A(z) \pmod{z^{2^{t+1}}}
\end{eqnarray*}
\subsection{多项式取模}
\subsection{多项式ln}
\subsection{多项式exp}
\subsection{多项式快速幂}
使用常规快速幂可以得到$O(n\lg n\lg k)$的复杂度。
但是通过如下变形:
\begin{displaymath}
    F^k(z)=e^{k \ln F(z)}
\end{displaymath}
使用多项式ln/exp可以得到$O(n\lg n)$的复杂度。
\subsection{多项式三角函数}
由欧拉公式可得
\begin{displaymath}
    e^{F(z)i}=\cos F(z)+\sin F(z) i
\end{displaymath}
在复数域上做多项式exp即可。
\subsection{进制转换}
\subsection{多项式多点求值/插值}
\subsection{组合数取模}
以上内容参考了picks
\footnote{Newton's Method of Polynomial « Picks's Blog
\url{http://picks.logdown.com/posts/209226-newtons-method-of-polynomial}}
和Miskcoo\footnote{牛顿迭代法在多项式运算的应用 – Miskcoo's Space
\url{http://blog.miskcoo.com/2015/06/polynomial-with-newton-method}}的博客。


\appendix
\chapter{资料推荐}
\printindex
\begin{thebibliography}{999}
	\bibitem{NFTGC} Even, Shimon; Tarjan, R. Endre (1975).
	"Network Flow and Testing Graph Connectivity".
    \emph{SIAM Journal on Computing}.
    \bibitem{DSNA} Tarjan, R. E. (1983).
    \emph{Data structures and network algorithms.}
    \bibitem{MCIOI}胡伯涛(2007). \emph{最小割模型在信息学竞赛中的应用}
    \bibitem{ITA3}  Thomas H.Cormen / Charles E.Leiserson /
     Ronald L.Rivest / Clifford Stein.
     \emph{Introduction to Algorithms Third Edition}
    \bibitem{kdTree}n+e. \emph{K-D Tree 在信息学竞赛中的应用}
    \bibitem{huffman}M. J. Golin and G. Rote.
    \emph{A dynamic programming algorithm for constructing optimal
    prefix-free codes with unequal letter costs}
    \bibitem{LLH}L. L. Larmore and D. S. Hirschberg.
    \emph{A fast algorithm for optimal length-limited Huffman codes}
\end{thebibliography}

\backmatter
\chapter{后记}
\end{document}
