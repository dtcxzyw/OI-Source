\documentclass[10pt,AutoFakeBold,AutoFakeSlant,b5paper]{book}
%\usepackage{titletoc}
%\usepackage{titlesec}
%\usepackage{ctexcap}
\usepackage{xeCJK}
\setCJKmainfont{SimSun}
\usepackage[english]{babel}
\usepackage[T1]{fontenc}
\usepackage{listings}
\usepackage{xcolor}
\usepackage{ulem}
\usepackage{minitoc}
\setcounter{minitocdepth}{2}
\setlength{\mtcindent}{24pt}
\renewcommand{\mtcfont}{\small\rm}
\renewcommand{\mtcSSfont}{\small\bf}
\dominitoc[n]
\lstset{
    columns=fixed,
    frame=none,
    numbers=left,
    numbersep=5pt,
    backgroundcolor=\color[RGB]{255,255,255},
    keywordstyle=\color[RGB]{40,40,255},
    commentstyle=\it\color[RGB]{0,96,96},
    stringstyle=\rmfamily\slshape\color[RGB]{128,0,0},
    showstringspaces=false,
    escapeinside=``,
    language=c++,
    morekeywords={alignas,continute,friend,register,true,alignof,decltype,goto,
    reinterpret_cast,try,asm,defult,if,return,typedef,auto,delete,inline,short,
    typeid,bool,do,int,signed,typename,break,double,long,sizeof,union,case,
    dynamic_cast,mutable,static,unsigned,catch,else,namespace,static_assert,using,
    char,enum,new,static_cast,virtual,char16_t,char32_t,explict,noexcept,struct,
    void,export,nullptr,switch,volatile,class,extern,operator,template,wchar_t,
    const,false,private,this,while,constexpr,float,protected,thread_local,
    const_cast,for,public,throw}
}
\usepackage{amssymb}
\usepackage{amsmath}
\usepackage[hyperfootnotes=true]{hyperref}
\usepackage{tabularx}
\usepackage{url}
\usepackage{makeidx}
\usepackage[numbib,numindex,chapter]{tocbibind}
\usepackage{lastpage}
\usepackage{array}
\usepackage{shorttoc}
\makeindex
\pagestyle{headings}
\begin{document}
\newtheorem{theorem}{定理}[chapter]
\newtheorem{lemma}[theorem]{引理}
\newtheorem{property}[theorem]{性质}
\newtheorem{inference}[theorem]{推论}
\newcommand{\ud}{\mathrm{d}}
\newcommand{\binomial}[2]{\left(#1 \atop #2\right)}
\newcommand{\stirlingA}[2]{\left[#1 \atop #2\right]}
\newcommand{\stirlingB}[2]{\left\{#1 \atop #2\right\}}
\title{OI知识点复习笔记}
\author{dtcxzyw}
\frontmatter
\maketitle
\chapter{前言}
本人写复习笔记的目的有两个:
\begin{itemize}
	\item 系统地复习知识点并挖掘一些有用的性质。
	\item 学会使用\LaTeX{}。
\end{itemize}

当前共\pageref{LastPage}页,\input{../latex/charcnt}字。

项目地址:\url{https://github.com/dtcxzyw/OI-Source}

\shorttoc{简略目录}{0}
\renewcommand\contentsname{目录}
\tableofcontents
\mainmatter
\adjustmtc
\chapter{博弈论}
\section{SG函数}
一切Impartial Combinatorial Games都等价于Nim游戏,可以使用SG函数解决。

该类游戏拥有如下特征:

\begin{itemize}

	\item

\end{itemize}

\section{Nim系列游戏}

\subsection{Nim游戏}

普通Nim游戏的定义:
有两个玩家轮流从许多堆中移除对象。在每个回合中,玩家选择一个非空的堆,可以移除任何数量
的对象,但至少移除一个对象。无法操作的玩家为败者。

此类游戏可看做是Bash游戏的特殊化。

\begin{Theorem}
	$SG_{Nim}(x)=x$
\end{Theorem}

证明略。

\subsection{Bash游戏}

Bash游戏与普通Nim游戏的区别是增加了每次最多移除k个对象的限制。

\begin{Theorem}
	$SG_{Bash}(x)=x~mod~(k+1)$
\end{Theorem}

证明略。

\subsection{NimK游戏}

NimK游戏与普通Nim游戏的区别是每次可以从不超过k个堆中移除任意数目对象。

\begin{Theorem}\label{NimK}
	将每堆对象的数目拆位,若每位上1的个数mod(k+1)均为0,则必败,反之必胜。
\end{Theorem}

记忆:普通Nim游戏可理解为mod 2的情况。

算法正确性证明:



定理~\ref{NimK}得证。

\subsection{Anti Nim}

不能操作的玩家胜利。

\begin{Theorem}\label{AntiNim}
	先手必胜当且仅当满足以下条件之一:
	\begin{enumerate}
		\item $SG(x)=0$ 且所有堆的对象数都为1
		\item $SG(x)\not=0$ 且至少有一堆对象数大于1
	\end{enumerate}

\end{Theorem}

证明:
定义对象数为1的叫A堆,大于1的叫B堆。

\begin{enumerate}
	\item 若所有堆均为A堆,则奇数堆先手必败,反之必胜。
	\item 若B堆数等于1,显然$SG(x)\not=0$,则可根据堆的总数确定取该堆的数目,
	      使下一状态为情况1的奇数堆,所以先手必胜。
	\item 若B堆数大于1,则
	      \begin{enumerate}
		      \item 若$SG(x)=0$,则必须留下超过2个B堆并使$SG(x')\not=0$,否则会使
		            对方进入情况2的必胜态。
		      \item 若$SG(x)\not=0$,则根据Nim游戏的理论(必胜态->必败态),存在一种方法转移至情况3的子情况1。
	      \end{enumerate}
	      若玩家处于情况3的子情况2中,则可以在有限次回合内使对方无法转移至子情况2,
	      因此该状态为必胜态。
\end{enumerate}

定理~\ref{AntiNim}得证。

\subsection{阶梯博弈(Staircase Nim)}


\subsubsection{例题}

Luogu P3480 [POI2009]KAM-Pebbles\footnote{\url{https://www.luogu.org/problemnew/show/P3480}}

对于这题可将原条件通过差分转换为阶梯博弈模型($A_i \geq A_i-1 \Leftrightarrow
	A_i-A_i-1 \geq 0$)。

\lstinputlisting[title=Luogu P3480]{Source/'Game Theory'/3480.cpp}

出题灵感:Anti BashK游戏

以上内容参考了forezxl\footnote{anti-Nim游戏(反Nim游戏)简介
	\url{https://blog.csdn.net/a1799342217/article/details/78274410}}和
hehedad\footnote{关于nimk类型博弈的详细理解与解释
	\url{https://blog.csdn.net/chenshibo17/article/details/79783523}}的博客。

\section{Alpha-Beta剪枝}

\section{本章注记}
更多博弈论模型参见~博弈游戏的各种经典模型(备忘) - Randolph87 - 博客园
 \url{http://www.cnblogs.com/Randolph87/p/5804798.html},待补充。
\index{TODO!补充博弈论经典模型}

\chapter{网络流}
\minitoc
\section{二分图}
\index{B!Bipartite Graph}
\subsection{二分图判定}
\begin{property}
	二分图中不存在奇环。
\end{property}

如果存在奇环,则必有一条边的端点属于同一集合。所以可以使用DFS染色来判定二分图,
遇到矛盾则退出。

\lstinputlisting[title=BGJudge.cpp]{NetworkFlows/BGJudge.cpp}

\subsection{二分图最大匹配}

\subsubsection{匈牙利算法}
\index{H!Hungarian Algorithm}

匈牙利算法的主要步骤就是遍历左集合的每一个顶点,使得其尽可能找到一个匹配。
要为该顶点找到一个匹配,首先遍历边,如果右顶点已经有匹配,则递归尝试让该
匹配点重新找一个匹配,如果右顶点无匹配或者更换匹配成功,则这条边是一个匹配。

原则:有机会上,没机会创造机会也要上。
\footnote{Dark\_Scope 趣写算法系列之--匈牙利算法
	\url{https://blog.csdn.net/dark\_scope/article/details/8880547}}

感性的算法的正确性证明:每次递归时匹配数只增不减,且递归有权修改整个连通块
的着色情况。(似乎并没有什么说服力)。

匈牙利算法的时间复杂度为$O(VE)$,每次尝试匹配的复杂度为$O(E)$。

\index{*TODO!匈牙利算法标准描述与正确性证明}

\subsubsection{Greedy Matching}
可以先遍历一次图,贪心地连边,以减少尝试拆开匹配边的次数。
在图很大的时候有加速效果。

该方法参考了江任捷的演算法筆記\footnote{
    演算法筆記 - Matching
    \url{http://www.csie.ntnu.edu.tw/\~u91029/Matching.html\#4}
}。
\subsubsection{Hopcroft–Karp Algorithm}
\index{H!Hopcroft–Karp Algorithm}
暂时先坑着\CJKsout{为什么不写Dinic呢}。
\index{*TODO!Hopcroft–Karp算法}

\subsubsection{例题}

Luogu P1129 [ZJOI2007]矩阵游戏
\footnote{\url{https://www.luogu.org/problemnew/show/P1129}}

首先用二分图最大匹配找到n个不同行且不同列的黑格子(置换矩阵P),然后就可以操作得到
目标矩阵(单位矩阵I)了。

\lstinputlisting[title=Luogu P1129]{Source/Unclassified/Done/1129.cpp}

\subsection{二分图最大权匹配 Kuhn-Munkras Algorithm}
\index{K!Kuhn-Munkras Algorithm}
\CJKsout{先用费用流做吧,暂时先坑着。}
\subsubsection{起步}
维护每个左/右顶点的权值(称为顶标),所有节点的顶标和为答案上界。
令每个左顶点的顶标为出边边权最大值,右顶点顶标为0。

对每个顶点运行匈牙利算法,若左右顶点顶标之和等于边权,则考虑连边;
若无法为当前点找到匹配,则将访问到的左顶点顶标-1,右顶点顶标+1,
等价于使答案上界-1(DFS访问树中的叶子必为左顶点),重新为该点寻找匹配。
把任意二分图当做完全二分图(不存在的边权值为0),迭代必定会结束。

这种做法能够保证在找到最大匹配的情况下使权值和最大。
\subsubsection{优化1}
可以发现在左-1右+1后,原先边权等于左右顶点顶标之和的边仍然被经过,
一个简单的思路是一次性突破``瓶颈'',即令下次增广时终点位置处的某条边从
不可连边变为可连边,每次DFS增广时维护(顶标和-边权)的最小值$d$,
若匹配失败则左$-d$右$+d$。

这才是复杂度比较靠谱的算法($O(n^3)$)。
\subsubsection{优化2}
在匹配每个点时,初始化所有右顶点的松弛函数$slack$为$\infty$,然后
DFS时$slack$维护(顶标和-边权)的最小值。若匹配失败则令$d$为未访问右
顶点的$slack$函数最小值,左$-d$右$+d$,同时未访问节点的$slack-=d$。

该优化的复杂度不变,但实测该方法比优化1的效率更高(3x)。
\subsubsection{优化3}
考虑记录其增广时的路径,然后将递归算法转换为非递归算法。
\begin{lstlisting}
int w[size][size],lh[size],rh[size],pair[size],
    pre[size],slack[size];
bool flag[size];
void aug(int s) {
    reset(flag);
    reset(pre);
    reset(slack,0x3f);
    pair[0]=s;
    int u=0;
    do {
        int v=pair[u],minh=inf,nxt;
        flag[u]=true;
        // 再次DFS后新访问到了点u和它的匹配点
        // 为点v找新匹配点
        for(int i=1;i<=n;++i)
            if(!flag[i]){
                int delta=lh[v]+rh[i]-w[v][i];
                if(delta<slack[i])
                    slack[i]=delta,pre[i]=u;
                    //点i的匹配点有可能置换为u的匹配点,
                    //以腾出u的匹配点的空位
                if(minh>slack[i])
                    minh=slack[i],nxt=i;//点i下次将被访问
            }
        //松弛
        for(int i=0;i<=n;++i)
            if(flag[i])lh[pair[i]]-=minh,rh[i]+=minh;
            else slack[i]-=minh;
        u=nxt;
    } while(pair[u]);//直到找到未匹配点为止
    //置换匹配
    while(u) {
        int p=pre[u];
        pair[u]=pair[p];
        u=p;
    }
}
int KM(int n) {
    for(int i=1;i<=n;++i) {
        int maxh=0;
        for(int j=1;j<=n;++j)
            maxh=std::max(maxh,w[i][j]);
        lh[i]=maxh;
    }
    reset(rh);
    reset(pair);
    for(int i=1;i<=n;++i)
        aug(i);
    int res=0;
    for(int i=1;i<=n;++i)
        res+=w[pair[i]][i];
    return res;
}
\end{lstlisting}
实测该方法比优化2的效率更高(2x)。
\index{*TODO!解释KM算法优化的合理性}
\subsection{二分图常见模型}
\subsubsection{最小点覆盖}
\index{K!König's theorem}
\begin{theorem}[König's Theorem]
	最小点覆盖数=最大匹配数。
\end{theorem}

使用反证法证明:如果有一条边两端顶点都不在最大匹配上,那么这条边可以进入最大匹配
成为一个更大的匹配边集,所以与最大匹配的假设矛盾。

\subsubsection{最大独立集}

\begin{theorem}
	最大独立集大小=顶点数-最小点覆盖数=顶点数-最大匹配数
\end{theorem}

证明:容易发现去掉二分图中的最小点覆盖可得到一个独立集(若其不是独立集,则说明存在一条
边未被覆盖,与点覆盖的定义矛盾)。尝试以此独立集为基础扩展,可以发现若要使点覆盖
中的某个点变为独立集的点,由最小点覆盖数=最大匹配数可知,最小点覆盖的每个点都与$\geq 1$
的边相连,因此必须使不少于1个原独立集的点被删除。所以无论如何修改,最多得到与之大小
相等的独立集。

\subsubsection{DAG最小路径覆盖}

\paragraph{最小不相交路径覆盖}

将顶点拆成左右两点,若存在边$u\rightarrow v$则连边$Lu\rightarrow Rv$,求二分图最大匹配。

\begin{theorem}
	最小路径覆盖数=顶点数-二分图最大匹配数。
\end{theorem}

证明:二分图中每增加一个匹配,就意味着减少一条路径。

\paragraph{最小可相交路径覆盖}

先用Floyd求出传递闭包,转化为最小不相交路径覆盖问题。
因为如果要从a走到b,直接连边可以避开中间点的流量限制。

以上内容参考了罗茜\footnote{二分图详解及总结
	\url{https://www.cnblogs.com/alihenaixiao/p/4695298.html}},
justPassBy\footnote{有向无环图(DAG)的最小路径覆盖
	\url{https://www.cnblogs.com/justPassBy/p/5369930.html}}和
不可不戒\footnote{二分图:最大独立集\&最大匹配\&最小顶点覆盖
	\url{https://blog.csdn.net/lezg\_bkbj/article/details/9872189}}
的博客。
\subsection{Hall定理}
\index{H!Hall's Marriage Theorem}
Hall定理用于判断二分图是否存在完美匹配。
\begin{theorem}\label{Hall}
    二分图$G=\{V1,V2,E\},|V1|\leq|V2|$存在完美匹配当且仅当$V1$中任意$k$个顶点
    至少与$V2$中任意$k$个顶点相连。
\end{theorem}
\paragraph{证明}
充分性:假设二分图$G$不存在完美匹配,记$G$的最大匹配为$M$,$V1$上至少有一点
$u$不在$M$上。由条件可知点$u$有一条不在$M$上的边,记对面的点为$v$。若点$v$不在
$M$上,则与$M$为最大匹配矛盾;否则尝试使用匈牙利算法寻找增广路,记涉及到的$V1$的子集
为$S$,则右边至少有$|S|$个节点与其相连,因而存在增广路,与$M$为最大匹配矛盾。

必要性:由于二分图$G$有完美匹配,$V1$的$k$个顶点至少与各自的匹配相连。

还有一个比较有用的推论:
\begin{inference}
   对于二分图$G=\{V1,V2,E\},|V1|\leq|V2|$,若存在整数$t$,满足$V1$中
   任意节点的度数$\geq t$,$V2$中任意节点的度数$\leq t$,则$G$存在完美匹配。
\end{inference}

\paragraph{例题}
[POI2009]LYZ-Ice Skates

由定理~\ref{Hall}可以考虑枚举所有集合,但复杂度无法接受,考虑排掉一些显然不优的集合。
选出的集合可以分为3类:
\begin{itemize}
    \item 脚的大小连续;
    \item 脚的大小不连续但是鞋号区间连续,把中间未被选中的脚的大小选中,但是鞋号区间不变,
    可以有更充分的证据证明不存在完美匹配;
    \item 脚的大小不连续且鞋号区间不连续,这个集合可以根据鞋号区间的连续性分为
    前两种集合,每个集合是独立的子问题。
\end{itemize}
因此只需考虑脚的大小连续的集合。

记脚的大小为$i$的人数有$a_i$个,根据定理有$\displaystyle \sum_{i=l}^r{a_i}
\leq (r+d-l+1)*k$。让右端为常数,得$\displaystyle \sum_{i=l}^r{(a_i-k)}\leq d*k$,
可用线段树维护最大子段和。

代码:
\lstinputlisting{Source/Templates/Hall.cpp}

上述内容参考了Feynman1999的博客\footnote{
    Hall定理(二分图匹配问题,Hungary算法基础)
    \url{https://blog.csdn.net/feynman1999/article/details/76037603}
}。

\section{最大流}
Dinic与ISAP属于Ford-Fulkerson方法中的SAP(Shortest Augment Path)系。
而HLPP属于Push–Relabel算法。
\subsection{Dinic算法}
\index{D!Dinic}
个人比较喜欢使用Dinic算法\sout{(因为我只会这个)}。

Dinic的计算流程如下:
\begin{enumerate}
	\item BFS建分层图,若找不到增广路则退出;
	\item DFS在分层图上找增广路并修改流量,重复步骤1。
\end{enumerate}

时间复杂度证明:

\begin{enumerate}
	\item \begin{lemma}
		Dinic每次BFS后的阻塞流层数是递增的(即$d[t]$递增)。
	\end{lemma}
	\item 每次BFS的时间复杂度为$O(E)$。
	\item 每次DFS的时间复杂度为$O(VE)$。
\end{enumerate}

因此算法的时间复杂度为$O(V^2E)$。

在容量均为1的图上,Dinic的时间复杂度为$O(min \{ V^\frac{2}{3},E^\frac{1}{2} \} E)$,
证明:

留坑待填,参见\cite{NFTGC}。

做二分图最大匹配时Dinic跑得飞快,时间复杂度$O(\sqrt V E)$,证明:

留坑待填,参见\cite{DSNA}。

\index{*TODO!特殊图下Dinic的时间复杂度证明}

时间复杂度证明源自Wikipedia-EN\footnote{
	Dinic's algorithm - Wikipedia
	\url{https://en.wikipedia.org/wiki/Dinic\%27s\_algorithm}}以及
	permui的博客\footnote{ 最大流算法-ISAP - permui
		\url{https://www.cnblogs.com/owenyu/p/6852664.html}}
\subsubsection{优化}
\begin{itemize}
	\item 当前弧优化:每次从未遍历的边开始遍历,减少重复计算(就算前面的边没满,
	      下一次还可以增广)。
	\item 记录无法增广的点(将其深度设为-1),避免重复计算。
	\item (玄学,未测试)BFS找到一条增广路就退出,无法解释。
	\item 若图为分层图,在Dinic之前贪心预流(依旧玄学,未测试):
	      \begin{enumerate}
		      \item 从$s$开始逐层递推,计算能够流出节点$i$的流量$out[i]$;
		      \item 从$t$开始逐层倒推,计算每条边的实际流量。
	      \end{enumerate}
	      代码:

	      \lstinputlisting[title=PreFlow]{NetworkFlows/PreFlow.cpp}

	      该方法源自沐阳的博客。
	      \footnote{ZOJ-2364 Data Transmission 分层图阻塞流 Dinic+贪心预流 - 沐阳
		      \url{https://www.cnblogs.com/Lyush/p/3204099.html}}
\end{itemize}

\subsubsection{板子}

常规优化:
\lstinputlisting[title=DinicA]{Source/Templates/DinicA.cpp}

玄学优化(注意在随机数据下表现可能更差):

\begin{itemize}
	\item 伸缩操作:首先按照边的容量从大到小排序,然后按照
	$cap>=2^k,2^(k-1),\cdots,2^0$加边,每加一组边跑一次Dinic。
	时间复杂度$O(VE\lg C)$。
	\item 延迟加反向边:建图时仍然加正反向边,但是第一次Dinic
	时避开反向边,第二次Dinic时才考虑反向边。
	\item 不退流跑,一次性退流:BFS失败时才退流,若退流后仍然失败才退出迭代。
\end{itemize}

这些优化参见kczno1的博客\footnote{
	论如何用dinic ac 最大流 加强版
	\url{http://kczno1.blog.uoj.ac/blog/3375}}。

参考代码:

常规优化+伸缩操作+延迟加反向边(实践中还是这个比较好用):
\lstinputlisting[title=DinicB]{Source/Templates/DinicB.cpp}

kczno1的最新做法-不退流跑,一次性退流:
\lstinputlisting[title=DinicC]{Source/Templates/DinicC.cpp}

\subsubsection{当Dinic遇上LCT}

留坑待补。
\index{*TODO!Dinic with LCT}

\subsection{ISAP算法}
\index{I!Improved Shortest Augment Path}

Dinic每次BFS计算分层图的过程为找最短增广路的过程。每次BFS
重新计算层次编号$d$似乎有些浪费,因此ISAP在Dinic的基础上用
DFS直接修改层次编号的方式来优化算法。ISAP的时间复杂度仍然为$O(V^2E)$。
记数组$d[u]$为残存网络中点$u$到汇点的最短距离,为了编码方便让$d[T]=1$。

算法步骤如下:
\begin{itemize}
	\item 迭代DFS增广,若找不到满足$d[u]=d[v]+1$的可增广边则说明此时的最短路标号
	已经过时,为了让点$u$可增广,令$d[u]=min\{d[v]\}+1$。
	\item 若$d[S]>|V|$则说明已不存在简单增广路径,退出迭代。
\end{itemize}

\subsubsection{优化}
\begin{itemize}
	\item 若数组$d$被初始化为0,则DFS需要$O(n^2)$的时间来初始化
	数组$d$。可以在增广前从汇点开始BFS$O(n+m)$预处理数组$d$。
	\item gap优化:维护每种层次编号的数量$gap[d]$,若$gap[d]=0$则说明
	出现了断层,不存在新的增广路。此时简单地令$d[S]=n+1$结束算法。
	\item 类似Dinic可以使用当前弧优化,{\bfseries 但在层次标号被修改后要重置链头}。
	\item 层次标号的修改是连续的,每次增广完后$++d[u]$。
	\item 流量用完后直接退出。
\end{itemize}

板子(代码似乎比DinicA还短而且跑得比DinicB还快):
\lstinputlisting[title=ISAP]{Source/Templates/ISAP.cpp}

{\bfseries 注意$mf=0$时直接返回不要更新层次标号。}

ISAP算法参考了permui的博客\footnote{ 最大流算法-ISAP - permui
\url{https://www.cnblogs.com/owenyu/p/6852664.html}}。

\subsection{HLPP算法}
\index{H!Highest-label push–relabel\\ algorithm}

\sout{算法导论\cite{ITA3}~26.4节讲的推送-重贴标签算法是$O(V^3)$的。。。}

HLPP算法使用``推送-重贴标签''算法,其时间复杂度为$O(V^2\sqrt{E})$。虽然时间复杂度
比Dinic优,但由于HLPP算法上界较紧,在实践中往往跑不过Dinic(加了优化后表现还行)。

\subsubsection{推送-重贴标签算法}

以水流类比网络流,每条边都是一根有流量限制的水管,允许每个点暂时存储一些多余的水,
称为超额流。特别地,源汇点可以长期存储无限多的水。其它点需要伺机将自身的超额流推送
出去,这里给每个节点再引入一个``高度''参数,规定流量只能往低处走。固定源点的高度为$V$。
当某个节点高于源点时,它的超额流将退回给源点。{\bfseries 注意高度可以达到$2V-1$}

该算法由两个基本操作组成:
\begin{itemize}
	\item ``推送'':一个节点把自己的超额流推送给高度比自己低1的节点(源点无高度差限制)。
	\item ``重贴标签'':当一个节点无法推送完超额流时,将自身高度加到
	连边有残存流量的最低邻接点的高度+1。
\end{itemize}

首先令S的出边满流,然后维护超额流节点队列,每次取出节点对其进行推送或重贴标签操作。
直至不存在超额流节点。时间复杂度$O(V^2E)$。

\subsubsection{前置重贴标签算法}

每次重贴标签时将节点移至队首,可将时间复杂度优化至$O(V^3)$。

参见算法导论\cite{ITA3}~第26.5节。

\subsubsection{HLPP实现与优化}

使用优先队列以高度为关键字维护超额流节点,每次选取最高标号的节点进行``推送-重贴标签''。

优化:
\begin{itemize}
	\item gap优化:当一个点被重贴标签后,若没有其他点拥有其原来的高度,
	高于此高度的点就无法把流量推送到汇点。将这些点的高度全部设为$V+1$使其流量
	流回源点。
	\item 高度预计算(我因此而TLE多次):将$d$初始化为每个点到汇点的最短路径长。
	{\bfseries 注意源点的高度固定为$V$。}
	\item 使用桶维护优先队列:注意到高度值的范围不大,使用桶来维护较为快速。
\end{itemize}

板子:

优先队列版:
\lstinputlisting[title=HLPPA]{Source/Templates/HLPPA.cpp}

桶版(参考PM250的代码\footnote{
	R13845988 评测详情
	\url{https://www.luogu.org/record/show?rid=13845988}
},自己不会用vector然后就用set代替了,常数大好多):
\lstinputlisting[title=HLPPB]{Source/Templates/HLPPB.cpp}

HLPP算法参考了Mr\_Spade的博客\footnote{
	网络最大流——最高标号预流推进
	\url{https://www.cnblogs.com/Mr-Spade/p/9636935.html}
}。

\subsection{最大流与最小割}

\index{M!Max-flow min-cut theorem}
\begin{theorem}[Max-flow min-cut theorem]\label{MFMCT}
	最大流=最小割。
\end{theorem}

证明:
\begin{itemize}
	\item
	\begin{lemma}\label{MCA}
		最大流$\leq$最小割
	\end{lemma}
	由于流量被割边所限制,所以最大流$\leq$任意割,所以最大流$\leq$最小割。
	\item
	\begin{lemma}\label{MCB}
		最大流$\geq$最小割
	\end{lemma}
	证明:跑完最大流后残量网络内$s$与$t$不连通,所以得到了一个割,
	即最大流$\geq$最小割。
\end{itemize}

结合引理~\ref{MCA}与~\ref{MCB}可得最大流=最小割。
\subsection{无向图最小割}
\subsubsection{Stoer-Wagner Algorithm}
\index{S!Stoer-Wagner Algorithm}
若需要求全局最小割,使用Stoer-Wagner Algorithm。

算法步骤如下:
\begin{enumerate}
	\item 任意指定一个节点作为初始点集;
	\item 查询到点集内的点边权和的最大的点集外的点;
	\item 合并最后加入的两个节点$s,t$并更新最小割;
	\item 重复第一步直至整个图被合并。
\end{enumerate}
具体做法见代码。边权可用优先队列维护,时间复杂度$O(|V||E|\lg |E|)$。

模板(SP12056 FZ10B - Nubulsa Expo):
\lstinputlisting{Source/Templates/Stoer-Wagner.cpp}

这题$|V|$比较小所以可以用邻接矩阵存图,$O(|V|^3)$解决。
\lstinputlisting{Source/Templates/Stoer-WagnerV3.cpp}

不知为何两种方法在SPOJ上都TLE了。
\index{*TODO!证明无向图最小割算法的正确性并修改模板}
上述内容参考了Oyking的博客\footnote{
	全局最小割StoerWagner算法详解
	\url{https://www.cnblogs.com/oyking/p/7339153.html}
}。
\subsubsection{流量构造法}
若指定源汇点,连边时给正反向边的残余流量都初始化为割边代价,然后跑Dinic。

\section{费用流}
\index{M!MCMF}
从普通的EK算法扩展,既然每次增加的流量是一样的,那么我们就选择费用最小(大)
的增广路径,从而保证在得到最大流的前提下费用最小(大)。

求最短路时使用SPFA,若没有负权边尽量使用Dijkstra。

一般的建图思路是通过流量限制来保证方案合法,然后设计边的费用引导至最优代价。

\subsection{使用Dijkstra实现费用流}\label{DijMCMF}
其实即使有负权边,也是可以使用Dijkstra来求费用流的\CJKsout{(但是仍然需要SPFA)}。

核心思想是对原图适当地修改变为不带负权边的图。首先在迭代外用SPFA求从源点到每个点的最短路,
记距离为$h[u]$,满足三角不等式$h[u]+w[u][v]\geq h[v]$。将该式变形得$w[u][v]+h[u]-h[v]
\geq 0$,令左式为边的新权值,就可以在迭代中使用Dijkstra求最短路了,记实际最短距离为
$md[i]$,计算得到的最短距离为$dis[i]$。容易发现最短路径边权和中的$h[]$抵消后,可以得到
$dis[i]=md[i]+h[S]-h[i]$,其中$h[S]=0$,那么实际距离比计算距离多$h[i]$。由此可得
本次迭代产生的费用贡献为$(dis[T]+h[T])*minf$,并且需要在当前迭代结束前更新最短距离
$h'[u]=h[u]+dis[u]$。

上述内容参考了Mogician的博客\footnote{
    最大流与Dijkstra做费用流 - Mogician's blog - 洛谷博客
    \url{https://www.luogu.org/blog/Mogician/Network-Flow-Guide}}。
\subsection{多路增广费用流}
一般使用该方法作为费用流模板。

与普通费用流的差别如下:
\begin{itemize}
    \item 使用vector存边对缓存友好
    \item SPFA使用SLF带容错优化
    \item SPFA从T开始找增广路
    \item DFS多路增广从S沿着最短路跑,可以使用当前弧优化,注意不要走环
\end{itemize}

参考代码:
\lstinputlisting{Source/Templates/MCMF.cpp}

该方法来自Melacau的博客(翻fjsdfzoj时发现的)\footnote{
    【模板】板子的集合\\
    \url{https://www.cnblogs.com/Melacau/p/ban.html}
}。

\section{带上下界网络流}
\section{常见网络流/最小割模型}
\subsection{平面图转对偶图}

平面图与对偶图的定义:
\begin{itemize}
	\item 平面图(Planar Graph):在平面上画出来可以使边与边只在顶点上相交的图。
	      \index{P!Planar Graph}
	\item 对偶图(Dual Graph):将平面图的每条边两边的区域连边而成的新平面图。
	      \index{D!Dual Graph}
\end{itemize}

记平面图$G$的对偶图为$G^*$,平面集合为$P_G$。

对偶图$G^*$有两个性质:
\begin{itemize}
	\item
	      \begin{character}
		      $G^*$中的环对应$G$中的一个割。
	      \end{character}
	\item
	      \begin{character}
		      $|P_G|=|V_{G^*}|,|E_G|=|E_{G^*}|$
	      \end{character}
\end{itemize}

实际应用时,首先连接$(s,t)$使得外部平面被分为两个平面,以获得源汇点$s',t'$(同时连
到一个点上并没有什么用),然后按照定义建图即可(注意不要加入边$s',t'$)。

那么$s,t$的最小割=$s'->t'$的最短路(即拆点前的最小环),时间复杂度降低不少。

\subsubsection{例题}

Luogu P4001 [BJOI2006]狼抓兔子\footnote{【P4001】[BJOI2006]狼抓兔子 - 洛谷
\url{https://www.luogu.org/problemnew/show/P4001}}

根据定理~\ref{MFMCT}转换为求最大流,将右上角当做起点,右下角当做终点,然后使用上述
方法连边即可。

\lstinputlisting[title=Luogu P4001]{Source/Unclassified/Done/4001.cpp}

\subsection{最大权闭合子图}

$S$向非负权点连容量为权值的边,负权点向$T$连容量为权值相反数的边,如果选择点$u$必须
选择点$v$,就从$u$向$v$连容量为$\infty$的边。

答案=正权值之和-最小割。

简单理解:如果割去正权点的权值,则说明舍弃该正权点,权值从答案中扣除;如果割去负权点
的权值,则说明选择之前的正权点并从答案扣除该负权值。

严格的正确性证明待补充。\index{TODO!最大权闭合子图算法的正确性}

\subsubsection{板子}

Luogu P4174 [NOI2006]最大获利\footnote{【P4174】[NOI2006]最大获利 - 洛谷
\url{https://www.luogu.org/problemnew/show/P4174}}

\lstinputlisting[title=Luogu P4174]{Source/Source/'Network Flows'/4174.cpp}

\subsubsection{输出方案}

\begin{theorem}
    Dinic最后一次增广时可访问到的点就是最终方案。
\end{theorem}

简单理解:最后一次增广后BFS必然找不到增广路,此时割掉的边无法继续增广,对应的点无法被
访问到,剩余的点就是最终方案了。

上述内容参考了appgle\footnote{网络流算法基本模型 - appgle
	\url{https://www.cnblogs.com/hyl2000/p/6618519.html}},
MaxMercer\footnote{关于平面图到对偶图的转化 \\
	\url{https://blog.csdn.net/MaxMercer/article/details/77976666}}和
Cold\_Chair\footnote{网络流——最大权闭合子图 \\
	\url{https://blog.csdn.net/Cold\_Chair/article/details/52841351}}
的博客。

\section{最小割树}
\index{G!Gomory–Hu Tree}
\subsection{构造}
考虑单次求最小割的过程,最小割将顶点集合一分为二,设求$u-v$的最小割$cut(u,v)$,
顶点被分割为集合$U,V$。
\begin{lemma}
	$\forall x\in U,y\in V,cut(x,y)\leq cut(u,v)$
\end{lemma}
\paragraph{证明}
若存在$x,y$使得$cut(x,y)>cut(u,v)$,则$cut(u,v)$无法把$x,y$分开,也就意味着无法把
$u,v$分开,$cut(u,v)$不是割。

然后对每个点集再次选择两个点求最小割,将其切分为两个集合,直到所有集合都只有一个点为止。
每次求完最小割后给$u-v$连一条权重为$cut(u,v)$的边,这样做最后能得到一棵树。
\begin{theorem}
	$u-v$在树上的链上最小边权等于$cut(u,v)$。
\end{theorem}
\subsection{询问}
建出树后可以使用倍增法或线段树+树链剖分$O(\lg n)$查询点对答案。

板子:
\lstinputlisting{Source/Templates/GHT.cpp}
上述内容参考了UranusITS的博客\footnote{
	[学习笔记]最小割树(Gomory-Hu Tree)
	\url{http://www.cnblogs.com/coder-Uranus/p/9771919.html}
}。

\subsection{Gusfield算法}
\index{G!Gusfield Algorithm}

\subsection{应用}
\subsubsection{k小割}
\subsubsection{最小割计数}

\section{技巧总结}
\subsection{最大流}
\begin{itemize}
    \item 若一个点只能被经过有限次,将其拆为入点和出点,入点到出点连流量为
    经过次数限制的边。
    \item 树形最大流可以贪心解决。
\end{itemize}
\subsection{最小割}
\begin{itemize}
    \item 最大化收益可以理解为已经拿到所有收益,最小化损失。然后将其转化为最小割解决。
    \item 使用$+\infty$边描述依赖关系,可以保证这条边不出现在最小割中。
    \item 用$S,T$与点的连边来表示点的权。
\end{itemize}
\subsection{费用流}
\begin{itemize}
    \item 要求费用最小且边数最小:类比进制的思想,实际费用乘以一个大于总边数的因子,再加上
    1作为该边边权。
    \item 若已知走一条边之前必定已经走完了另外几条边,则考虑动态加边。
    \item 对于层数较少,结构简单的图,考虑使用其它数据结构贪心模拟费用流。
    \item (待验证)判断一条边是否一定被选:在残量网络上跑SPFA,若距离差不等于边权则必选。
    \item 餐巾计划问题:
\end{itemize}

上述内容参考了胡伯涛的2007年国家集训队论文《最小割模型在信息学竞赛中的应用》
\cite{MCIOI}和租酥雨的博客\footnote{
    网络流总结\\
    \url{https://www.cnblogs.com/zhoushuyu/p/8137534.html}
}。


\chapter{数据结构}
\section{树状数组}
\subsection{标号管辖范围}

\subsection{lowbit函数原理}

lowbit函数定义为:$lowbit(i)=i\&-i$。

由于$i$始终为正,所以$-i$的补码表示是$i$的位取反再加1。$i$末尾的0对应取反后的1,
再加1后就会变成$1000$的形式,1的位置就是$i$末尾1的位置,而$i$与$-i$之前的位均不同,
所以位与后为0,因此$i\&-i$仅保留末尾1的位。


\section{线段树}
\subsection{技巧}
\subsubsection{全局最优值剪枝}
可以使用全局变量维护自己当前遍历到的最优值,若父节点维护的信息表明管辖范围内不可能
出现更优值,则直接返回减少递归深度。(在kd-tree中比较有效)
\subsubsection{标记永久化}
直接将对整个区间的操作存到标记中而不下放,统计时加回去,减少常数。
\subsection{zkw线段树}
\index{Z!zkw's Segment Tree}
留坑待填。
\index{*TODO!zkw线段树}
\subsection{势能分析线段树}
对于某种无法打标记的区间操作(例如区间开根号),若该操作对某个元素施加少数次该操作就会使其
趋于稳定或区间内的值相等,同样可以使用线段树。每次区间操作暴力修改,合并时维护下次操作是否
可以跳过/缩点。

更多应用需要SegmentTreeBeats,留坑待填。
\index{*TODO!Segment Tree Beats}
\index{S!Segment Tree Beats}
\subsection{线段树分治}
留坑待填。
\index{*TODO!线段树分治}

\section{划分树}
划分树是一种类似于线段树但很少使用的数据结构,用来求解区间第k大问题,可用
主席树代替,权当了解。
\subsection{构建}
和线段树类似,将每一段区间在下一层分为两个子区间,即以这段区间的中位数将区
间内的数划分为左右两部分,并且同一边的数之间相对次数不变。为了支持查询,还
需要记录区间内每一个数及之前的数有多少个划分到左区间中。
\subsubsection{注意事项}
\begin{itemize}
    \item 由于每一层都只有n个数,所以只要每一层开n个数的数组即可,记$A[d]$为
    第$d$层的划分情况,$cnt[d][i]$为第$d$层的第$i$个数及同区间之前的数有多
    少个数进入了左子树。
    \item 中位数可预先对数组排序获得,记排序后数组为$B$。
    \item 注意有多个中位数的情况(否则左区间的数将覆盖到右区间的存储区域上)。
\end{itemize}
\subsection{查询}

\begin{enumerate}
    \item 当到达叶子节点时直接返回$A[d][i]$或$B[i]$。
    \item 利用$cnt$数组差分可以查询到$[l,r]$之间有多少数进入了左子树,
    决定往哪棵子树走。
    \item 根据$cnt$数组重新计算目标范围在下一层的区间,递归查询。
\end{enumerate}

\subsection{板子}

SP3946 MKTHNUM - K-th Number

\lstinputlisting[title=PartitionedTree.cpp]
{DataStructure/PartitionedTree.cpp}

以上内容参考了hchlqlz\footnote{划分树讲解 - hchlqlz
\url{https://www.cnblogs.com/hchlqlz-oj-mrj/p/5744308.html}}的博客。

\section{平衡二叉树}
\subsection{FHQTreap}\label{FHQTreap}
\index{F!FHQTreap}
FHQTreap是一种比较好写的平衡二叉树,虽然效率不太高
(不如子节~\ref{splay}\\的splay),但其易于理解,不需要旋转,
且对可持久化友好。
FHQTreap的使用基于两个基本操作:split与merge。

对于二叉搜索树的操作,由于FHQTreap本来就能够满足二叉搜索树的定义
,因此操作方法相同(在对效率要求不高的情况下灵活地使用split和merge
还可以减少代码量)。

对于序列操作,使用子节~\ref{split}所述的splitKth即可完成区间提取。

\subsubsection{split}\label{split}

split函数的作用是将一棵树按照权值或位置划分成两棵子树。

\begin{itemize}
\item
按位置划分:$split(rt,k,x,y)$表示将以$rt$为根的树分为以$x$为根的
左子树和以$y$为根的右子树,其中左子树内的节点是原树的前$k$个,需要维护
每个节点的子树大小$siz$。

代码如下:
\begin{lstlisting}[title=splitKth]
    void split(int u,int k,int& x,int& y) {
        if(u) {
            push(u);
            int lsiz=T[T[u].ls].siz;
            //`决定当节点u为第k个时被分到哪棵子树`
            if(k<=lsiz) {
                y=u;
                split(T[u].ls,k,x,T[u].ls);
            }
            else{
                x=u;
                split(T[u].rs,k-lsiz-1,T[u].rs,y);
            }
            update(u);
        }
        else x=y=0;
    }
\end{lstlisting}
\item
按权值划分:$split(rt,k,x,y)$表示将以$rt$为根的树分为以$x$为根的
左子树和以$y$为根的右子树,其中左子树内的节点值均小于等于$k$。

代码如下:
\begin{lstlisting}[title=splitKey]
    void split(int u,int k,int& x,int& y) {
        if(u) {
            push(u);
            //`=决定当T[u].val==k时被分到哪棵子树`
            if(T[u].val<=k) {

                x=u;
                split(T[u].rs,k,T[u].rs,y);
            }
            else{
                y=u;
                split(T[u].ls,k,x,T[u].ls);
            }
            update(u);
        }
        else x=y=0;
    }
\end{lstlisting}
\end{itemize}

根据树的实际意义(二叉搜索树还是序列)以及实际需要来决定使用哪种split。

\subsubsection{merge}

merge将两棵树按照左右顺序(中序遍历)合并。
和treap一样,merge使用随机权重来保持树的平衡。

代码如下:

\begin{lstlisting}[title=merge]
    int merge(int u,int v) {
        if(u && v) {
            if(T[u].pri<T[v].pri) {
                push(u);
                T[u].rs=megre(T[u].rs,v);
                update(u);
                return u;
            }
            else{
                push(v);
                T[v].ls=merge(u,T[v].ls);
                update(v);
                return v;
            }
        }
        return u|v;
    }
\end{lstlisting}

\subsubsection{指示权重的伪随机数生成器}\label{WRG}

显然FHQTreap也是一个Treap,所以需要一个表现良好的伪随机数生成器
来指示该节点的权。

最易于实现的伪随机数生成算法就是线性同余法(LCG)了。
\index{L!Linear Congruential\\ Generator}
C++11中<random>的$std::linear\_congruential\_engine$给出
了两组预置的参数:
\begin{itemize}
    \item $minstd\_rand0:(a=16807, c=0, m=2147483647)$

    Discovered in 1969 by Lewis, Goodman and Miller, adopted as
    "Minimal standard" in 1988 by Park and Miller.
    \item $minstd\_rand:(a=48271, c=0, m=2147483647)$

    Newer "Minimum standard", recommended by Park, Miller, and Stockmeyer in 1993.

\end{itemize}

通常选择第2个即$a=48271$,代码如下:

\begin{lstlisting}[title=minstd\_rand]
int getRand() {
    static int seed = 347;
    return seed = seed * 48271LL % 2147483647;
}
\end{lstlisting}

现在来试试说明它的优越性:

\begin{itemize}
    \item 根据节~\ref{PrimitiveRoot}所述,如果$g$是模数$P$的一个原根,
    则$g$的幂模$P$可以取到$[1,P-1]$内的每一个数,且循环周期长度为$P-1$。
    由于2147483647是梅森素数,所以它一定存在原根。
    下列程序可证明48271是2147483647的一个原根:
    \lstinputlisting[title=RandomTestA.cpp]{DataStructure/RandomTestA.cpp}
    \item 48271可以较早地使int溢出,从而避免出现$a$过小而导致``锯齿波''。
    下列程序可证明48271可以满足OI考试的需要:
    程序输出minc=7884 maxc=8515 except=8192 s2=88.492,可见数据还是蛮均匀的。
\end{itemize}

参见cppreference\footnote{
    \url{https://en.cppreference.com/w/cpp/numeric/random/
    linear\_congruential\_engine}}与
    Wikipedia-EN\footnote{
    \url{https://en.wikipedia.org/wiki/Linear\_congruential\_generator}}。

如果需要更均匀的随机数,可以使用如下方案(质量从低到高):
\begin{enumerate}
    \item 使用比LCG更好的梅森旋转算法
    \footnote{std::mersenne\_twister\_engine - cppreference.com
    \url{https://en.cppreference.com/w/cpp/numeric/random/
    mersenne\_twister\_engine}};
    \item Intel指令集内置RDRAND;
    \item 使用由一些机构提供的真随机数生成器SDK,如\url{https://www.random.org/};
    \item 在需要蒙特卡洛采样的场合使用低差异序列如Halton,Sobol等。
\end{enumerate}

\subsection{splay}\label{splay}
\index{S!splay}

由于Treap做LCT复杂度多一个log(而且我还看不懂),所以还是学一下好了。

splay主要由$rotate$和$splay$函数组成:

\subsubsection{rotate}

$rotate(u)$表示将节点$u$旋转到$u$的父亲上。



在实践中可使用$connect(u,p,c)$把节点$u$挂到节点$p$的位置$c$下,$getPos(u)$获得
节点$u$相对于父亲的位置。

代码如下:

\begin{lstlisting}[title=rotate]
int getPos(int u) {
    return u == T[T[u].p].c[1];
}
void connect(int u, int p, int c) {
    T[u].p = p;
    T[p].c[c] = u;
}
void rotate(int u) {
    int ku = getPos(u);
    int p = T[u].p;
    int kp = getPos(p);
    int pp = T[p].p;
    int t = T[u].c[ku ^ 1];
    T[u].p = pp;
    if (!isRoot(p))
        connect(u, pp, kp);
    connect(t, p, ku);
    connect(p, u, ku ^ 1);
    update(p);
    update(u);
}
\end{lstlisting}

\subsubsection{splay}

\subsubsection{具体应用}

对于二叉搜索树的操作,同子节~\ref{FHQTreap}相同,直接用二叉搜索树的
操作即可。

对于序列操作,可使用splay来提取区间:

\begin{enumerate}
    \item
\end{enumerate}

上述内容参考了自为风月马前卒的博客\footnote{splay详解(一) - 自为风月马前卒
\url{http://www.cnblogs.com/zwfymqz/p/7896036.html}}。

\section{动态树}
\subsection{常规操作}
\subsection{技巧}
\subsubsection{DSU代替连通性检测}

\section{并查集}\label{DSU}
\index{D!Disjoint Set Union}
\subsection{路径压缩}
路径压缩的原理很简单,即把找到的最新的祖先存储下来,于是该节点的深度被缩为1。
注意路径压缩会使树的形状改变,若维护的数据与树的形状有关则只能使用
LCT或本节~\ref{RankMerge}所述的按秩合并。设$find$次数为$f$,时间复杂度
$O(n+f\cdot (1+\log_{2+\frac{f}{n}}n))$。
\subsection{按秩合并}\label{RankMerge}
对于每个节点维护秩,代表该节点高度的上界。合并时按照启发式策略将较小秩的连通块
并到较大的连通块。设总操作数为$m$,时间复杂度为$O(m+n\lg n)$。
\subsection{复杂度证明}
留坑待填,参见算法导论\cite{ITA3}中的21.3与21.4节。
\index{*TODO!并查集复杂度证明}
\subsection{并查集的分裂}
若要将集合中的某个点从原集合剥离,且不需要可持久化(即不查询历史信息),则可以
考虑``金蝉脱壳'',即保留原节点不动,但消除其对集合的影响,另建新点代表该点。
具体步骤为为每个点维护一个$id$,指向当前实际所指向的点,分裂时先消除原$id$指向的点
对原集合的影响,再创建新的点并更新$id$。

\subsubsection{例题}

UVA11987 Almost Union-Find \footnote{
    【UVA11987】Almost Union-Find - 洛谷
    \url{https://www.luogu.org/problemnew/show/UVA11987}}

步骤同上所述,{\bfseries 注意只有同时使用路径压缩和按秩合并才能达到
$O(m\alpha(n))$的复杂度}。

代码如下:
\lstinputlisting[title=UVA11987]{DataStructure/UVA11987.cpp}

\subsection{并查集重构树}
类似于Kruskal重构树,当两个连通块相连时建一个新的父亲节点并连边,
可以发现两个节点第一次被连接的时间点就是它们的LCA建立的时间点。

\paragraph{例题~BZOJ 3712: [PA2014]Fiolki}

可以发现两种物质在同一瓶内当它们在并查集重构树上的LCA建立时。
以LCA深度为第一关键字,反应优先级为第二关键字对反应进行排序。
按照排序后的顺序模拟反应。

代码:
\lstinputlisting{Source/Source/DSU/BZOJ3712.cpp}

该内容参考了小蒟蒻yyb的博客\footnote{
    【BZOJ3712】Fiolki(并查集重构树)
    \url{https://www.cnblogs.com/cjyyb/p/9368629.html}
}。

\section{K-D Tree}
K-D Tree是一棵二叉树,每一层按照某个轴将本空间内的所有点分为较为均匀的两部分,
该节点保存划分的中点。查询时依靠不断剪枝来提高查询速度。
\subsection{构树}
具体步骤如下:
\begin{enumerate}
	\item 对于当前子空间,选取一个轴来划分(使用$std::nth\_element$)出中点;
	\item 将中点存储在当前节点上;
	\item 递归建左右子树;
	\item 更新子树信息。
\end{enumerate}
$std::nth\_element$的复杂度为$O(n)$,因此构树的复杂度为$O(nlgn)$。
\subsection{插入}
\subsubsection{离线标记}
构树时将所有点加入,记录每个点的id,然后加入点时打标记一路更新即可,
不过这样做影响了查询的复杂度。
\subsubsection{替罪羊树}
与二叉搜索树的插入相同,注意需要确定每一个节点的划分轴。当二叉树不平衡时会影响
查询复杂度,采用替罪羊树的策略,维护每棵子树的size,如果
$max(siz_l,siz_r)>=siz_u \cdot fac$则暴力重构子树(注意每次插入只要找到最高
的不平衡子树重构即可),一般$fac$取0.75。
\subsection{删除}
删除节点后的处理方法与插入相同。
注意被删除的节点可以gc。

\begin{lstlisting}[title=gc]
    std::vector<int> pool;
    int newNode() {
        static int cnt=0;
        int id;
        if(pool.size()) {
            id=pool.back();
            pool.pop_hack();
        }
        else id=++cnt;
        return id;
    }
    void freeNode(int u) {
        pool.push_back(u);
    }
\end{lstlisting}

\subsection{查询}
\begin{enumerate}
	\item 如果整棵子树均不满足要求,就直接返回;
	\item 如果整棵子树均满足要求且可以不需要继续递归,就记录答案(或者打标记)后返回;
	\item 计算当前节点;
	\item 递归左右子树。
\end{enumerate}
在随机数据下,查询的时间复杂度是$O(lgn)$,在构造数据下复杂度约
是$O(n^\frac{d-1}{d})$。
证明待补充。
\index{TODO!K-D Tree查询复杂度证明}
\subsection{估值}
下列为一些常见估值函数:

由于每个方向上是的独立的,对每个方向贪心后加起来即可。
\subsubsection{曼哈顿距离最小}
$w=\sum_{i=1}^d{max(mind_i-p_i,0)+max(p_i-maxd_i,0)}$
当$p_i$在区域内时估值为0,在一边时估值为到最近一边的值(另一边由于符号问题值为0)。
\subsubsection{曼哈顿距离最大}
$w=\sum_{i=1}^d{max(abs(mind_i-p_i),abs(maxd_i-p_i))}$
选择距离最大的一边。
\subsubsection{欧几里得距离最小}
$w=\sum_{i=1}^d{max(mind_i-p_i,p_i-maxd_i,0)^2}$
\subsubsection{欧几里得距离最大}
$w=\sum_{i=1}^d{max((mind_i-p_i)^2,(p_i-maxd_i)^2)}$
\subsection{技巧}
\subsubsection{全局最优值剪枝}
如果通过该节点维护的子树信息可以确定子树内不存在更优解,搜索该子树
已经没有意义了。还可以搭配另一个优化:先求两棵子树的估价函数值,
选择最优的先进入(更有可能获得最优值然后减少在另一棵子树上的计算)。
\subsubsection{预处理降维}
如果插入与查询离线,则可以对某一维排序,边插入边查询,降低kd-Tree查询复杂度。

以上内容参考了n+e的课件\emph{K-D Tree 在信息学竞赛中的应用}\cite{kdTree}。

\section{堆}
\subsection{左偏树}
\index{L!Leftist Tree}
左偏树(Leftist Tree)也是一种二叉堆,核心操作是$merge$函数,
它可以以$O(\lg n)$合并两棵左偏树。

定义外节点为没有左子树或右子树的节点。对于左偏树的每一个节点,维护其到子树外节
点的最近距离,其中外节点的$dist=0$,$null$的$dist=-1$(其实没必要太严格,
差不多平衡就够了)。

左偏树具有左偏性质:
\begin{property}\label{LTC}
    $dist_l \geq dist_r$
\end{property}

由此定义可得到一个推论:

\begin{inference}
    $dist_u=min(dist_l,dist_r)+1=dist_r+1$
\end{inference}

考虑一棵距离为$k$的左偏树的最小节点数,得到以下定理:

\begin{theorem}
    一棵距离为$k$的左偏树为满二叉树时节点数最少,有$2^{k+1}-1$个节点。
\end{theorem}

由此得到推论~\ref{LTI}:

\begin{inference}\label{LTI}
    一棵节点数为$n$的左偏树,距离最大为$[lg(n+1)-1]$。
\end{inference}

先给出引理~\ref{LTL}:

\begin{lemma}\label{LTL}
    左偏树的最右链恰好有一个外节点。
\end{lemma}

证明:由于左偏树是一棵树,最右链至少有一个外节点;若存在两个及以上的外节点,则
对于某个非深度最深的点,必有右子树(否则链就断了),却没有左子树(由外节点定义可知),
与性质~\ref{LTC}矛盾。

由推论~\ref{LTI}与引理~\ref{LTL}可得如下定理:

\begin{theorem}\label{LTT}
    一棵由$n$个节点组成的左偏树最右链最多有$[lg(n+1)]$个节点。
\end{theorem}

以上证明参考了阿波罗2003的博客\footnote{
    浅谈左偏树 - 阿波罗2003
    \url{https://www.cnblogs.com/yyf0309/p/LeftistTree.html}
}。

我原来的简单理解:对于左偏树中的每一个节点,维护其子树高度。每次$merge$时先往右子树塞,
若右子树的深度比左子树的深度更大,就把左子树换过来塞。以此保证树的高度尽可能小。

$merge(u,v)$的操作如下:

\begin{enumerate}
    \item 如果$u$或$v$有一个为$null$则返回另一个节点;
    \item 若$v$应该在$u$的上一层则$swap(u,v)$;
    \item 递归将节点$v$的子树与$u$的右子树合并;
    \item 若$dist_l<dist_r$则$swap$左右子树;
    \item 更新节点$u$的距离$dist_u=dist_r+1$;
    \item 返回该树的根$u$。
\end{enumerate}

根据定理~\ref{LTT}可得$merge$的复杂度为$O(lgn)$。

代码如下(以大根堆为例):

\begin{lstlisting}[title=merge]
int merge(int u, int v) {
    if (u && v) {
        if (T[u].val < T[v].val)
            std::swap(u, v);
        T[u].r = merge(T[u].r, v);
        if (T[T[u].l].dis < T[T[u].r].dis)
            std::swap(T[u].l, T[u].r);
        T[u].dis = T[T[u].r].dis + 1;
    }
    return u | v;
}
\end{lstlisting}

\subsubsection{修改}
\begin{itemize}
    \item 插入节点时,新建一个只有插入元素的堆,然后合并两个堆。
    \item 删除节点时,合并堆顶左右儿子表示的子堆。
\end{itemize}

\subsection{斜堆}
斜堆的操作与左偏树差不多,它们的区别是斜堆不维护到外节点的最近距离,
而是在每一次$merge$时简单地$swap$左右子树。
\subsection{可删堆}\label{MultiSet}
一个简单的方法是使用$std::multiset$,但是其常数很大;
更保险的做法是使用两个优先队列(已加入/已删除)来完成操作:
\begin{itemize}
	\item 加入时将元素加入``已加入堆'';
	\item 删除时将元素加入``已删除堆'';
	\item 取堆顶时,若两堆堆顶相等则弹出,直到两堆堆顶不相等,返回``已加入堆''
	      的堆顶。
\end{itemize}

\section{可持久化数据结构}
可持久化数据结构的核心思想就是\emph{Copy On Write}(写时复制),当一个对象将
被改变时,简单地复制其整体,未修改的部分仍引用原对象的数据,达到节省拷贝时间与
空间的目的。

可持久化数据结构有主席树(可持久化线段树),可持久化可并堆,可持久化Trie,
可持久化数组,可持久化并查集,可持久化平衡树等。
\subsection{主席树}
用主席树做的经典模型有:
\begin{itemize}
    \item 差分
    \item 对于每一个节点为左节点,维护其右边节点为右节点时的答案
    \item 将某一维离散化后不断插入新数据进行预处理以回答在线询问
\end{itemize}
\subsection{可持久化Trie}
若遇到求区间xor最大值之类的问题,使用可持久化Trie。
\subsection{可持久化数组}
可持久化数组有两种实现:
\begin{itemize}
    \item 块状数组
    \item 主席树
\end{itemize}
可持久化并查集可使用可持久化数组实现。
\subsection{优化}
\subsubsection{标记永久化}
将对整个区间的操作记录在管理此区间的节点,标记不下传,统计时参与计算。
此法节约了$push$的时间且对可持久化友好。
\subsubsection{克隆开关}
若已知按照原方法有一个节点不再被任何时间的数据结构引用时,直接在该节点上修改即可
(当然也可以gc,比较麻烦)。
因此在操作前可以设置一个$enableClone$开关,若为$false$则直接返回原节点即可。
代码如下:
\begin{lstlisting}[title=cloneA]
bool enableClone=true;
int cloneNode(int src) {
    if(enableClone) {
        int id=allocNode();
        T[id]=T[src];
        return id;
    }
    return src;
}
\end{lstlisting}
对于可持久化并查集,若使用路径压缩,则不好判断是否$clone$,在每个节点上记录其被
创建时的时间戳,与当前版本时间戳比较即可。
代码如下:
\begin{lstlisting}[title=cloneB]
int timeStamp=0;
int cloneNode(int src) {
    if(T[src].ts!=timeStamp) {
        int id=allocNode();
        T[id]=T[src];
        T[id].ts=timeStamp;
        return id;
    }
    return src;
}
\end{lstlisting}
此法节约了复制节点时的时间与空间。

\section{DLX舞蹈链}
\index{D!Dancing Links X}
DLX用来求解精确覆盖问题。

\paragraph{精确覆盖问题} 给定一个01矩阵,求使得每一列恰好有1个1的行集合。
\subsection{X算法}
X算法使用递归+回溯搜索可行解。

算法步骤如下:
\begin{enumerate}
	\item 从矩阵中选取一行;
	\item 将该行和该行所有1对应的列以及与该行冲突的行从矩阵中删除得到一个新矩阵。
	\item 若该矩阵为空矩阵,则跳到步骤4;否则递归求解新矩阵的精确覆盖,若返回false则
	      返回步骤1选取下一行;
	\item 若选取的行全部为1,则返回true,否则返回false。
\end{enumerate}
\subsection{DLX}
递归+回溯使得存储与维护矩阵既麻烦又费时。Donald E.Knuth使用双向链表
来维护矩阵,这个数据结构被称为Dancing Links。它利用了双向链表删除与恢复的方便性。

对于矩阵内的每一个1(此种矩阵一般为稀疏矩阵),维护其上下左右元素标号和自身坐标。
每个元素既是所属行的链表元素,又是所属列的链表元素。每个列的链表还有链头$C_i$(即0行元素),
这些链头又与总链头$head$串在一起,以便检查覆盖情况。

算法步骤如下:

记标示列链表链头$C$为将元素$C$所在列元素以及这些元素所在行元素删除,回标$C$为其逆操作。
\begin{enumerate}
	\item 检查$head.right$是否为自身,若是则覆盖完毕,输出答案栈内所有元素,返回true;
	\item 记$C=head.right$,标示$C$,枚举$C$所在链表内的行$D$:
	      \begin{enumerate}
		      \item 标示元素$D$所在链表行元素对应列链表链头。
		      \item 将其压入答案栈中;
		      \item 递归求解,若返回true则退出,否则逆序回标,枚举下一行。
	      \end{enumerate}
	\item 回标$C$,返回false。
\end{enumerate}

{\bfseries 为了提高搜索效率可以维护每列1的个数,每次选取1个数最少的列遍历。}

板子:
\lstinputlisting{Source/Templates/DLX.cpp}

上述内容参考了万仓一黍的博客\footnote{
	跳跃的舞者,舞蹈链(Dancing Links)算法——求解精确覆盖问题
	\url{http://www.cnblogs.com/grenet/p/3145800.html}
}。


\chapter{数论}
\subsection{辗转相除法GCD}
\subsubsection{裴蜀定理}
\index{B!Bézout's Theorem}
\begin{theorem}[Bézout's Theorem]\label{BT}
    对于任意$a,b\in \mathbb{Z}$,关于$x,y$的线性不定方程(裴蜀方程)
    $ax+by=c$有无穷多整数解$(x,y)$当且仅当$(a,b)|c$。特别地,
    一定存在$(x,y)$使得$ax+by=(a,b)$成立。
\end{theorem}

由此可得推论:

\begin{inference}
    $a,b$互质的充要条件是存在整数$(x,y)$使得$ax+by=1$。
\end{inference}

接下来证明一定存在$(x,y)$使得$ax+by=(a,b)$成立:

设$s$是$a$和$b$线性组合集中的最小正元素,对于某个整数组$(x,y)$有$ax+by=s$,
令$q=[a/s],r=a mod s=a-q(ax+by)=a(1-qx)+b(-qy)$,所以$r$也是一个线性组合。
因为$s$是线性组合集中的最小正元素,且$0\leq r \le s$,所以$r=0$,可得$s|a$。
同理$s|b$,因此$s$是$a,b$的公约数,可得$(a,b) \geq s$。因为$(a,b)|ax+by$
且$s>0$,所以$(a,b) \leq s$。结合$(a,b) \geq s$与$(a,b) \leq s$可得
$s=(a,b)$。

然后证明对于任意$a,b\in \mathbb{Z}$,$ax+by=c$有整数解
$(x,y) \Leftrightarrow (a,b)|c$:

充分性:

必要性:

至于无穷多整数解嘛。。。拿最小公倍数调一调初始解$(x,y)$即可。

证明参考了霜刃未曾试的博客\footnote{关于裴蜀定理的一些证明\\
\url{https://blog.csdn.net/discreeter/article/details/69833579}}与
算法导论\cite{ITA3}第31.1节定理31.2的证明。
\subsubsection{exgcd}
由定理~\ref{BT}可知一定存在整数解$(x,y)$满足$ax+by=(a,b)$,如何构造
出一组解呢?

$exgcd$(扩展欧几里得算法)可求出一组特殊的整数解。

$exgcd$构造出的解特殊性在于

\subsubsection{位运算gcd}
原理

位扫描优化

如果某个数末尾有多个0,则可以直接使用右移k位代替不断右移1位。
下面是统计末尾0的个数k的方法:

\begin{itemize}
    \item GCC自带了对位扫描指令的封装,即$\_\_builtin\_$系列函数,
    直接使用$\_\_builtin\_clz$函数即可。
    \item
\end{itemize}

实现

\section{欧拉定理}
\subsection{费马小定理}\label{FLTS}
\index{F!Fermat's Little Theorem}
\begin{theorem}[Fermat's Little Theorem]\label{FLT}
	~\\
	$\forall p \textrm{ is a prime number},a\in \mathbb{Z},a^p \equiv a \pmod{p}$
\end{theorem}

该定理是定理~\ref{ET}的特殊化,不证。

由定理~\ref{FLT}可得:

\begin{inference}
	$a^{-1} \equiv a^{p-2} \pmod{p}$
\end{inference}

可使用快速幂在$O(lgp)$的复杂度下求某个数的逆元。

\subsection{线性推逆元}

如果需要获得$a\in [1,p)$内模$p$的逆元,复杂度为$O(plgp)$逐个快速幂的方法
并不是最优的。

首先有$1^{-1}\equiv 1 \pmod{p}$。

令$p=qa+r$,其中$q=[\frac{p}{a}],r=p \bmod a$。

再把$p \equiv 0 \pmod{q}$中的$p$用$qa+r$代替,两边同时乘上$(ar)^{-1}$,
移项得$a^{-1}\equiv -qr^{-1} \pmod{q}$,即
$a^{-1}\equiv -[\frac{p}{a}](p \bmod a)^{-1} \pmod{p}$。

代码如下:
\begin{lstlisting}[title=inv]
inv[1]=1;
for(int i=2;i<=n;++i)
    inv[i]=asInt64(mod-mod/i)*inv[mod%i]%mod;
\end{lstlisting}

以上内容参考了Miskcoo的博客\footnote{[数论]线性求所有逆元的方法 – Miskcoo's Space\\
	\url{http://blog.miskcoo.com/2014/09/linear-find-all-invert}}

Update:还有更一般的做法,可以在$O(n+\lg p)$内推出任意$n$个非0数的逆元:

首先计算出前$i$个数的前缀积$M_i$,然后快速幂计算$M_n^{-1}$,最后从后往前倒推计算每个数的
逆元。比如要计算$A_i$的逆元,倒推维护$X=M_n^{-1}\prod_{j=i+1}^n{A_j}$,那么
$A_i^{-1}=M_{i-1}X$。

该方法源自WAAutoMaton的博客\CJKsout{(知识都是在乱翻他人博客中学到的)}\footnote{
	[loj ???] 乘法逆元2 题解
	\url{https://wa-am.com/2019/03/08/loj-乘法逆元2-题解}
}。
\subsection{欧拉定理}
\index{E!Euler's Theorem}
\begin{theorem}[Euler's Theorem]\label{ET}
	~\\
	对于任意互质正整数对$(a,n)$,有$a^{\varphi(n)} \equiv 1 \pmod{n}$
\end{theorem}
证明:

令$S=\{[x]_n\in Z_n|(a,n)=1\}$(由与$n$互质的模$n$剩余类组成的集合),
它与$\cdot_n$构成整数模$n$乘法群,$(S,\cdot_n)$的阶为$\varphi(n)$。

接着有两种证明思路:
\begin{itemize}
	\item 对于任意一个与$n$互质的正整数$a$,$a$的幂模$n$的值$a,a^2,\cdots,a^k$
	      构成了一个子群,其中$a^k\equiv 1 \pmod{n}$。

	      根据定理~\ref{LT},有$k|\varphi(n)$,令$M=\varphi(n)/k$,有
	      $a^{\varphi(n)}=a^{kM}=(a^k)^M\equiv 1^M\equiv 1 \pmod{n}$。
	\item 根据定义得对于$[x]_n\in S$和$S$中的所有元素$[a_1]_n,[a_2]_n,\cdots,
		[a_{\varphi(n)}]_n$,$[x]_n \cdot_n [a_i]_n$\\组成的集合仍然是$S$,
		因此有$x^{\varphi(n)}[a_1]_n[a_2]_n\cdots[a_{\varphi(n)}]_n=
		(x[a_1]_n)(x[a_2]_n)\cdots\\(x[a_{\varphi(n)}]_n)\equiv[a_1]_n
		[a_2]_n\cdots[a_{\varphi(n)}]_n\pmod{n}$,两边消去可得
		$x^{\varphi(n)}\equiv 1\pmod{n}$。
\end{itemize}

上述证明源自Wikipedia-EN\footnote{Euler's theorem - Wikipedia
	\url{https://en.wikipedia.org/wiki/Euler's\_theorem}}和Eden Harder
的博客\footnote{RSA 加密周边 - Eden Harder
	\url{http://edenharder.is-programmer.com/posts/43247.html}}。
\subsection{扩展欧拉定理}
\begin{theorem}\label{exEuler}
	$\forall a\in \mathbb{Z},x,m\in \mathbb{Z^+},x\geq \varphi(m)
		,a^x\equiv a^{x \bmod \varphi(m)+\varphi(m)} \pmod{m}$
\end{theorem}

\begin{lemma}\label{EEL1}
	\begin{displaymath}
		\left\{
		\begin{array}{l}
			x\equiv y \pmod{m_1} \\
			x\equiv y \pmod{m_2}
		\end{array}
		\right.
		\Rightarrow x\equiv y \pmod{lcm(m_1,m_2)}
	\end{displaymath}
\end{lemma}

证明:
\begin{displaymath}
	\left\{
	\begin{array}{l}
		x\equiv y \pmod{m_1} \\
		x\equiv y \pmod{m_2}
	\end{array}
	\right.
	\Rightarrow
	\left\{
	\begin{array}{l}
		x+c_1m_1=y \\
		x+c_2m_2=y
	\end{array}
	\right.
\end{displaymath}
\begin{displaymath}
	\Rightarrow
	c_1m_1=c_2m_2=k\cdot lcm(m_1,m_2)
	\Rightarrow
\end{displaymath}
\begin{displaymath}
	x \equiv y \pmod{lcm(m_1,m_2)}
\end{displaymath}

\begin{inference}\label{EEL1I}
	当$a,b$互质时,$x\equiv y \pmod{ab}$
\end{inference}

\begin{inference}
	\begin{displaymath}
		\left\{
		\begin{array}{l}
			x\equiv y \pmod{m_1} \\
			\cdots               \\
			x\equiv y \pmod{m_n}
		\end{array}
		\right.
		\Rightarrow x\equiv y \pmod{lcm(m_1,\cdots,m_n)}
	\end{displaymath}
\end{inference}

\begin{lemma}\label{EEL2}
	\begin{displaymath}
		\forall p\textrm{ is a prime number},q\in \mathbb{Z^+},q>1,
		\varphi(p^q)\geq q.
	\end{displaymath}
\end{lemma}

证明:首先有$\varphi(p^q)=(p-1)p^{q-1}$,当$p$固定时,$q$取2使得$\varphi(p^q)-q$
最小,但该值仍非负。当且仅当$p=2,q=2$时,$\varphi(p^q)=q$。

接下来证明定理~\ref{exEuler}:

首先证明当$m$为素数$p$的幂$(m=p^q)$时成立:
\begin{itemize}
	\item 若$gcd(a,p)=1$,则$gcd(a,p^q)=1$,根据欧拉定理可证在该情况下成立;
	\item 若$gcd(a,p)=p$,由适用范围可知$x\geq q$,由引理~\ref{EEL2}可知
	      $x \bmod \varphi(p^q) + \varphi(p^q) \geq q$,因此
	      $a^x\equiv 0 \equiv a^{x \bmod \varphi(p^q)+\varphi(p^q)} \pmod{p^q}$
\end{itemize}

对于任意$m$,可根据算术基本定理将其分解为素数幂之积。因为$\varphi(p^q)|\varphi(m)$,所以有
$a^x\equiv a^{x \bmod \varphi(p^q)+\varphi(p^q)}
	\equiv a^{x \bmod \varphi(m)+\varphi(m)} \pmod{p^q}$。
根据引理~\ref{EEL1}及其推论合并这些式子可证明该定理。

以上证明源自后缀自动机·张的文章\footnote{微小的欧拉定理EXT证明
	\url{https://zhuanlan.zhihu.com/p/24902174}}。

{Warning:扩展欧拉定理在模意义下矩阵幂的应用中,有时正确,但是已经出现被Hack的例子。
尽可能使用矩阵乘法以外的递推方式。}
\index{*TODO!扩展欧拉定理在矩阵幂中的应用}

\section{Miller Rabin素性测试}
\section{Pollard Rho启发式因子分解}
\section{RSA算法}
\subsection{原理}
\begin{theorem}[素数定理]\label{PT}
    $\lim_{n\rightarrow\infty}\frac{\pi(n)}{n/\ln n}=1$
\end{theorem}
RSA的安全性基于以下事实:寻找大素数很容易(根据定理~\ref{PT},素数密度还是挺大的),
但把一个数分解为两个质数之积却很难。

RSA算法的基本步骤如下:
\begin{enumerate}
    \item 随机选取两个大素数$p,q$,使得$p\neq q$,令$n=pq$;
    \item 选取一个与$\varphi(n)=(p-1)(q-1)$复制的小奇数$e$,
    计算出$e$的乘法逆元$d$;
    \item 将$P(e,n)$公开,作为{\bfseries RSA公钥};\\
          将$S(d,n)$保密,作为{\bfseries RSA私钥}。
\end{enumerate}

对于消息$M$,公钥持有者可进行运算:$P(M)=M^e \bmod n$;
私钥持有者可进行运算:$S(M)=M^d mod n$。
对于用公/私钥加密$M$得到的密文$C$,只有使用私/公钥才能得到$M$。
由于$\bmod n$的缘故,消息$M$的域为$Z_n$。

下面证明RSA算法的正确性,即证明:
\begin{displaymath}
    P(S(M))=S(P(M))=M^{ed}\equiv M \pmod{n}
\end{displaymath}

因为$e,d$是模$\varphi(n)$意义下的乘法逆元,所以有$ed=1+k(p-1)(q-1)$。

\begin{itemize}
    \item 若$M\not\equiv 0 \pmod{p}$,则有
    \begin{eqnarray*}
        M^{ed}&\equiv& M^{1+k(p-1)(q-1)} \pmod{p}\\
        &\equiv& M\cdot (M^{p-1})^{k(q-1)} \pmod{p}\\
        &\equiv& M\cdot 1^{k(q-1)} \pmod{p}\\
        &\equiv& M \pmod{p}
    \end{eqnarray*}
    \item 若$M\equiv 0 \pmod{p}$,上述等式仍成立。
\end{itemize}

同样地,对于$q$有$M^{ed}\equiv M \pmod{q}$。根据引理~\ref{EEL1},有
$M^{ed}\equiv M \pmod{n}$,证毕。

\subsection{应用}
\subsubsection{消息加密}
发送方使用接收方的公钥$P$把消息$M$加密得到密文$C$,将密文$C$发送给
接收方。接收方使用自己的私钥$P$解密得到消息$M$。
\paragraph{快速无公钥加密系统}
若消息过长,则仅用$P$加密对称加密算法的随机密钥$K$,同时用密钥$K$加密
$M$得到密文$C$,把$(P(K),C)$发送给接收方。接收方使用$P$解密得到$K$,
再用$K$对$C$解密即可。
\subsubsection{数字签名}
发送方使用自己的私钥$S$把消息$M$签署得到签名$C$,将消息$M$与签名$C$
发送给接收方。接收方使用发送方的公钥$P$解密得到消息$M$,验证消息是否正确。
\paragraph{快速数字签名}
同理把对称加密算法的密钥改为快速散列函数的值即可。
\paragraph{证书链}
以一个可信根为起点,大家都知道这个根的公钥。下一级可以将自己的公钥和被上一级
签署后的公钥作为签名证书,由此验证证书链上每一级的正确性,从而证明链尾端消息的
正确性。
以上内容参考了算法导论\cite{ITA3}第31.7节。

\section{中国剩余定理CRT}
\subsection{CRT}
\index{C!Chinese Remainder Theorem}
\begin{theorem}[Chinese Remainder Theorem]
	对于模线性方程组:
	\begin{displaymath}
		\left\{\begin{array}{l}
			x \equiv a_1 \pmod{n_1} \\
			x \equiv a_2 \pmod{n_2} \\
			\cdots                  \\
			x \equiv a_k \pmod{n_k}
		\end{array}\right.
	\end{displaymath}\\
	其中$n_1,n_2,\cdots,n_k$两两互质,令$\displaystyle N=\prod_{i=1}^k{n_i}$,
	该模线性方程组在$[0,N)$内有唯一解。
\end{theorem}
如何求解该线性方程组呢?和拉格朗日插值法的思路相同,对于每一个方程都给最终的解
贡献一个$x_i$,满足
\begin{displaymath}
	x_i \bmod n_j =
	\left\{\begin{array}{ll}
		0   & \textrm{if $i\neq j$} \\
		a_i & \textrm{if $i=j$}
	\end{array}\right.
\end{displaymath}
答案即为$\displaystyle \sum_{i=1}^n{x_i} \bmod N$。
考虑$i\neq j$时$x_i$应该整除$n_j$,因此$x_i$应该有系数$M=N/n_i$;当$i=j$时,
$x_i$应该有系数$a_i$,为了抵消$M$带来的影响,再乘上$M$模$n_i$的乘法逆元即可(
由于$n$两两互质,$M$与$n_i$也互质,根据定理~\ref{ET},保证其乘法逆元存在)。
\subsection{ExCRT}
当$n$不满足两两互质的条件时,可能会找不到其乘法逆元。
所以我们采用另一种思路求解方程:每次选择两个方程将其合并,直到只剩一个方程为止。

考虑两个方程组成的方程组:
\begin{displaymath}
	\left\{\begin{array}{l}
		x \equiv a_1 \pmod{n_1} \\
		x \equiv a_2 \pmod{n_2} \\
	\end{array}\right.
\end{displaymath}
等价于
\begin{eqnarray}
	x0=a_1+k_1n_1\label{CRTE}\\
	x0=a_2+k_2n_2
\end{eqnarray}
移项得$k_1n_1-k_2n_2=a_2-a_1$,可以使用
$exgcd$求出$c_1n_1+c_2n_2=gcd(n_1,n_2)$的各项参数。根据定理~\ref{BT},
若$gcd(n_1,n_2)\nmid(a_2-a_1)$则该方程组无解。等比例缩放方程求出$k1$,
带入方程~\ref{CRTE}反推出$x0$,得到新的模线性方程$x \equiv x0
	\pmod{lcm(n_1,n_2)}$。

\section{积性函数与线性筛}
\subsection{定义}
\index{A!Arithmetic Function}
\paragraph{数论函数(Arithmetic Function)}
若函数$f:\mathbb{Z^+}\rightarrow\mathbb{C}$,则称函数$f$为数论函数。

\index{M!Multiplicative Function}
\paragraph{积性函数(Multiplicative Function)}
若函数$f$为数论函数,且$f(1)=1$,对于任意互质的正整数$a,b$都有$f(ab)=f(a)f(b)$,
则称函数$f$为积性函数。

\index{C!Completely Multiplicative\\ Function}
\paragraph{完全积性函数(Completely Multiplicative Function)}
若函数$f$为积性函数且对于任意正整数$a,b$都有$f(ab)=f(a)f(b)$,
则称函数$f$为完全积性函数。
\begin{property}\label{MFC}
	若$f$为积性函数,对于正整数$\displaystyle n=\prod_{i=1}^m{{p_i}^{c_i}}$,有
	$\displaystyle f(n)=\prod_{i=1}^m{f({p_i}^{c_i})}$
\end{property}
\begin{property}
	若$f$为完全积性函数,对于正整数$\displaystyle n=\prod_{i=1}^m{{p_i}^{c_i}}$,
	有$\displaystyle f(n)=\prod_{i=1}^m{f(p_i)^{c_i}}$
\end{property}
\subsection{常见积性函数}
\subsubsection{积性函数}
\begin{itemize}
	\item 除数函数$\displaystyle \sigma_k(n)=\sum_{d|n}{d^k}$,\\
	      根据性质~\ref{MFC}可得
	      $\displaystyle \sigma_k(n)=\prod_{i=1}^m{\sum_{j=0}^{c_i}{p_i^{jk}}}$
	\item 约数个数函数$\tau(n)=\sigma_0(n)$
	\item 约数和函数$\sigma(n)=\sigma_1(n)$
	\item \index{E!Euler Totient Function}
	      欧拉函数(Euler Totient Function)
	      $\displaystyle \varphi(n)=\sum_{i=1}^n{[(n,i)=1]}=n\prod_{p|n}{(1-\frac{1}{p})}$,
	      且有$\displaystyle \sum_{i=1}^n{[(n,i)=1]*i}=\frac{n\varphi(n)+[n=1]}{2}$
	\item \index{M!Möbius function}
	      莫比乌斯函数定义为:
	      \begin{displaymath}
		      \mu(d)=
		      \left\{
		      \begin{array}{ll}
			      1      & \textrm{if $d=1$}                                \\
			      (-1)^k & \textrm{if $\displaystyle d=\prod_{i=1}^k{p_i}$} \\
			      0      & \textrm{otherwise}
		      \end{array}
		      \right.
	      \end{displaymath}

	      简单来说就是如果存在平方因子则$\mu(n)$为0,否则$\mu(n)=(-1)^\textrm{质因子数}$。
\end{itemize}
\begin{theorem}\label{MobiusT}
	\begin{displaymath}
		[n=1]=\sum_{d|n}{\mu(d)}
	\end{displaymath}
\end{theorem}
证明:当$n=1$时,该等式成立。
对于$n>1$的情况,将$n$分解为$\displaystyle \prod_{i=1}^m{{p_i}^{c_i}}$,令
$\displaystyle X=\prod_{i=1}^m{p_i}$,
仅考虑$\mu(d)\neq 0$的部分,$\mu(d)$有贡献当且仅当$d|X$,因此$d$可表示为一个长度为$m$
的01向量。
由排列组合知识可知选取奇数个1的向量方案数等于选取偶数个1的向量方案数,即正负贡献抵消。
\begin{theorem}\label{PhiT}
	\begin{displaymath}
		n=\sum_{d|n}{\varphi(d)}
	\end{displaymath}
\end{theorem}
证明:将$n$个分数$\frac{1}{n},\frac{2}{n},\cdots,\frac{n}{n}$化为最简分数,
$\varphi(x)$即表示分母为$x$的最简分数个数。
\begin{theorem}\label{SigmaT}
	\begin{eqnarray*}
		\sum_{i=1}^n{\tau(i)}&=&\sum_{i=1}^n{[\frac{n}{i}]}\\
		\sum_{i=1}^n{\sigma(i)}&=&\sum_{i=1}^n{i\cdot[\frac{n}{i}]}
	\end{eqnarray*}
\end{theorem}
证明:枚举因子$i$,$n$以内有$[\frac{n}{i}]$个因子。
\subsubsection{完全积性函数}
\begin{itemize}
	\item \index{U!Unit Function}
	      元函数(Unit Function)~$\epsilon(n)=[n=1]$
	\item \index{C!Constant Function}
	      恒等函数(Constant Function)~$1(n)=1$
	\item 单位函数$id(n)=n$
	\item 幂函数$id^k(n)=n^k$
\end{itemize}
以上内容参考了skywalkert的博客\footnote{浅谈一类积性函数的前缀和\\
	\url{https://blog.csdn.net/skywalkert/article/details/50500009}}与
Wikipedia-EN\footnote{Arithmetic function - Wikipedia
	\url{https://en.wikipedia.org/wiki/Arithmetic\_function}}。
\subsection{线性筛}
主要思路是每次拿当前的数和已经筛出的素数构造成新的合数并将其筛去。

代码如下:
\begin{lstlisting}[title=Euler]
int prime[size/log(size)],pcnt=0;
bool flag[size];
void pre(int n) {
    for(int i=1;i<=n,++i) {
        if(!flag[i])
            prime[++pcnt]=i;
        for(int j=1;j<=pcnt && prime[j]*i<=n;++j) {
            flag[prime[j]*i]=true;
            if(i%prime[j]==0)
                break;//case 1
        }
    }
}
\end{lstlisting}
注意到case 1中的优化,它保证了每个合数最多被筛1次,从而使时间复杂度变为$O(n)$,
并且增加了一个性质:合数只被其最小质因子筛去。
接下来证明该优化的正确性:当$i\bmod p_j=0$时,有$i=kp_j$,
若要用$p_{j+x}$筛去后面的合数$ip_{j+x}=kp_jp_{j+x}$,可知该合数未来将被合数$kp_{j+x}$与素数
$p_j$筛去,直接跳出不会影响结果,且保证合数被最小质因子筛除,便于质因数分解。
\subsection{积性函数筛}
\subsubsection{欧拉函数}
\begin{itemize}
	\item $\varphi(1)=1$;
	\item 若$i$为素数,则$\varphi(i)=i-1$;
	\item 若$i \bmod p_j=0$,则说明$ip_j$存在至少两个因子$p_j$,因此
	      $\varphi(ip_j)=\varphi(i)p_j$;
	\item 若$i \bmod p_j\neq 0$,则根据积性函数性质可得
	      $\varphi(ip_j)=\varphi(i)(p_j-1)$。
\end{itemize}
\subsubsection{莫比乌斯函数}
\begin{itemize}
	\item $\mu(1)=1$;
	\item 若$i$为素数,则$\mu(i)=-1$;
	\item 若$i \bmod p_j=0$,则说明$ip_j$存在至少两个因子$p_j$,因此
	      $\mu(ip_j)=0$。注意若数组已清零则不赋值;
	\item 若$i \bmod p_j\neq 0$,则根据积性函数性质可得
	      $\mu(ip_j)=-\mu(i)$。
\end{itemize}
\subsubsection{约数个数}
记数组$A_i$为$i$中最小质因子的次数。
\begin{itemize}
	\item $\tau(1)=1,A_1=0$;
	\item 若$i$为素数,则$\tau(i)=2,A_i=1$;
	\item 若$i \bmod p_j=0$,则说明$ip_j$存在至少两个因子$p_j$,因此
	      $\tau(ip_j)=\tau(i)\cdot\frac{A_i+2}{A_i+1}$且$A_{ip_j}=A_i+1$;
	\item 若$i \bmod p_j\neq 0$,则根据积性函数性质可得
	      $\tau(ip_j)=2\tau(i)$且$A_{ip_j}=1$。
\end{itemize}
\subsubsection{约数和}
由性质~\ref{MFC}可得
\begin{displaymath}
	\sigma(n)=\prod_{i=1}^m{\sum_{j=0}^{c_i}{p_i^j}}
\end{displaymath}
记数组$low_i$为$i$中最小质因子的幂,$sum_i$为$i$中最小质因子的贡献。
\begin{itemize}
	\item $\sigma(1)=1,low_1=1,sum_1=1$;
	\item 若$i$为素数,则$\sigma(i)=i+1,low_i=i,sum_i=i+1$;
	\item 若$i \bmod p_j=0$,则说明$ip_j$存在至少两个因子$p_j$,因此
	      $\sigma(ip_j)=\sigma(i)\cdot\frac{sum_{ip_j}}{sum_i}$且
	      $low_{ip_j}=low_i*p_j,sum_{ip_j}=sum_i+low_{ip_j}$;
	\item 若$i \bmod p_j\neq 0$,则根据积性函数性质可得
	      $\sigma(ip_j)=(p_j+1)\sigma(i)$且
	      $low_{ip_j}=p_j,sum_{ip_j}=p_j+1$。
\end{itemize}
\subsubsection{普通积性函数}
同约数和的思想,记数组$sum_i$为$i$中最小质因子的贡献。
要求能够快速推出$f({p_i}^{c_i})$的值。
\begin{itemize}
	\item $f(1)=1,sum_1=???$;
	\item 若$i$为素数,则$f(i)=sum_i=???$;
	\item 若$i \bmod p_j=0$,则说明$ip_j$存在至少两个因子$p_j$,因此
	      $f(ip_j)=f(i)\cdot\frac{sum_{ip_j}}{sum_i}$;
	\item 若$i \bmod p_j\neq 0$,则根据积性函数性质可得
	      $f(ip_j)=f(p_j)f(i)$。
\end{itemize}
以上内容参考了租酥雨的博客\footnote{积性函数与线性筛 - 租酥雨
	\url{https://www.cnblogs.com/zhoushuyu/p/8275530.html}}。
\subsection{因子分解}
通过在每次筛除时记录其最小质因子,可以于$O(\lg n)$复杂度内分解因子。

\section{狄利克雷卷积,狄利克雷逆与莫比乌斯反演}
\subsection{狄利克雷卷积}
\index{D!Dirichlet Convolution}
对于数论函数$f,g$,定义狄利克雷卷积
\begin{displaymath}
	(f*g)(n)=\sum_{d|n}{f(d)g(\frac{n}{d})}=\sum_{ab=n}{f(a)g(b)}
\end{displaymath}
由积性函数集合与狄利克雷卷积组成的群的乘法单位元为元函数$\epsilon$。

狄利克雷卷积有如下性质:
\begin{eqnarray*}
	\textrm{结合律} & (f*g)*h=f*(g*h)\\
	\textrm{分配律} & f*(g+h)=f*g+f*h\\
	\textrm{交换律} & f*g=g*f;\\
	\textrm{单位元} & f*\epsilon=\epsilon*f=f。
\end{eqnarray*}
\subsection{狄利克雷逆}
\index{D!Dirichlet Inverse}
已知数论函数$f$,求$g=f^{-1}$,满足$f*g=\epsilon$。
\begin{itemize}
	\item 当$n=1$时,有$(f*g)(1)=f(1)g(1)=\epsilon(1)=1$,
	      解得$g(1)=\frac{1}{f(1)}$。
	\item 当$n>1$时,
	      有$\displaystyle (f*g)(n)=\sum_{ab=n}{f(a)g(b)}=\epsilon(n)=0$,
	      解得$\displaystyle g(n)=\frac{-1}{f(1)}
		      \sum_{d|n,d<n}{f(\frac{n}{d})g(d)}$。
\end{itemize}
\subsubsection{狄利克雷逆性质}
\begin{property}
	积性函数的狄利克雷逆仍然是积性函数。
\end{property}
\begin{property}
	若数论函数$f,g$为积性函数,则$(f*g)^{-1}=f^{-1}*g^{-1}$。
\end{property}
\begin{property}\label{CMFP}
	积性函数$f$为完全积性函数当且仅当$f^{-1}(n)=\mu(n)f(n)$。
\end{property}
证明:\begin{eqnarray*}
	(f*f^{-1})(n)&=&\sum_{ab=n}{f(a)f^{-1}(b)}\\
	&=&\sum_{ab=n}(f(a)\mu(b)f(b))\\
	&=&f(n)\sum_{d|n}{\mu(d)}\\
	&=&f(n)\epsilon(n)\\
	&=&\epsilon(n)
\end{eqnarray*}
\subsubsection{常见数论函数及其狄利克雷逆}
\begin{itemize}
	\item $1*\mu=\epsilon$\\
	      参见定理~\ref{MobiusT}的证明。
	\item $id^\alpha*(\mu\cdot id^\alpha)=\epsilon$\\
	      根据性质~\ref{CMFP}可证明。
	\item $\displaystyle \varphi*(\sum_{d|n}{\mu(d)d})=\epsilon$\\
	      由定理~\ref{PhiT}可得$id=\varphi*1$,两边同时乘上$\mu$
	      可得$id*\mu=\varphi$,所以$\varphi^{-1}=id^{-1}*\mu^{-1}=id^{-1}*1$。
	\item $\sigma_\alpha*(\sum_{d|n}{\mu(d)\mu(\frac{n}{d})d^\alpha})=\epsilon$

	      $\sigma_\alpha=id^\alpha*1$可推出
	      $(\sigma_\alpha)^{-1}=(id^\alpha)^{-1}*\mu$
\end{itemize}
以上内容参考了Wikipedia-EN\footnote{Dirichlet convolution - Wikipedia\\
	\url{https://en.wikipedia.org/wiki/Dirichlet\_convolution}}。
\subsection{莫比乌斯反演}
\index{M!Möbius Inversion}
\begin{theorem}
	对于数论函数$f,g$,满足$\displaystyle g(n)=\sum_{d|n}f(d)$,则有
	\begin{displaymath}
		f(n)=\sum_{d|n}\mu(d)g(\frac{n}{d})
	\end{displaymath}
\end{theorem}
莫比乌斯反演可表示为若$g=f*1$则$f=\mu*g$。
证明:将$g=f*1$两边同时乘上$\mu$即可。
证明源自Wikipedia-EN\footnote{Möbius inversion formula - Wikipedia\\
	\url{https://en.wikipedia.org/wiki/Mobius_inversion}}。
\subsection{常见技巧}
\begin{itemize}
	\item
	      对于数论函数$g,f$,
	      \begin{displaymath}
		      g(n)=\sum_{n|d}{f(d)}\Rightarrow
		      f(n)=\sum_{n|d}{\mu(d)g(\frac{d}{n})}
	      \end{displaymath}
	\item
	      若$\displaystyle n=\prod_{i=1}^m{{p_i}^{c_i}},g(n)=\sum_{d|n}{f(d)}$
	      且$f$为积性函数,将$g$看做$f*1$可知$g$也是积性函数,则$g(n)=\prod_{i=1}^m
		      {\sum_{j=0}^{c_i}{f(p_i^j)}}$。
	\item 交换内外求和顺序。
	\item 枚举倍数,最大公约数等有共性的值并换元。
	\item 在化简前缀和函数时可能会遇到如下式子:
		\begin{eqnarray*}
			ans(n)&=&\sum_{i=1}^n{f(i)}\\
			&=&A(n)+B(n)\sum_{i=1}^n{\sum_{d|i}{f(d)}}\\
			&=&A(n)+B(n)\sum_{\frac{i}{d}=1}^n{\sum_{d=1}^{[\frac{n}{\frac{i}{d}}]}{f(d)}}\\
			&=&A(n)+B(n)\sum_{t=1}^n{\sum_{d=1}^{[\frac{n}{t}]}{f(d)}}\\
			&=&A(n)+B(n)\sum_{t=1}^n{ans([\frac{n}{t}])}
	      \end{eqnarray*}
		线性筛预处理一部分前缀和(一般预处理规模为$n^{2/3}$,最终时间复杂度
		$O(n^{2/3})$,大规模前缀和使用根号分块法递归计算。

		注意这里可以使用存储Trick来Cache计算结果(多次询问使用map或时间戳数组
		清零,下面只讨论单次询问的情况)。设预处理了前$k$个前缀和,其中$k\geq \sqrt{n}$。
		那么$[\frac{n}{t}]>k$的值不超过$\sqrt{n}$个,并且$t$不同对应的值也不同。所以
		可以以$t$为下标把计算结果存入另一个数组中。
	\item 同时除以最大公约数使其互质,然后套用$\varphi$。
	\item $\displaystyle [gcd(i,j)=1]=\sum_{k|gcd(i,j)}\mu(k)=
		      \sum_{k|i,k|j}\mu(k)$
	\item $\displaystyle \sum_{i=1}^n{i}=
		      \sum_{i=1}^n{\sum_{d|i}\varphi(d)}=
		      \sum_{d=1}^n{\varphi(d)\cdot[\frac{n}{d}]}$
	\item $(id\cdot\varphi)*id=id^2$
\end{itemize}
更多技巧待补充。

\section{低于线性时间复杂度筛法}
积性函数前缀和算法的思维难度:杜教筛>min\_筛>Powerful~Number。
\subsection{杜教筛}
杜教筛主要用于计算大数据规模积性函数求和。
\subsubsection{约数函数前缀和}
求$\displaystyle \sum_{i=1}^n{\sigma(i)},n\leq 10^{12}$。
\begin{eqnarray*}
    \sum_{i=1}^n{\sigma(i)}&=&\sum_{i=1}^n{\sum_{d|i}d}\\
    &=&\sum_{d=1}^n{d[\frac{n}{d}]}
\end{eqnarray*}
由于$[\frac{n}{d}]$存在许多连续相同的值,使用整除分块法可做到$O(\sqrt{n})$。
\subsubsection{欧拉函数前缀和}
求$\displaystyle \sum_{i=1}^n{\varphi(i)},n\leq 10^{11}$。
由定理~\ref{PhiT}可得
$\displaystyle \varphi(n)=n-\sum_{d|n,d<n}{\varphi(d)}$。
\begin{eqnarray*}
    ans(n)&=&\sum_{i=1}^n{\varphi(i)}\\
    &=&\sum_{i=1}^n{\left(i-\sum_{d|i,d<i}{\varphi(d)}\right)}\\
    &=&\frac{n(n+1)}{2}-\sum_{i=2}^{n}{\sum_{d|i,d<i}{\varphi(d)}}\\
    &=&\frac{n(n+1)}{2}-\sum_{\frac{i}{d}=2}^n
    {\sum_{d=1}^{[\frac{n}{\frac{i}{d}}]}{\varphi(d)}}\\
    &=&\frac{n(n+1)}{2}-\sum_{t=2}^n
    {\sum_{d=1}^{[\frac{n}{t}]}{\varphi(d)}}\\
    &=&\frac{n(n+1)}{2}-\sum_{t=2}^n{ans([\frac{n}{t}])}
\end{eqnarray*}
同理使用分块+递归询问区间和来计算答案。为了降低复杂度,应该先线性筛预处理前一部分值。
当预处理$k=n^\frac{2}{3}$时可以取到复杂度$T(n)=O(n^\frac{2}{3})$。
\subsubsection{莫比乌斯函数前缀和}
求$\displaystyle \sum_{i=1}^n{\mu(i)},n\leq 10^{11}$。
由定理~\ref{MobiusT}可得
$\displaystyle \mu(n)=[n=1]-\sum_{d|n,d<n}{\mu(d)}$。
\begin{eqnarray*}
    ans(n)&=&\sum_{i=1}^n{\mu(i)}\\
    &=&\sum_{i=1}^n{\left([i=1]-\sum_{d|i,d<i}{\mu(d)}\right)}\\
    &=&1-\sum_{i=1}^n{\sum_{d|i,d<i}{\mu(d)}}\\
    &=&1-\sum_{t=2}^n{ans([\frac{n}{t}])}
\end{eqnarray*}
\subsubsection{其它函数前缀和}
主要思路是使用狄利克雷卷积构造出一个简单的前缀和函数,且用于卷积的另一个函数也容易计算。

令$\displaystyle A(n)=\sum_{i=1}^n\frac{i}{(n,i)}$,求
$\displaystyle \sum_{i=1}^n{A(n)},n\leq 10^{9}$。

先化简$A(n)$:
\begin{eqnarray*}
    A(n)&=&\sum_{i=1}^n\frac{i}{(n,i)}\\
    &=&\sum_{d|n}{\sum_{i=1}^n{[(n,i)=d]\cdot\frac{i}{d}}}\\
    &=&\sum_{d|n}{\sum_{\frac{i}{d}=1}^{\frac{n}{d}}
    {[(\frac{n}{d},\frac{i}{d})=1]\cdot\frac{i}{d}}}\\
    &=&\frac{1}{2}\left(1+\sum_{d|n}{d\cdot\varphi(d)}\right)
\end{eqnarray*}

那么答案即为$\displaystyle \frac{1}{2}\left(n+\sum_{t=1}^n
    {\sum_{d=1}^{[\frac{n}{t}]}{d\cdot\varphi(d)}}\right)$。

考虑计算$\displaystyle \sum_{d=1}^n{d\cdot\varphi(d)}$的值:

易知$(id\cdot\varphi)*id=id^2$,因为\begin{displaymath}
    \sum_{d|n}d\cdot\varphi(d)\cdot\frac{n}{d}=
    n\cdot\sum_{d|n}\varphi(d)=n^2
\end{displaymath}

所以有\begin{eqnarray*}
    \frac{n(n+1)(2n+1)}{6}&=&\sum_{i=1}^n{(id\cdot\varphi)*id}\\
    &=&\sum_{t=1}^n{t\cdot\sum_{d=1}^{[\frac{n}{t}]}{d\cdot\varphi(d)}}
\end{eqnarray*}

\subsubsection{总结}
欲求积性函数$f(x)$的前缀和,构造$h=f*g$,其中$h(x)$和$g(x)$都是积性函数,且易求得
$h(x)$与$g(x)$的前缀和。

考虑$h(x)$前缀和的表达式(记大写形式为前缀和,即$F(n)=\displaystyle \sum_{i=1}^n{f(i)}$):
\begin{displaymath}
    H(n)=\sum_{i=1}^n{\sum_{d|i}{f(\frac{i}{d})g(d)}}=
    \sum_{d=1}^n{g(d)\sum_{i=1}^{\lfloor\frac{n}{d}\rfloor}{f(i)}}=
    \sum_{d=1}^n{g(d)F(\lfloor\frac{n}{d}\rfloor)}
\end{displaymath}

提出$F(n)$,即$F(n)=H(n)-\displaystyle \sum_{d=2}^n{g(d)F(\frac{n}{d})}$。

在递归时可以预处理一部分前缀和,同时使用HashTable缓存计算结果。
\subsubsection{时间复杂度分析}
使用整除分块法需要计算前$i$项前缀和,与前$\lfloor\frac{n}{i} \rfloor$项前缀和,
其中$i\leq \sqrt{n}$。后一部分比前一部分复杂度更高。考虑使用积分近似,有
\begin{displaymath}
    \int_0^{\sqrt{n}}{\sqrt{\frac{n}{x}} \ud x}=O(n^{3/4})
\end{displaymath}

预处理一部分前缀和可以有效降低算法时间复杂度:记预处理前$k$个前缀和,$k\geq \sqrt{n}$。

那么时间复杂度为
\begin{displaymath}
    O(k)+\int_0^{\frac{n}{k}}{\sqrt{\frac{n}{x}}\ud x}=O(k)+O(\frac{n}{\sqrt{k}})
\end{displaymath}

平衡两边的复杂度,解得$k=n^{1/3}$。

以上例题来自skywalkert的博客\footnote{浅谈一类积性函数的前缀和\\
    \url{https://blog.csdn.net/skywalkert/article/details/50500009}},
总结部分参考了国家集训队2016论文集任之洲的论文《积性函数求和的几种方法》。

\subsection{min\_25筛}
这里求和的积性函数$F$满足$F(p)$是一个关于$p$的低阶多项式且能够快速求出$F(p^k)$。
据说min\_25筛踩爆洲阁筛,那我就不学洲阁筛了。在此附上洲阁筛教程\footnote{
    洲阁筛学习 | \_\_debug's Home\\
    \url{http://debug18.com/posts/calculate-the-sum-of-multiplicative-function/}
}。

\subsubsection{预处理}
首先考虑求$\displaystyle \sum_{p\leq n}{F(p)}$。

记$g(n,j)$为满足$x$为$n$以内素数,或者$x$的最小质因子$>p_j$的$F(x)$之和,
所求值即为$g(n,|P|)$。考虑$g(n,j)$如何从$g(n,j-1)$转移。易知最小质因子为
$p_j$的合数为$p_j^2$,若其$>n$,则$g(n,j)$与$g(n,j-1)$都只求素数的积性函
数值之和,所以$g(n,j)=g(n,j-1)$。若$p_j^2\leq n$,则转移时会损失掉一些
$F(x)$,这些$x$的最小质因子为$p_j$。考虑提出$x$的$p_j$,满足$\frac{x}{p_j}$
的最小质因子$\geq p_j$,计算$\frac{x}{p_j}$的积性函数和,发现$g(\frac{n}{p_j},j-1)$
包括了它们,又因为$\frac{n}{p_j}\geq p_j > p_{j-1}$,$g(\frac{n}{p_j},j-1)$还有
$\displaystyle \sum_{p<p_j}F(p)$,需要扣除。{\bfseries 由于积性函数$F$的特殊性,
把不同次数的项拆开算,单项为完全积性函数,乘上$F(p_j)$即为需要减去的值。}

{\bfseries 若拆开后某一项系数不为1,这一项就不是完全积性。算这一项的贡献时先去除系数,
最后整体乘以系数。}

综上,有\begin{displaymath}
    g(n,j)=\left\{\begin{array}{lr}
        g(n,j-1)        & p_j^2>n   \\
        g(n,j-1)-F(p_j)(g(\frac{n}{p_j},j-1)-\displaystyle \sum_{p<p_j}{F(p)}) & p_j^2\leq n \\
    \end{array}\right.
\end{displaymath}

预处理素数时只需要筛$\sqrt{n}$内的素数,边界$g(n,0)$是所有数按照素数的计算方式计算
的值之和。由于最后只需要$g(n,|P|)$,非质数的贡献会被筛掉。

实质上$g(n,j)$就是埃氏筛法筛完$p_j$后未被筛的合数以及素数的积性函数值之和。

接下来尝试求出所有的$g(x,|P|),x=\lfloor \frac{n}{i}\rfloor$。
这里有一个存储上的trick:由于$\lfloor \frac{n}{i}\rfloor$有连续重复项,
最多$2\sqrt{n}$个,对于$x=\lfloor \frac{n}{i}\rfloor>\sqrt{n}$,把它
映射到$\lfloor \frac{n}{x}\rfloor$上存储,这样保证了空间复杂度为$O(\sqrt{n})$。

由于最后只要$g(x,|P|)$,$g$数组只要开1维滚动更新。

伪代码如下:
\begin{lstlisting}
int g[2][sqsiz],q[2*sqsiz];
int& getG(int x) {
    if(x<=sqr) return g[0][x];
    return g[1][n/x];
}
void calcG() {
    int m=0,i=1;
    while(i<=n) {
        int val=n/i;
        q[++m]=val;
        getG(val)=sumf(val);
        i=n/val+1;
    }
    for(int i=1;i<=psiz;++i) {
        int cp=p[i],cp2=cp*cp;
        for(int j=1;j<=m && cp2<=q[j];++j) {
            int k=q[j],&val=getG(k);
            val=sub(val,mul(f(cp),getG(k/cp)-sumpf[i-1]));
        }
    }
}
\end{lstlisting}

计算$G$时始终不考虑1,在求和时才加入。
\subsubsection{求和}
记$S(n,j)$为$n$以内最小质因子$\geq p_j$的积性函数值和,所求答案即为$S(n,1)+f(1)$。

把$S(n,j)$分为素数和合数求解:
\begin{itemize}
    \item 对于素数部分,$g(n,|P|)$代表了素数积性函数值和,再扣去不满足
    最小质因子要求的素数,最终贡献为$g(n,|P|)-\displaystyle \sum_{p<p_j}F(p)$。
    \item 对于合数部分,枚举其最小质因子$p_k$及其幂次$c$,单独贡献为\\
    $F(p_k^c)S(\frac{n}{p_k^c},k+1)+F(p_k^{c+1})$。注意此处的$F\cdot S$直接利用
    了积性函数的定义,因为$S$部分无$p_k$因子。由于$S$不处理$j=k$的部分,需要另外加上
    $F(p_k^{c+1})$。
\end{itemize}

递归的边界条件为$n\leq 1 \vee n<p_j$,无需记忆化。

时间复杂度为$O(\frac{n^\frac{3}{4}}{\lg n})$,空间复杂度为$O(\sqrt{n})$。

模板(LOJ\#6053. 简单的函数):
\lstinputlisting{Source/Templates/min_25.cpp}

上述内容参考了小蒟蒻yyb\footnote{
    min\_25筛
    \url{https://www.cnblogs.com/cjyyb/p/9185093.html}
}和租酥雨\footnote{
    Min\_25 筛
    \url{https://www.cnblogs.com/zhoushuyu/p/9187319.html}
}的博客。

Min\_25筛似乎也被称作``通用筛法''、``扩展埃拉托斯特尼筛法'',严格证明参见
zbh2047的文章\footnote{关于一种积性函数前缀和的通用筛法的时间复杂度证明

    \url{https://www.cnblogs.com/zbh2047/p/8552551.html}

    \url{https://zhuanlan.zhihu.com/p/33544708}}。
\subsection{Powerful~Number}
定义Powerful~Number为所有质因子的指数都$\geq 2$的数,那么每个Powerful~Number都可以
被表示为$a^2b^3$的形式(若指数为奇数则分配一个立方给$b$,其余分给$a$)。
{\bfseries 注意1也是Powerful~Number。}

\begin{theorem}
    $n$以内的Powerful~Number个数为$O(\sqrt{n})$。
\end{theorem}

证明:枚举$a$,将$b$的个数累积,可得式子
\begin{displaymath}
    \sum_{i=1}^{\lfloor\sqrt{n}\rfloor}{\lfloor\sqrt[3]{\frac{n}{i^2}}\rfloor}
\end{displaymath}

使用积分近似求出其上界为$\int_1^{\sqrt{n}+1}{\sqrt[3]{\frac{n}{(x-1)^2}} \ud x}=O(\sqrt{n})$。
根据Wikipedia-EN\footnote{Powerful~number
    \url{https://en.wikipedia.org/wiki/Powerful\_number}}的描述,其上界常数为
    $\frac{\zeta(\frac{3}{2})}{\zeta(3)}\approx 2.173$。

对于某个复杂的积性函数$f(x)$,若$f(p^e)$易于计算且存在一个简单(易于求$g(p^e)$与前缀和)
的积性函数$g(x)$,满足对于所有素数$p$,有$f(p)=g(p)$,称函数$g$拟合了函数$f$。

设$h=f/g$,这里的除法是狄利克雷除法,等价于$h=f*g^{-1}$。由于狄利克雷逆$g^{-1}$是积性
函数,狄利克雷卷积$h$也是积性函数。那么对于所有素数$p$,有$f(p)=h(1)g(p)+h(p)g(1)$,
由于$h(1)=1,f(p)=g(p)$,可得$h(p)=0$。由于$h(x)$是积性函数,$h(x)$可能非0当且仅当$x$
是Powerful~Number。

现在要求$\displaystyle Ans=\sum_{i=1}^n{f(n)}$,由于$f=h*g$,有
$\displaystyle Ans=\sum_{ab\leq n}{h(a)g(b)}$。由上文的推导可知$h(x)$仅在
Powerful~Number处有贡献,且Powerful~Number的个数是$O(\sqrt{n})$的,可以
$O(\sqrt{n})$DFS暴力枚举质因子组合得到$a$。记$n$以内的Powerful~Number组成的集合为$S$,
原式变为$\displaystyle Ans=\sum_{a\in S}{h(a)\sum_{b=1}^{\lfloor \frac{n}{a} \rfloor}{g(b)}}$。
易求$g(x)$的前缀和,问题在于如何快速推得$h(a)$的值。

由于$h(x)$是积性函数且$x$是Powerful~Number,在DFS时仅需计算$h(p^e),e>1$的值。
使用$f(p^e)$展开式,快速求得$f(p^e)$与$g(p^e)$,再根据历史信息$h(p^{e'}),e'<e$,
可以快速得到$h(p^e)$。如果$g(x)$是完全积性函数,可以对$f(p^e)$展开式平移得到$f(p^{e+1})$
的展开式。由于$e$很小,$h(p^e)$的求值不是瓶颈。预处理可以节省DFS时的重复计算。

{\bfseries 十分有效的优化:DFS递归时会遇到大量的0次项,这些不必要的递归会导致实际运行缓慢。
可以DFS钦定一些质因子必选,使得每层DFS都对最终的$a$有贡献,杜绝爆栈。}

具体实现参考Project~Euler~484:
\lstinputlisting{Source/Source/'Number Theory'/PE484.cpp}

{\bfseries 记得要计算$a=1$时的贡献!!!}

上述内容参考了fjzzq2002的博客\footnote{
利用powerful~number求积性函数前缀和
    \url{https://www.cnblogs.com/zzqsblog/p/9904271.html}
}。

Min\_25使用Powerful~Number得到新的做法,参见
Sum~of~Multiplicative~Function~on~Powerful~Numbers
\footnote{\url{https://min-25.hatenablog.com/entry/2018/11/11/172228}}。
\subsection{素数k次幂前缀和}
可以使用Min\_25筛的前半部分在$O(\frac{n^{3/4}}{\lg n})$内解决,但这不是最优的。

一般使用Meissel Lehmer方法在$O\left(\left(\frac{n}{\lg n}\right)^{2/3}\right)$
内解决,留坑待补。
\index{*TODO!Meissel Lehmer}
\subsection{约数个数函数前缀和}
使用整除分块可以在$O(\sqrt{n})$内解决,但还有更优算法。

考虑变化后的式子$S(n)=\displaystyle \sum_{i=1}^n{\lfloor\frac{n}{i}\rfloor}$,
可以将其理解为在$[1,n]$内在双曲线$xy=n$与$x$轴内的整点数。使用Stern-Brocot Tree可以
在$O(n^{1/3}\lg n)$内解决,留坑待补。

\index{*TODO!约数个数函数前缀和}

\section{BSGS}
\index{B!Baby Step Giant Step}
BSGS法(Baby Step Giant Step)用来求解类似于$a^x\equiv b\pmod{P}$的方程。
\subsection{BSGS}
普通BSGS仅考虑$P$为素数的情况。

以下为求解最小非负整数解的方法:

首先根据定理~\ref{ET}可知$x$的最小非负整数解小于$\varphi(P)=P-1$。将x表示为
$\sqrt{P}$进制数,分别用$O(\sqrt{P})$的复杂度枚举值,一半存入HashTable,另一半
查询是否有匹配值。注意参数的枚举顺序。

\begin{enumerate}
    \item 若$a$为$P$的倍数,则特判$b$是否为$0$,算法结束;
    \item 令$m=\lceil\sqrt{P}\rceil,x=im-j$,移项得$a^{im}\equiv ba^j\pmod{P}$;
    \item 枚举$ba^j$的值,按$j$从小到大{\bfseries 覆盖}存入HashTable;
    \item 枚举$(a^m)^i$的值,按$i$从小到大在HashTable中查询,存在则返回$im-j$;
    \item 返回无解。
\end{enumerate}

\subsection{ExBSGS}
ExBSGS可解决$a,P$不互质的问题。主要思路是将原方程化为普通BSGS可解决的方程。

记化简后方程为$Aa^{x-B}\equiv b\pmod{P}$,化简步骤如下:
\begin{enumerate}
    \item 将$A,B$初始化为$1,0$;
    \item 令$d=(a,P)$,
    \begin{itemize}
        \item 若$d\mid b$,则提出一个因子$d$,即$A*=a/d,b/=d,P/=d,++B$;
        \item 若$d\nmid b$,则特判$b$是否为$A$,$b=A$则$x=B$,$b\neq A$则无解;
    \end{itemize}
    \item 重复第2步直至$d=1$。
\end{enumerate}
令$x=im-j+B$转化为普通BSGS,{\bfseries 注意在BSGS前要暴力检查$x\in[0,B)$是否可行}。

代码如下:
\lstinputlisting{Source/Templates/exBSGS.cpp}

以上内容参考了ZigZagK的博客\footnote{BSGS及扩展BSGS
\url{https://blog.csdn.net/zzkksunboy/article/details/73162229}}。

\section{原根}\label{PrimitiveRoot}
\index{P!Primitive Root}
\subsection{基本定义与定理}
\subsubsection{数论阶}
设$n>1,(a,n)=1$,记$\delta_n(a)$为使得$a^r\equiv 1 \pmod{n}$
成立的最小正整数$r$,称其为$a$模$n$的阶。

\begin{theorem}
	设$n>1,(a,n)=1$,若$a^x\equiv 1 \pmod{n}$,则有$\delta_n(a)\mid x$。
\end{theorem}

\subsubsection{原根}
若$\delta_n(a)=\varphi(n)$,则称$a$为模$n$的一个原根。

若$a$为模$n$的原根,根据定理~\ref{ET},对于$0\leq i< \varphi(n)$,$a^i\bmod{n}$两两不同。

\begin{theorem}
	如果模$n$有原根,则它一共有$\varphi(\varphi(n))$个原根。
\end{theorem}

\begin{theorem}
	$n=2,4,p^i,2p^i\Leftrightarrow$模$n$有原根,其中$p$为奇素数。
\end{theorem}

\subsection{求模n的原根}

对$\varphi(n)$进行质因数分解,
对于$\displaystyle \varphi(n)=\prod_{i=1}^m{p_i^{c_i}}$,若
恒有$g^\frac{\varphi(n)}{p_i}\not\equiv 1 \pmod{n}$,则$g$为模$n$的原根。

以上内容参考了mosquito\_zm的博客\footnote{原根
	\url{https://blog.csdn.net/mosquito\_zm/article/details/77227570}}。
\subsection{原根的应用}
\begin{itemize}
	\item 在NTT中用于推算主单位根。
	\item 将乘积恒定转换为幂次和恒定后NTT。
	\paragraph{例题} [SDOI2015]序列统计\footnote{【P3321】[SDOI2015]序列统计 - 洛谷
	\url{https://www.luogu.org/problemnew/show/P3321}}

	将$x$映射为$g^i$,使用生成函数推导出多项式幂的形式,最后对答案进行逆映射。

	\lstinputlisting[title=luogu P3321]{Source/Source/'FFT NTT'/3321.cpp}

\end{itemize}

%\section{二次剩余与三次剩余}
\subsection{勒让德符号}
\index{L!Legendre Symbol}
定义勒让德符号:
\begin{displaymath}
	\Legendre{a}{p}=
	\left\{\begin{array}{lr}
		0  & a\equiv 0 \pmod{p}                  \\
		1  & \exists x,x^2\equiv a \pmod{p}      \\
		-1 & \not \exists x,x^2\equiv a \pmod{p}
	\end{array}\right.
\end{displaymath}
勒让德符号是完全积性函数,即
\begin{displaymath}
	\Legendre{ab}{p}=\Legendre{a}{p}\Legendre{b}{p}
\end{displaymath}
\subsubsection{与斐波那契数列的关系}
\begin{theorem}
	若$p$为素数,则
	\begin{displaymath}
		F_{p-\Legendre{p}{5}}\equiv 0 \pmod{p}
	\end{displaymath}
	\begin{displaymath}
		F_p\equiv \Legendre{p}{5} \pmod{p}
	\end{displaymath}
\end{theorem}
该定理用于求超大斐波那契数取模。
\subsubsection{二次互反律}
\index{Q!Quadratic Reciprocity Law}
\begin{theorem}
	若$p,q$为不同的奇素数,则
	\begin{displaymath}
		\Legendre{p}{q}\Legendre{q}{p}=(-1)^\frac{(p-1)(q-1)}{4}
	\end{displaymath}
\end{theorem}
此外有两个补充结论:
\begin{theorem}
    \begin{displaymath}
        \Legendre{-1}{p}=(-1)^\frac{p-1}{2}=\left\{\begin{array}{lr}
            1 & \textrm{if~} p\equiv 1\pmod{4}\\
            -1 & \textrm{if~} p\equiv 3\pmod{4}\\
        \end{array}\right.
    \end{displaymath}
\end{theorem}
\begin{theorem}
    \begin{displaymath}
        \Legendre{2}{p}=(-1)^\frac{p^2-1}{8}=\left\{\begin{array}{lr}
            1 & \textrm{if~} p\equiv 1,7\pmod{4}\\
            -1 & \textrm{if~} p\equiv 3,5\pmod{4}\\
        \end{array}\right.
    \end{displaymath}
\end{theorem}
\subsection{二次剩余}
\index{Q!Quadratic Residue}
求解二次剩余即求解下列同余方程:
\begin{displaymath}
	x^2\equiv a \pmod{p}
\end{displaymath}
\subsubsection{欧拉判别准则}
\index{E!Euler's Criterion}
\begin{theorem}[Euler's Criterion]
    若$p$为奇素数且$p\nmid a$,则
    \begin{displaymath}
        \Legendre{a}{p}\equiv a^\frac{p-1}{2}\pmod{p}
    \end{displaymath}
\end{theorem}
\subsubsection{模奇素数}
这里使用ACdreamer介绍的Cipolla随机化算法。
\index{C!Cipolla's Algorithm}
\begin{theorem}
    设$b$满足$\omega=b^2-a$不是模$p$的二次剩余,则
    $x\equiv (b+\sqrt{\omega})^\frac{p-1}{2}\pmod{p}$是
    方程$x^2\equiv a\pmod{p}$的解。
\end{theorem}
证明:

\subsubsection{模奇素数幂}
\subsubsection{模2的幂}
\subsubsection{模合数}

上述内容参考了Miskcoo\footnote{
	[数论]二次剩余及计算方法
	\url{http://blog.miskcoo.com/2014/08/quadratic-residue}
}和ACdreamer\footnote{
    二次同余方程的解
    \url{https://blog.csdn.net/acdreamers/article/details/10182281}
}的博客与百度百科(勒让德符号,二次互反律与欧拉判别准则)。
\subsection{三次剩余}


\chapter{集合论~群论}
\minitoc
\section{集合论定理}
\begin{theorem}[对称差]
	\begin{displaymath}
		A\oplus B=(A-B)\cup(B-A)=A\cup B - A\cap B
	\end{displaymath}
\end{theorem}
\index{D!De Morgan's Laws}
\begin{theorem}[De Morgan's Laws]\label{DML}
	\begin{eqnarray*}
		\overline{A\cup B}=\overline{A}\cap \overline{B} \\
		\overline{A\cap B}=\overline{A}\cup \overline{B}
	\end{eqnarray*}
\end{theorem}
\index{I!Inclusion–exclusion Principle}
\begin{theorem}[Inclusion–exclusion Principle]\label{IEP}
	\begin{displaymath}
		\left|\bigcup_{i=1}^n{A_i}\right|=
		\sum_{\emptyset \neq J\subseteq \{1,2,\cdots,n\}}{(-1)^{|J|-1}
			\left|\bigcap_{j\in J}{A_j}\right|}
	\end{displaymath}
\end{theorem}

容斥原理用来求集合并的大小,为了求集合交的大小,可以使用补集转换思想,
由定理~\ref{DML}与~\ref{IEP}可得

\begin{theorem}\label{ExDML}
	\begin{displaymath}
		\left|\bigcap_{i=1}^n{A_i}\right|=
		\left|\overline{\bigcup_{i=1}^n{\overline{A_i}}}\right|=
		|U|+\sum_{\emptyset \neq J\subseteq \{1,2,\cdots,n\}}{(-1)^{|J|}
			\left|\bigcap_{j\in J}{\overline{A_j}}\right|}
	\end{displaymath}
\end{theorem}

以上内容参考了Wikipedia-EN\footnote{Inclusion–exclusion principle - Wikipedia\\
	\url{https://en.wikipedia.org/wiki/Inclusion\%E2\%80\%93exclusion\_principle}

	De Morgan's laws - Wikipedia
	\url{https://en.wikipedia.org/wiki/De\_Morgan\%27s\_laws}}以及
国家集训队2013论文集《浅谈容斥原理》。
\subsection{模意义统计方案}
若要求恰好满足$k$个条件的方案数,且这些条件是等价的。考虑使用至少满足$k$个条件的方案数
容斥求出,后者由于限制条件较松,很容易使用NTT等方法得到。

记$g(x)$为至少满足$x$个条件的方案数,那么$g(i),i\geq k$构成的每种方案都对应
$\binomial{i}{k}$种满足$k$个条件的方案。由容斥可得
$ans=\displaystyle \sum_{i=k}^n{(-1)^{i-k}\binomial{i}{k}g(i)}$。

\section{拉格朗日定理}
\index{L!Lagrange's Theorem}
\begin{theorem}[Lagrange's Theorem]\label{LT}
	若$(S,\oplus)$是一个有限群,$(S',\oplus)$是$(S,\oplus)$的子群,则
	$|S'|$是$|S|$的约数。
\end{theorem}
证明留坑待补。
\index{*TODO!拉格朗日定理证明}

\section{置换群}
\index{P!Permutation Groups}
{\bfseries 置换}是从$[1,n]$到$[1,n]$的一一映射。

置换可以分解为多个循环,计算循环相关数据的方法为:枚举每一个节点
\begin{enumerate}
    \item 若该节点已被访问,则跳过;
    \item 顺着该节点对应的目标节点不断跳跃,标记已访问,直至跳跃到已访问点(即出发点)为止。
    \item 这个环就是一个循环。
\end{enumerate}
\begin{theorem}
    若对于一个置换有$n$个循环,长度分别为$l_1,l_2,\cdots,l_n$,
    则该置换的循环节长度为$lcm(l_1,l_2,\cdots,l_n)$。
\end{theorem}
\paragraph{不动点}
若一个状态$S$经由置换$f$置换后的状态与原状态相同,则状态$S$为$f$的不动点。
\paragraph{等价关系}
对于一个置换集合$F$,若状态$S$能经由$F$中的置换变为状态$S'$,则称$S$与$S'$等价。
\paragraph{等价类}
满足等价关系的状态属于同一等价类。

\subsubsection{Burnside引理}
\index{B!Burnside's Lemma}
\begin{lemma}[Burnside's Lemma]
    等价类数目为置换群$G$中所有置换的不动点数目的平均值。
    \begin{displaymath}
        |X/G|=\frac{1}{|G|}\sum_{g\in G}|X^g|
    \end{displaymath}
\end{lemma}
上述定理证明留坑待补。
\index{*TODO!证明Burnside引理}
\subsubsection{Polya定理}
\index{P!Pólya Enumeration Theorem}

\begin{theorem}[Pólya Enumeration Theorem]
    若对每一个节点进行$m$染色,置换$g$有$c(g)$个循环,则染色方案
    等价类数目为$\displaystyle \frac{1}{|G|}\sum_{g\in G}m^{c(g)}$。
\end{theorem}

证明:一个循环内所有的节点颜色相同,不同循环颜色的选择是独立的,每一个循环颜色选择
方案对应一个不动点,根据乘法原理可知$|X^g|=m^{c(g)}$。

以上内容参考了QAQqwe的博客\footnote{Burnside引理与Polya定理
\url{https://blog.csdn.net/liangzhaoyang1/article/details/72639208}}与
Wikipedia-EN\footnote{
    Burnside's lemma - Wikipedia
    \url{https://en.wikipedia.org/wiki/Burnside\%27s\_lemma}

    Pólya enumeration theorem - Wikipedia
    \url{https://en.wikipedia.org/wiki/P\%C3\%B3lya\_enumeration\_theorem}
}。

\subsubsection{常见题型}
题型来自My\_ACM\_Dream的博客\footnote{polya|burnside定理的一些总结\\
\url{https://blog.csdn.net/My\_ACM\_Dream/article/details/45312365}}。

\paragraph{正方形旋转}
n*n正方形染色:
\begin{itemize}
    \item 旋转$0^\circ$,循环节数$n^2$。
	\item 旋转$90^\circ/270^\circ$,若$n$为偶数,循环节数$\frac{n^2}{4}$;
	若$n$为奇数,循环节数$\frac{n^2-1}{4}+1$。
    \item 旋转$180^\circ$,若$n$为偶数,循环节数$\frac{n^2}{2}$;若$n$为奇数,循环
    节数$\frac{n^2-1}{2}+1$。
\end{itemize}
奇偶循环节数不同的原因是因为$n$为奇数时中间的点自成一个循环节。
\paragraph{正方形反射(对称)}
\begin{tabular}{|c|c|c|}
	\hline
			 & 对角反射& 对边中点反射\\
	\hline
	$n$为奇数 & $\frac{n^2-n}{2}+n$& $\frac{n^2-n}{2}+n$ \\
	\hline
	$n$为偶数 & $\frac{n^2-n}{2}+n$& $\frac{n^2}{2}$\\
	\hline
\end{tabular}
\paragraph{环形旋转}
对于一个有$n$个点的环,旋转$i$个点的置换的循环节数为$(n,i)$。

证明:$i$最小乘上$\frac{n}{(n,i)}$才会被$n$整除,所以每一个循环节的长度为
$\frac{n}{(n,i)}$,循环节个数为$(n,i)$。
\paragraph{环形对称翻转}
\begin{itemize}
	\item $n$为奇数:只有n种置换(以一点一边中点为对称轴),循环节数为
	$[\frac{n}{2}]+1$。
	\item $n$为偶数:\begin{itemize}
		\item 边边中点:$\frac{n}{2}$种,循环节数为$\frac{n}{2}$。
		\item 点点:$\frac{n}{2}$种,循环节数为$\frac{n}{2}+1$。
	\end{itemize}
\end{itemize}
\paragraph{正方体旋转}
注意是{\bfseries 棱边}置换。
\begin{itemize}
	\item 自身不变,置换1种,循环节12个,长度1;
	\item 以对面中心为轴,旋转角为$90^\circ,180^\circ,270^\circ$,
	轴有3种选择,共9种置换。
	\begin{itemize}
		\item $90^\circ/270^\circ$:循环节3个,长度4。
		\item $180^\circ$:循环节6个,长度2。
	\end{itemize}
    \item 以对边中点为轴,旋转角为$180^\circ$,有6对边,置换数为6,
    有5个长度为2的循环和2个长度为1的循环。
    \item 以对顶点为轴,旋转角为$120^\circ,240^\circ$,有4对点,置换数为8,
    均有4个长度为3的循环。
\end{itemize}
总置换数24。
\paragraph{$n$较小}
\begin{itemize}
    \item 颜色不限:裸Polya解决。
    \item 颜色限制:裸Burnside解决。
\end{itemize}
\paragraph{环形旋转且$n$较大}
枚举循环节数(即$d=(n,i)$),利用欧拉函数与容斥解决。
\paragraph{有染色限制}
使用dp与矩阵快速幂解决。


\chapter{组合数学}
\section{Catalan数}
\subsection{性质}
\index{C!Catalan Numbers}
Catalan数是组合数学中的常见数列
\footnote{A000108 - OEIS \url{http://oeis.org/A000108}},其前几项为
\begin{displaymath}
	1, 1, 2, 5, 14, 42, 132, 429, 1430, \ldots
\end{displaymath}

Catalan数(记为$C_n$)满足如下关系:
\begin{eqnarray}
	C_0&=&C_1=1\\
	C_{n+1}&=&\sum_{i=0}^n{C_iC_{n-i}}\label{CT2}\\
	&=&\sum_{i=1}^n{C_iC_{n+1-i}}\label{CT3}\\
	C_n&=&\frac{4n-2}{n+1}C_{n-1}\\
	C_n&=&{2n \choose n}-{2n \choose n+1}=\frac{1}{n+1}{2n \choose n}\\
	C_n&=&\prod_{k=2}^n\frac{n+k}{k}
\end{eqnarray}
根据Striling近似公式
\begin{displaymath}
	n!\sim\sqrt{2\pi n}\left(\frac{n}{e}\right)^n
\end{displaymath}
可得
\begin{displaymath}
	C_n\sim\frac{4^n}{\sqrt{\pi} n^\frac{3}{2}}
\end{displaymath}
\subsection{常见应用}
\subsubsection{括号序列,出栈序列,网格行走}
\paragraph{括号序列} 给定$2n$个位置填上左右括号使括号匹配(对于每一个位置之前的
左括号必须不少于右括号)。
\paragraph{出栈序列} 求将$n$个元素入栈一次(限制顺序)并出栈一次(不限制顺序)
的方案数(对于每一次操作都要保证栈不出现下溢,即入栈元素不少于出栈元素)。
\paragraph{网格行走} 在一个$n*n$的网格内从左下角移动到右上角,纵坐标必须不少于
横坐标,求方案数。
\paragraph{分析}
这三个问题是同构的,都满足操作数为$2n$且限制任意时刻操作A的数目不少于操作B的数目。
它们的答案都是$C_n$,以括号序列问题为例,通过等式~\ref{CT2}理解:
将括号序列看做由一个可分割的序列加上一个不可分割的序列(即最外层有一对配对括号)得来,
左边为$n_1+1$对,右边为$n_2$对,满足$n_1+n_2=n-1$,这种方案的贡献为
$C_{n_1}C_{n_2}$。
\subsubsection{二叉树构型计数}
\paragraph{有$n$个节点的二叉树}
通过等式~\ref{CT2}理解:枚举左右子树大小,满足左右子树节点数为$n-1$。
\paragraph{有$n+1$个叶子节点的满二叉树}
通过等式~\ref{CT3}理解:枚举左右子树叶子节点数,满足其和为$n+1$。
\subsubsection{阶梯填充}
用$n$个长方形填充$n*n$的阶梯的方案数为$C_n$。

不严格证明:填充一个以直角顶点与阶梯顶点为对顶点的长方形,使其分为大小为
$n_1*n_1,n_2*n_2$的两个小阶梯,满足$n_1+n_2=n-1$,分别分配$n_1,n_2$
个长方形的份额,就成为子问题了。该分析满足等式\ref{CT2}。
\subsubsection{凸包分割}
将$n+2$个顶点的凸包分为三角形的方案数为$C_n$。

猜想:最终将分为$n$个三角形。

证明留坑待补。
\subsubsection{圆上点连线}
将圆上的$2n$个点两两配对连线,所连$n$条线段不相交的方案数为$C_n$。

证明留坑待补。

\index{*TODO:Catalan应用证明}
上述内容参考了Wikipedia-EN\footnote{Catalan number - Wikipedia
	\url{https://en.wikipedia.org/wiki/Catalan\_number}}。

\section{Stirling数}
\index{S!Stirling Number}
\subsection{第一类Stirling数}
\newcommand{\stirlingA}[2]{\left[#1 \atop #2\right]}
令$\stirlingA{n}{k}$为把$n$个点放到$k$个环内的方案数,有
\begin{eqnarray*}
    \stirlingA{n}{0}&=&0 \textrm{~for all $n\geq 1$}\\
    \stirlingA{n}{n}&=&1 \textrm{~for all $n\geq 0$}\\
    \stirlingA{n}{k}&=&(n-1)\stirlingA{n-1}{k}+\stirlingA{n-1}{k-1}\\
    \sum_{i=0}^n{\stirlingA{n}{i}}&=&n!
\end{eqnarray*}
证明递推式:
\begin{itemize}
    \item 若将当前点丟给之前的环,则可以选择$p-1$个点在其右边,因此贡献
    $(n-1)\stirlingA{n-1}{k}$。
    \item 当前点自成一环,贡献$\stirlingA{n-1}{k-1}$。
\end{itemize}
\subsection{第二类Stirling数}
\newcommand{\stirlingB}[2]{\left\{#1 \atop #2\right\}}
令$\stirlingB{n}{k}$为把$n$个点放到$k$个集合内的方案数,有
\begin{eqnarray}
    \stirlingB{n}{0}&=&0 \textrm{~for all $n\geq 1$}\\
    \stirlingB{n}{n}&=&1 \textrm{~for all $n\geq 0$}\\
    \stirlingB{n}{k}&=&k\stirlingB{n-1}{k}+\stirlingB{n-1}{k-1}\label{SB3}\\
    \stirlingB{n}{k}&=&\frac{1}{k!}\sum_{i=0}^k{(-1)^{k-i}{k \choose i}i^n}
    \label{SB4}
\end{eqnarray}
证明等式~\ref{SB3}:
\begin{itemize}
    \item 若将当前点丟给之前的集合,则可以选择$k$个集合,因此贡献
    $k\stirlingA{n-1}{k}$。
    \item 当前点自成一集合,贡献$\stirlingB{n-1}{k-1}$。
\end{itemize}

证明~\ref{SB4}:
考虑计算将$n$个点放入$k$个带编号集合且无空集的方案数:
\begin{eqnarray*}
    N&=&k^n+\sum_{i=1}^k{(-1)^i{k \choose i}(k-i)^n}
    \textrm{~(根据定理~\ref{ExDML})}\\
    &=&k^n+\sum_{i=0}^{k-1}{(-1)^{k-i}{k \choose i}i^n}\\
    &=&\sum_{i=0}^k{(-1)^{k-i}{k \choose i}i^n}
\end{eqnarray*}
最后变为无标号时除以$k!$即可。

\section{Lucas/ExLucas}
\index{L!Lucas's Theorem}
\subsection{Lucas定理}
\begin{theorem}[Lucas's Theorem]
	对于非负整数$n,m$以及质数$p$,若
	\begin{eqnarray*}
		n&=&\sum_{i=0}^k{n_ip^i}\\
		m&=&\sum_{i=0}^k{m_ip^i}
	\end{eqnarray*}
	则
	\begin{displaymath}
		{n \choose m}\equiv\prod_{i=0}^k{n_i \choose m_i} \pmod{p}
	\end{displaymath}
\end{theorem}

以上内容参考了Wikipedia-EN\footnote{Lucas's theorem - Wikipedia
	\url{https://en.wikipedia.org/wiki/Lucas\%27s\_theorem}}
\subsection{ExLucas}


\chapter{多项式}
\section{快速傅里叶变换FFT}
\subsection{FFT原理}
FFT求多项式卷积的过程为:$\Theta(n\lg n)$求值->$\Theta(n)$点值乘法->
$\Theta(n\lg n)$插值。

$\Theta(n\lg n)$求值/插值的复杂度是在单位复数根上计算得到的。

\subsubsection{单位复数根}

定义{\bfseries $n$次单位复数根}是满足$\omega^n=1$的复数$\omega$,恰好有$n$个,即
$\omega_n^k=e^{2\pi ik/n},k=0,1,\cdots,n-1$。

定义{\bfseries 主$n$次单位根}$\omega_n=e^{2\pi i/n}$。

下面是关于$n$次单位复数根的性质:

\begin{lemma}[消去引理]\label{FFTL1}
	对于任意整数$n\geq 0,k \geq 0,d>0$,
	\begin{displaymath}
		\omega_{dn}^{dk}=\omega_n^k
	\end{displaymath}
\end{lemma}
证明:
\begin{displaymath}
	\omega_{dn}^{dk}=e^{2\pi i dk/dn}=e^{2\pi i k/n}=\omega_n^k
\end{displaymath}

\begin{inference}\label{FFTI2}
	对于任意偶数$n>0$,有
	\begin{displaymath}
		\omega_n^{n/2}=\omega_2=-1
	\end{displaymath}
\end{inference}

\begin{lemma}[折半引理]
	对于偶数$n>0$,$n$个$n$次单位复数根的平方组成的集合为$n/2$个$n/2$
	次单位复数根的集合。
\end{lemma}
证明:根据引理~\ref{FFTL1}可得$(\omega_n^k)^2=(\omega_n^{k+n/2})^2=
	\omega_{n/2}^k$,每个$n/2$次单位复数根恰好被得到2次。

\begin{lemma}[求和引理]
	对于任意整数$n\geq 1$与不能被$n$整除的非负整数$k$,有
	\begin{displaymath}
		\sum_{i=0}^{n-1}{(w_n^k)^i}=0
	\end{displaymath}
\end{lemma}
证明:
\begin{displaymath}
	\sum_{i=0}^{n-1}{(w_n^k)^i}=\frac{(w_n^k)^n-1}{w_n^k-1}=0
\end{displaymath}
$n$不整除$k$保证了分母不为0。

\subsubsection{DFT}
对于次数界为$n$的多项式
\begin{displaymath}
	A(x)=\sum_{i=0}^{n-1}{a_ix^i}
\end{displaymath}
其DFT为
\begin{displaymath}
	DFT_n(a)=(y_0,y_1,\cdots,y_{n-1})=
	(A(\omega_n^0),A(\omega_n^1),\cdots,A(\omega_n^{n-1}))
\end{displaymath}

\subsubsection{FFT}
FFT采用分治策略,假设$n$是2的幂(不足补0),其步骤如下:
\begin{enumerate}
	\item 若次数界为1,则返回$a_0$。
	\item 定义新的次数界为$n/2$多项式
	      \begin{eqnarray*}
		      A^{[0]}(x)&=&a_0+a_2x+\cdots+a_{n-2}x^{n/2-1}\\
		      A^{[1]}(x)&=&a_1+a_3x+\cdots+a_{n-1}x^{n/2-1}
	      \end{eqnarray*}
	      递归计算其在点$(\omega_n^0)^2,(\omega_n^1)^2,\cdots,(\omega_n^{n-1})^2$
	      的值(实际上递归只求了前一半)。
	\item 该多项式满足等式\begin{equation}\label{RFFTE}
		      A(x)=A^{[0]}(x^2)+xA^{[1]}(x^2)
	      \end{equation}
	      可利用递归计算的值合并。
	      对于$k=0,1,\cdots,n/2-1$,
	      \begin{eqnarray*}
		      y_k&=&y_k^{[0]}+\omega_n^ky_k^{[1]}\\
		      y_{k+n/2}&=&y_k^{[0]}-\omega_n^ky_k^{[1]}
	      \end{eqnarray*}
	      正确性证明:
	      \begin{eqnarray*}
		      y_k&=&y_k^{[0]}+\omega_n^ky_k^{[1]}\\
		      &=&A^{[0]}(\omega_{n/2}^k)+\omega_n^kA^{[1]}(\omega_{n/2}^k)\\
		      &=&A^{[0]}(\omega_n^{2k})+\omega_n^kA^{[1]}(\omega_n^{2k})
		      \textrm{~(根据引理~\ref{FFTL1})}\\
		      &=&A(\omega_n^k) \textrm{~(根据式~\ref{RFFTE})}\\
		      y_{k+n/2}&=&y_k^{[0]}-\omega_n^ky_k^{[1]}\\
		      &=&A^{[0]}(\omega_{n/2}^k)+\omega_n^{k+n/2}A^{[1]}(\omega_{n/2}^k)
		      \textrm{~(根据推论~\ref{FFTI2})}\\
		      &=&A^{[0]}(\omega_n^{2k+n})+\omega_n^{k+n/2}A^{[1]}(\omega_n^{2k+n})
		      \textrm{~(根据引理~\ref{FFTL1}与$\omega_n^n=1$)}\\
		      &=&A(\omega_n^{k+n/2}) \textrm{~(根据式~\ref{RFFTE})}\\
	      \end{eqnarray*}
\end{enumerate}
\subsubsection{逆DFT}
\begin{theorem}
	\begin{displaymath}
		DFT_n^{-1}(a)=\frac{1}{n}DFT_n(DFT_n(a))
	\end{displaymath}
\end{theorem}
证明:

以上内容来自算法导论\cite{ITA3}第30章 多项式与快速傅里叶变换。
\subsection{迭代FFT实现}
\subsubsection{单位复数根预处理}
\subsubsection{离线位逆序}
\subsubsection{在线位逆序}
\subsection{实序列DFT}
\subsection{MTT之拆系数FFT}

\section{快速数论变换NTT}
\subsection{NTT原理}
NTT的原理与FFT类似,即找到单位根$x$满足$x^n\equiv 1 \pmod{p}$。
NTT模数$p$需满足$p$为质数且$p=r\cdot 2^k+1$。

根据定理~\ref{FLT}可知若模数$p$为质数则有$x^{p-1}\equiv 1 \pmod{p}$,
所以当$n|(p-1)$时才能进行NTT。

根据~\ref{PrimitiveRoot}节所述,$p$必有原根,设$p$的原根为$g$,则
$g^\frac{p-1}{n}$就是{\bfseries 主$n$次单位根},$n$个单位根即为
$w_n^k=g^{k\cdot \frac{p-1}{n}}$。

其余部分与FFT相同。

\subsection{NTT实现}
NTT仅预处理单位复数根部分不同,以模998244353为例:
\begin{lstlisting}
int tot, root[size], invR[size];
void init(int n) {
    const int g = 3;
    tot = n;
    Int64 base = powm(g, (mod - 1) / n);
    Int64 invBase = powm(base, mod - 2);
    root[0] = invR[0] = 1;
    for (int i = 1; i < n; ++i)
        root[i] = root[i - 1] * base % mod;
    for (int i = 1; i < n; ++i)
        invR[i] = invR[i - 1] * invBase % mod;
}
\end{lstlisting}
\subsection{NTT常见模数}
\begin{itemize}
    \item $469762049=7*2^26+1$。
    \item $998244353=119*2^23+1$。
    \item $1004535809=479*2^21+1$,加起来不爆int。
    \item $2281701377=17*2^27+1$,平方恰好不爆long long。
\end{itemize}
\index{*Constant!NTT模数P=\{469762049,998244353,1004535809,2281701377\},g=3}
它们的原根均为3。
\subsection{MTT之三模数NTT}
选取3个模数,比如\{469762049,998244353,1004535809\},它们的乘积大于卷积
过程中最大的数,分别以这三个数为模数求NTT,最后解同余方程组即可。

但是使用CRT求解会爆long long,因此先合并前两项,得到
\begin{eqnarray*}
    x&\equiv&n_1 \pmod{p_1}\\
    x&\equiv&n_2 \pmod{p_2}
\end{eqnarray*}
设$x=k_1p_1+n_1=k_2p_2+n_2$,由于$k_1<p_2$,我们可以求解
$k_1p_1\equiv n_2-n_1 \pmod{p_2}$得到$k_1$,带入原式求出$x \bmod{p}$的值。

该方法来自AntiLeaf\footnote{COGS2294 释迦 - AntiLeaf
\url{http://www.cnblogs.com/hzoier/p/6441967.html}}

\section{快速沃尔什变换FWT}
\subsection{FWT原理}
FWT主要用来求下列三种卷积:
\begin{eqnarray*}
    z_n&=&\sum_{i\&j=n}{a_ib_j}\\
    z_n&=&\sum_{i|j=n}{a_ib_j}\\
    z_n&=&\sum_{i\land j=n}{a_ib_j}
\end{eqnarray*}
\subsection{FWT实现}

\section{多项式高级操作}
\subsection{牛顿迭代法}
已知函数$G(z)$,求函数$F(z) \bmod{z^n}$满足$G(F(z))\equiv 0 \pmod{z^n}$。

令$n=2^t$,若$n$不为2的幂,补齐后截断即可。

当$t=0$时,简单地令$F(z)$的常数项为0。

若已知$G(F_0(z)) \equiv 0\pmod{z^{2^t}}$,
可以计算$G(F(z))$在$F_0(z)$上的泰勒展开:
\begin{displaymath}
    G(F(z))=\sum_{i=0}^\infty{\frac{G^{(i)}(F_0(z))}{i!}\cdot (F(z)-F_0(z))^i}
\end{displaymath}
由于$F(z)$与$F_0(z)$后$2^t$项均相等,所以它们之差的平方的最小非0项次数$\geq 2^{t+1}$,
因此仅前两项有效,即
\begin{displaymath}
    G(F(z))\equiv G(F_0(z))+G'(F_0(z))(F(z)-F_0(z)) \pmod{z^{2^{t+1}}}
\end{displaymath}
结合$G(F(z))\equiv 0 \pmod{z^{2^{t+1}}}$可得到新的$F_0(z)$:
\begin{displaymath}
    F(z)\equiv F_0(z)-\frac{G(F_0(z))}{G'(F_0(z))} \pmod{z^{2^{t+1}}}
\end{displaymath}
这就是牛顿迭代法。
\subsection{多项式开方}
对于给定的$A(z)$,求$F(z) \pmod z^n$,使得$F^2(z)\equiv A(z)\pmod{z^n}$。

构造方程$F^2(z)-A(z)\equiv 0\pmod{z^n}$,
同理可得
\begin{eqnarray*}
    F(z)&\equiv& F_0(z)-\frac{F_0(z)^2-A(z)}{2F_0(z)} \pmod{z^{2^{t+1}}}\\
    &\equiv& \frac{F_0(z)^2+A(z)}{2F_0(z)} \pmod{z^{2^{t+1}}}
\end{eqnarray*}

注意当$t=0$时可能需要用二次剩余在模意义下开根。

\subsection{多项式求逆}
对于给定的$A(z)$,求$F(z) \pmod z^n$,使得$F(z)\cdot A(z)\equiv 1\pmod{z^n}$。

构造方程$F(z)\cdot A(z)-1\equiv 0\pmod{z^n}$,
同理可得
\begin{eqnarray*}
    F(z)&\equiv& F_0(z)-\frac{F_0(z)\cdot A(z)-1}{A(z)} \pmod{z^{2^{t+1}}}\\
    &\equiv& F_0(z)-(F_0(z)\cdot A(z)-1)\cdot{F_0(z)} \pmod{z^{2^{t+1}}}
    \textrm{~(考虑最小非0项可知可乘$F_0(z)$代替$F(z)$)}\\
    &\equiv& 2F_0(z)-F_0^2(z)\cdot A(z) \pmod{z^{2^{t+1}}}
\end{eqnarray*}
\subsection{多项式取模}
\subsection{多项式ln}
\subsection{多项式exp}
\subsection{多项式快速幂}
使用常规快速幂可以得到$O(n\lg n\lg k)$的复杂度。
但是通过如下变形:
\begin{displaymath}
    F^k(z)=e^{k \ln F(z)}
\end{displaymath}
使用多项式ln/exp可以得到$O(n\lg n)$的复杂度。
\subsection{多项式三角函数}
由欧拉公式可得
\begin{displaymath}
    e^{F(z)i}=\cos F(z)+\sin F(z) i
\end{displaymath}
在复数域上做多项式exp即可。
\subsection{进制转换}
\subsection{多项式多点求值/插值}
\subsection{组合数取模}
以上内容参考了picks
\footnote{Newton's Method of Polynomial « Picks's Blog
\url{http://picks.logdown.com/posts/209226-newtons-method-of-polynomial}}
和Miskcoo\footnote{牛顿迭代法在多项式运算的应用 – Miskcoo's Space
\url{http://blog.miskcoo.com/2015/06/polynomial-with-newton-method}}的博客。


\chapter{线性代数}
\minitoc
\section{高斯消元}
\subsection{变换为上三角矩阵}
高斯消元的思路很简单:每次通过初等变换消去第$i$行第$i$列下面的项,
最终变换为一个上三角矩阵。
\begin{lstlisting}[title=gauss]
typedef double FT;
const FT eps=1e-8;
FT A[size][size];
bool gauss(int n) {
    for(int i=1;i<=n;++i) {
        int x=i;
        for(int j=i+1;j<=n;++j)
            if(fabs(A[j][i])>fabs(A[x][i]))
                x=j;
        if(fabs(A[x][i])<eps)return false;
        if(x!=i) {
            for(int j=i;j<=n;++j)
                std::swap(A[i][j],A[x][j]);
        }
        for(int j=i+1;j<=n;++j) {
            FT fac=A[j][i]/A[i][i];
            for(int k=i;k<=n;++k)
                A[j][k]-=fac*A[i][k];
        }
    }
    return true;
}
\end{lstlisting}
时间复杂度$O(n^3)$。
\subsubsection{精度}
对于实数运算,选择绝对值最大数作为主元,以减小误差。
\subsubsection{优化}
可以观察矩阵的特殊性来优化消元复杂度,对于方程间联系不大的情况使用
迭代解小规模线性方程组。
\subsection{求解线性方程组}\label{LSE}
即求解$Ax=b$的向量$x$。
注意到上三角矩阵的最后一行是一元方程,解出后倒数第二行还是是一元方程,所以不断逆推即可。
注意对矩阵$A$的操作也要应用到向量$b$上(两边同时乘上变换矩阵)。
时间复杂度$O(n^2)$。
\begin{lstlisting}
for(int i=n;i>=1;--i) {
    FT sum=B[i];
    for(int j=i+1;j<=n;++j)
        sum-=A[i][j]*X[j];
    X[i]=sum/A[i][i];
}
\end{lstlisting}
\subsection{求解逆矩阵}\label{InvMatGauss}
设将$A$变换为上三角矩阵的变换矩阵为$P$,求逆矩阵即求解方程$PAA^{-1}=PI$,
由矩阵乘法的定义可知,将$A^{-1}$与$PI$按列拆分,即得到$(PA)A_i^{-1}=(PI)_i$
的形式,按照求解线性方程组的方法做即可。

由于要额外维护$PI$,所以常数较LUP分解大。

\subsubsection{板子}

【P4783】【模板】矩阵求逆 - 洛谷
\footnote{\url{https://www.luogu.org/problemnew/show/P4783}}
\lstinputlisting[title=InvMatGauss]{LinearAlgebra/InvMatGauss.cpp}
时间复杂度$O(n^3)$。

\section{LUP分解}
LUP分解的数值稳定性较高斯消元法强。
\subsection{基本原理}
要求解线性方程组$Ax=b$,对系数矩阵$A$进行LUP分解:
\begin{displaymath}
	PA=LU
\end{displaymath}
其中$P$为置换矩阵,$L$为下三角矩阵,$U$为上三角矩阵。

将$Ax=b$左乘$P$,得$PAx=Pb$,然后用$PA=LU$代换得$LUx=Pb$。
设$y=Ux$,有$Ly=Pb$,可以使用类似于~\ref{LSE}节$O(n^2)$求解$y$,
然后再次使用该方法求出$x$。

\subsection{LUP分解}
\subsubsection{LU分解}
考虑不会出现不需要换主元的情况(比如对称正定矩阵),即$P=I$。

运用矩阵代数将A分解:
\begin{eqnarray*}
	A&=&\left[\begin{array}{c|ccc}
			a_{11} & a_{12} & \cdots & a_{1n} \\
			\hline
			a_{21} & a_{22} & \cdots & a_{2n} \\
			\vdots & \vdots & \ddots & \vdots \\
			a_{n1} & a_{n2} & \cdots & a_{nn}
        \end{array} \right]\\
    &=&\left[\begin{array}{cc}
        a_{11}&w^T\\
        v&A'
    \end{array}\right]\\
    &=&\left[\begin{array}{cc}
        1&0\\
        v/a_{11}&I_{n-1}
    \end{array}\right]
    \left[\begin{array}{cc}
        a_{11}&w^T\\
        0&A'-vw^T/a_{11}
    \end{array}\right]\\
    &=&\left[\begin{array}{cc}
        1&0\\
        v/a_{11}&I_{n-1}
    \end{array}\right]
    \left[\begin{array}{cc}
        a_{11}&w^T\\
        0&L'U'
    \end{array}\right] \textrm{~(递归分解子矩阵)}\\
    &=&\left[\begin{array}{cc}
        1&0\\
        v/a_{11}&L'
    \end{array}\right]
    \left[\begin{array}{cc}
        a_{11}&w^T\\
        0&U'
    \end{array}\right]\textrm{~(左右矩阵分别为L,U)}\\
    &=&LU
\end{eqnarray*}

求出矩阵的外围部分与子矩阵后将子矩阵递归分解。

\subsubsection{LUP分解实现}
设换主元时把第1行(因为要递归分解)与第$k$行交换的置换矩阵为$Q$,则$QA$
可以进行LU分解,即\begin{displaymath}
    QA=\left[\begin{array}{cc}
        1&0\\
        v/a_{k1}&I_{n-1}
    \end{array}\right]
    \left[\begin{array}{cc}
        a_{k1}&w^T\\
        0&A'-vw^T/a_{k1}
    \end{array}\right]
\end{displaymath}
设子矩阵满足$P'(A'-vw^T/a_{k1})=L'U'$,与$Q$相乘得到置换矩阵$P$,即
\begin{displaymath}
    P=\left[\begin{array}{cc}
        1&0\\
        0&P'
    \end{array}\right]Q
\end{displaymath}
继续变换:
\begin{eqnarray*}
    PA&=&\left[\begin{array}{cc}
        1&0\\
        0&P'
    \end{array}\right]
    \left[\begin{array}{cc}
        1&0\\
        v/a_{k1}&I_{n-1}
    \end{array}\right]
    \left[\begin{array}{cc}
        a_{k1}&w^T\\
        0&A'-vw^T/a_{k1}
    \end{array}\right]\\
    &=&\left[\begin{array}{cc}
        1&0\\
        P'v/a_{k1}&P'
    \end{array}\right]
    \left[\begin{array}{cc}
        a_{k1}&w^T\\
        0&A'-vw^T/a_{k1}
    \end{array}\right]\\
    &=&\left[\begin{array}{cc}
        1&0\\
        P'v/a_{k1}&I_{n-1}
    \end{array}\right]
    \left[\begin{array}{cc}
        a_{k1}&w^T\\
        0&P'(A'-vw^T/a_{k1})
    \end{array}\right]\\
    &=&\left[\begin{array}{cc}
        1&0\\
        P'v/a_{k1}&I_{n-1}
    \end{array}\right]
    \left[\begin{array}{cc}
        a_{k1}&w^T\\
        0&L'U'
    \end{array}\right]\\
    &=&\left[\begin{array}{cc}
        1&0\\
        P'v/a_{k1}&L'
    \end{array}\right]
    \left[\begin{array}{cc}
        a_{k1}&w^T\\
        0&U'
    \end{array}\right]\\
    &=&LU
\end{eqnarray*}

理解性记忆:按照《线性代数及其应用》的描述,LU分解的目的就是使用一系列行倍加变换把
$A$化为阶梯形$U$,同时构造单位下三角矩阵$L$使得对$L$施加相同的行变换后变为$I$。
记行变换矩阵为$P$,则$PL=I$且$PA=U$,由$L$可逆可得$LP=I$,继而得到$A=LU$。
行倍加变换的策略就是把每次把当前主元位置下方的元素消为0,由于$L$对应的主元位置为1,
该主元位置下方对应的就是行倍加的系数。

实际操作非常简单:选取非0主元置换到对角线上,然后令$l_{ji}=a_{ji}/a_{ii}$,
同时将其作为行倍加系数进行行倍加变换。{\bfseries 注意置换时要整行置换。}

\begin{lstlisting}
typedef double FT;
const FT eps=1e-8;
FT A[size][size],B[size];
bool LUP(int n){
    for(int i=1;i<=n;++i) {
        int x=i;
        for(int j=i+1;j<=n;++j)
            if(fabs(A[j][i])>fabs(A[x][i]))
                x=j;
        if(fabs(A[x][i])<eps)return false;
        if(i!=x){
            for(int j=1;j<=n;++j)
                std::swap(A[i][j],A[x][j]);
            std::swap(B[i],B[x]);
        }
        for(int j=i+1;j<=n;++j) {
            A[j][i]/=A[i][i];
            FT fac=A[j][i];
            for(int k=i+1;k<=n;++k)
                A[j][k]-=fac*A[i][k];
        }
    }
    return true;
}
\end{lstlisting}
{\bfseries 注意L与U同时覆盖于A数组上,即}
\begin{displaymath}
	a_{ij}=\left\{\begin{array}{ll}
		l_{ij} & \textrm{if $i>j$}     \\
		u_{ij} & \textrm{if $i\leq j$}
	\end{array}\right.
\end{displaymath}
\subsection{正向/反向替换}
原理同~\ref{LSE}节,代码如下:
\begin{lstlisting}
FT Y[size],X[size];
void solve(int n) {
    for(int i=1;i<=n;++i) {
        FT sum=B[i];
        for(int j=1;j<i;++j)
            sum-=A[i][j]*Y[j];
        Y[j]=sum;
    }
    for(int i=n;i>=1;--i) {
        FT sum=Y[i];
        for(int j=i+1;j<=n;++j)
            sum-=A[i][j]*X[j];
        X[i]=sum/A[i][i];
    }
}
\end{lstlisting}
\subsection{LUP分解求逆矩阵}
同~\ref{InvMatGauss}节所述,求解$n$次线性方程组得到逆矩阵。LUP分解的好处
在于只需维护置换矩阵$P$而不是维护变换矩阵$PI$(这两个矩阵的意义不同)。
\lstinputlisting[title=InvMatLUP]{Source/Templates/LUP.cpp}

以上内容参考了算法导论\cite{ITA3} 第28章 矩阵运算。

\section{行列式}
\subsection{定义与性质}
\subsection{求行列式}

\section{基础设施}
以下内容主要讨论二维空间中的计算几何。
\subsection{点,向量,直线,半平面的表示}
点与向量由2个坐标表示;半平面和直线由直线上一点$ori$与
直线的方向向量$dir$表示;直线上的一点可表示为$ori+dir*t$,通过控制参数$t$的取值
还可以表示射线或线段;可以人为规定半平面的顺时针/逆时针$180^\circ$为半平面所在点集,
通过叉积来判断点在半平面的哪一边。
\begin{lstlisting}
typedef double FT;
struct Vec {
    FT x,y;
    //constructor
    //operator+-*
};
struct Line {
    Vec ori,dir;
    Vec operator()(FT t) const {
        return ori+dir*t;
    }
};
\end{lstlisting}
\subsection{点乘与叉乘}
\subsubsection{点乘}
向量点乘$dot(a,b)=a.x*b.x+a.y*b.y=|a||b|cos<a,b>$,一般用来
判断与法向量的夹角以及在某个向量上的投影长度。点乘满足加法分配律和交换律。
\subsubsection{叉乘}
向量叉乘$cross(a,b)=a.x*b.y-b.x*a.y=|a||b|sin<a,b>$,这是两向量
构成的平行四边形的有向面积。一般用来判断向量的相对方向以及计算多边形的面积。
叉乘满足$cross(a,b)=-cross(b,a)$和加法分配律(将其视作线性变换
$T:a\rightarrow cross(a,b)$或$T:b\rightarrow cross(b,a)$可证)。

\begin{theorem}[拉格朗日公式]
	cross(a,cross(b,c))=b*dot(a,c)-c*dot(a,b)
\end{theorem}

三维向量的叉乘计算了垂直于这两个向量的向量(两向量组成平面的法向量),即
\begin{displaymath}
	cross(a,b)=\left(\begin{array}{c}
		a.y*b.z-b.y*a.z \\
		a.z*b.x-b.z*a.x \\
		a.x*b.y-b.x*a.y
	\end{array}\right)
\end{displaymath}
其方向满足右手定则(右手四指与大拇指垂直,食指指向向量$a$,其余三指指向向量$b$,
大拇指方向即为叉乘方向),模长满足$|cross(a,b)|=|a||b|sin<a,b>$。
\subsubsection{体积计算}
结合点乘和叉乘可得点$A$与点$B,C,D$所组成的三棱锥$A-BCD$的有向体积为
\begin{displaymath}
	V=\frac{1}{6}|dot(\overrightarrow{BA},cross(\overrightarrow{BC},
	\overrightarrow{BD}))|
\end{displaymath}
其中$\overrightarrow{N}=cross(\overrightarrow{BC},\overrightarrow{BD})$
的方向为法向,模长为底面面积的2倍。$|dot(\overrightarrow{BA},\overrightarrow{N})|$
又给其模长增加了高的因子。套椎体体积公式可得上式。
\subsubsection{极角计算}
点乘可以得到$x=|a||b|cos<a,b>$,叉乘可以得到$y=|a||b|sin<a,b>$,将$(x,y)$
看做在半径为$|a||b|$的圆上的点,极角为$atan2(x,y)$。
\subsection{点到直线的距离}
设偏移向量$delta=p-ori$:
\begin{itemize}
	\item 计算$dot(delta,dir)/|dir|$可得到投影长度$d'$,根据
	      勾股定理得到$d^2=|delta|^2-d'^2$。
	\item 计算$|cross(delta,dir)|$可得偏移向量与方向向量构成的平行四边形的面积,
	      根据面积公式得到$d=|\frac{cross(delta,dir)}{|dir|}|$。
\end{itemize}
叉乘法的运算量少且精度较高,能够指示半平面方向,建议选用。
\subsection{直线、线段的交点}
\begin{lstlisting}
Vec intersect(const Line& a,const Line& b) {
    Vec delta=a.ori-b.ori;
    FT t=cross(b.dir,delta)/cross(a.dir,b.dir);
    return a.ori+a.dir*t;
}
\end{lstlisting}
证明留坑待补。
\index{*TODO!直线相交正确性证明}

线段相交也是如此,但首先要判断两线段是否相交。将该问题转换为
两线段是否互相平分。设线段为$a-b,c-d$,首先判断$a-b$平分$c-d$,
即$c,d$分别位于$a-b$两边,有$cross(c-a,b-a)*cross(d-a,b-a)\leq 0$,
同理对$c-d$也做一遍。
\subsection{判定点是否在多边形内}
\subsubsection{随机射线法}
从点P开始随机引出一条射线,计算其与多边形的边的交点个数,若为奇数次则
在多边形内。注意射线恰好经过点时要重新选择方向。
\subsubsection{旋转角法}
从点P与多边形的一个点开始,不断旋转到下一个点,直至转完一圈为止。
此时若点P旋转了$0^\circ$,则在多边形外;若点$P$旋转了$360^\circ$,
则在多边形内。旋转角可以使用前文所述方法计算,为了避免精度问题,以
$\pi(180^\circ)$为界进行比较。
\subsubsection{半平面法}
若该多边形为凸多边形,以每条边构造半平面,使用叉积判断是否在半平面内,
若点在所有半平面内则在多边形内。
\subsection{向量的旋转}
根据复数乘法的规律:模长相乘,幅角相加。构造逆时针旋转角度$\theta$的单位旋转向量
$e^{\theta i}=\cos \theta+\sin \theta i$,将原向量乘以该旋转向量得到结果:
$(x\cos \theta-y\sin \theta,x\sin \theta+y\cos \theta)$。

有时对点坐标以某点为原点进行旋转变换,然后按照x轴顺序贪心计算是一个不错的骗分方法。
\subsection{坐标系的切换}
有同一$d$维坐标空间中的点$P$与单位正交向量组$b_1,b_2,\cdots,b_d$,以
该向量组为新坐标空间的基向量,那么点$P$在新坐标空间的坐标值为它在
这些向量上的投影长度。

事实上,将点$P$在新坐标系上的坐标$P'$变换回旧坐标系意味着$P'$左乘矩阵
$T={b_1,b_2,\cdots,b_d}$,即$TP'=P$。由于矩阵$T$的特殊性,
有$T^{-1}=T^T$。因此$P'=T^TP$,即投影长度。
\subsection{点、向量、法向量的坐标变换}
可以使用一个矩阵$R^{d*d}$来表示$d$维空间中的旋转,缩放,坐标系切换,
引入齐次坐标(即给向量再加一维,非透视投影时恒为1)可支持平移(仅对于点的变换有意义)。

在三维空间下使用$4*4$的矩阵来表示对坐标的变换。

\subsubsection{平移}
将坐标平移$(x,y,z)$:
\begin{displaymath}
	\left(\begin{array}{cccc}
		1 & 0 & 0 & x \\
		0 & 1 & 0 & y \\
		0 & 0 & 1 & z \\
		0 & 0 & 0 & 1 \\
	\end{array}\right)
\end{displaymath}
\subsubsection{旋转}
以基于$z$轴旋转$\theta$为例:

思路是对$x,y$轴坐标进行二维旋转,$z$轴坐标不变。
\begin{displaymath}
	\left(\begin{array}{cccc}
		\cos \theta & -\sin \theta & 0 & 0 \\
		\sin \theta & \cos \theta  & 0 & 0 \\
		0           & 0            & 1 & 0 \\
		0           & 0            & 0 & 1 \\
	\end{array}\right)
\end{displaymath}
\subsubsection{缩放}
对坐标分别缩放$x,y,z$:
\begin{displaymath}
    \left(\begin{array}{cccc}
        x & 0 & 0 & 0 \\
        0 & y & 0 & 0 \\
        0 & 0 & z & 0 \\
        0 & 0 & 0 & 1 \\
    \end{array}\right)
\end{displaymath}
\subsubsection{向量变换}
注意平移不会影响向量的变换,因此将矩阵截断为3*3矩阵$T'=mat3(T)$。
\subsubsection{法向量变换}
若法向量$N$变换按照向量变换计算,若遇到缩放则会发生变换后不垂直于变换后平面
的情况。因为缩放矩阵的基向量不是单位向量。考虑一个与该法向量垂直的向量
$V$,满足$N^T\cdot V=0$。那么对于变换后的两向量仍然
要保持垂直,即$N^TT'^{-1} \cdot T'V=0$,对左边进行转置得到$N'=(T'^{-1})^TN$。
\subsection{反射与折射}
以下的入射向量、出射向量与法向量均为单位向量。
\subsubsection{反射}
根据反射定律,反射向量由水平方向的向量$I_x$减去垂直方向的向量$I_y$。
列出方程组:
\begin{eqnarray*}
    I&=&I_x+I_y\\
    I_y&=&N\cdot dot(I,N)\\
    O&=&I_x-I_y\\
\end{eqnarray*}
解得$O=I-2N\cdot dot(I,N)$。
\subsubsection{折射}
斯涅尔定律描述了折射率与角度的关系:
\begin{theorem}
    $\eta_1\sin \theta_1=\eta_2\sin \theta_2$
\end{theorem}
同样以法向量和切向量为基向量进行正交分解,
记$\eta=\frac{\eta_1}{\eta_2}$,有
\begin{eqnarray*}
    I&=&I_x+I_y\\
    I_y&=&N\cdot dot(I,N)\\
    I_x&=&\sin \theta_1T\\
    O&=&O_x+O_y=\sin \theta_2T-\cos \theta_2N
\end{eqnarray*}
化简:
\begin{eqnarray*}
    O&=&\sin \theta_2T-\cos \theta_2N\\
    &=&\frac{\sin \theta_2}{\sin \theta_1}I_x - \cos \theta_2 N\\
    &=&\frac{\eta_1}{\eta_2}(I-dot(I,N)\cdot N)-\cos \theta_2 N\\
    &=&\eta\cdot I-(\eta dot(I,N)+\cos \theta_2)N
\end{eqnarray*}

代码如下,注意全反射的情况(即$\sin \theta_2$超出值域):
\begin{lstlisting}
Vec refract(Vec I,Vec N,FT eta) {
    FT idn=dot(I,N);
    FT cosO2=1.0-eta*eta*(1.0-idn*idn);
    if(cosO2<0.0)return Vec();
    FT k=eta*idn+sqrt(cosO2);
    return I*eta-k*N;
}
\end{lstlisting}
上述内容参考了Milo Yip的文章\footnote{
    用 C 语言画光(五):折射
    \url{https://zhuanlan.zhihu.com/p/31127076}
}和glm库的代码\footnote{
    glm/func\_geometric.inl at master · g-truc/glm · GitHub
    \url{https://github.com/g-truc/glm/blob/master/glm/detail/func\_geometric.inl}
}。
\subsection{pick定理}
\index{P!Pick's Theorem}
\begin{theorem}[Pick's Theorem]
    若格点多边形内的点数为$a$,落在边上的点数为$b$,则
    该多边形的面积为$a-\frac{b}{2}+1$。
\end{theorem}
\subsection{切比雪夫距离}
切比雪夫距离是两点坐标之差的最小值。
分别考虑到原点曼哈顿距离和切比雪夫距离为1的点$P,Q$,
发现将$P$绕原点旋转$45^\circ$再缩放$\sqrt{2}$倍后等于$Q$。

因此$P(x,y)\rightarrow Q(x+y,x-y)$,$Q(x,y)
\rightarrow P(\frac{x+y}{2},\frac{x-y}{2})$。
\subsection{精度处理}
一般引入$eps=1e-8$来避免精度问题。

常见问题:
\begin{itemize}
    \item 判断两个值相等:$fabs(a-b)<eps$。
    \item 要输出$1.00$却输出$0.99$或者要输出$0.0$却输出$-0.0$:
        对正值$+eps$,负值$-eps$。
    \item 已知$\sin \theta$求$\cos \theta$:$sqrt(1.0-cosTheta*cosTheta)$
    可能会因为传入$sqrt$的参数小于0而返回$nan$,在调用数学函数前要clamp到定义域内
    或特判。
    \item 若乘积表示的范围越界,则可以使用对数加表示数乘。
    \item 技巧:若要维护$std::set<FT>$,在比较器中引入$eps$自动完成去重工作。
    \item 多个量级相差巨大的浮点数相加减时要尽量使每次相加的两个数量级差不多。
    比如连加应该要从小到大累加。
    \item 二分时尽量固定迭代次数而不是用eps比较。
    \item 一些判定性问题可以使用模意义下的数代替浮点数解决。
\end{itemize}

为了辅助判断两个值的大小关系,引入一个符号函数$sign$:
\begin{lstlisting}
int sign(FT x) {
    return (x>eps)-(x<-eps);
}
\end{lstlisting}

该内容参考了Ac\_smile的博客\footnote{计算几何中的精度问题\\
    \url{https://www.cnblogs.com/acsmile/archive/2011/05/09/2040918.html}
}与Oyking的博客\footnote{
    IO/ACM中来自浮点数的陷阱(收集向)
    \url{https://www.cnblogs.com/oyking/p/3959905.html}
}。

\section{最小二乘逼近}
\index{L!Least Squares}
该方法用来确定基函数的系数以拟合曲线。

设有$m$个数据点$(x_1,y_1),(x_2,y_2),\cdots,(x_m,y_m)$,
$n$个基函数$f_1,f_2,\cdots,f_n$
记向量$y=(y_1,y_2.\cdots,y_m)$,基函数值矩阵$A$
为\begin{displaymath}
    A=\left[
    \begin{array}{cccc}
    f_1(x_1)&f_2(x_1)&\cdots&f_n(x_1)\\
    f_1(x_2)&f_2(x_2)&\cdots&f_n(x_2)\\
    \vdots&\vdots&\ddots&\vdots\\
    f_1(x_m)&f_2(x_m)&\cdots&f_n(x_m)
    \end{array}
    \right]
\end{displaymath}
设系数向量为$c$,则有误差向量$\eta=Ac-y$。

使用最小二乘的思想,令$\|\eta\|^2$最小,
即最小化
\begin{displaymath}
    \|Ac-y\|=\sum_{i=1}^m{\left(\sum_{j=1}^n{a_{ij}c_j}-y_i\right)}^2
\end{displaymath}

对向量$c$中的每个元素微分并让结果为0,
即对于单个元素$c_k$,有
\begin{displaymath}
    \frac{\ud \|\eta\|^2}{\ud c_k}=\sum_{i=1}^m{2\left(
        \sum_{j=1}^n{a_{ij}c_j-y_i}\right)a_{ik}}=0
\end{displaymath}
将$n$个等式结合在一起,发现这是个向量*矩阵的形式,可得新的矩阵方程$(Ac-y)^TA=0$,
记$A^+=(A^TA)^{-1}A^T$为矩阵$A$的伪逆,有$c=A^+y$。

以上内容参考了算法导论\cite{ITA3} 28.3节。

\section{常系数齐次线性递推}
\index{L!Linear Difference Equation}
给定系数$a_0,a_2,\cdots,a_k$,有一个信号${f_n}$,满足$k$阶齐次线性差分方程
$\displaystyle \sum_{i=0}^k{a_if_{n-i}}=0$对所有$n$成立。现在给定信号
${f_n}$中的连续$k$项,求信号的任意一项。

一般$a_0$取1,其余系数取反,那么有$\displaystyle f_n=\sum_{i=1}^k{a_if_{n-i}}$,
假设给定了$f_0,f_1,\cdots,f_{k-1}$的值,现在要求出$f_n$的值。

如果$k$足够小,那么很容易构造转移矩阵,记向量$F_i$表示$(f_{i+k-1},f_{i+k-2},\cdots,f_i)$,
很容易构造$k*k$的转移矩阵$A$,其中$i$向$i+1$转移系数为1,$i$向$1$转移系数为$a_i$。即
$A[i+1][i]=1,A[1][i]=a_i$,那么有$F_n=A^nF_0$。使用矩阵快速幂可得到
$O(k^3\lg n)$的算法。

注意原等式左右向量取第$k$项仍然成立,即$(F_n)_{[k]}=(A^nF_0)_{[k]}$。左边就是$f_n$,
右边就是以$A^n$的第$k$行为系数的$f_{0,\cdots,k-1}$的线性组合。记这些系数为
$c_{0,\cdots,k-1}$,同样将$f_{0,\cdots,k-1}$表示为$A^nF_0$的形式,有
$A^n=\displaystyle \sum_{i=0}^{k-1}{c_iA^i}$,记右式为矩阵$A$的多项式表达$R(M)$。
设存在矩阵多项式$F(M),G(M)$,满足$G(M)$的次数为$k$,$F(M)G(M)$的次数为$n$,
且$M^n=F(M)G(M)+R(M)$。根据Cayley-Hamilton定理,若$G(M)$为矩阵$A$的特征多项式,
其次数恰好为$k$且$G(A)=0$。由于$G(M)$的次数为$k$,$R(M)\equiv M^n\pmod{G(M)}$,
多项式取模可求出$R(M)$。并且由于$G(A)=0$,此时有$A^n=R(A)$。拿到$R(M)$后就可以直接$O(k)$
求值。

接下来讨论如何构造出矩阵$A$的特征多项式$G(M)$。根据定义有
$G(\lambda)=\textrm{det}(\lambda I_k-A)$,多项式高斯消元求行列式十分麻烦。
注意到矩阵$\lambda I_k-A$的特殊性,在第一行展开求和,去除第一行第$i$
列后都会得到一个下三角矩阵,行列式值为对角线上元素之积,同时代数余子式的$(-1)^{i+1}$项
与子矩阵的$-1$项恰好抵消。综上所述,$G(\lambda)=\lambda^k-a_1\lambda^{k-1}-\cdots-a_k$。

因此该方法的性能瓶颈在多项式取模上,时间复杂度$O(k\lg k\lg n)$。引入$\lg n$的原因是
我们无法直接构造一个多项式$x^n$然后以$n$的规模做取模,由于该多项式较为简单,可以使用类似
模意义快速幂的方法做模$R(M)$意义下的多项式快速幂。当$k$较小时,可以考虑暴力取模,时间
复杂度$O(k^2\lg n)$。

好写的优化技巧:
\begin{itemize}
    \item 预处理出$G(x)$与$G_{rev}^{-1}(x)$的点值表达。
    \item 需要取模时才实例化取模。
\end{itemize}

参考代码:
\lstinputlisting{Source/Templates/LR.cpp}

上述内容参考了《线性代数及其应用》\cite{LAIA5}4.8节以及shadowice1984的博客
\footnote{
    题解 P4723 【【模板】线性递推】
    \url{https://www.luogu.org/blog/ShadowassIIXVIIIIV/solution-p4723}
}。


\chapter{数学杂项}
\section{泰勒级数展开}
\index{T!Taylor Series}
\subsection{形式}
函数$f$在点$a$上的泰勒级数展开为
\begin{displaymath}
    \sum_{n=0}^\infty{\frac{f^{(n)}(a)}{n!}(x-a)^n}
\end{displaymath}
\index{M!Maclaurin Series}
$a=0$时称为麦克劳林级数。
\subsection{常见泰勒级数展开}
通常使用麦克劳林级数推导。
\begin{eqnarray*}
    e^x&=&\sum_{n=0}^\infty{\frac{x^n}{n!}}\\
    \ln(1-x)&=&-\sum_{n=1}^\infty{\frac{x^n}{n}} \textrm{~for all $|x|<1$}\\
    \frac{1}{1-x}&=&\sum_{n=0}^\infty{x^n}\\
    \frac{1}{(1-x)^k}&=&\sum_{n=0}^\infty{(x^n)^{(k)}}\\
    (1+x)^\alpha&=&\sum_{n=0}^\infty{{\alpha \choose n}x^n}\\
    \sin(x)&=&\sum_{n=0}^\infty{\frac{(-1)^n}{(2n+1)!}x^{2n+1}}\\
    \cos(x)&=&\sum_{n=0}^\infty{\frac{(-1)^n}{(2n)!}x^{2n}}\\
\end{eqnarray*}

\section{Simpson积分}
\subsection{形式与推导}
Simpson积分是用二次函数来拟合等距三点$(l,f(l)),(m,f(m)),(r,f(r))\\
(m=\frac{l+r}{2})$来积分的。推导如下:
\begin{eqnarray*}
    \int_l^r{f(x)\ud x}&\approx&\int_l^r{(ax^2+bx+c)\ud x}\\
    &=&\frac{a}{3}(r^3-l^3)+\frac{b}{2}(r^2-l^2)+c(r-l)\\
    &=&\frac{r-l}{6}(2a(r^2+rl+l^2)+3b(r+l)+6c)\\
    &=&\frac{r-l}{6}(f(l)+4f(m)+f(r))
\end{eqnarray*}

\subsection{自适应Simpson}
当Simpson未能精确拟合函数值时,将其区间二分拟合。判断拟合程度
可以将当前积分与子区间积分之和相比较。

\begin{lstlisting}
typedef double FT;
const FT eps=1e-8;
FT f(FT x);
FT simpson(FT l,FT r,FT fl,FT fm,FT fr) {
    return (r-l)*(fl+4.0*fm+fr)/6.0;
}
FT SAAImpl(FT l,FT m,FT r,FT fl,FT fm,FT fr,FT sm) {
    FT lm=(l+m)*0.5,flm=f(lm),sl=simpson(l,m,fl,flm,fm);
    FT rm=(m+r)*0.5,frm=f(rm),sr=simpson(m,r,fm,frm,fr);
    FT esm=sl+sr;
    if(fabs(sm-esm)<eps)return esm;
    return SAAImpl(l,lm,m,fl,flm,fm,sl)+
    SAAImpl(m,rm,r,fm,frm,fr,sr);
}
FT SAA(FT l,FT r) {
    FT m=(l+r)*0.5;
    FT fl=f(l),fm=f(m),fr=f(r);
    return SAAImpl(l,m,r,fl,fm,fr,simpson(l,r,fl,fm,fr));
}
\end{lstlisting}

\paragraph{警告}
\begin{itemize}
    \item {\bfseries 最好不要在圆上做Simpson积分。}
    \item {\bfseries 注意初始定义域端点的位置(比如函数只在$[0,50]$有较大贡献,
其余区域贡献几乎为0,若对$[0,10000]$Simpson积分则误差较大,考虑分析函数图像后
倍增区间大小)。}
\end{itemize}

\section{概率与期望}
\subsection{全概率公式}
\begin{theorem}\label{FP}
    若事件$A$可分为独立事件$A_1,A_2,\cdots,A_n$,则
    \begin{displaymath}
        P(B)=\sum_{i=1}^n{P(B|A_i)P(A_i)}
    \end{displaymath}
\end{theorem}
\subsection{贝叶斯定理}
由\begin{displaymath}
    P(A\cap B)=P(B)P(A|B)=P(A)P(B|A)
\end{displaymath}
可得贝叶斯定理
\begin{theorem}
    \begin{displaymath}
        P(A|B)=\frac{P(A)P(B|A)}{P(B)}
    \end{displaymath}
\end{theorem}
结合定理~\ref{FP}得
\begin{inference}
    \begin{displaymath}
        P(A_i|B)=\frac{P(A_i)P(B|A_i)}{\displaystyle \sum_{j=1}^n{P(B|A_j)P(A_j)}}
    \end{displaymath}
\end{inference}
\subsection{期望}
\begin{theorem}[期望的线性性质]
    $E[X+Y]=E[X]+E[Y]$
\end{theorem}
\begin{theorem}
    $E[aX]=aE[X]$
\end{theorem}
\begin{theorem}
    若随机变量X,Y独立且期望$E[XY]$有定义时,$E[XY]=E[X]E[Y]$。
\end{theorem}
有一个十分常用的优化:
\begin{displaymath}
    E[X]=\sum_{i=0}^n{i\cdot P(X=i)}
    =\sum_{i=0}^n{i(P(X\geq i)-P(X\geq i+1))}
    =\sum_{i=1}^n{P(X\geq i)}
\end{displaymath}
\index{J!Jensen's Inequality}
\begin{theorem}[Jensen's Inequality]
    若函数$f(x)$为凸函数(即对于任意$x,y$和$\lambda\in [0,1]$,有
    $f(\lambda x+(1-\lambda)y)\geq\lambda f(x)+(1-\lambda)f(y)$)
    ,则$E[f(X)]\geq f(E[X])$。
\end{theorem}
\begin{theorem}
    对于在$[0,1]$上均匀分布的随机变量$X$,$n$个随机变量的第$k$小值
    的期望为$\frac{k}{n+1}$。
\end{theorem}
证明:
随机变量$X$的概率分布函数cdf(Cumulative Distribution Function
\index{C!Cumulative Distribution\\ Function})
为$\displaystyle cdf(x)=P(X\leq x)=\int_0^1{pdf(x)\ud x}$。
对其求导得到概率密度函数pdf(Probability Density Function)
\index{P!Probability Density Function}:
\begin{displaymath}
    pdf(x)=P(X=x)=k\binomial{n}{k}x^{k-1}(1-x)^{n-k}
\end{displaymath}
乘上随机变量后积分即为期望:
\begin{displaymath}
    \int_0^1{xpdf(x)\ud x}=\frac{k}{n+1}
\end{displaymath}
对于求解其它连续区间的期望问题也是这个思路。
\subsubsection{随机变量的方差}
$Var[X]$表示随机变量$X$的方差。
\begin{theorem}
    若随机变量$X$的均值为$E[X]$,则有$E[X^2]=Var[X]+E^2[X]$。
\end{theorem}
\begin{theorem}
    $Var[aX]=a^2Var[X]$
\end{theorem}
\begin{theorem}
    若随机变量$X_1,X_2,\cdots,X_n$两两独立,则有
    \begin{displaymath}
        Var[\sum_{i=1}^n{X_i}]=\sum_{i=1}^n{Var[X_i]}
    \end{displaymath}
\end{theorem}
\subsubsection{高斯消元求期望}
对于一般的图,可以列出每个点期望之间的线性关系,构造出
方程组后高斯消元求解。注意要判断矩阵是否为稀疏矩阵,若为稀疏矩阵,继续研究其特殊性质,
一般可以不断地递推求解一元一次方程得到所有解。

求{\bfseries 树}上期望时,一般的套路是将方程表达为$dp[u]=k\cdot dp[p]+b$
的形式,自底向上把方程推到根后再从根往下推数值解。
\subsection{伯努利试验}\label{Bernoulli}
每次伯努利试验有2种结果:以$p$的概率成功,或者以$1-p$的概率失败。
\subsubsection{几何分布}
不断进行伯努利试验,第$k$次成功的概率为$q^{k-1}p$。
试验次数的期望为
\begin{eqnarray*}
    E[X]&=&\sum_{k=1}^\infty {kq^{k-1}p}\\
    &=&\frac{p}{q} \sum_{k=1}^\infty{kq^{k-1}}\\
    &=&\frac{p}{q} \cdot \frac{q}{(1-q)^2} \textrm{参见~\ref{GF}}\\
    &=&\frac{1}{p}
\end{eqnarray*}
\subsubsection{二项分布}
进行$n$次伯努利试验,成功$k$次的概率为$\binomial{n}{k}p^kq^{n-k}$。
根据期望的线性性质,$n$次伯努利试验的期望成功次数为
$E[X]=E[\displaystyle \sum_{i=1}^n{X_i}]=\sum_{i=1}^np=np$。

以上内容参考了算法导论\cite{ITA3}附录C.4。

\section{生成函数}

\section{拉格朗日插值}
\subsection{原理}
拉格朗日插值法的思路是将每个点对应一个基函数,
使得点$(x_i,y_i)$对应的基函数$F_i(x)$满足
\begin{displaymath}
F_i(x)=\left\{\begin{array}{ll}
    y_i& \textrm{~if~$x=x_i$}\\
    0&\textrm{otherwise}
\end{array}\right.
\end{displaymath}
如何构造$F_i(x)$呢?首先可以提出$y_i$令$F_i(x)=y_iG_i(x)$,
然后显然$G_i(x)$有$\displaystyle \prod_{j\neq i}{x-x_j}$项,
为了让$G_i(x_i)=1$,再除以$\displaystyle \prod_{j\neq i}{x_i-x_j}$。

综上所述,插值函数为
\begin{displaymath}
    F(x)=\sum_{i=1}^n{y_i\prod_{j\neq i}{\frac{x-x_j}{x_i-x_j}}}
\end{displaymath}
\subsection{插值多项式计算}
时间复杂度$\Theta(n^2)$,步骤如下:
\begin{enumerate}
    \item $O(n^2)$计算出多项式$\displaystyle F(x)=\prod_{i=1}^i{x-x_i}$;
    \item $\Theta(n^2)$计算每个$i$的多项式$\frac{F(x)}{x-x_i}$;
    \item $\Theta(n^2)$将每个多项式乘以系数
    $\frac{y_i}{\displaystyle \prod_{i\neq j}{x_i-x_j}}$
    (分母部分直接将$x_i$带入$i$对应的分子的多项式求出);
    \item $\Theta(n^2)$将多项式相加合并。
\end{enumerate}
单点插值也可以使用这种优化。
\subsection{多点插值}
\begin{itemize}
    \item 预处理出多项式,然后每次使用霍纳法则$O(n)$求值;
    \item 预处理出每个$i$的$\displaystyle y_i \prod_{i\neq j}\frac{1}{x_i-x_j}$,
    每次$O(n^2)$求分子后相加。
\end{itemize}
\subsection{缺点}
\begin{itemize}
    \item 拉格朗日插值法的给出的曲线的确经过了样本点,但是这个曲线有可能十分曲折,而且
    受单个点的影响大。可以考虑使用最小二乘逼近来消除噪声,或者使用样条。不过拉格朗日插值
    一般用于已确定多项式次数的插值,比如自然数幂和。
    \item $\Theta(n^2)$拉格朗日插值法的时间复杂度并没有比FFT插值方法优。
\end{itemize}
\subsection{求解自然数幂和}\label{psum}
易知$F_k(n)=\displaystyle \sum_{i=0}^n{i^k}$是一个$k+1$次多项式。
预处理出$F_k$在$1,\cdots,k+2$上的值,插值出多项式。

在预处理时直接处理出多项式而不是点值表示,插值的复杂度要$O(n)$而不是$O(n^2)$。

代码模板:
\begin{lstlisting}
struct Poly {
    Int64 fac[maxk];
    int end;
    void init(int k) {
        int val[maxk];
        end = k + 1;
        for(int i = 0; i <= end; ++i) {
            val[i] = 1;
            for(int j = 0; j < k; ++j)
                val[i] = asInt64(val[i]) * i % mod;
        }
        for(int i = 1; i <= end; ++i)
            val[i] = (val[i - 1] + val[i]) % mod;
        Int64 kt[maxk];
        for(int i = 0; i <= end; ++i)
            fac[i] = 0;
        for(int i = 0; i <= end; ++i) {
            kt[0] = val[i];
            for(int j = 1; j <= end; ++j)
                kt[j] = 0;
            for(int j = 0; j <= end; ++j)
                if(i != j) {
                    for(int k = end; k >= 1; --k)
                        kt[k] =
                            (kt[k - 1] - j * kt[k]) %
                            mod;
                    kt[0] = -j * kt[0] % mod;
                    Int64 div = powm(i - j, mod - 2);
                    for(int k = 0; k <= end; ++k)
                        kt[k] = kt[k] * div % mod;
                }
            for(int j = 0; j <= end; ++j)
                fac[j] = (fac[j] + kt[j]) % mod;
        }
    }
    Int64 operator()(Int64 x) const {
        x%=mod;//!!!
        Int64 res = 0;
        for(int i = end; i >= 0; --i)
            res = (res * x + fac[i]) % mod;
        return res;
    }
}
\end{lstlisting}

{\bfseries 血泪史:写Project~Euler~639时,由于插值处没有对$x$预先取模,导致在保持系数在$(-mod,mod)$
与$[0,mod)$的答案不一致,且它们的答案都是错误的,我因此调试了一天。}

{\bfseries Update:注意到由于插值点是连续的,
$\displaystyle \prod_{i\neq j}{i-j}$实际上可以表示为阶乘之积,可以$O(n)$预处理,
因此预处理复杂度也可以达到$O(n)$。}

\section{反演}
\subsection{反演定义}
若数列${f_n}$与${g_n}$满足
\begin{displaymath}
	f_n=\sum_{i=0}^n{a_{ni}g_i}
\end{displaymath}
反演就是已知数列${f_n}$(函数较简单),
求数列${g_n}$的过程(关键是要推出$b_{ni}$):
\begin{displaymath}
	g_n=\sum_{i=0}^n{b_{ni}f_i}
\end{displaymath}
这其实是一个求解线性方程组的过程。
\subsection{二项式反演}\label{BI}
\begin{theorem}
	\begin{displaymath}
		f(n)=\sum_{i=0}^n{(-1)^i\binomial{n}{i}g(i)}
		\Leftrightarrow g(n)=\sum_{i=0}^n{(-1)^i\binomial{n}{i}f(i)}
	\end{displaymath}
\end{theorem}
使用容斥证明:

设集合$S$中拥有性质$P_1,P_2,\cdots,P_n$的集合分别为$A_1,A_2,\cdots,A_n$,
根据定理~\ref{ExDML}可得不具有这$n$个性质的对象的集合大小为
\begin{displaymath}
	f(n)=|S|+\sum_{\emptyset \neq J\subseteq{1,2,\cdots,n}}
	{(-1)^{|J|}\left|\bigcap_{j\in J}{A_j}\right|}
\end{displaymath}
若集合$A_1,A_2,\cdots,A_n$满足任意$i$个集合的并集大小相等,记为$g(i)$,
定义$g(0)=|S|$,有
\begin{displaymath}
	f(n)=\sum_{i=0}^n{(-1)^i \binomial{n}{i}g(i)}
\end{displaymath}
同样对$g(i)$使用容斥可以得到右式。
\begin{inference}\label{BII}
	\begin{displaymath}
		f(n)=\sum_{i=0}^n{\binomial{n}{i}g(i)}
		\Leftrightarrow g(n)=\sum_{i=0}^n{(-1)^{n-i}\binomial{n}{i}f(i)}
	\end{displaymath}
\end{inference}
把$g(i)$代入后把外面的$(-1)^i$丢进去可证。
\subsubsection{错位排序问题}
求$n$个人均站错位置的方案数。

记$f(n)$为$n$个人任意站的方案数,$g(n)$为$n$个人都站错的方案数。

显然$f(n)=n!$且$\displaystyle f(n)=\sum_{i=0}^n{\binomial{n}{i} g(i)}$,
由推论~\ref{BII}得
\begin{eqnarray*}
	g(n)&=&\sum_{i=0}^n{(-1)^{n-i}\binomial{n}{i}i!}\\
	&=&\sum_{i=0}^n{(-1)^{n-i}\frac{n!}{(n-i)!}}\\
	&=&n!\cdot\sum_{i=0}^n{\frac{(-1)^i}{i!}}
\end{eqnarray*}
\subsubsection{球染色问题}
求用$k$种颜色给$n$个排成一排的球染色,满足相邻球不同色且必须用上所有颜色的方案数。

记$f(k)$为使用$k$种颜色,相邻球不同色,不要求用上所有颜色的染色方案数,
$g(k)$为使用$k$种颜色的方案数。

那么有$f(k)=k(k-1)^{n-1}$且$\displaystyle f(k)=\sum_{i=0}^k{\binomial{k}{i}g(i)}$。

同理可得$\displaystyle g(k)=\sum_{i=2}^k{(-1)^{k-i}\binomial{k}{i}k(k-1)^{n-1}}$。

\subsection{斯特林反演}
\begin{theorem}
    \begin{displaymath}
        f(n)=\sum_{i=1}^n{\stirlingB{n}{i}g(i)}
        \Leftrightarrow
        g(n)=\sum_{i=1}^n{(-1)^{n-i}\stirlingA{n}{i}f(i)}
    \end{displaymath}
\end{theorem}
\subsection{子集反演}
\begin{theorem}
	\begin{displaymath}
		f(S)=\sum_{T\subseteq S}{(-1)^{|T|}g(T)}
		\Leftrightarrow
		g(S)=\sum_{T\subseteq S}{(-1)^{|T|}f(T)}
	\end{displaymath}
\end{theorem}
\begin{inference}
	\begin{displaymath}
		f(S)=\sum_{T\subseteq S}{g(T)}
		\Leftrightarrow
		g(S)=\sum_{T\subseteq S}{(-1)^{|S|-|T|}f(T)}
	\end{displaymath}
\end{inference}
证明同~\ref{BI}节所述。
\subsection{多重子集反演}
一个数的质因数分解可以对应一个多重子集,考虑莫比乌斯反演。

以上内容参考了vfleaking的幻灯片
\footnote{炫酷反演魔术 \url{http://vfleaking.blog.uoj.ac/slide/87}}
和Miskcoo的博客\footnote{反演魔术:反演原理及二项式反演 – Miskcoo's Space.
	\url{http://blog.miskcoo.com/2015/12/inversion-magic-binomial-inversion}
}。
\subsection{最值反演(minmax容斥)}
\subsubsection{一般形式}
记集合为$S$,$max(S)$为集合$S$的最大元素,$min(S)$为集合$S$的最小元素。
实际上这两个函数可推广为关于最值的函数。

最值反演的公式如下:
\begin{eqnarray*}
	max(S)&=&\sum_{T\subseteq S}{(-1)^{|T|+1}min(T)}\\
	min(S)&=&\sum_{T\subseteq S}{(-1)^{|T|+1}max(T)}
\end{eqnarray*}

证明(以第一个式子为例):记$x=max(S)$(如果集合可重,就给相同元素编号使其不相等,
下面仅考虑不可重集合的情况)。当且仅当$T=\{x\}$时贡献才含有$x$项。对于$T\neq \{x\}$,
设$y=min(T)$,设$S$中$\geq y$的元素有$k$个,有$k>1$,由于在这$k$个中选择奇数个与
选择偶数个的方案数相同(使用二项式定理可证明),最终$y$的贡献会被抵消。

\subsubsection{期望形式}
由期望的线性性质可以推出在$max(S/T)$与$min(S/T)$外再套上一层
期望后等式仍然成立。

一般的套路是每位都有一定的概率从0变为1,求全部变为1的期望步数。
最值反演后转换为至少1位变为1的期望步数。

\paragraph{例题~「HAOI2015」按位或}
记$max(S)$为状态S中最晚出现的1的出现的期望时间,$min(S)$为状态S中最早出现
的1的出现的期望时间。$min(S)$很容易求得其表达式,考虑与$S$的交不为空集的$T$,表达式为
$min(S)=\frac{1}{\displaystyle \sum_{T\cap S \neq \emptyset}{P_T}}$(根据
伯努利试验中几何分布的期望求得,参见第~\ref{Bernoulli}节)。
这个式子仍然不好求,可以考虑补集转换,即考虑$T$与$S$的交为空集的情况。这个条件蕴含了
$T\subseteq C_US$,使用第~\ref{FMT}节的FMT可快速求出。

代码:
\lstinputlisting{Source/Source/MinMax/LOJ2127.cpp}

\subsubsection{推广}
$S$的第$k$大$max_k(S)$满足如下等式:

\begin{displaymath}
	max_k(S)=\sum_{T\subseteq S}{(-1)^{|T|+k}\binomial{|T|-1}{k-1}min(T)}
\end{displaymath}

证明留坑待补。
\index{*TODO!最值反演推广证明}

上述内容参考了Mr\_Spade的博客\footnote{
	min-max容斥/最值反演及其推广
	\url{http://www.cnblogs.com/Mr-Spade/p/9636968.html}
}。
\subsection{单位根反演}
单位根反演用来解决已知生成函数$F(x)=\displaystyle \sum_{i=0}^n{f(i)x^i}$,
求$\displaystyle \sum_{i=0}^n{f(i)[k|(i+b)]}$,$k,b$为定值。

考虑$b$为0的情况,有下列定理:
\begin{theorem}
	\begin{displaymath}
		[k|i]=\frac{1}{k}\sum_{j=0}^{k-1}{\omega_k^{ij}}
	\end{displaymath}
\end{theorem}

证明:与IDFT的证明类似,若$[k\nmid i]$,则根据求和引理~\ref{FFTL4},该式的值为0。
否则$\omega_k^{ij}$恒为1,该式的值为1。

该定理在模意义下仍然成立。\sout{一般来说模意义下的给定$k$都是2的幂,因为给定的模数的欧拉
函数值要求2的指数很大。}

接下来考虑将要求值的式子展开,有$\displaystyle \sum_{i=0}^n{f(i)\cdot \frac{1}{k}
\sum_{j=0}^{k-1}{\omega_k^{ij}}}$,交换求值顺序化简为$\frac{1}{k}\displaystyle
\sum_{i=0}^{k-1}{F(\omega_k^i)}$。需要用到单位根反演的场合肯定无法快速计算$F$的每一项
系数,但是容易快速求出$F(x)$,目前遇到的题目基本上结合二项式展开快速求出$F(x)$的值。对于
$b\neq 0$的情况,可以将$F(x)$乘以$x^{k-b}$平移。

上述内容参考了czyhe的博客\footnote{
	单位根反演
\url{https://czyhe.me/algorithm/unit-root-inv/unit-root-inv/}}。
\subsection{拉格朗日反演}
\index{L!Lagrange inversion Theorem}
如果要求某个函数的反函数的某一项,可以使用拉格朗日反演:
\begin{theorem}
	若多项式$F(x),G(x)$都没有常数项且一次项系数互为逆元,满足$F(G(x))=x$,
	则\begin{displaymath}
		[x_n]F(x)=\frac{1}{n}[x^{n-1}](\frac{x}{F(x)})^n
	\end{displaymath}
\end{theorem}
\subsubsection{扩展拉格朗日反演}
\begin{theorem}
	\begin{displaymath}
		[x_n]H(F(x))=\frac{1}{n}[x^{n-1}]H'(x)(\frac{x}{F(x)})^n
	\end{displaymath}
\end{theorem}
证明留坑待补。

这两个定理一般用在生成函数解决图的计数问题中。

上述内容参考了ZJT的博客\footnote{
	拉格朗日反演
	\url{http://zjt-blog.cc/articles/63}
}。

\section{常见数学公式与应用}
\subsection{常见公式}
\begin{eqnarray*}
	\textrm{平方和}&~\sum_{k=1}^n{k^2}&=\frac{n(n+1)(2n+1)}{6}\\
	\textrm{立方和}&~\sum_{k=1}^n{k^3}&=\frac{n^2(n+1)^2}{4}\\
	\textrm{无穷递减几何级数}&~\forall |x|<1,\sum_{k=0}^\infty{x^k}
	&=\frac{1}{1-x}\\
	\textrm{调和级数}&~H_n=\sum_{k=1}^n{\frac{1}{k}}&=\ln n+\gamma+\varepsilon_n\\
	\textrm{Leibniz formula for π}&~\sum_{k=0}^\infty{\frac{(-1)^k}{2k+1}}
	&=\arctan 1=\frac{\pi}{4}\\
	\textrm{Wallis' product}&~\prod_{k=1}^\infty{\frac{2k}{2k-1}
		\cdot\frac{2k}{2k+1}}&=\frac{\pi}{2}\\
	\textrm{Basel problem}&~\zeta(2)=\sum_{k=1}^\infty{\frac{1}{k^2}}
	&=\frac{\pi^2}{6}\\
	\textrm{素数倒数和}&~\sum_{p\in P,p\leq n}{\frac{1}{p}}&=
	\ln \ln x+M+O(\frac{1}{\ln x}),
	M\approx 0.26149721\index{M!Meissel–Mertens Constant}
\end{eqnarray*}
\subsection{调和级数应用}
$\varepsilon_n\approx\frac{1}{2n}$,
欧拉-马歇罗尼常数$\gamma\approx 0.5772156649\ldots$。
\index{*Constant!$\gamma\approx 0.5772156649\ldots$}
\subsubsection{$O(n\ln n)$dp}
\begin{itemize}
	\item 两层循环分别枚举区间长度和左端点。
	\item Eratosthenes筛法:枚举因子和因子的倍数。
\end{itemize}
\subsubsection{书籍堆叠问题}
有$n$本长度为$l$,质量分布均匀的书,将书叠成一摞,放在桌边,求书最多能伸出多长。

首先书籍从下到上肯定是不断伸出的,从上往下编号,设第$i-1$本书比第$i$本书多伸长$x_i$,
记最上面的书为第$0$本书,桌子为第$n$本书。

若要使书不倒下,需满足上面的书的重心在当前书上,
即
\begin{displaymath}
	\frac{\sum_{i=1}^k{\sum_{j=i}^k{x_i}}}{k}=\frac{\sum_{i=1}^k{ix_i}}{k}\leq \frac{l}{2},k=1,2,\cdots,n
\end{displaymath}

当$x_i=\frac{\frac{l}{2}}{i}$时等号恒成立,此时
伸长量$\displaystyle L=\sum_{i=1}^n{x_i}=\frac{l}{2}H_n$。

注意在计算调和级数时,小规模用暴力,大规模用$O(1)$近似。

如果允许在一层上放多本书,则最大伸出量与$\sqrt[3]{n}$成正比。

该问题源自Wikipedia-EN\footnote{Block-stacking problem - Wikipedia
	\url{https://en.wikipedia.org/wiki/Block-stacking\_problem}}。
\subsubsection{吉普车问题}

给定$n$个单位的燃料,吉普车只能携带$1$单位的燃料,$1$单位燃料可行驶$1$单位距离,
吉普车可以在沙漠的任意地方留下燃料,最大化最后一次的行驶距离。

该问题有两种类型:
\begin{itemize}
    \item 探索沙漠:最后一次要返回基地,答案为$H_n$。
    \item 穿越沙漠:最后一次不返回基地,答案为$\displaystyle
    \sum_{k=1}^n{\frac{1}{2k-1}}=H_{2n-1}-\frac{1}{2}H_{n-1}$。
\end{itemize}

该问题源自Wikipedia-EN\footnote{Jeep problem - Wikipedia
	\url{https://en.wikipedia.org/wiki/Jeep\_problem}}。
\subsubsection{蚂蚁在橡胶绳上}

一只蚂蚁在$1km$长的橡胶绳上以$1cm/s$的速度爬行,同时绳子以$1km/s$的速度拉伸,
蚂蚁是否可以到达绳子的另一端?

答案为是,可用离散方法或积分方法解决,时间$T=\frac{c}{v}(e^{\frac{v}{\alpha}}-1)$。
$c$为初始绳长,$v$为伸长速度,$\alpha$为蚂蚁运动速度。

该问题源自Wikipedia-EN\footnote{Ant on a rubber rope - Wikipedia
	\url{https://en.wikipedia.org/wiki/Ant\_on\_a\_rubber\_rope}}。

以上内容参考了Wikipedia-EN\footnote{Harmonic series (mathematics) - Wikipedia
	\url{https://en.wikipedia.org/wiki/Harmonic\_series\_(mathematics)}}。

\section{线性时间复杂度插值}


\chapter{动态规划}
\section{背包优化}
\subsection{完全背包优化}
\subsubsection{排序筛选}
若一种物品比另一种物品代价更大,收益更低,直接排除。
\subsection{多重背包优化}
\subsubsection{完全背包转换}
若该物品的数量已经超过最大需求,直接转换为完全背包,单种物品$O(V)$转移。
\subsubsection{二进制优化}
将数量按$2^0,2^1,2^2,\cdots,2^k,rem$拆分为多个物品,然后做01背包,
单种物品$O(V\lg c)$转移。
\subsubsection{单调队列优化}
朴素的多重背包状态转移方程为:
\begin{eqnarray*}
    end&=&min(c[i],j/v[i])\\
    dp[i][j]=max(dp[i-1][j-k*v[i]]+k*w[i]),0\leq k \leq end
\end{eqnarray*}
令$a=j/v[i],b=j\%v[i]$,将方程转换为:
\begin{displaymath}
    dp[i][j]=max(dp[i-1][b+k*v[i]]-k*w[i])+a*w[i],a-end\leq k \leq a
\end{displaymath}
此时$k$表示比$a$少取$k$件。

此时$max$部分只与$k$的取值有关,使用单调队列优化。
每次转移时,枚举$b$,对$k$做单调队列,根据转移区间的移动弹出队列。
单种物品$O(V)$转移。

该方法参考了soloier的博客\footnote{单调队列优化多重背包\\
\url{https://blog.csdn.net/sinat\_34943123/article/details/52857327}}。

\subsubsection{例题}
Luogu P1776 宝物筛选\_NOI导刊2010提高(02)
\footnote{【P1776】宝物筛选\_NOI导刊2010提高(02) - 洛谷
\url{https://www.luogu.org/problemnew/show/P1776}}

使用了上述的完全背包转换和单调队列优化。

\lstinputlisting[title=Luogu 1776]{DP/MultiBag.cpp}

\section{数位动规}
问题一般是询问区间内数位满足指定要求的数的个数。
区间计数可转换为前缀和差分,因此原可问题转换为询问$n$
以内的满足指定要求的数的个数。

一般思路如下:
\begin{enumerate}
	\item 将$n$拆位为$n_k,n_{k-1},\cdots,n_1$;
	\item 从最低位开始dp一直做到最高位,第一维一般是首位数字;
	\item 从最低位开始统计到次高位,因为这部分答案是满的;
	\item 从最高位开始做到最低位,假设做到第$i$位,表示处理该位以前
	      的位都固定,枚举当前位$j<n_i$,使用预处理的dp值统计入答案。
\end{enumerate}

\paragraph{优化} 若固定的位不满足要求(比如对相邻位有要求)则直接返回
当前统计的答案(因为后续dp的数字都是不合法的,可以忽略)。

\section{基于连通性状态压缩的动态规划/轮廓线DP/插头DP}
\subsection{简单回路问题}
例题:ural 1519

给定一个棋盘,某些格子不能经过,其余格子必须经过,求有多少条简单回路。
\subsubsection{最小表示法}
最小表示法用于表示行内格子的连通性。有两种方法:
\begin{itemize}
    \item 有障碍的格子标记为0,连通的格子标记为同一个数,并且$i$比$i+1$更早出现。
    预处理连通状态时使用DFS计算,同构于集合划分问题,状态总量为贝尔数。
    \item 有障碍的格子标记为0,其余格子标记为与其连通的最左格子的列号。
\end{itemize}

下文均使用第一种方法。存储时将其编码为$k$进制数,$k\geq$连通块个数+1。为了提高运算效率,
将$k$对齐至2的幂。
\subsubsection{状态的表示}
在例题中,每个格子与4个相邻格子都有可能连通,连接的部分称为``插头''。由于转移是
逐行进行的,下一行的插头受到上一行下插头的影响,因此每行需要用二进制表示当前行对应位置
是否有下插头。由于最后需要围成一个回路,还要维护行内格子之间的连通性。因此使用$F(i,j,k)$
表示前$i$行,下插头状态为$j$,连通性为$k$的方案数。注意到若该格子没有下插头,则它的连通性
不影响下一行的插头。那么干脆直接记录下插头的连通性,若不存在下插头则标记为0。

鉴于逐格递推比逐行递推更有优势,这里仅记录逐格递推。

用状态$F(i,j,k)$表示当前处理到第$i$行,已经处理完前$i-1$行和第$i$行前$j$列的格子,插头
连通性为$k$的方案数。已决策的格子与未决策格子之间的分界线称为``轮廓线''。在转移的过程中,
除了$n$个下插头外,第$i$格到第$i+1$格还有一个右插头。$k$从左到右表示$n+1$个插头的连通性。

考虑$k$需要使用的进制,由于一个回路最多有$m/2$个连通块,在例题中至少要用7进制,使用8进制
更加快速。
\subsubsection{状态的转移}
\paragraph{一般转移}
逐格递推,可以发现每移动一格最多有2个插头被改动。

枚举当前格子的插头,状态共有3种转移:

下文的``相接位置''指代轮廓线与当前格子的两条邻接边,``对应位''指转移前相接位置的位与
转移后新边的位。
\begin{itemize}
    \item 新建连通分量:相接位置没有右插头和下插头,当前格子有右插头和下插头,将对应位置
    置为新的标号,然后重新$O(n)$扫描以保持最小表示法。
    \item 合并连通分量:当前格子有左插头和上插头。若对应的右插头和下插头未连通,则将含有
    这两个标号的所有位置标记为同一标号,并重新扫描,再将对应位置0。若其已连通,则只允许在
    最后一个非障碍格子中合并为一条回路。
    \item 保持连通分量:相接位置有右插头或下插头恰好一个,当前格子有上或左
    插头恰好一个,有下或右插头恰好一个。不需要重新扫描,可以$O(1)$转移。
\end{itemize}
\paragraph{障碍格子处理}
当遇到障碍格子时,相接位置必须没有插头,由于该格子不能铺线,对应位置上没有插头,仍然置为0,
所以状态不变。
\paragraph{跨行处理}
当从上一行的最后一个格子转移到下一行的第一个格子时,不能从有右插头的状态转移。实现时
保证上一行的最后一个格子不能产生右插头,转移到首格时移位处理。
\paragraph{始末状态}
初始状态和终止状态都没有插头,值为0。
\paragraph{小结}
在推导状态转移时需要注意以下方面:
\begin{itemize}
    \item 状态表示是否存储足够的信息
    \item 连通分量的三种变化情况
    \item 移动格子时要保证所有裸露的插头都在轮廓线上
    \item 转移的目标状态表示的是{\bfseries 轮廓线上插头}的连通性。
    \item 转移完状态后是否需要重新扫描以保持最小表示法的性质
    \item 障碍格子和行首尾格子的处理
    \item 始末状态
    \item 从可行性与最优性对无效状态进行剪枝
\end{itemize}
\subsubsection{程序的实现与优化}
直接枚举状态会产生大量的无效状态,因此使用队列从初始状态开始转移。枚举每个格子,
枚举循环队列中的状态,计算出可以转移的目标状态。使用Hash表存储当前已经转移过的状态的
dp值,Hash表的size不必太大,每转移一个格子开一个新的Hash表。

参考代码:
\lstinputlisting{Source/Templates/Link.cpp}
\subsubsection{简单回路问题与括号表示法}
事实上对于简单回路问题,裸露的插头必须在轮廓线上,轮廓线上的插头必定两两匹配。
又因为求的是简单回路,匹配的插头不会交叉。那么可以使用一个括号序列来表示连通性,
0表示没有插头,1表示左端,2表示右端,使用4进制表示状态。

再次根据连通分量的变化讨论转移:
\begin{itemize}
    \item 新建连通分量:要求相接位置没有插头,转移时将对应位分别置为1和2。
    \item 合并连通分量:需要按照相接位置的插头是左括号还是右括号讨论,记
    右插头为$A$,下插头为$B$:
    \begin{itemize}
        \item $A$左$B$左:将$B$对应的右括号修改为左括号
        \item $A$左$B$右:此时将连成一条回路,当其为最后一个无障碍格子时才转移
        \item $A$右$B$左:不需要额外修改
        \item $A$右$B$右:将$A$对应的左括号修改为右括号
    \end{itemize}
    最后将对应位都置为0。
    \item 保持连通分量:直接继承有插头位置的左右标号
\end{itemize}

此法思维难度低,实现简单,程序速度快。但是括号表示法的局限性较大,参见下文的广义表示法。

参考代码:
\lstinputlisting{Source/Source/Link/5056.cpp}
\subsubsection{非回路问题转化为简单回路问题}
求从棋盘中一个特殊点经过所有非障碍点走到另一个特殊点的方案数。

可以尝试额外构造一条宽度为1的路径使其连通,新棋盘的一条回路对应了原棋盘的一条路径。
\subsection{简单路径问题}
给定一个棋盘,某些格子不能经过,其余格子必须经过,求有多少条简单路径。

此时非轮廓线上有不超过2个裸露插头。若使用最小表示法,则还要记录每个插头与路径端点的连通情况。
若使用括号表示法,则再引入标号3指示独立插头,说明这个插头连接着路径的一端,转移时需要保证任意
时刻轮廓线上的独立插头不超过2个。
\subsection{最大化回路/路径/连通块点权和}
\begin{itemize}
    \item 回路:注意仅在连为回路的情况更新答案。由于允许有些格子不经过,会导致回路外还可能
    存在一些孤立的路径,需要特判连为回路后轮廓线上是否均没有插头。
    \item 路径:额外使用2bit记录轮廓线上方(不含边界)裸露插头的个数,同时引入独立插头标记。
    注意最后连为一条路径时要保证轮廓线上插头+裸露插头的个数$\leq 2$。
    \item 连通块:仅记录$m$个格子的连通性。特别考虑移动轮廓线后消失的格子(即当前格子的上方)
    ,若它被选中且当前格子不选,需要保证它至少与轮廓线上的其它格子连通。
\end{itemize}
\subsection{广义括号表示法}
括号表示法的局限性在于其只能表示在轮廓线上最多有2个插头的连通分量。

将最左插头标记为``('',最右插头标记为``)'',中间的插头标记为``)('',独立插头标记为``()'',
能够匹配的括号对应的插头是连通的。
\subsection{棋盘染色问题}
棋盘上格子的连通性取决于棋盘格子的颜色,因此需要额外记录轮廓线上棋盘格子的颜色。
\subsection{局部连通性加速DP}
有些题目会给出一个特殊的构图方法,其连边具有局部性质。那么就可以讨论局部连通性计算转移矩阵,
使用矩阵快速幂加速DP。

上述内容参考了IOI2018国家集训队论文集中陈丹琦的《基于连通性状态压缩的动态规划问题》。

\section{单调队列优化}
当状态转移方程为如下形式时
\begin{displaymath}
    dp[i][j]=max(dp[i-1][j-k]+w(i-1,j-k))+c(i,j),l \leq k \leq r
\end{displaymath}
可以使用单调队列优化。

该状态转移方程满足如下性质:
\begin{itemize}
	\item 转移区间是连续的。
	\item 当前转移位置到转移区间有距离限制。
\end{itemize}

对每个$i$进行dp的步骤如下:
\begin{enumerate}
	\item 初始化空队列;
	\item 计算自己到队首的距离,弹出超出转移距离的队首;
	\item 计算新进入转移区间的转移点的权(即$max$的内容);
	\item 为了维护队列的单调性,不断地从队尾弹出不比当前转移点更优的旧转移点
	      (即使同样优也要弹出,因为当前转移点肯定比旧转移点更晚弹出);
	\item 此时队首为最优值,加上常数后即为dp最优值。
\end{enumerate}

\section{斜率优化}\label{Slope}
\subsection{推导}
当状态转移方程为如下形式时:
\begin{displaymath}
    dp[i]=min(a[i]*b[j]+c[i]+d[j])
\end{displaymath}
考虑使用斜率优化。

以下推导假设$a[i]$单调递减且$b[j]$单调递增:

设决策点$j<k<i$,且从点$k$转移到$i$不差于从点$j$转移到$i$,
易证从点$k$转移到$i+1$同样不差于从点$j$转移到$i+1$。称该性质为决策单调性。

接下来考虑点$k$不比点$j$更差的条件:
\begin{eqnarray*}
    a[i]*b[j]+c[i]+d[j]&\geq&a[i]*b[k]+c[i]+d[k]\\
    \Rightarrow -a[i]&\geq&\frac{d[k]-d[j]}{b[k]-b[j]}
\end{eqnarray*}

记斜率
\begin{displaymath}
    slope(i,j)=\frac{d[j]-d[i]}{b[j]-b[i]}
\end{displaymath}

斜率可以使用单调队列维护,记$q[b]$为队首,$q[e-1]$为队尾:
\begin{itemize}
    \item 若$-a[i]\geq slope(q[b],q[b+1])$,则表明$q[b+1]$不比$q[b]$更差,
    弹出$q[b]$。
    \item 若$slope(q[e-2],q[e-1])\geq slope(q[e-1],i)$,则说明若
    $q[e-2]$被弹出,$q[e-1]$一定被弹出,所以$q[e-1]$无效,可以先弹出。
\end{itemize}

从``形''的角度理解,单调队列维护了一个下凸壳。
\subsection{应用}
主要过程就是研究决策单调性满足的条件,然后选取适当的数据结构维护信息,快速dp。

实际应用中需注意以下几点:
\begin{itemize}
    \item 比较斜率时尽量使用乘法避免精度误差,提高效率,要考虑变号时的符号问题,最好
    保持分母为正。\CJKsout{(反正也就两处符号,面向样例编程就行了)}。
    \item 若$a[i]$单调,使用单调队列,否则使用~\ref{BSDP}所述的决策二分栈/队列。
    \item 若$b[i]$单调,使用单调队列,否则使用平衡树维护凸壳/CDQ分治(留坑待补)/李超线段树。

    {\bfseries 血泪史:「CEOI2017」Building Bridges

    事实上动态凸壳的维护不是很好处理,因为浮点数的精度问题不好解决。我调了3个多小时还是
    WA(更悲惨的是总共只WA一半,但每组捆绑测试都有测试点WA)。可以考虑维护动态半平面交,
    毕竟HPI还是比较成熟的,由于半平面只有插入,使用第~\ref{BinIns}节所述的二进制分组
    解决。事实证明HPI+二进制分组法数值稳定性比较好(一遍AC,速度比动态凸包快,代码比动态
    凸包短)。
    }
\end{itemize}

以上内容参考了MashiroSky的博客\footnote{斜率优化学习笔记 - MashiroSky
    \url{https://www.cnblogs.com/MashiroSky/p/6009685.html}
}。
\subsection{树上斜率优化}
例题:NOI2014 购票

我原先的做法:维护每条重链的完整单调队列,查询时二分出单调队列的一段,然后二分
转移点。最后讨论该段前后不在队列上的点取出暴力转移。虽然最终得到满分,但是这种
做法严重依赖于数据强度,容易被Hack,并且细节很多。\CJKsout{2019.3.17:为什么
又是rank2。。。}
\subsubsection{点分治}
考虑树退化成链的情况,由于转移长度有限制且不单调,无法使用单调队列维护。使用CDQ分治解决:
\begin{enumerate}
    \item 递归处理$[l,m]$
    \item 计算$[l,m]$到$[m+1,r]$的转移:
    \begin{enumerate}
        \item 将$[m+1,r]$按照更新左边界点降序排序,忽略左边界点超过$m$的点
        \item 将$[l,m]$从右到左加入,维护凸包,加入点$i$后,二分转移所有左边界点
        恰好为$i$的点
    \end{enumerate}
    \item 递归处理$[m+1,r]$
\end{enumerate}

现在考虑树的情况,树上分治一般使用点分治:

记当前分治过程的重心为$g$,连通块的根为$u$,节点$u$的还未尝试转移的深度最浅的祖先$top_u$:
\begin{enumerate}
    \item 由于这是有根树,$u$所在的子连通块与其它连通块并不平等,
    且$g$需要从$u$处转移,因此首先递归计算$u$的子连通块。
    \item 若$u\neq g$,此时$g$的所有祖先都已经计算完毕,尝试从$u$到$g$的父亲转移$g$。
    \item DFS遍历子连通块的节点,序列化点的编号,记录所属连通块的根$bel$。
    \item 将节点按照$top$排序,从$u$到$u$的祖先逐个加入凸包,尝试更新dp值,
    同时更新$top=bel$。
    \item 递归分治子连通块。
\end{enumerate}

时间复杂度$O(n\lg^2 n)$。
\subsubsection{可持久化单调队列}
对于可以使用单调队列解决的树上斜率优化问题,在插入一个点时在该点存储被该点
弹出的节点编号,回溯时删除自身,同时恢复被弹出的节点。由于不能确定每个点被
弹出和恢复的次数,无法保证这个方法的复杂度。

可以发现在加入一个决策点后,队列只要覆盖一个决策点。只需在修改前记录被覆盖位置的决策点和
原队列的长度,回溯时恢复该位置与队列长度。插入时使用二分快速计算覆盖位置。不执行队列的弹出
操作,同样使用二分计算转移点(这样也可以应对自变量$x$不单调的情况)。

对于本题,由于还有$l$的限制,如果只维护单个凸包,最优决策点有可能被删去
\CJKsout{(我的做法就是暴力处理这种情况)}。由于凸包是可并的,可以像线段树那样维护深度
区间凸包。查询时每个连续区间都二分找到最优决策点,合并时选取这些决策点的最优解,单次查询
时间复杂度$O(\lg^2 n)$。修改时按照上文所述修改$O(\lg n)$个区间,单次修改时间复杂度
$O(\lg^2 n)$。

该内容参考了xyz32768\footnote{
    [BZOJ3672][Noi2014]购票(斜率优化+点分治)\\
    \url{https://blog.csdn.net/xyz32768/article/details/82709944}
}
和Sakits\footnote{
    bzoj3672: [Noi2014]购票(树形DP+斜率优化+可持久化凸包)\\
    \url{https://www.cnblogs.com/Sakits/p/8215297.html}
}的博客。
\subsection{CDQ分治维护凸包}
例题:NOI2007 货币兑换

记第$i$天最大收益为$c_i$,将其兑换为AB券的数量为$(a_i,b_i)$。那么$j$买入到$i$卖出的
转移就相当于计算$A_ia_j+B_ib_j$的最大值。

有两种分析方法:
\begin{itemize}
    \item 设$a_j<a_k$,在转移点$i$时,点$j$比点$k$优蕴含着
    $(a_j-a_k)A_i+(b_j-b_k)B_i>0$,化简为
    $\frac{b_j-b_k}{a_j-a_k}<-\frac{A_i}{B_i}$。将$(a,b)$视作点,
    同上文的分析,这里维护$i$之前所有转移点$(a,b)$的上凸包,二分寻找两边斜率包含
    $-\frac{A_i}{B_i}$的点作为决策点。
    \item 将$A_ia_j+B_ib_j$视作点$(a_j,b_j)$到直线$A_ix+B_iy=0$的距离*常数,
    那么最远点可以使用这条直线的平行线夹逼得到。
\end{itemize}

上述分析是等价的,由于$a_i$不单调,需要维护动态凸包(这里的$a_i$不是预先知道的,不能用李超树)。
考虑使用CDQ分治,即每次递归处理完左边的答案后,使用左边的凸包转移右边的点。递归前预排序斜率,
在递归时分发到左右区间,这样就保证右边点的斜率是有序的,可以$O(区间长度)$扫描转移。至于凸包,
可以在回溯时归并排序保证水平序。

上述内容参考了2008年陈丹琦的集训队作业《从〈Cash〉谈一类分治算法的应用》。

\section{四边形不等式优化}
在区间动规时通常会推出以下状态转移方程:
\begin{displaymath}
    dp[i][j]=min(dp[i][k]+dp[k+1][j])+w[i][j],i\leq k <j
\end{displaymath}
此时可考虑使用四边形不等式将$O(n^3)$优化到$O(n^2)$。

接下来定义两个性质:
\paragraph{区间包含的单调性}
对于$a\leq b\leq c\leq d$,有$f(b,c)\leq f(a,d)$。
\paragraph{四边形不等式}
对于$a\leq b\leq c\leq d$,有$f(a,d)+f(b,c)\geq f(a,c)+f(b,d)$。

\paragraph{优化} 计算$w[i][j]$时要考虑区间统计方面的优化(如前缀和)。

以上内容参考了XDU\_Skyline的博客\footnote{动态规划专题小结:四边形不等式优化
    \url{https://blog.csdn.net/u014800748/article/details/45750737}
}。

\section{矩阵快速幂优化}
\subsection{常规矩阵快速幂}
若dp状态转移方程满足如下形式:
\begin{displaymath}
    dp[i]=\sum_{j=1}^k{c_idp[i-j]}
\end{displaymath}
或对于图满足如下形式:
\begin{displaymath}
    dp[d][i][j]=\sum_{(i,k)\in E,(k,j)\in E}{dp[d-1][i][k]\cdot dp[d-1][k][j]}
\end{displaymath}
则可以使用矩阵快速幂优化。

计算$k*k$的转移矩阵$A$,dp初始值为$1*k$的向量$v_0$。
构造$A$,使其每乘一次$A$,向量表示的区间后移一格,那么
$A[i][j]$表示其在做一次乘法后将第$i$点的值贡献到第$j$点中的权值。

矩阵乘法满足结合律,因此可以使用矩阵快速幂进行计算。

\paragraph{例题} 「NOI2007」 生成树计数

使用基尔霍夫定理计算肯定是不现实的(\CJKsout{不过+k=3时的斐波那契数列可以水到80分})。
注意到连边是局部的,且$k$很小。可以使用状态来表示最近$k$个节点的连通性,然后枚举树边
转移。至于连通性的表示,可以使用最小表示法DFS预处理,即每个节点对应一个编号,且$i$必须
在$i+1$之前出现,相同编号的节点连通,并且这些状态保证了之前的链合法。当$k=5$时,对应的
集合划分数目(贝尔数)为52,使用矩阵快速幂转移。

参考代码:
\lstinputlisting{Source/Source/DP/LOJ2356.cpp}

\subsection{矩阵链乘秩分解}

若大小为$n*n$的矩阵$A$可表示为大小为$n*k$的矩阵$B$与大小为
$k*n$的矩阵$C$的乘积,其中$k\ll n$。
那么可以将$A$的幂拆开,错位结合,计算$k*k$的矩阵$D=CB$,对$D$快速幂后
计算需要的值{\bfseries (答案向量为$v_0AD^{c-1}B$,尽量按需计算结果)}。

\paragraph{例题} bzoj3583 杰杰的女性朋友

使用上述方法优化矩阵快速幂的效率。此外还存在一个问题,矩阵$A$的$k$次幂求的是走$k$
步的方案数的转移矩阵,但是答案要的矩阵为矩阵幂求和。因此我们可以对于每次询问再加一个
累加计数器,自己向自己连边,对应点向自己连边,最后单独求出起点到自己的方案数。即再开
一个``信道''?,然后在新开的信道上,终点的出边,计数器的入边与出边均设为1。

还有另一种方法:注意到这里求的是矩阵的等比数列之和。可以将数列对半分,然后可以提出
$(1+A^d)$的因子,子问题的规模减半。细节比较多,这里不详细写。

参考代码:
\lstinputlisting{Source/Source/DP/bzoj3583.cpp}

\subsection{dp步伐不一致时的解决方案}
例题~LOJ\#2180. 「BJOI2017」魔法咒语

如果单次转移最多需要跳跃$k$步($k$为小常数),可以给每个状态$S_0$引入$k-1$个``延迟状态''
$S_i$,若有状态$S$跳跃$k$步到达状态$T$,则实际转移为
$S_0\rightarrow T_{k-1} \rightarrow T_{k-2} \rightarrow \cdots \rightarrow T_0$。
注意后面的转移链是固定的,可以单独预处理。而第一个转移与原来的处理方式相同,只要根据跳跃
步数计算需要延迟转移的时间,然后连到链上对应的节点。统计时仍然只统计$S_0$,因为延迟状态
的贡献是不完整的。

\subsection{矩阵对角化加速快速幂}
有时会推出一些比较简单的矩阵(尤其是三角矩阵与对称矩阵),这时可以快速找到矩阵的特征值,
计算出特征向量,将矩阵表示为$A=PDP^{-1}$的形式,其中$P$是可逆矩阵,由特征向量组成,
$D$为对应特征值组成的对角矩阵。

那么$A^n$可以表示为$PD^nP^{-1}$的形式,中间的部分可以快速幂求出,两边的部分要根据具体
情况讨论。

\paragraph{例题} CF923E Perpetual Subtraction

记初始向量为$P_0$,转移矩阵为$A$,若用矩阵左乘表示转移,则答案为$A^nP_0$。

现在写出转移矩阵$A$:
\begin{displaymath}
    \left[
    \begin{array}{ccccc}
        1&\frac{1}{2}&\frac{1}{3}&\cdots&\frac{1}{n+1}\\
         &\frac{1}{2}&\frac{1}{3}&\cdots&\frac{1}{n+1}\\
         &           &\frac{1}{3}&\cdots&\frac{1}{n+1}\\
         &           &           &\ddots&\vdots\\
         &           &           &      &\frac{1}{n+1}\\
    \end{array}
    \right]
\end{displaymath}

这是一个上三角矩阵,不过即使有稀疏矩阵优化,$O(n^3\lg M)$的快速幂仍然无法承受。

由于该矩阵的特殊性,考虑对角化该矩阵以加速矩阵幂的计算。易知该矩阵的特征向量为
$1,\frac{1}{2},\cdots,\frac{1}{n+1}$。接下来代入$Av=\lambda v$求出
特征向量$v$并组出$P$:经过小矩阵的推导得特征值$\frac{1}{1+x}$的特征向量为
$[(-1)^i\binomial{i}{0},(-1)^{i+1}\binomial{i}{1},\cdots (-1)^{i+n}\binomial{i}{n}]^T$。
继续打表求出逆矩阵,然后将矩阵乘法展开,得到卷积的形式,使用NTT解决。具体证明参见参考链接。

该方法参考了yhx-12243的博客\footnote{
    [Codeforces923E/947E]Perpetual Subtraction\\
    \url{https://yhx-12243.github.io/OI-transit/records/cf923E\%3Bcf947E.html}
}。

对角化很简单,但是寻找特征向量与逆矩阵是困难的。



\chapter{树}
\minitoc
\section{最小公共祖先}
\subsection{倍增法}
预处理:DFS时计算每个节点的深度和$2^k$级祖先。

查询:首先将较深节点跳到同一高度,若原节点在一条链上,则
较浅的点为LCA,算法结束。否则按$k$从大到小尽量跳,保持不
跳到同一祖先。最后这两个节点的父亲就是原节点的LCA。

\begin{lstlisting}
int d[size],p[size][20];
void DFS(int u) {
    for(int i=1;i<20;++i)
        p[u][i]=p[u][i-1][i-1];
    for(int i=last[u];i;i=E[i].next) {
        int v=E[i].to;
        if(p[u][0]!=v) {
            p[v][0]=u;
            d[v]=d[u]+1;
            DFS(v);
        }
    }
}
int lca(int u,int v){
    if(d[u]<d[v])std::swap(u,v);
    int delta=d[u]-d[v];
    for(int i=0;i<20;++i)
        if(delta&(1<<i))
            u=p[u][i];
    if(u==v)
        return u;
    for(int i=19;i>=0;--i)
        if(p[u][i]!=p[v][i])
            u=p[u][i],v=p[v][i];
    return p[u][0];
}
\end{lstlisting}
预处理$O(n\lg n)$,查询$O(\lg n)$。
\subsection{树链剖分}
树链剖分后,如果在同一条链上则返回较浅者,否则令链头较深的节点向上跳。

\begin{lstlisting}
int lca(int u,int v) {
    while(top[u]!=top[v]) {
        if(d[top[u]]>d[top[v]])u=p[top[u]];
        else v=p[top[v]];
    }
    return d[u]<d[v]?u:v;
}
\end{lstlisting}

两趟DFS预处理$O(n)$。
由于树链剖分后重链不超过$\lg n$条,所以查询也是$O(\lg n)$的,常数比倍增法小。
\subsection{欧拉序+ST表}
考虑DFS序,显然两个来自节点$i$的不同子树的点的LCA为节点$i$,那么可以在
DFS完一棵子树后加入节点$i$的深度作为隔板,按访问时间戳查询ST表即可。为了
应对在一条链上的情况,同时也为了给节点$i$一个时间戳,在遍历节点$i$之初就插入一个隔板。
\begin{lstlisting}
int L[size],d[size],A[20][size*2],siz=0;
void DFS(int u,int p) {
    A[0][++siz]=d[u];
    L[siz]=u;
    for(int i=last[u];i;i=E[i].next) {
        int v=E[i].to;
        if(v!=p) {
            d[v]=d[u]+1;
            DFS(v,u);
            A[0][++siz]=d[u];
        }
    }
}
void buildST() {
    for(int i=1;i<20;++i) {
        int end=siz-(1<<i)+1,off=1<<(i-1);
        for(int j=1;j<=end;++j)
            st[i][j]=std::min(st[i-1][j],st[i-1][j+off]);
    }
}
int query(int l,int r) {
    int siz=r-l+1;
    int p=ilg2(siz);
    return std::min(A[p][l],A[p][r-(1<<p)+1]);
}
int lca(int u,int v) {
    u=L[u],v=L[v];
    if(u<v)std::swap(u,v);
    return query(u,v);
}
\end{lstlisting}

预处理$O(n\lg n)$,查询$O(1)$。
\subsection{Tarjan}
当查询离线时,可使用Tarjan算法。

\lstinputlisting[title=Tarjan]{Tree/Tarjan.cpp}

原理与欧拉序+ST表法类似,当节点分别在两棵不同的子树时,若另一节点已处理完毕,
他的祖先肯定是LCA(因为LCA处还没有遍历完,未合并到更高的祖先上去)。

\paragraph{优化} 可以把find改为用队列实现,迭代更新父亲。

\section{链剖分}
以下方法只适用于静态树。
\subsection{轻重链剖分}
对于每个节点,选取其子树最大的儿子作为重儿子,其余节点为轻儿子。
然后将连续的重儿子串成一条链(由DFS序体现),每条链上最浅的节点为链头,
链内每个节点都指向链头方便跨越轻边跳跃到上一条链。

\begin{lstlisting}
int d[size],p[size],siz[size],son[size];
void buildTree(int u) {
    siz[u]=1;
    for(int i=last[u];i;i=E[i].nxt) {
        int v=E[i].to;
        if(v!=p[u]) {
            p[v]=u;
            d[v]=d[u]+1;
            buildTree(v);
            siz[u]+=siz[v];
            if(siz[son[u]]<siz[v])
                son[u]=v;
        }
    }
}
int top[size];
void buildChain(int u) {
    if(son[u]) {
        top[son[u]]=top[u];
        buildChain(son[u]);
    }
    for(int i=last[u];i;i=E[i].nxt) {
        int v=E[i].to;
        if(!top[v]) {
            top[v]=v;
            buildChain(v);
        }
    }
}
\end{lstlisting}

{\bfseries 注意调用buildChain前要先$top[1]=1$。}

\begin{property}
    剖分后的任意点到根经过的轻边与重链的数目为$O(\lg n)$级别。
\end{property}
证明:
\begin{itemize}
    \item 由于每个轻儿子的大小不超过父亲大小的一半,所以每次经过一次
    轻边时,所在子树大小增加一半,所以最多经过$O(\lg n)$条轻边。
    \item 重链是由轻边连接的,所以重链数=轻边数+1,因此经过重链数也是
    $O(\lg n)$级别的。
\end{itemize}

\subsubsection{在链上统计中的应用}
在线段树中需要保证操作链上的节点尽可能连续,
使用轻重链剖分后的DFS序保证了经过重链数为$O(\lg n)$级别,
搭配线段树可在$O(\lg^2 n)$内单次修改/查询。

模板与求LCA类似:
\begin{lstlisting}
int top[size],id[size],pid[size],icnt=0;
void buildChain(int u) {
    id[u]=++icnt;
    pid[icnt]=u;//for build
    //...
}
void build(int l,int r,int id) {
    if(l==r) {
        int u=pid[l];
        //...
    }
    //...
}
typedef void (*Func)(int,int,int);
template<Func func>
void applyImpl(int l,int r) {
    nl=l,nr=r;
    func(1,icnt,1);
}
template<Func func>
void apply(int u,int v) {
    while(top[u]!=top[v]) {
        if(d[top[u]]<d[top[v]])
            std::swap(u,v);
        applyImpl<Func>(id[top[u]],id[u]);
        u=p[top[u]];
    }
    if(d[u]>d[v])
        std::swap(u,v);
    applyImpl<Func>(id[u],id[v]);
}
\end{lstlisting}

\subsection{长链剖分}

顾名思义就是把子树深度最深的儿子当做重儿子进行剖分。

\subsubsection{快速合并以深度为下标的消息}

\subsubsection{快速求k级祖先}

长链剖分参考了MoebiusMeow的博客\footnote{长链剖分随想 - MoebiusMeow
    \url{https://www.cnblogs.com/meowww/p/6403515.html}
}。

\section{树的直径}
\section{Dsu On Tree}\label{DOT}
Dsu On Tree用于求解{\bfseries 静态树}的{\bfseries 子树统计}问题。
主要思路是在重链剖分后,令父亲直接继承重儿子的信息,暴力枚举轻儿子子树节点更新信息。

步骤如下:
\begin{enumerate}
    \item DFS轻儿子计算轻儿子子树节点的子树信息,统计完毕后消去影响;
    \item DFS重儿子计算重儿子子树节点的子树信息,统计完毕后保留影响;
    \item 暴力枚举轻儿子子树节点更新信息;
    \item 计算当前节点自身的影响;
    \item 此时维护的信息为该节点的子树信息,记录答案;
    \item 若标记为消去影响,则重置维护的信息({\bfseries 注意清空数组的复杂度
    不能高于其余部分的复杂度})。
\end{enumerate}

\begin{lstlisting}[title=Dsu On Tree]
void DFS(int u, int p, bool erase) {
    for(int i = last[u]; i; i = E[i].nxt) {
        int v = E[i].to;
        if(v != p && v != son[u])
            DFS(v, u, true);
    }
    if(son[u])
        DFS(son[u], u, false);
    for(int i = last[u]; i; i = E[i].nxt) {
        int v = E[i].to;
        if(v != p && v != son[u]) {
            for(int j = L[v]; j <= R[v]; ++j)
                add(idc[j]);//idc数组为已在DFS序上展开的数据
        }
    }
    add(c[u]);
    //记录信息
    if(erase) {
        for(int i = L[u]; i <= R[u]; ++i)
            erase(idc[i]);//为了保证复杂度按需清空
        //重置其余维护信息
    }
}
\end{lstlisting}

若可以$O(1)$计算加上/删除一个点的影响,则时间复杂度为$O(n\lg n)$。

证明:因为每个节点到根经过轻边条数为$O(\lg n)$,所以每个节点被作为轻儿子的
子树节点合并的次数为$O(\lg n)$,总时间复杂度为$O(n\lg n)$。

以上内容参考了Arpa的博客\footnote{[Tutorial] Sack (dsu on tree) - Codeforces
    \url{http://codeforces.com/blog/entry/44351}
}。

\section{Purfer Sequence}
\index{P!Purfer Sequence}
Purfer Sequence用来表示树的结构(注意这里的树{\bfseries 带标号})。
\subsection{构造Purfer Sequence}
\begin{enumerate}
	\item 在当前树中找到标号最小的叶子节点;
	\item 向序列加入与该叶子节点相连的节点标号,并将该叶子节点删除。
	\item 重复步骤1直至只剩2个节点。
\end{enumerate}
由此可得两个性质:
\begin{property}[唯一性]
	一个大小为$n$的树对应一个长度为$n-2$的Purfer Sequence。
\end{property}
\begin{property}
	一个度数为$d$的节点在Purfer Sequence中出现$d-1$次。
\end{property}
\subsection{恢复原树}
\begin{enumerate}
	\item 统计节点在Purfer Sequence中出现的次数得到每个点的度数,记为$d[u]$;
	\item 对于序列中的每一个编号$v$,选取$d[u]=1$且标号最小的节点$u$,连接$(u,v)$,
		  然后$--d[u],--d[v]$(实际上只需用平衡树维护度数为1的节点,
		  在平衡树上删除点$u$,$--d[v]$后判断是否要插入点$v$);
	\item 将剩下两个$d[u]=1$的点连边。
\end{enumerate}
\subsection{计数应用}
可根据度数与出现次数的关系计算满足度数要求的树的数目。
{\bfseries 注意无解和$n\leq 2$时的情况。}

以上内容参考了JMJST的博客\footnote{
	BZOJ 1005 [HNOI2008] 明明的烦恼(组合数学 Purfer Sequence) - jianzhang.zj
	\url{http://www.cnblogs.com/zhj5chengfeng/archive/2013/08/23/3278557.html}
}。

\section{虚树}
在多次询问的树型dp问题中,若遇到询问总点数与树的大小同数量级的情况,
可以在每次询问中将询问节点建成一棵``虚树'',然后对虚树做树型dp。
这样做可以有效地降低dp规模($\displaystyle qT(n)->
\sum_{i=1}^q{\left(O(m_i\lg m_i)+T(O(2m_i))\right)}$)。

\subsubsection{构造过程}

首先预处理从根节点DFS遍历整棵树,记录DFS序与深度,预处理计算LCA需要的信息,
同时维护其他信息。

对于每一次询问:
\begin{enumerate}
    \item 将询问节点按DFS序排序;
    \item 将第一个节点加入栈中,栈上维护的是当前节点到根的链;
    \item 对于剩下每一个节点$u$:
    \begin{enumerate}
        \item 计算自己与栈顶节点$v$的$lca$;
        \item 若栈中第二个节点$p$的深度比$lca$大,则连接$p->v$,弹出$v$。
        重复该步骤直至不满足条件;
        \item 若$lca$比$v$浅,连接$lca->v$,弹出$v$。
        \item 若栈为空或$v$比$lca$浅,加入节点$lca$。
        \item 加入节点$u$。
    \end{enumerate}
    \item 此时栈上还有一条链,将链加入树中,栈底就是根节点。
\end{enumerate}
\subsubsection{算法解释}

$lca$有两种可能:
\begin{itemize}
    \item $lca$为$v$:此时的操作只有加入节点$u$,
    就是简单地将自己挂在虚树中的父亲下。
    \item $u$和$v$在$lca$的两棵子树下:
    首先不断地折叠链直至$u$和$v$在虚树上的直接祖先为$lca$,
    然后继续分类:
    \begin{itemize}
        \item $p$为$lca$,连接$lca->v$后把$v$换成$u$。
        \item $p$为$lca$的祖先,连接$lca->v$后把$v$换成$lca$与$v$。
        \item $lca$为原链头的祖先,把链全部折叠后加入节点$lca$与$u$。
    \end{itemize}
\end{itemize}

\subsubsection{算法实现}

\begin{lstlisting}
int buildTree(int k) {
    g2.cnt = 0;
    int top = 1;
    std::sort(id + 1, id + k + 1, cmp);
    st[1] = id[1];
    for (int i = 2; i <= k; ++i) {
        int u = id[i];
        int lca = getLca(u, st[top]);
        while (top > 1 && d[lca] < d[st[top - 1]]) {
            g2.addEdge(st[top - 1], st[top]);
            --top;
        }
        if (d[lca] < d[st[top]]) {
            g2.addEdge(lca, st[top]);
            --top;
        }
        if (top == 0 || d[st[top]] < d[lca])
            st[++top] = lca;
        st[++top] = u;
    }
    while (top > 1) {
        g2.addEdge(st[top - 1], st[top]);
        --top;
    }
    return st[1];
}
\end{lstlisting}

{\bfseries 注意有时需要把根节点强制加入到虚树中去。}

\section{点分治}
点分治就是每次选择树的重心(儿子的子树大小的最大值最小的节点)作为分治点,
以分治点为根将整棵树分为多棵子树,统计当前节点的答案,
然后递归每棵子树分治。这种方法一般用来解决{\bfseries 路径统计问题}。
\subsection{常规点分治}
\subsubsection{重心性质}
\begin{property}\label{WPP}
    重心的儿子子树大小不超过整棵树大小的一半。
\end{property}
\begin{property}
    所有点到重心的距离和最小,到两个重心的距离和相等。
\end{property}
\begin{property}
    两棵树合并后的重心在这两棵树的重心的路径上。
\end{property}
\begin{property}
    添加或减少一个叶子节点,重心最多偏移一条边。
\end{property}
\subsubsection{重心选择}
以当前联通块内任意一点为根(当然是分治点的儿子)DFS,然后计算
其儿子的子树大小的最大值,然后与除自己子树外的节点数求最大值
(儿子为有根树意义下的父亲时的子树),
就得到当前节点的权重了。

使用$vis$数组来标记是否已成为分治点,顺便原来分割联通块。

\begin{lstlisting}[title=getRoot]
bool vis[size];
int root,tsiz,msiz,siz[size];
void getRootImpl(int u,int p) {
    int maxs=0;
    siz[u]=1;
    for(int i=last[u];i;i=E[i].nxt) {
        int v=E[i].to;
        if(!vis[v] && v!=p) {
            getRootImpl(v,u);
            siz[u]+=siz[v];
            maxs=std::max(maxs,siz[v]);
        }
    }
    maxs=std::max(maxs,tsiz-siz[u]);
    if(maxs<msiz) {
        msiz=maxs;
        root=u;
    }
}
int getRoot(int u,int csiz) {
    msiz=1<<30;
    tsiz=csiz;
    getRootImpl(u,0);
    return root;
}
\end{lstlisting}

\subsubsection{分治与统计}
每次把重心作为分治点,统计从分治点出发的信息,两两合并后对于每棵子树去除
来自同一棵子树的信息(因为这已经不是简单路径了)。

\begin{lstlisting}[title=divide]
void divide(int u) {
    //count u->child
    vis[u]=true;
    for(int i=last[u];i;i=E[i].nxt) {
        int v=E[i].to;
        if(!vis[v]) {
            //minus v->u->v
            divide(getRoot(v,siz[v]));
        }
    }
}
\end{lstlisting}

\subsubsection{时间复杂度}
点分治会带来$O(\lg n)$的复杂度,证明:

根据性质~\ref{WPP},点分治的层数为$O(\lg n)$,而且每层的总规模都是$n$。

\subsection{动态点分治}
动态点分治就是在分治时连接当前分治点与子联通块的分治点,
这些点构成了一棵点分树,多次查询时使用点分树来计算。

\subsubsection{例题}

Luogu P4115 Qtree4\footnote{【P4115】Qtree4 - 洛谷
\url{https://www.luogu.org/problemnew/show/P4115}}



\section{图上路径}
\subsection{欧拉回路}
\subsection{哈密尔顿回路}

\chapter{生成树}
\section{最小生成树}
\subsection{Kruskal与LCT}
Luogu P4234 最小差值生成树\footnote{【P4234】最小差值生成树 - 洛谷
	\url{https://www.luogu.org/problemnew/show/P4234}
}

按照Kruskal的处理顺序处理生成树,当遇到环时删掉环上权值最小的边,然后连上自己。

\subsection{Prim算法}

从一个点开始贪心地连接最短的可扩展边,直至每个点都被连接。

代码:
\begin{lstlisting}
struct Info{
    int u,d;
    Info(int u,int d):u(u),d(d){}
    bool operator<(const Info& rhs) const {
        return d>rhs.d;
    }
};
bool flag[size];
int prim(int n) {
    flag[1]=true;
    std::priority_queue<Info> heap;
    for(int i=last[1];i;i=E[i].nxt)
        heap.push(Info(E[i].to,E[i].w));
    int cnt=1,sum=0;
    while(cnt<n && heap.size()) {
        int u=heap.top().u;
        int d=heap.top().d;
        heap.pop();
        if(!flag[u]) {
            flag[u]=true;
            sum+=E[i].w;
            for(int i=last[u];i;i=E[i].nxt) {
                int v=E[i].to;
                if(!flag[v])
                    heap.push(Info(v,E[i].w));
            }
        }
    }
    return cnt==n?sum:-1;
}
\end{lstlisting}

当边数较大时,可以考虑使用Prim或优先队列+Kruskal(参见第~\ref{PQS}\\节)。

\subsection{次小生成树}
步骤如下:
\begin{enumerate}
	\item 构造最小生成树,标记并连接树边;
	\item 对生成树进行树链剖分,构造线段树,使其可以$O(\lg n)$查询到链上最大权;
	\item 枚举非树边,计算其替代链上的最大边后的代价,更新答案。
\end{enumerate}

对于严格次小生成树,需要维护链上最大权与严格次大权。

\subsection{生成树性质}
以下性质用于解决最小生成树计数问题:
\begin{property}
	最小生成树中,不同权值边的数量固定。
\end{property}
\begin{property}
	Kruskal算法中,处理完某一权值的边后,连通块相同。
\end{property}
\subsubsection{例题}

Luogu P4208 [JSOI2008]最小生成树计数\footnote{
【P4208】[JSOI2008]最小生成树计数 - 洛谷
\url{https://www.luogu.org/problemnew/show/P4208}
}

根据以上两个性质可以把每种权值的边分别计算,然后使用乘法原理组合即为答案。

步骤如下:
\begin{enumerate}
	\item 首先使用常规Kruskal计算每种权值边的数量$c_i$以及等权边的分布区间;
	\item 对于每一种权值,计算用完$c_i$条边的方案数。
	      由于题中每种权值的边数不超过10,可以枚举每条边选还是不选,
          在回溯时直接令连接的两点的根的父亲重设为自己,时间复杂度近似为$O(2^{c_i})$。
          模拟对该权值边做Kruskal的过程。
	\item 把每种权值的方案数乘起来就是方案数。
\end{enumerate}

代码如下:
\lstinputlisting{Source/Unclassified/Done/4208.cpp}

另一种方法:每次计算小连通块->大连通块的过程,对于每个大连通块单独使用
Matrix-Tree定理求方案数,方案数之积即为答案。

最小生成树计数问题参考了clover\_hxy的博客\footnote{
bzoj 1016: [JSOI2008]最小生成树计数 (矩阵树定理+最小生成树)
\url{https://blog.csdn.net/clover\_hxy/article/details/69397184}
}。

\section{Kruskal重构树}

\section{曼哈顿距离MST}
\section{Matrix-Tree定理}\label{MatrixTree}
\subsection{基本定义}
\index{K!Kirchhoff's Matrix Tree Theorem}
\begin{theorem}[Kirchhoff's Matrix Tree Theorem]
	一个无向图的生成树个数为度数矩阵(第$u$行第$u$列为点$u$的度数)减
	邻接矩阵(第$u$行第$v$列为$u,v$之间的边数)去掉第$i$行第$i$列后
	的行列式值。
\end{theorem}
根据这个定理,$O(n^3)$便可以求解无向图的生成树计数问题。{\bfseries 注意
高斯消元时交换两行会使行列式值取反,处理时记录符号或者直接返回其绝对值
(仅限于非模意义下求值)。}
\subsection{扩展}
\subsubsection{完全图生成树}
\index{C!Cayley's Formula}
\begin{theorem}[Cayley's Formula]
	大小为$n$的完全图的生成树个数为$n^{n-2}$。
\end{theorem}
套用Matrix-Tree定理或者使用Purfer序列可证明。若点与点之间的边数为$m$,
方案再乘上$m^{n-1}$。
\subsubsection{有向图生成树计数}
邻接矩阵只记录有向边,度数矩阵只记录入度,以$s$为根时删去第$s$行第$s$列后求行列式。
\subsubsection{边权乘积和}
把度数矩阵改为与某点相连的边的边权和,把邻接矩阵的边数改为该边的边权和。
若边权为整数则可以将其理解为将权值为$w$的边拆成$w$条边后求生成树数。

在「长乐集训 2017 Day10」生成树求和 加强版 这道题中,按位拆分后发现需要做不进位
加法,但是矩阵树定理只能做乘法。考虑使用生成函数表示和为0,1,2的种类数,最后做高斯消元。
但是多项式高斯消元并不好做,可以代入多个点值求高斯消元,然后插值。注意生成函数的乘法需要
做循环卷积,可以使用3个单位根作为代入点值。最后计算以3为基的IDFT插值(暴力计算以点值为
多项式系数,在3个单位根的逆上的值,最后将结果除以3)。注意到由于模数
1e9+7模3余2,在整数域上只有1个单位根,因此只能使用模意义下的复数根作为单位根。

\subsubsection{概率扩展}
Luogu P3317 [SDOI2014]重建\footnote{【P3317】[SDOI2014]重建 - 洛谷
\url{https://www.luogu.org/problemnew/show/P3317}
}

图中的每条边都有出现的概率,求图恰好连成一棵生成树的概率。

答案为每种方案的树边出现的概率和非树边不出现的概率之积的和。
将答案除以所有边都不出现的概率,转化为每种方案的树边出现的概率除以
树边不出现的概率的和。由此可以将其转化为边权乘积问题,即令出现概率为$p$的边的边权为
$\frac{p}{1-p}$,求完行列式后乘以$\displaystyle \prod_{i=1}^m{(1-p)}$。
注意使用偏移$\varepsilon$来防止除零。

代码如下:
\lstinputlisting{Source/Unclassified/Done/3317.cpp}

\subsubsection{限制边数}
图上的边有两种颜色,限制生成树中一种颜色的边的数量,求方案数。

令该颜色的边对应$x$,另一种颜色对应$1$,构造多项式,最后求出的行列式多项式的
$x^k$项的系数就对应使用$k$条边的方案数。多项式高斯消元不太方便,
可以先预处理$|V|$个$x$所对应的行列式值,然后插值出多项式。

上述内容参考了MoebiusMeow的博客\footnote{
	康复计划\#5 Matrix-Tree定理(生成树计数)的另类证明和简单拓展
	\url{https://www.cnblogs.com/meowww/p/6485422.html}
}和Wikipedia-EN\footnote{
	Kirchhoff's theorem - Wikipedia\\
	\url{https://en.wikipedia.org/wiki/Kirchhoff\%27s\_theorem}
}。

\section{斯坦纳树}
\index{S!Steiner Tree}
最小斯坦纳树求的是将指定的大小为$k$的点集连通的最小代价。

计算步骤如下:
使用状态压缩来描述指定点集的连通状态,
令$dp[i][s]$为以$i$为根,连通状态为$s$的最小代价。

转移方法有两种:
\begin{itemize}
    \item 两个不相交集合与同一个点连接,即
    \begin{displaymath}
        dp[i][s]=min\left\{dp[i][t]+dp[i][s-t]\right\},t\in s
    \end{displaymath}
    \item 给集合连入一条新边:
    \begin{displaymath}
        dp[i][s]=min\left\{dp[j][s]+w(i,j)\right\},(i,j)\in E
    \end{displaymath}
\end{itemize}

首先令$dp[i][1<<j]=0$,初始化单个点的情况。

可以从小到大枚举连通集合:
\begin{enumerate}
    \item 枚举子集更新$dp[i][]$。
    \item 将$dp[][s]\neq \infty$的点入队,使用SPFA或Dijkstra更新,注意不要
    改变连通状态。
\end{enumerate}

枚举子集更新的复杂度为$\displaystyle n\sum_{i=0}^k{\binomial{k}{i}2^i}=
n\cdot(1+2)^k=n\cdot 3^k$。

代码如下:
\begin{lstlisting}
int dp[1<<maxk][size];
int solve(int n,int k) {
    memset(dp,0x3f,sizeof(dp[0])<<k);
    for(int j=0;j<k;++j)
        for(int i=1;i<=n;++i)
            dp[1<<j][i]=0;
    int end=1<<k;
    for(int s=0;s<end;++s) {
        for(int t=s&(s-1);t;t=s&(t-1))
            for(int i=1;i<=n;++i)
                dp[s][i]=std::min(dp[s][i],
                    dp[t][i]+dp[s^t][i]);
        for(int i=1;i<=n;++i)
            if(dp[i][j]!=inf)
                //push
        SSSP(s);
    }
    int ans=inf;
    for(int i=1;i<=n;++i)
        ans=std::min(ans,dp[end-1][i]);
    return ans;
}
\end{lstlisting}


\chapter{图论}
\minitoc
\section{割点与桥}
\subsection{割点}
若删除{\bfseries 无向连通图}中的一个点及与它相连的边,使得整个图不连通,
那么称这个点为割点。

查找割点的步骤如下:
\begin{enumerate}
    \item 从一个点开始DFS遍历未被遍历的点;
    \item 对于每个点维护其访问时间$dfn$和不经过与父亲相连的边所能访问到的点的访问时间
    最小值$low$;
    \item 如果自己不是DFS树的根,若DFS树中儿子的$low$不超过自己的$dfn$,则说明删掉
        自己后DFS树上自己的父亲与自己的儿子不连通,自己为割点;
    \item 如果自己为DFS树的根,并且自己在DFS树上有两个及以上的儿子,说明自己也是割点。
\end{enumerate}

代码如下(求的是每个连通图的割点):
\lstinputlisting{Source/Review/Graph/CutVertex.cpp}

\subsection{桥}
若删除{\bfseries 无向连通图}中的一条边,使得整个图不连通,那么称这条边为桥。

同样维护$dfn$与$low$,在DFS树上处理$u->v$的过程中,
$(u,v)$为桥当且仅当$(u,v)$无重边且$dfn[u]<low[v]$。

为了判断无重边的情况,在DFS过程中让其儿子返回是否存在重边。

\begin{lstlisting}
struct EdgeT {
    int u,v;
    EdgeT(int u,int v):u(u),v(v) {}
} E[maxm];
int dfn[size], low[size], ccnt = 0, ecnt = 0;
bool DFS(int u, int p, int e) {
    static int icnt = 0;
    dfn[u] = low[u] = ++icnt;
    int pcnt = 0;
    for(int i = last[u]; i; i = E[i].nxt) {
        int v = E[i].to;
        if(v != p) {
            if(dfn[v])
                low[u] = std::min(low[u], dfn[v]);
            else {
                bool flag = DFS(v, u, i);
                low[u] = std::min(low[u], low[v]);
                if(flag && dfn[u] < low[v])
                    E[++ecnt] = EdgeT(u, v);
            }
        }
        else ++pcnt;
    }
    return pcnt==1;
}
\end{lstlisting}

\section{强连通分量}
\index{S!Strongly Connected\\ Component}
\subsection{定义}
\paragraph{强连通} 如果一对点存在路径互相可达,就称这对点强连通。
\paragraph{强连通子图} 强连通子图的点两两互相可达。
\paragraph{强连通分量} 有向图的极大强连通子图。

一般可以将强连通分量缩成一个点,然后对缩点后的图进行dp。
\subsection{Tarjan算法}
Tarjan算法求SCC的步骤如下:

\begin{enumerate}
	\item DFS遍历每个未遍历的点。
	\item 对于每个点,维护其访问时间$dfn$,可访问到的栈上的
	      最早的点的访问时间$low$。DFS处理该点时,将该点加入栈中。
	\item 若$dfn[u]==low[u]$,则说明栈上从栈顶到自己的点构成了强连通分量,
	      新建一个强连通分量,记录每个点所属的强连通分量。
\end{enumerate}

正确性证明留坑待补。
\index{*TODO!Tarjan算法正确性证明}

代码如下:
\begin{lstlisting}
int dfn[size],low[size],st[size],top=0,col[size],ccnt=0;
bool flag[size];
void DFS(int u) {
    static int icnt=0;
    dfn[u]=low[u]=++icnt;
    flag[u]=true;
    st[++top]=u;
    for(int i=last[u];i;i=E[i].nxt) {
        int v=E[i].to;
        if(dfn[v]) {
            if(flag[v])
                low[u]=std::min(low[u],dfn[v]);
        }
        else {
            DFS(v);
            low[u]=std::min(low[u],low[v]);
        }
    }
    if(dfn[u]==low[u]) {
        int c=++ccnt,v;
        do {
            v=st[top--];
            flag[v]=false;
            col[v]=c;
        } while(u!=v);
    }
}
for(int i=1;i<=n;++i)
    if(!dfn[i])
        DFS(i);
\end{lstlisting}

\section{双连通分量}
\subsection{点双连通分量}
\subsection{边双连通分量}

\section{2-SAT}
\subsection{问题描述}
有若干个布尔变量,对于由多个AND连接的若干个OR子表达式,且子表达式的操作数为
布尔变量或者其否定,
判断是否存在一组对布尔变量的赋值,使得这个布尔表达式的值为1。

这类问题一般描述为一个变量为真/假限制了另一个变量必须为真/假,求使得
所有限制被满足的一组变量赋值方案。对每个布尔变量拆点,可以将这些限制
表示为有向图。将命题中的条件向结论连边,同时对逆否命题的条件和结论连边。

比如要求变量$X$为真时变量$Y$必定为真,首先连边$X_1\rightarrow Y_1$;其逆否命题为
$Y$为假时$X$必定为假,连边$Y_0\rightarrow X_0$。

对于强制某一变量为真/假的需求,独立出来不好做,考虑沿用连边的思路。如果强制变量$X$为真,就
连边$X_0\rightarrow X_1$。
\subsection{可行性判定}
对这个图求强连通分量,发现同一强连通分量的点同时被选或不被选(分量里有一个点被选,根据
边的意义和强连通分量的定义,它可以把这个强连通分量内的点染色为被选)。

因此可以对该图求强连通分量,若$X_0$与$X_1$在同一个强连通分量内则为无解。
\subsection{构造方案}
以下方法的正确性证明留坑待补。
\index{*TODO!2-SAT构造方案正确性证明}
\subsubsection{DFS染色法}
首先对其进行缩点,标记每个强连通分量的对立分量,并连反图。

按照拓扑序处理每个强连通分量,DFS对立分量:
\begin{enumerate}
    \item 若自己已被标记则返回;
    \item 将自己标记为不选择,将对立分量标记为选择;
    \item DFS递归标记所连的点。
\end{enumerate}
查看每个点所在强连通分量的标记来输出方案。
\subsubsection{更简洁的方法}
由于tarjan算法标记强连通分量的顺序为自底向上,而上述方法的顺序同样也是
自底向上。因此对于每个布尔变量,选取所在强连通分量标号小的布尔值。

该方法参考了TRTTG的博客\footnote{2-SAT问题的方案输出
\url{https://www.cnblogs.com/TheRoadToTheGold/p/8436948.html\#\_label1}}。

{\bfseries 由于01变量的染色过程是连续的,因此当$X_0$与$X_1$较后一个的颜色确定后
就可以进行判断。这种方式可以避免在非法解上浪费时间。判断过程要放在if(dfn[i])外!!!}
\subsection{前后缀优化建图}
例题 PA2010 Riddle

有$n$个城镇,隶属于$k$个郡,城镇之间有$m$条连边。现在要将这些城镇中设立首都,
满足任意一条边之间至少有一个城镇为首都,且每个郡最多有一个首都,求是否存在合法方案。

点覆盖的约束很好解决,但是限制每个郡最多有一个首都比较困难,因为常规思路的连边达到
$O(n^2)$级别。考虑如何每个点只连常数条边就可以把不选的约束传递到同一个郡内的所有节点。
记布尔变量$x_i$表示城镇$i$是否为首都,记布尔变量$x_i'$表示在其所在郡的城镇序列中,
城镇$i$及其之前的城镇是否被选。根据单点与前缀、前缀与前缀的关系连边,就可以把约束快速
传递。由于前缀与前缀之间的约束关系可以将约束后传,不必对后缀连边。

该方法参考了ZigZagK的博客\footnote{
    【前后缀优化建图+2-SAT】BZOJ3495(PA2010)[Riddle]题解\\
    \url{https://blog.csdn.net/zzkksunboy/article/details/76285426}
}。

\section{仙人掌与圆方树}
\subsection{仙人掌}
\paragraph{仙人掌} 任意一条边最多只存在于一个环中的无向连通图叫做仙人掌。
\subsubsection{仙人掌的判定}
假设整个图已经连通,可以先对仙人掌进行DFS,记录DFS序与点的深度。然后
树上差分求出经过这条边的环数。判定的同时还可以把仙人掌去环变成森林。
\begin{lstlisting}
void initGraph(int n) {
    memset(last, n);
    cnt = 1;
}
bool flag[size];
int p[size], d[size], id[size], icnt;
void DFS(int u) {
    id[++icnt] = u;
    flag[u] = true;
    for(int i = last[u]; i; i = E[i].nxt) {
        int v = E[i].to;
        if(!flag[v]) {
            d[v] = d[u] + 1;
            p[v] = u;
            DFS(v);
        }
    }
}
int tag[size];
bool graph2Forest(int n) {
    icnt = 0;
    memset(flag, n);
    DFS(1);
    memset(tag, n);
    for(int i = 2; i <= cnt; i += 2) {
        int u = E[i].to, v = E[i ^ 1].to;
        if(d[u] < d[v])
            std::swap(u, v);
        if(p[u] != v)
            ++tag[u], --tag[v];
    }
    initGraph(n);
    for(int i = n; i >= 1; --i) {
        int u = id[i];
        tag[p[u]] += tag[u];
        if(tag[u] == 0) {
            if(p[u]) {
                addEdge(u, p[u]);
                addEdge(p[u], u);
            }
        } else if(tag[u] > 1)
            return false;
    }
    return true;
}
\end{lstlisting}
如果graph2Forest返回false则说明这个图不是仙人掌。若返回true
则建出一个去掉环的森林。
\subsubsection{DFS树dp法}
对于简单的仙人掌问题可以使用DFS树做法:

首先可以使用类似Tarjan的算法判断是否出现了环,然后对于环和桥分别dp,
将环的信息记录在环上节点在DFS树上最浅的节点上(即在回到环上最浅点时
另外dp),这样向上转移时就和桥一样了。

模板:
\begin{lstlisting}
int d[size], p[size], dfn[size], low[size], icnt = 0;
void DFS(int u) {
    dfn[u] = low[u] = ++icnt;
    for(int i = last[u]; i; i = E[i].nxt) {
        int v = E[i].to;
        if(v == p[u])
            continue;
        if(!dfn[v]) {
            p[v] = u;
            d[v] = d[u] + 1;
            DFS(v);
        }
        low[u] = std::min(low[u], low[v]);
        if(low[v] > dfn[u])
            //'转移桥边' v->u
    }
    for(int i = last[u]; i; i = E[i].nxt) {
        int v = E[i].to;
        if(p[v] != u && dfn[u] < dfn[v])
            solveRing(u, v);//'计算环' v->u
    }
}
\end{lstlisting}
\subsection{圆方树}
圆方树是解决仙人掌问题的利器,主要思想是把仙人掌构造为一棵树,然后使用
熟悉的树上操作来处理。
\subsubsection{构造}
在Tarjan时连树边(圆圆边),在环上深度最浅的点上处理环,把环上的节点(圆点)
都连到新建的点(方点)上(圆方边)。

\begin{lstlisting}
int dfn[size],low[size],d[size],p[size],ncnt;
void solveRing(int u,int v) {
    int siz=d[v]-d[u]+1;
    int id=++ncnt;
    for(int i=v;i!=u;i=p[i])
        //add id->i
    //add id->u
}
void tarjan(int u) {
    static int icnt=0;
    dfn[u]=low[u]=++icnt;
    for(int i=last[u];i;i=E[i].nxt) {
        int v=E[i].to;
        if(v==p[u])
            continue;
        if(!dfn[v]) {
            p[v]=u;
            d[v]=d[u]+1;
            DFS(v);
            low[u]=std::min(low[u],low[v]);
        }
        else low[u]=std::min(low[u],dfn[v]);
        if(dfn[u]<low[v])
            //add u->v
    }
    for(int i=last[u];i;i=E[i].nxt) {
        int v=E[i].to;
        if(v!=p[u] && p[v]!=u && dfn[u]<dfn[v])
            solveRing(u,v);
    }
}
ncnt=n;
\end{lstlisting}

还有一种简洁的建树方法:
参见\url{https://blog.csdn.net/qq\_36551189/article/details/81047872}

圆方树具有以下性质:
\paragraph{子仙人掌} 以$r$为根的仙人掌上点$p$的子仙人掌为去掉$p$到$r$的简单路径的边后
点$p$所在的连通块。
\begin{property}
    以$r$为根仙人掌上点$p$的子仙人掌对应圆方树上点$p$的子树。
\end{property}
\begin{property}
    圆方树不存在方点与方点的连边。
\end{property}

\subsubsection{应用}
\paragraph{最短路}
把圆圆边的权值设为原边权值,圆方边的权值设为当前点到方点父亲的最短距离
(可以维护到DFS树根的距离来计算),方圆边的权值设为0。
重新DFS准备树链剖分时计算即可。如此可以保证经过环的时候走的是最短路。

类比树上距离的做法,树链剖分后求LCA。
\begin{itemize}
    \item 若LCA为圆点,则按照树上距离的方法解决。
    \item 若LCA为方点,则说明该路径经过了环上的两个点,但走哪一侧还未知。
    首先计算询问的这两个点分别在哪棵子树中,计算两条路径的答案,接下来只要考虑
    环上路径。
    询问子树时需要对在LCA环上的祖先是否为重儿子进行分类,jump函数计算该点的编号:
    \begin{lstlisting}
    int jump(int u,int lca) {
        int res;
        while(top[u]!=top[lca]) {
            res=u;
            u=p[top[u]];
        }
        return u==lca?res:son[lca];
    }
    \end{lstlisting}
    预先在计算方圆边距离时标记走的方向,然后分类讨论计算环上两点的最短路。
\end{itemize}
\paragraph{点分治}
注意处理方点时的复杂度,一般将方点设为环的大小,点分治时找带权重心。
\index{*TODO!仙人掌典型应用}
\subsection{广义圆方树}
对于每个点双,将点双内的点(称为圆点)连到新点(称为方点)。
\begin{lstlisting}
int dfn[size], low[size], st[size], timeStamp = 0,
    top = 0, nsiz;
void tarjan(int u) {
    dfn[u] = low[u] = ++timeStamp;
    st[++top] = u;
    for(int i = g1.last[u]; i; i = g1.E[i].nxt) {
        int v = g1.E[i].to;
        if(dfn[v])
            low[u] = std::min(low[u], dfn[v]);
        else {
            tarjan(v);
            low[u] = std::min(low[u], low[v]);
            if(dfn[u] <= low[v]) {
                int s = ++nsiz, p;
                g2.addEdge(u, s);
                do {
                    p = st[top--];
                    g2.addEdge(s, p);
                } while(p != v);
            }
        }
    }
}
nsiz = n;
\end{lstlisting}
广义圆方树的点与边数的最大规模都是$2V$级别的(圆点与方点相间,
并且连成一棵树)。

关键在于考虑圆点和方点的权值和边权,注意更新方点时不要从计算父亲的贡献,
仅计算子树的贡献,在另外的计算过程中考虑父亲的贡献,以保证更新的复杂度。

上述内容参考了小蒟蒻yyb\footnote{仙人掌\&圆方树学习笔记
    \url{https://www.cnblogs.com/cjyyb/p/9098400.html}
}和immortalCO\footnote{圆方树——处理仙人掌的利器
    \url{http://immortalco.blog.uoj.ac/blog/1955}
}的博客。

\section{图上路径}
\subsection{欧拉回路}
\subsection{哈密尔顿回路}

\section{弦图}
\index{C!Chord}
\subsection{相关概念}
\subsubsection{团}
\index{C!Clique}
图$G$的子图$G'=(V',E')$,$G'$是$V'$的完全图。
\subsubsection{极大团}
\index{M!Maximal Clique}
一个不是其它团的子集的团。
\subsubsection{最大团}
\index{M!Maximum Clique}
点数最大的团,记团数为$\omega(G)$。
\subsubsection{最小染色}
\index{M!Minimum Coloring}
$\chi(G)$为使得相邻点颜色不同的最小颜色数。

\subsection{完美消除序列}
\subsection{最大势算法}
\subsection{弦图判定}
\subsection{弦图的极大团}
\subsection{弦图的点染色}
\subsection{弦图的最大独立集与最小点覆盖}
\subsection{弦图的点染色}
\subsection{区间图}
上述内容参考了陈丹琦在WC2009上的讲稿《弦图与区间图》\cite{chord}。

\section{带花树}
\index{B!Blossom Algorithm}
带花树主要用来解决无向图最大匹配与完美匹配问题。
\subsection{无向图最大匹配}
尝试使用匈牙利算法计算无向图最大匹配,即每次选取一个未匹配点,
以此为树根,DFS遍历出交错树。

定义交错树上的``奇点''为距树根距离为奇数的点,``偶点''为距树根
距离为偶数的点。

那么对于二分图有如下两种情况:
\begin{itemize}
    \item 奇点连到偶点:一定走匹配边。
    \item 偶点连到奇点:一定走未匹配边。
\end{itemize}

但是对于无向图来说有一点不同:
\begin{itemize}
    \item 奇点连到偶点:一定走匹配边。
    \item 偶点连到奇点:一定走未匹配边。
    \item 偶点连到偶点:一定走未匹配边。此时我们把交错树上的两个偶点的
    交错路径以及偶点到偶点的边形成的奇环称为``花'',树根称为``花托''。
\end{itemize}

考虑跨过偶点到偶点的边,可以发现原先的奇点可以变为偶点,然后可以走未匹配边延伸出
更多的交错路径。既然花上的点都可以是偶点,不妨将其直接缩为一个偶点,这就是``缩花''。
一朵花至少有3个点,因此最多缩花$\frac{|V|}{2}$次,这里使用并查集缩花。

其余步骤与匈牙利算法类似:

遍历未匹配点尝试建立交错树以寻找增广路径:

使用BFS遍历,队列仅维护偶点。
\begin{itemize}
    \item 若走到未访问点:
    \begin{itemize}
        \item 若该点已匹配:
        \item 若该点未匹配:找到增广路径,翻转链上边的匹配状态后返回true。
    \end{itemize}
    \item 若走到已访问点:
    \begin{itemize}
        \item 若该点为奇点:由于该点已访问,什么也不要做。
        \item 若该点为偶点:缩花,参数为两个偶点标号$u,v$。
    \end{itemize}
\end{itemize}

缩花:
\begin{enumerate}
    \item 使用DSU找出花托,即$LCA(u,v)$。
    \item 两遍调用缩花的一边并把奇点标记为偶点,加入队列。
\end{enumerate}

假设并查集复杂度为$O(1)$,算法时间复杂度$O(|V||E|)$。
二分图最大匹配中的Greedy Matching仍然适用。

模板:
\lstinputlisting{Source/Templates/Blossom.cpp}

\subsection{Micali-Vazirani Algorithm}
\index{M!Micali-Vazirani Algorithm}
该算法也用于求无向图最大匹配,时间复杂度$O(\sqrt{|V|}|E|)$。

由于找不到中文资料,暂时先坑着。
参考Silvio Micali与Vijay V. Vazirani的论文\cite{MMG}。
\subsection{近线性复杂度的随机匹配算法}
这里介绍的是2011年由Anant Jindal, Gazal Kochar, Manjish Pal
提出的基于马尔科夫链和Glauber Dynamics的随机算法,参见论文\cite{AGM}。

算法步骤如下(调用下述算法$10\lg n$次):
\begin{enumerate}
    \item 令$M_0$为任意匹配,$p=2^{|E|}$。
    \item 迭代$k=10|E|\lg|V|$次更新$M_i$。
    \begin{enumerate}
        \item 随机取边$e$;
        \item 令$M'$有$\frac{p}{1+p}$的概率为$M_i\cup \{e\}$,
        有$\frac{1}{1+p}$的概率为$M_i \backslash \{e\}$。
        \item 若$M'$为匹配则令$M_{i+1}=M'$,否则$M_{i+1}=M_i$。
    \end{enumerate}
    \item 输出$M_k$。
\end{enumerate}

时间复杂度$O(|E|\lg^2|V|)$,算法正确率为
$1-\left(\frac{169}{189}\right)^{10\lg n}$。

模板(不正确,待修复):

为了防止错过最优解,我还加入了备份操作,最坏时间复杂度$O(|V|^2)$。

\lstinputlisting{Source/Templates/AGM.cpp}

\index{*TODO!最大匹配随机算法实现}
\subsection{无向图最大权匹配}

上述内容参考了江任捷的演算法筆記\footnote{
    演算法筆記 - Matching
    \url{http://www.csie.ntnu.edu.tw/\~u91029/Matching.html}
}、permui\footnote{
    一般图最大匹配-带花树算法
    \url{https://www.cnblogs.com/owenyu/p/6858508.html}
}和hcsoso\footnote{
    Maximum matching in general graphs
    \url{https://finiteplayground.wordpress.com/2011/07/15/
    maximum-matching-in-general-graphs/}
}的博客。

\section{支配树}
\index{D!DominatorTree}
\subsection{定义}
支配树基于某个原点$s$,若$s$到某个点$v$的节点必定经过$u$,则称$u$支配$v$。
起点$s$与自己$v$称为平凡支配点。

若$v$的某个支配点$w$满足其被$v$的其余非平凡支配点支配,则$w$为$v$的最近支配点,
记作$idom(v)=w$。每个点到自己的$idom$连边,则构造出了一棵树,称为支配树。
\subsection{DAG的支配树}
DAG的支配树构造较为简单。考虑按照top序加入节点,当前节点的$idom$就是其前驱节点在支配树上的LCA。
由于树的形态是固定的,使用倍增可以$O((n+m)\lg n)$实现。
\subsection{一般图的支配树}
该内容留坑待补。
\index{*TODO!一般图支配树}

上述内容参考了MoebiusMeow的博客\footnote{
    康复计划\#4 快速构造支配树的Lengauer-Tarjan算法\\
    \url{https://www.cnblogs.com/meowww/archive/2017/02/27/6475952.html}
}。

\section{杂讲}
\subsection{竞赛图}
\index{T!Tournament}
竞赛图是一个无向完全图被定向后得到的图。
\begin{theorem}
	竞赛图缩点后是一条链。
\end{theorem}
\subsubsection{竞赛图判定}
竞赛图可以用来指示两两选手比赛的胜负,判定比分是否
合法即判定是否存在合法的竞赛图。
\index{L!Landau's Theorem}
\begin{theorem}[Landau's Theorem]
	对于一个有序的竞赛图度数序列/得分序列$0\leq s_1 \leq s_2 \leq \cdots \leq s_n$,
	有$\displaystyle \forall 1\leq k\leq n,\sum_{i=1}^k{s_i}\geq \binomial{k}{2}$
	,当$k=n$时等号必须成立。
\end{theorem}
对度数/胜利场数排序后逐个判断即可。
\subsection{最小平均值环}
对于一个有向图,找出平均值最小的环。

类似于分数规划的思想,对平均值进行二分,将所有边权减去二分值,
若存在负环则说明存在环的平均值小于该二分值,不断二分即可。
\subsection{平面图性质}
\index{P!Planar Graph}
以下仅讨论$V\geq 3$的情况:
\begin{property}
	$E\leq 3V-6$
\end{property}
\begin{property}
	$F\leq 2V-4$
\end{property}
使用这些性质可以限制边数以加速平面图判定。

上述内容参考了Wikipedia-EN\footnote{Planar graph - Wikipedia
	\url{https://en.wikipedia.org/wiki/Planar\_graph}
}
\subsection{拓扑排序判环}
若拓扑排序无法使得所有点都入队则说明存在环。
可以通过枚举点,将其度数-1取得删掉一条边的效果。
例如CF915D Almost Acyclic Graph:
\lstinputlisting{Source/Source/TopSort/CF915D.cpp}
\subsection{Lindström–Gessel–Viennot Lemma}
\index{L!Lindström–Gessel\\–Viennot Lemma}
给定一个DAG,以及$n$个起点$a_1,a_2,\cdots,a_n$和对应终点$b_1,b_2,\cdots,b_n$,
求这$n$条点不相交(包括终点)路径的方案数。

根据Lindström–Gessel–Viennot Lemma,记$e(a_i,b_j)$为$a_i\rightarrow b_j$的
路径方案数,答案为\begin{displaymath}
	det\left(\left[\begin{array}{cccc}
			e(a_1,b_1) & e(a_1,b_2) & \cdots & e(a_1,b_n) \\
			e(a_2,b_1) & e(a_2,b_2) & \cdots & e(a_2,b_n) \\
			\vdots     & \vdots     & \ddots & \vdots     \\
			e(a_n,b_1) & e(a_n,b_2) & \cdots & e(a_2,b_n) \\
		\end{array}\right]\right)
\end{displaymath}
\subsubsection{推广}
实际上$e(a_i,b_j)$为$a_i\rightarrow b_j$的所有路径上边权积之和,
类似Matrix-Tree定理的讨论可扩展到边权相关问题。

上述内容参考了Wikipedia-EN\footnote{
	Lindström–Gessel–Viennot lemma - Wikipedia
	\url{https://en.wikipedia.org/wiki/Lindstr\%C3\%B6m\%E2\%80\%93Gessel\%E2\%80\%93Viennot\_lemma}
}。


\chapter{字符串}
\minitoc
\section{Hash}
\subsection{BKDRHash}
BKDRHash在一般情况下表现良好,速度快,Hash质量高。
代码如下:
\begin{lstlisting}
typedef unsigned long long HashT;
HashT BKDRHash(const char*  str) {
    HashT res=0;
    while(*str) {
        res=res*131+*str;
        ++str;
    }
    return res;
}
\end{lstlisting}
\subsection{混合Hash}
若要快速求得子串Hash,预处理前缀Hash值,查询时化到同一幂次再差分。

若要判断两个串是否为同一子串或对称子串,可以同时预处理后缀Hash值,
然后将正向Hash值与逆向Hash值相乘得到新的Hash值。

若要忽略子串的某一连续区间,可以将其前缀的前缀Hash与后缀的后缀Hash
加权混合([CTSC2014]企鹅QQ)。
\subsection{子串Hash}
有时会遇到与不重复连续子串的问题,可以$O(n\lg n)$枚举(调和级数)。
一般来说二分长度+Hash是个不错的思路。

\section{Trie字典树}
Trie字典树利用了字符串的公共前缀信息,一般用作
搭配AC自动机或者实现可持久化01Trie回答xor最值问题。

Trie树上每个节点对应一个字符串的某个前缀,节点的LCA为
这两个节点所代表字符串的最大公共前缀。

代码如下:
\begin{lstlisting}
struct Node{
    int nxt[26],cnt;
} T[tsiz];
void insert(const char* str) {
    static int icnt=0;
    int p=0;
    while(*str) {
        int& id=T[p].nxt[*str-'a']
        if(id==0)
            id=++icnt;
        p=id;
        ++str;
    }
    ++T[p].cnt;
}
\end{lstlisting}

有时Trie的空间需求很大,此时需要考虑压缩Trie的空间,尤其是01Trie。

以01Trie的值$\leq 2^{30}$,串数$n\leq 10^6$,单个节点16Byte为例:

保守估计,开大小$nL$的Trie:457MB。

注意到最坏情况下是$2^{19}$层被填满,剩下一堆长链,此时数组大小为$2^{20}+11n$:183MB。

如果剩余链的答案很好统计,那么干脆将无分支的长链压缩为一点,该标记往往不需要多余空间。若遇到
新分支才把该剩余链向下推。此时数组大小为$2^{20}+n$:31MB。这个方法需要更多的代码,其实现要求精细。

由此可见,Trie的空间优化潜力很大,但需要冒不小的风险(比如惰性扩展的代码写挂了)。

\section{AC自动机}
\index{A!Aho–Corasick Algorithm}
AC自动机基于Trie树实现,原理类似于KMP算法,即在Trie树上匹配字符串,
失配时根据fail指针跳到下一匹配位置。
\subsection{构造}
首先建出Trie,然后对Trie按BFS序确定fail指针,fail指针指向它的在Trie中的
最长后缀(根据Trie树的性质,这个后缀也是某个模式串的前缀),BFS序遍历可以
保证其长度最长。

为了简化找fail指针的代码,可以设一个虚拟节点0,root编号为1,令root的
fail为0,0的所有儿子全设为root,这样就可以避免判断是否到达root,
同时令找不到后缀的节点的fail为root。

\begin{lstlisting}
const int root=1;
int q[size];
void cook() {
    for(int i=0;i<26;++i)
        T[0].nxt[i]=root;
    T[root].fail=0;
    int b=0,e=1;
    q[b]=root;
    while(b!=e) {
        int u=q[b++];
        for(int i=0;i<26;++i) {
            int v=T[u].nxt[i];
            if(v) {
                int p=T[u].fail;
                while(!T[p].nxt[i])
                    p=T[p].fail;
                T[v].fail=T[p].nxt[i];
                q[e++]=v;
            }
        }
    }
}
\end{lstlisting}

还有一种更加简洁快速的cook过程。让无儿子$v$的节点$u$继承其fail的对应儿子,
省去了在Trie上不断跳fail指针的时间。这种cook方式不会影响fail树的建立。
\begin{lstlisting}
int q[size];
void cook() {
    int b = 0, e = 0;
    for(int i = 0; i < 26; ++i)
        if(T[0].nxt[i])
            q[e++] = T[0].nxt[i];
    while(b != e) {
        int u = q[b++];
        for(int i = 0; i < 26; ++i) {
            int& v = T[u].nxt[i];
            if(v) {
                T[v].fail = T[T[u].fail].nxt[i];
                q[e++] = v;
            } else
                v = T[T[u].fail].nxt[i];
        }
    }
}
\end{lstlisting}
\subsection{查询}
AC自动机的经典功能是多模匹配。
将主串的字符按顺序在自动机上跳,失配就走fail指针,
对于每一次匹配后的状态,不断跳fail到根查找是否存在
节点为某个模式串。
\begin{lstlisting}
void match(const char* str) {
    int p=0;
    while(*str) {
        int c=*str-'a';
        //`注意如果使用继承fail优化就不需要跳fail了`
        while(p && !T[p].nxt[c])
            p=T[p].fail;
        p=T[p].nxt[c];
        //`查找匹配的模式串`
        int cp=p;
        while(cp) {
            if(T[cp].cnt)
                //`处理匹配字符串`
            cp=T[cp].fail;
        }
        ++str;
    }
}
\end{lstlisting}
继承方式不影响查询的正确性(继承fail节点的儿子相当于跳到fail上),
而且同样避免了跳fail。
\subsection{fail树}
容易发现fail指针所指的节点所代表的字符串是自己的最长公共后缀。
从fail向自己连边,可以得到一棵fail树。将代表某个模式串的节点称为终结点,
那么fail树上某个终结点是它子树内的终结点的后缀。

\paragraph{例题~NOI2011 阿狸的打字机}
对于多次询问求第$x$个字符串在第$y$个字符串中出现次数的问题,模拟字符串$y$在AC
自动机上跑的过程,查询匹配时跳fail扫一遍相当于在fail树上将自己的权值传给父亲。
因此先建出fail树,按DFS序平铺为序列,(尽量按照字典序)让$y$在AC自动机上跑,
记录节点被经过的次数,最后求询问中所有$x$的子树权值和,使用树状数组很容易实现。

\section{Huffman编码}
\subsection{常规Huffman}
\subsection{k叉Huffman}
\section{KMP算法}
\index{K!Knuth–Morris–Pratt Algorithm}

KMP算法通过预处理nxt数组来避免重复匹配。考虑在长度为$i$的模式串前缀$S_i$
的非自身前缀$P_j$中,满足$P_j$是$S_i$后缀的$P_j$。定义$nxt[i]$为最长的满足条件的
$P_j$长度。

以下字符串下标从0开始,nxt数组下标从1开始。
\subsubsection{预处理}
可以利用之前的预处理信息跳nxt来找到最长后缀。
\begin{lstlisting}
int nxt[size];
void cook(const char* P) {
    int p=0;
    nxt[1]=0;
    for(int i=1;P[i];++i) {
        while(p && P[p]!=P[i])
            p=nxt[p];
        if(P[p]==P[i])
            ++p;
        nxt[i+1]=p;
    }
}
\end{lstlisting}
\subsubsection{匹配}
匹配和预处理的过程十分相似。
\begin{lstlisting}
void match(const char* str,int len) {
    int p=0;
    for(int i=0;str[i];++i) {
        while(p && P[p]!=str[i])
            p=nxt[p];
        if(P[p]==str[i])
            ++p;
        if(p==len)
            //match str[i-len+1...i]
    }
}
\end{lstlisting}

有时nxt数组会被用来辅助dp转移,构造出dp转移方程后使用矩阵快速幂加速。

ExKMP已被更好理解的Z Algorithm取代,故不再补充该内容。
Z Algorithm参见第~\ref{ZA}节。

\section{Manacher算法}
\index{M!Manacher's Algorithm}
Manacher算法用来求解最长回文串问题。主要思想是利用之前的计算结果
来加速回文串的计算,从而达到线性时间复杂度。

算法步骤如下:
\begin{enumerate}
    \item 为了统一奇回文串和偶回文串,向每对相邻字符间插入一个未出现过的字符,
    接下来算法仅讨论奇回文串;
    \item 维护当前访问到的最右位置$maxr$和最右位置所对应的字符串中心$pos$,
    以及以每个位置为中心向右扩展长度$RL[i]$(从中心开始数);
    \item 对于每一个位置:
    \begin{enumerate}
        \item
        \begin{itemize}
            \item 若当前位置$i\geq maxr$,设$RL[i]=1$;
            \item 否则令$RL[i]=min(RL[pos-(i-pos)],maxr-i+1)$(因为
            此时$i$到$maxr$的部分和$i$以$pos$为轴的对称部分对称,而那个
            部分已经被处理过了)。
        \end{itemize}
        \item 不断向两端扩展增大$i$;
        \item 更新$maxr$与$pos$。
    \end{enumerate}
    \item 答案即为$RL[i]-1$的最大值。
\end{enumerate}

\paragraph{小trick}
\begin{itemize}
    \item 在字符串开头再加另一个特殊字符,可以不用越界检查。
    \item 令maxr为当前匹配最右边的位置的右边一位,避免+1-1的麻烦。
\end{itemize}

这两个trick源自小蒟蒻yyb的博客\footnote{
    【BZOJ3160】万径人踪灭(FFT,Manacher)
    \url{https://www.cnblogs.com/cjyyb/p/8435460.html}
}。

代码如下:
\begin{lstlisting}
char buf[size],str[2*size];
int RL[2*size];
int manacher() {
    int cnt=0;
    str[cnt++]='#';
    str[cnt++]='@';
    for(int i=0;buf[i];++i) {
        str[cnt++]=buf[i];
        str[cnt++]='@';
    }
    int maxr=0,pos=0,ans=0;
    for(int i=1;i<cnt;++i) {
        RL[i]=(maxr>i?std::min(RL[2*pos-i],maxr-i):1);
        while(str[i-RL[i]]==str[i+RL[i]])
            ++RL[i];
        if(i+RL[i]>maxr) {
            maxr=i+RL[i];
            pos=i;
        }
        ans=std::max(ans,RL[i]);
    }
    return ans-1;
}
\end{lstlisting}

\section{回文自动机}
\index{P!Palindromic Tree}
\subsection{构造}
回文自动机需要维护每个节点所对应的回文串的长度$len$,两端加入字符$c$
后的后继节点$nxt[c]$,自身的最长后缀回文串所对应的节点$fail$,以及
自身代表的回文串数(需要最后在fail树上上传才是完整的)。
还要维护当前已加入自动机的字符$buf$,以及最后一次加入字符后的状态$last$。

构造PAM的复杂度为$O(n\lg |\Sigma|)$。
\subsubsection{初始化}
首先在空PAM中加入两个根:偶数长度的根0和奇数长度的根1。其中节点0的fail
指向1,len为0,节点1的len为-1(避免特判)。同时令$buf[0]=-1$(避免特判)。
\subsubsection{状态转移}
构造PAM时,按顺序向PAM加入字符。
首先向$buf$加入该字符,然后在fail树上跳最长后缀回文串,直至找到对称点
与自身相同为止。注意如果找不到这样的对称点,就会到达奇数长度的根,它的len
为-1,其对称点就是自己,所以迭代必定会结束。

接下来查看是否有该节点的后继节点,如果没有就新建一个节点,len比父亲多2,
fail指针重新在父亲的fail链上找。然后把该节点挂在父亲下。

最后重置last,++T[last].cnt结束加入。
\subsubsection{后处理}
注意自身的fail肯定是自己的子回文串,最后做一次后处理,按照节点编号的逆序
往上累加cnt即可。
\subsubsection{代码}
\lstinputlisting{Source/Templates/PAM.cpp}
\subsection{应用}
\subsubsection{统计串S的前缀本质不同的回文串个数}
extend该前缀的所有字符后,PAM的$siz-1$就是本质不同的回文串个数。
\subsubsection{统计串S的每个本质不同的回文串的出现次数}
由于PAM中每个节点代表一个回文串,后处理后每一个节点的
cnt就是它的出现次数。
\subsubsection{统计串S的回文串的个数}
答案即为后处理后的cnt之和。
\subsubsection{统计以节点$i$结尾的回文串个数}
对每个节点记录其在fail树上的深度,fail链上的节点所代表的回文串都是
自己的后缀回文串,所以答案即为深度。

以上内容参考了poursoul的博客\footnote{
    Palindromic Tree——回文树【处理一类回文串问题的强力工具】
    \url{https://blog.csdn.net/u013368721/article/details/42100363}
}。

\section{后缀树}
\index{S!Suffix Tree}
后缀树是由串S的所有后缀组成的压缩Trie。

为了避免压缩过程隐藏了某个后缀,插入每个后缀后再
插入一个不出现的字符。
\subsection{构造}
常规的构造方法是$O(n^2)$的,一般采用Ukkonen算法在线性时间内完成构造。
\index{U!Ukkonen's Algorithm}
\index{*TODO!线性构造后缀树}
留坑待补(还是去学后缀仙人掌吧)。
\subsection{广义后缀树}
广义后缀树在插入不同的串时用的结尾字符不同,以区分不同的字符串。

查询最长公共子串时找到最深的拥有2种结束标记的节点即可。
\subsection{应用}
后缀树满足如下性质:
\begin{itemize}
    \item 每个节点代表一个子串。

    查询串P在串S的出现次数可以按照Trie的方法匹配,然后
    统计其所在节点的子树中的叶子节点个数。
    \item 每个非叶节点至少有两个儿子。

    统计串S的最长重复子串只要找到其最深非叶子节点即可。
\end{itemize}
以上内容参考了的博客\footnote{
	从Trie树(字典树)谈到后缀树(10.28修订)
	\url{https://blog.csdn.net/v\_july\_v/article/details/6897097}
}。

\section{后缀数组}
\index{S!Suffix Array}
\subsection{倍增构造}
后缀数组$SA[i]$表示排名为$i$的后缀位置,相应地可以
得到起始位置为$i$的后缀排名$rk[i]$。

数组$rk$一般使用倍增法$O(n\lg n)$构造,然后对应地初始化$SA$。

构造步骤如下:
\begin{enumerate}
    \item 对于每一个字符初始化其排序关键字(即字符);
    \item 对关键字进行基数排序,算出此时字符串$[i\ldots i+2^k-1]$的排名;
    \item 若所有排名都不相同,直接跳出;否则将$rk[i]$与$rk[i+2^k]$作为以$i$
    为起始位置的字符串的关键字,重复步骤2;
    \item 根据$rk$数组初始化$SA$数组。
\end{enumerate}
代码如下:
\lstinputlisting{Source/Templates/SA.cpp}
\subsection{最长公共前缀}
\index{L!Longest Common Prefix}
定义$LCP(i,j)$为第$i$个后缀与第$j$个后缀(指排名而不是位置)的最长公共前缀。

显然LCP函数有两个性质:
\begin{itemize}
    \item $LCP(i,j)=LCP(j,i)$
    \item $LCP(i,i)=n-SA[i]+1$
\end{itemize}
所以只要考虑$LCP(i,j),i<j$的情况。
\begin{theorem}[LCP Theorem]
    \begin{displaymath}
        LCP(i,j)=min\{LCP(k-1,k)\},i<k\leq j
    \end{displaymath}
\end{theorem}
要证明该定理可先证明下列引理:
\begin{lemma}[LCP Lemma]
    \begin{displaymath}
       LCP(i,j)=min(LCP(i,k),LCP(k,j)),i<k<j
    \end{displaymath}
\end{lemma}
\paragraph{证明}
首先根据传递性显然有$LCP(i,j)\geq min(LCP(i,k),LCP(k,j))$。

其次考虑到$i,j,k$是后缀的排名且$i<k<j$,有$LCP(i,j)\leq LCP(i,k)$且
$LCP(i,j)\leq LCP(k,j)$,即$LCP(i,j)\leq min(LCP(i,k),LCP(k,j))$。
该引理得证。

因此设$height[i]=LCP(i-1,i),height[1]=0$,求出height数组后构建ST表$O(1)$询问LCP。

按照原串顺序预处理$height$数组。考虑在原串上相邻后缀的关系,显然两个后缀右移1位的
$LCP'\geq LCP-1$,用此性质快速转移后继续向后暴力匹配。

\begin{lstlisting}
void cook(int n) {
    int h=0;
    for(int i=1;i<=n;++i) {
        if(rk[i]==1)h=0;
        else {
            int k=SA[rk[i]-1];
            if(h)--h;
            while(buf[i+h]==buf[k+h])
                ++h;
        }
        height[rk[i]]=h;
    }
}
\end{lstlisting}
以上内容参考了Angel\_Kitty的博客\footnote{后缀数组(一堆干货) - Angel\_Kitty
    \url{https://www.cnblogs.com/ECJTUACM-873284962/p/6618870.html}
}。
\subsection{应用}
\subsubsection{一般思路}
\begin{itemize}
    \item 二分LCP长度,对height数组进行分组。
    \item 若遇到多串则将其用未出现字符连接后求后缀数组。
    \item 对于回文串/翻转系列问题则将其与反串用特殊字符相连后求后缀数组。
    \item 连续重复子串问题使用错位匹配解决。
    \item 「TJOI / HEOI2016」字符串:若LCP的一个端点固定,可以从它开始在height
    数组上左右暴力遍历,利用min单调性剪枝。
\end{itemize}
\subsubsection{可重叠最长重复子串}
该子串一定是某两个后缀的LCP,而LCP在height数组中取最小值,因此
答案为height最大值。
\subsubsection{不可重叠最长重复子串}
二分LCP长度k,按k对height数组进行划分,满足每块内的height值
$\geq k$,判断是否存在块内$SA[i]$的极差$\geq k$(此时子串不重叠)。
\subsubsection{可重叠k次最长重复子串}
二分LCP长度,对其分组,询问是否存在大小$\geq k$的块。
\subsubsection{本质不同的子串个数}
考虑按照后缀字典序加入每个后缀的前缀,每个后缀贡献了$n-SA[i]+1$个前缀,
去掉重复的$height[i]$个重复前缀。答案为每个后缀的贡献之和。可以把这两部分
分开考虑,答案为子串数-height数组和。
\subsubsection{最长回文子串}
将串与反串用未出现字符连接后求后缀数组,按照长度奇偶分类讨论(或者类似于Manacher算法
处理原串),枚举对称中心,求以其为首的后缀与以其在反串上的对称位置为首的后缀的LCP。
\subsubsection{连续重复子串}
已知字符串$S$由某个字符串多次重复得到,求最大重复次数。

枚举串长$n$的因子$k$,询问字符串$[1\ldots n-k]$与$[k+1\ldots n]$是否
相等,即判断$LCP(rk[1],rk[k+1])=n-k$是否成立。由于LCP的一端是固定的,
没有必要构建ST表支持RMQ,可以直接$O(n)$扫描处理。
\subsubsection{重复次数最多的连续重复子串}
首先枚举连续重复子串长度$L$,仅考虑重复2次以上的情况,那么整个连续重复串
每隔$L$个必相同,可以枚举起始位置$L*i,L*(i+1)$错位匹配求LCP。若起点
不为$L$的倍数,尝试计算LCP判断左边剩余部分是否相等。
\subsubsection{最长公共子串}
将两个串拼接在一块,求满足对应后缀起点来自不同字符串的height最大值。
\subsubsection{长度$\geq k$的公共子串数(可以相同)}
按$k$对height划分,在块内统计每个后缀之前的来自另一个字符串的后缀
与该后缀产生的贡献,这里可以用单调栈维护以当前位置结尾的height后缀最小值之和。分别对
两个串的后缀扫一遍累积贡献即为答案。
\subsubsection{出现于不少于$k$个字符串的最长子串}
二分长度对height分组,然后检查每组内是否存在来自$k$个字符串的后缀。
\subsubsection{在每个字符串重复但不重叠的最长子串}
二分长度对height分组,对每个字符串检查重复且不重叠的条件。
\subsubsection{出现或翻转后出现在每个字符串中的最长子串}
将每个串的原串+反串连起来,二分长度对height分组,判断是否满足存在来自所有字符串的后缀。

以上内容参考了罗穗骞的论文《后缀数组——处理字符串的有力工具》。

\section{后缀仙人掌}
\index{S!Suffix Cactus}
\subsection{概述}
后缀仙人掌将后缀Trie的每个非叶子结点与它的一个儿子合并,
形成``树枝''。每个树枝表示的是从根节点到自身叶子节点的后缀。
树枝的深度为其顶端节点的父节点的深度,根树枝的深度为0。记后缀排名为$s$
的后缀为$s$,深度为$depth(s)$。一个树枝的父亲树枝为包含其顶端节点左兄弟
的树枝。

后缀仙人掌有如下性质:
\begin{property}
	$depth(s)=LCP(SA[s],SA[s-1])$
\end{property}
显然$depth(s)$是它和它前一个后缀的最小公共祖先的深度。
$depth(s)$其实就是$height[s]$。
\begin{property}
	树枝$r(r>1)$的父亲树枝是满足$depth(s)\leq depth(r),s<r$的
	编号最大的树枝$s$。
\end{property}
\subsection{构造}
根据$depth(s)$与$height[s]$的联系,首先预处理后缀数组与$height$数组。

然后按照字典序从小到大考虑,维护一个单调栈,栈内存储可能的父亲树枝编号,
找到需要的父亲树枝$k$后将边$(k,depth(s))=s$插入HashTable。

代码如下:
\begin{lstlisting}
int st[size];
void buildSC(int n) {
    int top=1;
    st[top]=1;
    for(int i=2;i<=n;++i) {
        while(height[st[top]]>height[i])
            --top;
        addEdge(st[top],height[i],i);
        st[++top]=i;
    }
}
\end{lstlisting}
\subsection{应用}
多次询问串T的前缀为串S的子串的最大长度。

维护当前所在树枝$id$,与匹配长度$d$。每字符匹配时尽可能沿树枝走,
走不到就跳到子树枝,若没有子树枝则返回。
\begin{lstlisting}
int match(const char* str,int n) {
    int id=1,d=0,i;
    for(i=0;str[i];++i) {
        char c=str[i];
        bool flag=false;
        while(sa[id]+d>n || buf[sa[id]+d]!=c) {
            int nxt=find(id,d);
            if(!nxt) {
                flag=true;
                break;
            }
            id=nxt;
        }
        if(flag)
            break;
        ++d;
    }
    return i;
}
\end{lstlisting}
上述内容参考了WC2014营员交流课件《Suffix Cactus》。

\section{后缀自动机}
\index{S!Suffix Automaton}
\subsection{描述}
后缀自动机(SAM)可用来识别母串的后缀,使用最简状态表示。SAM是一个DAG,任何从初始状态出发的路径
对应母串的某个子串。与其他自动机不同的是,SAM中一个状态对应多个子串。

\subsection{构造}
假设现在已经构造出了串$S$的SAM,考虑如何构造串$Sx$的SAM以识别新的子串。如果能够快速完成这个任务,
就可以逐位将字符插入SAM以构造出整个串的SAM。

为了加强理解,这里不直接按照最终的SAM描述,而是需要什么元素加入什么元素。\CJKsout{兵来将挡,水来土掩。}
记$SAM(P)$为串$P$的SAM,$S_P(i)$表示$P$的第$i$个后缀。

首先用根节点表示空串,记为$R$。

考虑由$SAM(P)$构造出$SAM(Px)$,添加字符$x$使得当前串新增后缀$x,S_P(1)x,\\\cdots,S_P(|P|)x$。那么
可以考虑新增一个节点,将$SAM(P)$表示后缀的节点向它连一条$x$转移边。注意到上一次增加的节点
(记为$last(P)$)表示了$S_P(1),\cdots,S_P(|P|)$,需要连边的只有与$R$与$last(P)$。此时每个节点
需要存储转移边$nxt[|\Sigma|]$。

注意到每个节点的相同字母转移边只能有一条,而上述方法中节点$R$一直在连边,当已有转移边时就无法转移了。
考虑向串$ab$后加字符$a$,$R$已有字符$a$的转移边。注意到若不连该边,原有的连边可以识别后缀$a$,不过
下次转移时表示后缀的节点不再是$R$与$last(P)$,而是$R,last(P)$与表示$a$的节点。可以考虑用一条链将它们
串起来,从$last(Px)$开始按照最长串长度连向上一个表示后缀的节点,此时每个节点需要存储该节点的$link$。

如何程序化地描述向新点$id$连转移边的过程呢?那就是从$last(P)$开始,若该节点没有转移边则将其连向新点,然后
根据$link$指针跳到下一个表示后缀的节点。若跳到$R$处也没有转移边,则设置新点的$link=R$,更新$last(Px)=id$,
退出程序。否则说明$SAM(P)$已经能够识别该后缀,只需设置$id$的$link$。记当前有转移边的节点为$p$,转移边
指向$q$。注意到节点$q$不一定表示$p$表示的后缀+字符$x$,需要进行判断。

考虑按照最长串长度排列的$link$链,既然链上所有节点对应了所有后缀,且链上节点代表的最长串长度从$last(P)$
递减,那么每一个节点代表的后缀长度是一个连续区间,且区间之间无缝。此外节点还有一个性质:节点编号+串长度
唯一对应一个子串。证明:若节点编号与串长度相同,而代表的子串不同,则说明这两条不重叠的路径等长,将这两条路径
从当前节点延伸到某个叶子,对应了两个长度相等且不相同的后缀,与事实矛盾。

有了这两个性质,可以推出一个结论:若节点$q$的最长串长度$len_q$恰好等于节点$p$的最长串长度$len_p+1$,
则节点$q$及其在$link$链上往$R$的方向的元素代表了剩余的新后缀
($S_P(0\cdots len_p)+x\rightarrow S_{Px}(1 \cdots len_q)$,还有链头$R$代表空串,而长度
$\geq len_p+2$的后缀已经在连转移边后可被$id$识别)。此时每个节点需要存储最长串长度$len$。若
$len_p+1=len_q$,则令$link_{id}=q$。否则考虑分割出我们需要的后缀,即把节点$q$代表的子串切割
为两部分,其中一部分的最长串长度为$len_p+1$。可以新建一个节点$cq$,由于切出的这一个子串也能转移到
新的节点,所以需要拷贝$q$的转移边。由于$len_{cq}<len_q$,需要修改$link$链为
$link_q\leftarrow cq \leftarrow q$,最后将$id$的$link$指向$cq$。注意分割后原来转移边指向$q$的
节点$p$及它在$link$链上的节点都应该改指向$cq$,因为原本它们指向的就是这一部分子串(只是因为未分离前
合并到指向$q$)。无论是修改转移边到$id$还是$cp$,都是从$last(P)$或$p$开始的连续子链,因为之前修改
转移边的操作影响的都是连续子链。

SAM的点数不超过2n-2,边数不超过3n-3(转移边+Parent树边),构造复杂度为$O(n|\Sigma|)$,证明参见
文末引用。

代码如下:
\begin{lstlisting}
struct SAM {
    struct State {
        int nxt[26],link,len;
    } S[size*2];
    int last,siz;
    SAM():last(1),siz(1) {}
    void extend(int c) {
        int id=++siz;
        S[id].len=S[last].len+1;
        int p=last;
        while(p && !S[p].nxt[c]) {
            S[p].nxt[c]=id;
            p=S[p].link;
        }
        if(p) {
            int q=S[p].nxt[c];
            if(S[p].len+1==S[q].len)
                S[id].link=q;
            else {
                int cq=++siz;
                S[cq]=S[q];
                S[cq].len=S[p].len+1;
                while(p && S[p].nxt[c]==q) {
                    S[p].nxt[c]=cq;
                    p=S[p].link;
                }
                S[q].link=S[id].link=cq;
            }
        }
        else S[id].link=1;
        last=id;
    }
}
\end{lstlisting}


\subsection{Parent树的应用}
将$link$链并在一起,就得到了一棵树,称为Parent树。
\subsubsection{Right集合}
节点$u$表示的所有子串的结束位置集合称为节点$u$的Right集合。

\subsubsection{Parent树性质}
\begin{theorem}
    父节点的Right集合是儿子Right集合的并。
\end{theorem}

证明:父节点表示的子串同时也是儿子表示的子串的后缀。

\begin{theorem}
    不在同一条$link$链上的节点Right集合不相交。
\end{theorem}

证明留坑待补。
\index{*TODO!SAMRight集合性质证明}

这个性质保证了可持久化线段树合并的复杂度。

Parent树自底向上Right集合逐渐变大,匹配子串长度逐渐变小。据此性质可贪心倍增跳到满足条件的最高的祖先
后利用该位置的数据查询。
\subsubsection{子串匹配}
注意Parent树上的父亲是儿子的后缀,因此匹配子串时可以
在转移边上跑,失配就跳Parent树的link(等同于fail树)。
\subsubsection{与后缀树的联系}
Parent树是反串的后缀树,因为父亲是儿子的后缀,等同于父亲的反串是儿子反串的前缀,且该树可以识别
反串的所有后缀。
\subsubsection{计数问题}
对于每一个状态维护一个$right$值表示当前状态的Right集合大小。
新增状态时该状态贡献了1,但注意克隆状态并没有贡献,所以克隆后令$cq.right=0$。
最后拓扑排序dp在Parent树上自底向上更新就可以得到真实right值。

状态$s$表示了$s.len-s.link.len$个本质不同的子串,每种子串有$s.right$个。

优化:拓扑排序时可以按照len进行分层基数排序,但是广义SAM不能使用这种方法。
\subsubsection{可持久化线段树合并维护Right集合}
有时需要判定某个终点是否在某个状态的Right集合内,可以在extend时给新建状态添加
对应的Right值,然后拓扑排序进行线段树合并计算出
每个状态真正的Right集合。时间复杂度$O(n\lg n)$。

{\bfseries 若该方法用于匹配母串的某个子串的子串,在失配时沿着Parent树跳跃,注意要逐位跳,当
匹配长度等于父亲的最长后缀时才跳到父亲。例题:NOI2018 你的名字}
\subsubsection{倍增自匹配}
例题~ 「TJOI / HEOI2016」字符串

通过前缀$\rightarrow$后缀以及二分LCP长度将原问题转换为:给定子串$a$与$b$,
求长度为$m$的$b$的后缀是否出现在子串$a$中。

由于$a$与$b$在同一个串内,该问题即为判定Right集合中含有$b$的Right值且$r_S\geq m$
的状态$S$与$a$对应的Right区间是否有交。可以在构建SAM时保存每个字符对应的$last$,这些
状态一定含有对应字符的Right值。由于Parent树上Right集合的包含关系,其祖先也有该Right值
且Right集合更大,$r_S$递减,可以使用倍增计算出极大的Right集合所对应的状态(要求有$b$的
Right值且$r_S\geq m$),然后线段树查询该状态的Right集合是否与指定区间有交。

代码:
\lstinputlisting{Source/Source/SAM/LOJ2059.cpp}
\subsection{线性构造后缀数组}
首先构造出SAM,发现last到根的链上的状态分别代表每一个后缀。对这些状态进行
标记,按照字典序DFS,维护DFS子串的长度$d$,通过遍历顺序得到$sa$数组。

注意对于跑单条链的情况要使用路径压缩优化。

代码如下:
\lstinputlisting{Source/Templates/SAM2SA.cpp}

还可以利用Parent树的性质得到更优的做法。反串的SAM的Parent树就是原串的后缀树,对后缀树按照
字典序遍历后就可以得到rank与sa数组。

这个算法的关键在于如何计算出字典序顺序,即每个节点的转移边(转移边不会重复,如果重复就会开新的公共点)。
首先沿着转移边DFS,仅找$len$比自身大1的后继$v$以避免重复遍历。然后将转移字符压入栈中,
设置$link_v$的第$len_v-len_{link_v}$个字符转移为$v$,这恰好是$link_v$到$v$转移边上的首字符。
第二次按照字典序贪心DFS一遍就可以得到后缀数组。

该方法参考了z1j1n1的博客\footnote{
    使用后缀自动机求后缀数组\\
    \url{https://www.cnblogs.com/zhujiangning/p/7999381.html}
}。
\subsection{广义SAM}
有两种构造方法:
\begin{itemize}
    \item 在线:插入一个字符串之前将$last$重置,时间复杂度为O(Trie大小*字符集大小
    +叶子状态深度和)。
    \item 离线:先建出Trie,BFS插入,插入时把父亲在SAM上的编号当做$last$,
    时间复杂度为O(Trie大小*字符集大小)。
\end{itemize}

\subsubsection{统计状态对应的模板串数}
为每个节点记录最后匹配的模板串编号,每次extend后从$last$开始沿着parent树暴力上跳,将这些节点的
$count$值+1并标记,直到遇到被同模板串标记过节点的为止。设状态总数为$S$,最坏时间复杂度$O(S\sqrt{2S})$,
一般情况下不会达到最坏情况,有时跑得比$\lg$做法还快。在想不出更优做法时,这是一个简单有效的方法。

以上内容参考了WC2012陈立杰的讲课课件《后缀自动机 Suffix Automaton》
与Candy?\footnote{[后缀自动机]【学习笔记】
    \url{https://www.cnblogs.com/candy99/p/6374177.html}
}、dwjshift\footnote{
    用SAM建广义后缀树 $\ll$ dwjshift's Blog
    \url{http://dwjshift.logdown.com/posts/304570}
}、Mangoyang\footnote{
    一个用SAM维护多个串的根号特技
    \url{https://www.cnblogs.com/mangoyang/p/10155185.html}
}的博客。Menci的博客写得更详细,一些性质的证明请移步
\url{https://oi.men.ci/suffix-automaton-notes/}。

\subsection{序列自动机}
类比后缀自动机,序列自动机上的每条路径对应一个子序列。

\subsubsection{构造}
序列自动机的构造比较简单,即预处理$nxt[i][j]$表示位置$i$后的第一个字符$j$
出现的位置。存在一个简单的$O(n|\alpha|)$DP:
\begin{lstlisting}
for(int i=n;i>=1;--i) {
    for(int j=0;j<26;++j)
        nxt[i-1][j]=nxt[i][j];
    nxt[i-1][P[i]-'a']=i;
}
\end{lstlisting}

维护可持久化数组可以把时间复杂度降到$O(n\lg |\alpha|)$,在字符集比较大的时候
使用。
\subsubsection{应用}
下列序列数统计均指本质不同的序列。
\paragraph{子序列个数}
记$dp[i]$为从位置$i$开始的子序列个数,位置$i$的字符自成一个子序列,
并且它与$nxt[i][j]$位置的方案构成了本质不同的子序列,因此有
\begin{displaymath}
    dp[i]=1+\sum_j{dp[nxt[i][j]]}
\end{displaymath}
使用记忆化搜索或者逆序dp。
\paragraph{公共子序列个数}
例题~「FJOI2016」所有公共子序列问题

预处理出两个字符串的序列自动机后,使用记忆化搜索在序列自动机上跑。
\paragraph{回文子序列个数}
对原串和反串构建序列自动机,求这两个串的公共子序列数。

记记忆化搜索调用为$DFS(x,y)$,$x,y$分别为在这两个串上的匹配位置,有
$x\leq n+1-y$,等号成立意味着该回文序列为奇序列。但是奇序列不一定满足
其等号成立,如果记忆化搜索搜索到一个偶回文序列,删掉该回文序列中心的一个字符,
就会出现新的奇回文序列,因此在搜索时若满足$x+y<n+1$,要补上该奇序列的贡献。

上述内容参考了pig\_dog\_baby的博客\footnote{
    序列自动机(一个数组而已...)及经典例题
    \url{https://blog.csdn.net/pig\_dog\_baby/article/details/81145857}
}。

\section{表达式解析}
\subsection{算术表达式解析}
此类表达式由三种元素构成:操作数,运算符和括号。

\subsection{LL(1)递归下降法}
留坑待补。
\index{*TODO!LL(1)递归下降}

\section{Z Algorithm}
\index{Z!Z Algorithm}\label{ZA}
给定一个字符串$P$,记$Z(i)$为以$P$的第$i$个字符为首(从0开始)的后缀与$P$
的LCP长度。
\subsection{求解}
核心思想类似于Manacher算法,利用之前的计算结果尽可能减少
暴力匹配操作。

算法步骤如下:
\begin{enumerate}
    \item 首先有$Z(0)=|P|$。
    \item 维护当前已匹配的最右端,即$max\{i+Z(i)-1\}$,记为$R$;
    同时维护$R$对应的$i$,记为$L$。
    \item 考虑已经维护了当前$L,R$和$Z[0\cdots i-1]$,求$Z(i)$并更新$L,R$。
    \begin{itemize}
        \item 若$i>R$,直接从$i$开始暴力匹配,然后令$L=i,R=i+Z(i)-1$;
        \item 否则$i\leq R$,那么有$P[i-L\cdots R-L]=P[i\cdots R]$。
        然后有$Z(i)\geq min(Z(i-L),R-i+1)$。
        考虑$Z(i-L)$与$R-i+1$的关系:
        \begin{itemize}
            \item 若$Z(i-L)<R-i+1$,则不必继续匹配,令$Z(i)=Z(i-L)$。
            \item 否则从$Z(i)=R-i+1$开始暴力匹配,更新$L,R$。
        \end{itemize}
    \end{itemize}
\end{enumerate}

模板:
\begin{lstlisting}
int L=0,R=0;
for(int i=1;i<n;++i) {
    if(i>R) {
        L=R=i;
        while(R<n && P[R-L]==P[R])
            ++R;
        Z[i]=R-L;
        --R;
    }
    else {
        if(Z[i-L]<R-i+1)
            Z[i]=Z[i-L];
        else {
            L=i;
            while(R<n && P[R-L]==P[R])
                ++R;
            Z[i]=R-L;
            --R;
        }
    }
}
\end{lstlisting}

时间复杂度$O(|P|)$。
\subsection{应用}
Z Algorithm等价于ExKMP。ExKMP用来求解串$S$的每一个后缀$S_i$与另一个串$T$的
LCP长度。

使用Z Algorithm可以解决:构造新串$T+\textrm{分隔符}+S$,运行Z Algorithm,
$S$部分的$Z$值就是所求答案。实际上无需构造新串,由于分隔符的存在,可以分两次对
$T$与$S$运行算法。

上述内容参考了yashem66的译文\footnote{
    译文:Z-function/Z Algorithm的构造与应用

    \url{https://blog.csdn.net/qq\_33330876/article/details/72844491}

    原文:Z Algorithm

    \url{http://codeforces.com/blog/entry/3107}
}。

\section{卷积法解决字符串匹配问题}
\subsection{回文子序列}
以某个位置为对称轴的回文子序列的个数可以由关于这个位置对称的字符对数计算。
每一对都有选与不选两种选择,除去全不选的情况,记对称字符对数为$k$,方案为$2^k-1$。

接下来考虑如何计算出对称字符对数。若字符串按照Manacher算法处理,对于每个对称中心$i$,
以它为对称中心的字符对满足$S[i-x]=S[i+x]$,注意到$i-x+i+x=2i$为定值,可以联系到卷积。
枚举字符集的字符,将有该字符的位置标为1,其余标为0,做一遍自卷积,位置$2i$的系数指示了以
$i$为对称中心的当前字符对数。由于同一个位置上会被统计1次,不同位置的对会被统计2次,所以
(系数+1)/2才是实际对数。时间复杂度$O(|\Sigma|n\lg n)$。

事实上卷积时不一定用Manacher算法预处理,将偶回文序列的对称轴看做$x.5$,其两倍仍然是
整数,可直接统计$[1,2n]$全部系数。

\subsubsection{例题} BZOJ3160: 万径人踪灭

本题要求的是回文子序列数,去掉是连续一段的回文子串。回文子序列数可以使用FFT卷积或者
序列自动机实现,回文子串数可以用Manacher或者PAM实现。

参考代码(NTT+Manacher):
\lstinputlisting{Source/Source/'FFT NTT'/BZOJ3160.cpp}

由于卷积出的值很小(在$n$的范围内),FFT、NTT均可,注意控制FFT的精度(做完除法操作后
使用固定eps,如果想要省去除法操作,需要将eps乘以FFT规模作为实际eps)。
\subsection{带通配符匹配}
给定母串$S$与带通配符的模板串$P$,求$P$在$S$中的出现位置。

首先考虑不带通配符匹配的问题,可以使用KMP解决(带通配符则无法保持nxt的性质),但也有卷积
的方法。考虑如何将其表示为卷积的形式。如果母串$S$在位置$i$处匹配了$P$,那么有
$S[i+k-1]=P[k],1\leq k \leq|P|$。等式两边的下标之和不为定值,但它们的差为定值。那么
可以将$P$取反为$P_{rev}$,有$S[i+k-1]=P_{rev}[|P|-k+1],1\leq k \leq |P|$,两边
下标之和为$i+|P|$,可以进行卷积。同样考虑枚举字符集的字符,将有该字符的位置置为1,其余置0。
将每次卷积的结果累加,若位置$i+|P|$上的系数为$|P|$,则说明母串$S$在位置$i$匹配上了$P$。

有通配符的情况类似,每个有通配符的位置强制置1。

这种方法的时间复杂度仍为$O(|\Sigma|n\lg n)$。
\subsection{大字符集处理}
对于$|\Sigma|$较大的情况(比如26个字母),26次DFT的时间无法被接受。考虑如何把它们
放在一个式子内计算。考虑不带通配符的情况,将字母表示为数字,对应位相等则数字差为0。用
区间内差的绝对值之和为0表示整段对应区间数字差为0比较麻烦,索性使用平方和。那么有
$V[x]=\displaystyle \sum_{i=1}^{|P|}{(S[x+i-1]-P[i])^2}=0$,将平方展开,$P$
取反得到
\begin{displaymath}
    V[x]=\displaystyle \sum_{i=1}^{|P|}{S[x+i-1]^2+P_{rev}[|P|-i+1]^2-2S[x+i-1]P_{rev}[|P|-i+1]}
\end{displaymath}
仅需做一次卷积。考虑带通配符的情况,通配符无法表示为与26个数字都相等的数字,但是可以令其为0,
作为平方和的系数,也可以使整个式子的值为0。将式子拆开后可表示为两个卷积+一个常数的形式。
如果母串也带通配符,则再乘一个系数,表示为三个卷积之和。

参考代码:
\lstinputlisting{Source/Source/'FFT NTT'/BZOJ4503.cpp}

{\bfseries 使用FFT时,若最后不做除法,eps要开大些,比如0.5*p。
可以在做点值乘法时直接求和,仅需一次IDFT。}

上述内容参考了小蒟蒻yyb的博客\footnote{
    [复习]多项式和生成函数相关内容
    \url{https://www.cnblogs.com/cjyyb/p/10132855.html}
}。

\subsection{广义模式匹配}
例题:「THUPC2018」赛艇 / Citing

给定母方阵与模式方阵,求出所有匹配位置。

将母方阵按行拼接为一个串。模式方阵也如此拼接,但是行长要对齐到母方阵的行长便于匹配,
不在模式方阵的部分填为通配符。如此将该问题转化为带通配符的字符串匹配问题。

注意有可能出现一些匹配位置导致模式方阵在母方阵上面展开后错位的情况,因为实际上这个
匹配位置会导致模式方阵放上去后越界。为了避免这种情况,需要根据模式方阵的大小确定这个
匹配位置是否合法。

\section{最小表示法}
最小表示法用来解决字符串的循环同构问题。一个字符串的最小表示就是它的字典序
最小的循环串。如果两个字符串的循环表示相同,说明这两个字符串循环同构。

考虑朴素算法:维护当前最小表示的起点$i$以及用于比较的起点$j$,初始$i=0,j=1$。
然后按照$S[i]$与$S[j]$的大小分类,若$S[i]=S[j]$,则逐个比较直至$S[i]\neq S[j]$;
若$S[i]<S[j]$,则说明起点$j$不可能成为答案,令$j$后移;若$S[i]>S[j]$,则说明$j$
是更优的答案,令$i=j$,$j$后移。

上述算法的低效性在于$S[i]=S[j]$的比较无法重复利用,比如串aaaaaa可以将其卡到
$O(n^2)$。考虑记录当前起点$i$与$j$的前$k$位都相同,当$S[i+k]\neq S[j+k]$时,
需要移动某一个指针,若只移动一位会导致下一次匹配时仍然重新匹配。那么需要增加指针跳跃
的幅度。设$S[i+k]>S[j+k]$,那么$i$不是最优解,由于
$S[i\ldots i+k-1]=S[j\ldots j+k-1]$,$(i,i+k]$范围内的起点也不是最优解,
因此$i$需要后移$k+1$位。由于在当前阶段内已经扫描了$k+1$次,所以$i,j$的总偏移
等于扫描次数,由于总偏移不超过$4n$,算法的时间复杂度是$O(n)$的。

算法退出的条件为$i<n \land j<n \land k<n$,若$i$或$j\geq n$,则说明扫描完毕,
另一个为合法解;否则有$k=n$,两个均为最小表示。因此可以使用$min(i,j)$作为最终解。

参考代码:
\begin{lstlisting}
int scan(int n, const char* A) {
    int i = 0, j = 1, k = 0;
    while(i < n && j < n && k < n) {
        char ci = A[add(i, k, n)],
                cj = A[add(j, k, n)];
        if(ci == cj)
            ++k;
        else {
            (ci < cj ? j : i) += k + 1;
            if(i == j)
                ++j;
            k = 0;
        }
    }
    return std::min(i, j);
}
\end{lstlisting}

上述内容参考了zy691357966的博客\footnote{
    字符串最小表示法 O(n)算法
    \url{https://blog.csdn.net/zy691357966/article/details/39854359}
}。


\chapter{计算几何}
\section{基础设施}
以下内容主要讨论二维空间中的计算几何。
\subsection{点,向量,直线,半平面的表示}
点与向量由2个坐标表示;半平面和直线由直线上一点$ori$与
直线的方向向量$dir$表示;直线上的一点可表示为$ori+dir*t$,通过控制参数$t$的取值
还可以表示射线或线段;可以人为规定半平面的顺时针/逆时针$180^\circ$为半平面所在点集,
通过叉积来判断点在半平面的哪一边。
\begin{lstlisting}
typedef double FT;
struct Vec {
    FT x,y;
    //constructor
    //operator+-*
};
struct Line {
    Vec ori,dir;
    Vec operator()(FT t) const {
        return ori+dir*t;
    }
};
\end{lstlisting}
\subsection{点乘与叉乘}
\subsubsection{点乘}
向量点乘$dot(a,b)=a.x*b.x+a.y*b.y=|a||b|cos<a,b>$,一般用来
判断与法向量的夹角以及在某个向量上的投影长度。点乘满足加法分配律和交换律。
\subsubsection{叉乘}
向量叉乘$cross(a,b)=a.x*b.y-b.x*a.y=|a||b|sin<a,b>$,这是两向量
构成的平行四边形的有向面积。一般用来判断向量的相对方向以及计算多边形的面积。
叉乘满足$cross(a,b)=-cross(b,a)$和加法分配律(将其视作线性变换
$T:a\rightarrow cross(a,b)$或$T:b\rightarrow cross(b,a)$可证)。

\begin{theorem}[拉格朗日公式]
	cross(a,cross(b,c))=b*dot(a,c)-c*dot(a,b)
\end{theorem}

三维向量的叉乘计算了垂直于这两个向量的向量(两向量组成平面的法向量),即
\begin{displaymath}
	cross(a,b)=\left(\begin{array}{c}
		a.y*b.z-b.y*a.z \\
		a.z*b.x-b.z*a.x \\
		a.x*b.y-b.x*a.y
	\end{array}\right)
\end{displaymath}
其方向满足右手定则(右手四指与大拇指垂直,食指指向向量$a$,其余三指指向向量$b$,
大拇指方向即为叉乘方向),模长满足$|cross(a,b)|=|a||b|sin<a,b>$。
\subsubsection{体积计算}
结合点乘和叉乘可得点$A$与点$B,C,D$所组成的三棱锥$A-BCD$的有向体积为
\begin{displaymath}
	V=\frac{1}{6}|dot(\overrightarrow{BA},cross(\overrightarrow{BC},
	\overrightarrow{BD}))|
\end{displaymath}
其中$\overrightarrow{N}=cross(\overrightarrow{BC},\overrightarrow{BD})$
的方向为法向,模长为底面面积的2倍。$|dot(\overrightarrow{BA},\overrightarrow{N})|$
又给其模长增加了高的因子。套椎体体积公式可得上式。
\subsubsection{极角计算}
点乘可以得到$x=|a||b|cos<a,b>$,叉乘可以得到$y=|a||b|sin<a,b>$,将$(x,y)$
看做在半径为$|a||b|$的圆上的点,极角为$atan2(x,y)$。
\subsection{点到直线的距离}
设偏移向量$delta=p-ori$:
\begin{itemize}
	\item 计算$dot(delta,dir)/|dir|$可得到投影长度$d'$,根据
	      勾股定理得到$d^2=|delta|^2-d'^2$。
	\item 计算$|cross(delta,dir)|$可得偏移向量与方向向量构成的平行四边形的面积,
	      根据面积公式得到$d=|\frac{cross(delta,dir)}{|dir|}|$。
\end{itemize}
叉乘法的运算量少且精度较高,能够指示半平面方向,建议选用。
\subsection{直线、线段的交点}
\begin{lstlisting}
Vec intersect(const Line& a,const Line& b) {
    Vec delta=a.ori-b.ori;
    FT t=cross(b.dir,delta)/cross(a.dir,b.dir);
    return a.ori+a.dir*t;
}
\end{lstlisting}
证明留坑待补。
\index{*TODO!直线相交正确性证明}

线段相交也是如此,但首先要判断两线段是否相交。将该问题转换为
两线段是否互相平分。设线段为$a-b,c-d$,首先判断$a-b$平分$c-d$,
即$c,d$分别位于$a-b$两边,有$cross(c-a,b-a)*cross(d-a,b-a)\leq 0$,
同理对$c-d$也做一遍。
\subsection{判定点是否在多边形内}
\subsubsection{随机射线法}
从点P开始随机引出一条射线,计算其与多边形的边的交点个数,若为奇数次则
在多边形内。注意射线恰好经过点时要重新选择方向。
\subsubsection{旋转角法}
从点P与多边形的一个点开始,不断旋转到下一个点,直至转完一圈为止。
此时若点P旋转了$0^\circ$,则在多边形外;若点$P$旋转了$360^\circ$,
则在多边形内。旋转角可以使用前文所述方法计算,为了避免精度问题,以
$\pi(180^\circ)$为界进行比较。
\subsubsection{半平面法}
若该多边形为凸多边形,以每条边构造半平面,使用叉积判断是否在半平面内,
若点在所有半平面内则在多边形内。
\subsection{向量的旋转}
根据复数乘法的规律:模长相乘,幅角相加。构造逆时针旋转角度$\theta$的单位旋转向量
$e^{\theta i}=\cos \theta+\sin \theta i$,将原向量乘以该旋转向量得到结果:
$(x\cos \theta-y\sin \theta,x\sin \theta+y\cos \theta)$。

有时对点坐标以某点为原点进行旋转变换,然后按照x轴顺序贪心计算是一个不错的骗分方法。
\subsection{坐标系的切换}
有同一$d$维坐标空间中的点$P$与单位正交向量组$b_1,b_2,\cdots,b_d$,以
该向量组为新坐标空间的基向量,那么点$P$在新坐标空间的坐标值为它在
这些向量上的投影长度。

事实上,将点$P$在新坐标系上的坐标$P'$变换回旧坐标系意味着$P'$左乘矩阵
$T={b_1,b_2,\cdots,b_d}$,即$TP'=P$。由于矩阵$T$的特殊性,
有$T^{-1}=T^T$。因此$P'=T^TP$,即投影长度。
\subsection{点、向量、法向量的坐标变换}
可以使用一个矩阵$R^{d*d}$来表示$d$维空间中的旋转,缩放,坐标系切换,
引入齐次坐标(即给向量再加一维,非透视投影时恒为1)可支持平移(仅对于点的变换有意义)。

在三维空间下使用$4*4$的矩阵来表示对坐标的变换。

\subsubsection{平移}
将坐标平移$(x,y,z)$:
\begin{displaymath}
	\left(\begin{array}{cccc}
		1 & 0 & 0 & x \\
		0 & 1 & 0 & y \\
		0 & 0 & 1 & z \\
		0 & 0 & 0 & 1 \\
	\end{array}\right)
\end{displaymath}
\subsubsection{旋转}
以基于$z$轴旋转$\theta$为例:

思路是对$x,y$轴坐标进行二维旋转,$z$轴坐标不变。
\begin{displaymath}
	\left(\begin{array}{cccc}
		\cos \theta & -\sin \theta & 0 & 0 \\
		\sin \theta & \cos \theta  & 0 & 0 \\
		0           & 0            & 1 & 0 \\
		0           & 0            & 0 & 1 \\
	\end{array}\right)
\end{displaymath}
\subsubsection{缩放}
对坐标分别缩放$x,y,z$:
\begin{displaymath}
    \left(\begin{array}{cccc}
        x & 0 & 0 & 0 \\
        0 & y & 0 & 0 \\
        0 & 0 & z & 0 \\
        0 & 0 & 0 & 1 \\
    \end{array}\right)
\end{displaymath}
\subsubsection{向量变换}
注意平移不会影响向量的变换,因此将矩阵截断为3*3矩阵$T'=mat3(T)$。
\subsubsection{法向量变换}
若法向量$N$变换按照向量变换计算,若遇到缩放则会发生变换后不垂直于变换后平面
的情况。因为缩放矩阵的基向量不是单位向量。考虑一个与该法向量垂直的向量
$V$,满足$N^T\cdot V=0$。那么对于变换后的两向量仍然
要保持垂直,即$N^TT'^{-1} \cdot T'V=0$,对左边进行转置得到$N'=(T'^{-1})^TN$。
\subsection{反射与折射}
以下的入射向量、出射向量与法向量均为单位向量。
\subsubsection{反射}
根据反射定律,反射向量由水平方向的向量$I_x$减去垂直方向的向量$I_y$。
列出方程组:
\begin{eqnarray*}
    I&=&I_x+I_y\\
    I_y&=&N\cdot dot(I,N)\\
    O&=&I_x-I_y\\
\end{eqnarray*}
解得$O=I-2N\cdot dot(I,N)$。
\subsubsection{折射}
斯涅尔定律描述了折射率与角度的关系:
\begin{theorem}
    $\eta_1\sin \theta_1=\eta_2\sin \theta_2$
\end{theorem}
同样以法向量和切向量为基向量进行正交分解,
记$\eta=\frac{\eta_1}{\eta_2}$,有
\begin{eqnarray*}
    I&=&I_x+I_y\\
    I_y&=&N\cdot dot(I,N)\\
    I_x&=&\sin \theta_1T\\
    O&=&O_x+O_y=\sin \theta_2T-\cos \theta_2N
\end{eqnarray*}
化简:
\begin{eqnarray*}
    O&=&\sin \theta_2T-\cos \theta_2N\\
    &=&\frac{\sin \theta_2}{\sin \theta_1}I_x - \cos \theta_2 N\\
    &=&\frac{\eta_1}{\eta_2}(I-dot(I,N)\cdot N)-\cos \theta_2 N\\
    &=&\eta\cdot I-(\eta dot(I,N)+\cos \theta_2)N
\end{eqnarray*}

代码如下,注意全反射的情况(即$\sin \theta_2$超出值域):
\begin{lstlisting}
Vec refract(Vec I,Vec N,FT eta) {
    FT idn=dot(I,N);
    FT cosO2=1.0-eta*eta*(1.0-idn*idn);
    if(cosO2<0.0)return Vec();
    FT k=eta*idn+sqrt(cosO2);
    return I*eta-k*N;
}
\end{lstlisting}
上述内容参考了Milo Yip的文章\footnote{
    用 C 语言画光(五):折射
    \url{https://zhuanlan.zhihu.com/p/31127076}
}和glm库的代码\footnote{
    glm/func\_geometric.inl at master · g-truc/glm · GitHub
    \url{https://github.com/g-truc/glm/blob/master/glm/detail/func\_geometric.inl}
}。
\subsection{pick定理}
\index{P!Pick's Theorem}
\begin{theorem}[Pick's Theorem]
    若格点多边形内的点数为$a$,落在边上的点数为$b$,则
    该多边形的面积为$a-\frac{b}{2}+1$。
\end{theorem}
\subsection{切比雪夫距离}
切比雪夫距离是两点坐标之差的最小值。
分别考虑到原点曼哈顿距离和切比雪夫距离为1的点$P,Q$,
发现将$P$绕原点旋转$45^\circ$再缩放$\sqrt{2}$倍后等于$Q$。

因此$P(x,y)\rightarrow Q(x+y,x-y)$,$Q(x,y)
\rightarrow P(\frac{x+y}{2},\frac{x-y}{2})$。
\subsection{精度处理}
一般引入$eps=1e-8$来避免精度问题。

常见问题:
\begin{itemize}
    \item 判断两个值相等:$fabs(a-b)<eps$。
    \item 要输出$1.00$却输出$0.99$或者要输出$0.0$却输出$-0.0$:
        对正值$+eps$,负值$-eps$。
    \item 已知$\sin \theta$求$\cos \theta$:$sqrt(1.0-cosTheta*cosTheta)$
    可能会因为传入$sqrt$的参数小于0而返回$nan$,在调用数学函数前要clamp到定义域内
    或特判。
    \item 若乘积表示的范围越界,则可以使用对数加表示数乘。
    \item 技巧:若要维护$std::set<FT>$,在比较器中引入$eps$自动完成去重工作。
    \item 多个量级相差巨大的浮点数相加减时要尽量使每次相加的两个数量级差不多。
    比如连加应该要从小到大累加。
    \item 二分时尽量固定迭代次数而不是用eps比较。
    \item 一些判定性问题可以使用模意义下的数代替浮点数解决。
\end{itemize}

为了辅助判断两个值的大小关系,引入一个符号函数$sign$:
\begin{lstlisting}
int sign(FT x) {
    return (x>eps)-(x<-eps);
}
\end{lstlisting}

该内容参考了Ac\_smile的博客\footnote{计算几何中的精度问题\\
    \url{https://www.cnblogs.com/acsmile/archive/2011/05/09/2040918.html}
}与Oyking的博客\footnote{
    IO/ACM中来自浮点数的陷阱(收集向)
    \url{https://www.cnblogs.com/oyking/p/3959905.html}
}。

\section{凸包}
\index{C!Convex Hull}
\subsection{极角序凸包}
经典算法是Graham扫描法\index{G!Graham Scan}。
算法步骤如下:
\begin{itemize}
	\item 选择一个纵坐标最低的点(若有多个选横坐标最小)加入凸包,以此为
	      原点按极角对其他点排序;
	\item 按照极角序加入每一个节点,保持凸包相邻3个节点的凸性质,注意三点
	      在一条直线上时选择距离较远的点。
\end{itemize}
\subsection{水平序凸包}
极角序计算凸包容易由于$atan2$的精度问题而造成错误,并且不易处理共线问题
(始边要求从近到远,终边要求从远到近)。
考虑对横坐标升序排序(若横坐标相等则对纵坐标升序比较,主要用于解决左右边缘出现竖线的问题,
当然也可以使用旋转扰动法避免),分别计算其凸包的上凸壳和下凸壳,最后合并两部分。

代码如下(CCW):
\begin{lstlisting}
Vec P[size],C[size];
void convexHull(int n) {
    std::sort(P+1,P+n+1,[](const Vec& a,const Vec& b) {
        return a.x!=b.x?a.x<b.x:a.y<b.y;
    });
    int top=1;
    C[1]=P[1];
    for(int i=2;i<=n;++i) {
        while(top>=2 && cross(C[top]-C[top-1],
            P[i]-C[top-1])<eps)
            --top;
        C[++top]=P[i];
    }
    for(int i=n-1;i>=1;--i) {
        while(top>=2 && cross(C[top]-C[top-1],
            P[i]-C[top-1])<eps)
            --top;
        C[++top]=P[i];
    }
}
\end{lstlisting}
\subsection{在线凸包}
在线凸包即每次向点集中加入新点,维护当前凸包的某些信息。

经典思路是按照极角序(选择凸包内的定点作为基准点,因为凸包会越来越大,所以可以取前3个点的
重心)将凸包上的点存储在set上。加入点时在set上查询极角序相邻的点组成的边,判断加入该点后
新增的两条边是否是凸的,如果是凸的则继续更新左右两边,直至局部全为凸。{\bfseries 注意跨
x负半轴的情况(极角为$-pi$与$+pi$附近的点相邻)}。
\subsubsection{水平序在线凸包}
同理维护上下凸壳,根据dwjshift的博客\footnote{
	实现水平序动态凸包的小技巧 $\ll$ dwjshift's Blog
	\url{http://dwjshift.logdown.com/posts/285072}
}所述,可以将其横纵坐标取负再求一次凸壳,因此只要考虑维护下凸壳。
\subsection{凸包矢量和(闵可夫斯基和)}
已知点集$A,B$,求点集$C={P_1+P_2|P_1\in A \land P_2\in B}$的凸包。

显然该凸包与点集$A,B$凸包相加的凸包相同,容易想到预处理点集$A,B$的凸包后对
将每对凸包上的点之和加入集合做凸包。在点随机分布的情况下,这种方法的时间复杂度为
$O(n \lg n)$,因为凸包的期望规模为$\Theta(\lg n)$。但这种方法会被卡成
$O(n^2 \lg n)$,当所有点都在凸包上时(比如输入为一个圆)。

考虑优化对两个凸包做加法的过程,容易发现按照一个方向构造新凸包时,在原凸包选择相加
点的顺序是单调的。因此可以使用双指针法维护当前选择的点,每次判断哪个凸包上的指针后移
(类似归并排序),就能在$O(n)$复杂度内完成合并。注意在两个候选点的对应的累加点
与上一个累加点在一条直线上时,两个指针都向后移动1位,取这两个指针指向点之和。

事实上这正是旋转卡壳的应用之一。
\subsection{凸包合并}
合并后的凸包有一部分是两个凸包上点的连边,一部分是凸包上一段链。
$P_i,Q_j$连边当且仅当:
\begin{itemize}
	\item $P_i$与$Q_j$是并踵点对;
	\item $P_{i-1},P_{i+1},Q_{i-1},Q_{i+1}$在$P_i-Q_j$的同一侧。
\end{itemize}
使用旋转卡壳法可在线性时间内合并凸包。
该方法参考了ACMaker的博客\footnote{
	旋转卡壳——合并凸包
	\url{https://blog.csdn.net/ACMaker/article/details/3561150}
}。
\subsection{稀疏包分布}
\begin{itemize}
	\item 若点在圆面上均匀分布,则凸包期望规模为$\Theta(n^{1/3})$。
	\item 若点在凸多边形内部取得,凸包期望规模为$\Theta(\lg n)$。
	\item 若点根据二维正态分布取得,凸包期望规模为$\Theta(\sqrt{\lg n})$。
\end{itemize}
该内容来自算法导论\cite{ITA3}思考题33-5。
\subsection{二维最小乘积生成树}
选择$n-1$条边使得图连通,最小化所选边第一权值和与第二权值和的乘积。

解法:将第一权值和当做横坐标,第二权值和当做纵坐标,每一棵生成树对应一个点。
首先分别求出第一权值和和第二权值和的MST,标记这两棵MST对应点$A,B$。那么最优解
肯定在以$A,B$为端点的下凸壳上,考虑计算到$AB$距离最远的点$C$,用它更新答案。
它肯定在凸壳上并且排除了大部分解。然后对$AC,CB$进行递归计算。

接下来讨论如何计算最远点$C$:

首先由于$AB$长度固定,可以转化为求$S_{\triangle ABC}$的面积最大值。
使用叉积可得
\begin{eqnarray*}
	2S_{\triangle ABC}&=&cross(\overrightarrow{AC},\overrightarrow{AB})\\
	&=&(C.x-A.x)*(B.y-A.y)-(B.x-A.x)*(C.y-A.y)\\
	&=&C.x*(B.y-A.y)-C.y*(B.x-A.x)+Constant
\end{eqnarray*}
计算权值后可转化为计算最大生成树,为了简便可将其系数取反同样求MST。
递归结束的条件为找不到满足要求的点(即最大面积不为正)。

最小乘积最大匹配也使用类似做法。对于高维情况,将其改为求到超平面上最远距离。

事实上这是快速凸包算法的过程。

上述内容参考了空灰冰魂的博客\footnote{
	【BZOJ2395】【Balkan 2011】Timeismoney 最小乘积生成树
	\url{https://blog.csdn.net/vmurder/article/details/46828379}
}。
\subsection{三维凸包}
这里使用$O(n^2)$增量法计算三维凸包。通过链表来维护面以便快速删除,以及
使用邻接矩阵维护点的有向连接对应的面的编号便于更新。

该算法的主要思想是检查并删除``可视面'',然后将未封闭的边缘与新点连边重新构成封闭立体。
\begin{itemize}
	\item 首先选择四个不共面的点,组成四面体,将面加入链表(注意顶点顺序朝外
	      为CCW。可以计算重心到该面的有向体积,使其为负)。
	\item 将其余点逐个加入凸包。枚举每一个面,若该面到新点的四面体的有向体积为正,则
	      删除原来的面,并检查三条边的邻接面与其组成的四面体,直到其有向体积为负,
	      才加入新面。
\end{itemize}

在实现时使用$fid[a][b]$来维护边$a\rightarrow b$对应的面,注意边是有向的,
下面的代码人为规定为CCW序。

代码如下:
\lstinputlisting{Source/Unclassified/Done/4724.cpp}

该方法参考了\_sunshine的博客\footnote{
	hdu 4266 三维凸包(增量法)
	\url{https://www.cnblogs.com/-sunshine/archive/2012/08/25/2656794.html}
}。
\subsection{快速凸包}
\index{Q!Quick Hull}
高维凸包不易理解且使用增量法的$O(n^2)$复杂度较大,所以在此
介绍快速凸包算法。在平均情况下复杂度为$O(n\lg n)$,最坏情况$O(n^2)$,与
快速排序类似。

主要思想是每次选择到超平面的最远点,该点肯定在凸包上,并且该点与超平面将整个集合分割为$d+1$个子集,
其中该点与超平面组成的体内的点直接被排除,以达到快速求解的目的。

构造初始超平面时先选择一个点,再选择离这个点最远的点构成线,再选择离这条线最远的
点构成三角平面,以此类推。构造线也可以使用两个坐标为极值的点。还有一种随机化方法:
不断随机构造一个超平面,计算到这个超平面的有向距离的最大值与最小值,取出对应的点,
直到选出$d+1$个不重复的点(二维情况下只需选取2个初始点,三维情况则必须选取4个初始点)。

二维凸包代码:
\lstinputlisting{Source/Templates/QuickHull2D.cpp}
三维凸包代码(不正确):
\lstinputlisting{Source/Templates/QuickHull3D.cpp}
计算凸包/表面积时四点共面的情况不太好处理。为了处理多于3点共面的情况,
可以给每个点的坐标值加一些``扰动''。

\index{*TODO!修复三维快速凸包模板}

三维凸包实现参考了Valve Software的Dirk Gregorius在GDC2014
上的文章Implementing QuickHull\footnote{
	PowerPoint Presentation - DirkGregorius\_ImplementingQuickHull.pdf
	\url{http://media.steampowered.com/apps/valve/2014/DirkGregorius\_ImplementingQuickHull.pdf}
}和lloyd的代码\footnote{
	QuickHull3D: A Robust 3D Convex Hull Algorithm in Java
	\url{https://www.cs.ubc.ca/\~lloyd/java/quickhull3d.html}
}。

\subsubsection{快速凸包的精度控制}
根据Dirk Gregorius的文章,$\varepsilon$应为$d*\varepsilon_{machine}*max\{max_i-min_i\}$。

上述内容参考了Wikipedia-EN\footnote{
	Quickhull - Wikipedia
	\url{https://en.wikipedia.org/wiki/Quickhull}
}。

\section{圆}
处理圆的交并问题一般考虑圆之间的覆盖区间。
\subsection{圆的并}
求圆的并的面积。

首先去除被其他圆覆盖的圆。然后对于每个圆与其它圆求交点,得到每个圆被覆盖的弧度区间
(逆时针为正方向)。注意跨越弧度$\pi$的区间要分为两个区间。对每个圆的覆盖区间排序
得到不覆盖的区间。

画图可以发现圆并的面积等于圆弧面积+多边形的有向面积。由于多边形需要计算有向面积(直接
以原点作为基准点),不能直接计算扇形面积和。

对于半径为$R$的圆,弧度为$x$的圆弧的面积为$\frac{1}{2}(x-\sin x)R^2$。
当$x> \pi$时该等式也满足。

两圆求交时用圆心、交点弦中点、某一交点构造两个直角三角形,然后根据勾股定理列出方程组,
解出一个圆心到交点弦中点的距离,进而解出该三角形各边长,使用向量偏移计算交点弦中点和交点。

时间复杂度$O(n^2 \lg n)$。

代码如下:
\lstinputlisting{Source/Templates/CIRU.cpp}

\subsubsection{扩展}
求恰好被$i$个圆覆盖的区域面积。

同样对于每个圆考虑其覆盖区间,发现常规算法求出的被覆盖$i$次的区域对被覆盖$<i$次的区域
有贡献,因此计算完毕后要差分输出以扣除多余面积。
对覆盖分界点排序,每个区间差分标记覆盖,使用前缀和计算统计区间覆盖次数,
计算贡献。{\bfseries 注意覆盖次数要加上整个圆被覆盖的次数。}

\lstinputlisting{Source/Source/CG/CIRUT.cpp}

上述内容参考了Oyking的博客\footnote{
	SPOJ 8073 The area of the union of circles(计算几何の圆并)(CIRU)
	\url{https://www.cnblogs.com/oyking/p/3424999.html}
}。
\subsection{圆的交}
与求圆并同理,对于每个圆求出覆盖区间的交,答案贡献为圆弧面积+$\frac{1}{2}$区间交
两端点的叉积。
\subsection{最小圆覆盖}
\index{M!Minimum Covering Circle}
一般使用随机增量法:
\begin{enumerate}
	\item 将输入点随机重排列;
	\item 构造一个初始空圆(退化为点且没有点在这个点上);
	\item 不断加入点更新当前最小覆盖圆,设当前点为$P_i$:
	\begin{enumerate}
		\item 若该点已经在该圆内,跳出;
		\item 否则该点必在新圆上。将当前圆重置为一个空圆,固定该圆必有点$P_i$。
		按顺序加入点$P_j,j<i$,若$P_j$不在当前圆内,则$P_j$也在新圆上。
		于是将当前圆重置为以$P_i,P_j$为直径的圆,找出不在当前圆上的点$P_k,k<j$,
		则$P_i,P_j,P_k$三点可确定一个圆,保证该圆是$P_i,P_j,P_x,x\leq k$的
		最小覆盖圆,更新当前圆后继续迭代。
	\end{enumerate}
\end{enumerate}

看似时间复杂度为$O(n^3)$,事实上该算法的时间复杂度为$O(n)$。

\paragraph{实现细节}
在求三角形外接圆时注意对三点共线的点取最远点对作为直径,三点不共线则使用中垂线求交
求外接圆。

代码如下:
\lstinputlisting{Source/Templates/Cover.cpp}

上述内容参考了Wikipedia-EN\footnote{
	Smallest-circle problem - Wikipedia\\
	\url{https://en.wikipedia.org/wiki/Smallest-circle\_problem}
}。
\subsection{圆的反演}
\paragraph{定义} 已知$\odot O$的半径为$r$,若点$P_1,P_2$在以点$O$为端点的射线上,
且$OP_1\cdot OP_2=r^2$,则称$P_1,P_2$关于$\odot O$互为反演,称点$O$为反演中心。

\begin{property}
	一条不经过反演中心的直线的反演图形是一个经过反演中心的圆。
\end{property}
可以根据直线到反演中心的距离与反演圆的半径的关系来互推。

\begin{property}
	一个不经过反演中心的圆的反演图形还是一个不经过反演中心的圆。
\end{property}
可以根据圆心与反演中心的直线上与两圆的四个交点(两两对应)解出反演图形的
半径和到反演中心的距离。

\subsubsection{求过定点且与两个圆相切的圆}
以定点为反演中心,计算两个圆的反演圆,求反演圆的公切线,再做一次反演就可以得到
经过反演中心的圆了。

以上内容参考了ACdreamer的博客\footnote{
	圆的反演变换 - ACdreamer
	\url{https://blog.csdn.net/acdreamers/article/details/16966369}
}。

\section{半平面交}
\index{H!Half-Plane Intersection}
\subsection{基本算法}
基本思路是对半平面极角排序,然后按照极角序加入半平面,同时将无效的半平面(交点
在其它半平面外)删除。这里使用双端队列实现,同时对已加入的半平面的两个端口进行
剔除。注意两半平面平行的情况,此时保留较近的半平面。注意最后要用处理两端口``交叉''
的情况,应当使用一端直线消去另一端,然后确定首尾交点作为最后一个交点。

代码如下:
\lstinputlisting{Source/Templates/HPI.cpp}

一般可将线性不等式转换为半平面,然后使用半平面交来判断是否有解/计算最优解,
除去排序后时间复杂度$O(n)$。较复杂的最优化问题应当作为线性规划问题使用单纯形算法
解决,参见第~\ref{LP}节。
\subsection{线性判空集}
半平面交的时间复杂度为$O(n\lg n)$,但当只要求
半平面交是否为空集时,可以使用期望时间复杂度为$O(n)$的随机化算法。

该算法维护当前半平面交的纵坐标最高点$P$。每次{\bfseries 随机}加入新半平面$L$时,
若$P$在$L$内,则直接跳过;否则新的$P'$必然在$L$与之前的半平面的交点上,用
之前的半平面在当前直线上截出可行线段,保留最高点作为新的点$P$。

该方法源自WC2012上钟诚的讲稿《概率与随机化算法》与解轶伦的文章《随机增量算法》。

\section{旋转卡壳}
\index{R!Rotating Calipers}
\subsection{思想}
\subsection{典例}

\section{平面最近点对}

\chapter{最优化问题}
\minitoc
\section{最优化方法}
下列方法主要用于解决提答题中的最优化问题\CJKsout{以及乱搞骗分}。
\subsection{爬山法}
\index{H!Hill Climbing}
每次迭代时选取与$x$点临近的点$x'$计算,若其值更优,则接受$x'$(也可以选择临近最优点)
,进入下一步迭代;若所有临近点都比$x$要差,那么此时$x$局部最优。重新随机选取初始点进行
迭代,将迭代过的点中最优解作为答案。
\subsubsection{随机爬山法}
\index{S!Stochastic Hill Climbing}
在选取下一迭代点时,还可以对更优点集按照变化值建立分布,然后离散随机采样。
\subsection{模拟退火}
\index{S!Simulated Annealing}
模拟退火为比赛中的常用方法。
步骤如下:
\begin{enumerate}
	\item 选取一个初始温度$T$,与冷却系数$d\in (0,1)$,初始化一个解$s$,计算$f(s)$;
	\item 稍微修改该解得到新解$s'$,计算$f(s')$;
	\item \begin{itemize}
		      \item 如果$f(s')>f(s)$,则直接接受新解$s'$;
		      \item 否则按照Metropolis准则,以$e^{-\frac{\Delta f}{T}}$的概率接受新解。
	      \end{itemize}
	\item 冷却,即$T\leftarrow Td$;
	\item 当$T$足够小时退出,否则返回步骤2。
\end{enumerate}
$T$与$d$的作用是让变化越来越``保守''。在整个模拟退火的过程中要另外维护一个搜索到
的最优解,以增加正确率。

\subsubsection{参数选取}
参数$T$与$d$的选取是玄学。。。可以结合从某解到最优解的最大/期望/最小步数考虑。

Update:在FlashHu的博客里找到了人肉调参方法\footnote{
	模拟退火总结(模拟退火)\\
	\url{https://www.cnblogs.com/flashhu/p/8884132.html}
}:调参时二分初始温度与冷却系数(主要调冷却系数),每个二分值测试多次(因为随机化)。
迭代输出当前解与对应温度,\CJKsout{用心}感受解的下降速度是否均匀,如果下降速度太快或
太慢就调整对应时间段内的冷却系数。或许有自动化的算法完成这一过程?启发式模拟退火?

\index{*TODO!启发式模拟退火}
\subsubsection{新解生成}
注意在计算新解的价值时,尽量由旧解的价值修改得出,以降低更新解的复杂度。
在变化时要尽量避免价值差变化过大(比如TSP中交换相邻城市比交换任意两个城市更好)。
当然为了防止解在一个``盆地''中过分逗留,可以重新选取初始解或者选择恰当时机
对当前解进行比较大的改变。随着温度的减小,对解的变动也应该减小。
\subsubsection{分块模拟退火}
对于多峰函数,将区间分成几块,在块内使用模拟退火(因为模拟退火利用相邻解的关系)。

以上内容参考了Wikipeida-EN\footnote{
	Simulated annealing - Wikipedia\\
	\url{https://en.wikipedia.org/wiki/Simulated\_annealing}
}
\subsection{遗传算法}
\index{G!Genetic Algorithm}
遗传算法引入了生物学中的概念:遗传、变异、竞争与淘汰。

算法步骤如下:
\begin{enumerate}
	\item 生成一些初始解,计算这些解的``适应度'';
	\item 选取这些解的一部分较优解;
	\item 通过较优解之间的配对交叉把自己的``基因''(编码)遗传给
		  子代(可以加权),子代的基因以一定的概率变异(只要考虑$p<0.5$),
		  加入下一代群体;
	\item 迭代固定次数后退出,否则返回步骤2;
	\item 输出计算历史中适应度最高的解。
\end{enumerate}
\subsection{A*}
A*算法通过引入一个估价函数$h(x)$表示到目标点长度的估计值,
在做最短路时使用$f(x)+h(x)$作为权重更新距离,可以达到比Dijkstra
更好的性能。
\subsubsection{IDA*}
IDA*比A*多了一个迭代加深的操作,即每次DFS/BFS搜索时限制搜索深度。
若使用当前深度参数无法找到可行解,就加深深度重新DFS,注意这里
不一定要保存上一次的搜索树,因为新搜索树比原搜索树大得多。
\subsection{梯度下降}
\index{G!Gradient Descent}
梯度下降法用来找局部最优值,其主要思想是利用当前点的梯度来引导寻找更优值。
梯度在笛卡尔坐标系中被定义为一个向量,其坐标值为目标函数在对应基向量上的偏导数。
若要寻找$f(x)$最小值,迭代时令$x_{n+1}=x_n-\gamma \nabla f(x_n)$。

注意$\gamma$要小且不需要更改,因为随着算法的收敛,$\nabla f(x)$会越来越小。
迭代时达到指定精度或者固定迭代次数后直接退出,注意要不断更换$\gamma$来
保证算法的收敛($\gamma$过大会导致波动幅度大,过小会导致收敛速度慢)。

\paragraph{Barzilai-Borwein Method}
\index{B!Barzilai-Borwein Method}
这是一种自适应设置步长$\gamma$的方法,其表达式如下:
\begin{eqnarray*}
	D&=&\nabla f(x_n)-\nabla f(x_{n-1})\\
	\gamma_n&=&(x_n-x_{n-1})^T\cdot \hat{D}\\
\end{eqnarray*}

$\hat{D}$表示$D$对应的单位向量。

例如求$f(x)=x^4-3x^3+2$的在$x_0=6$附近的最小值所对应的$x$:

\lstinputlisting{Optmize/GD.cpp}

上述内容参考了Wikipedia-EN\footnote{
	Gradient descent - Wikipedia
	\url{https://en.wikipedia.org/wiki/Gradient\_descent}
}。
\subsection{解集存储}
在解决最优化问题时,保存新解{\bfseries 当且仅当该解比存储的最优解更优且
该解即将被比自己更差的解作为当前解},在迭代退出再执行一遍该逻辑。此法延迟了解集保存时间,
可以减少冗余拷贝。Lazy Evaluation大法好!!!
\subsection{拉格朗日乘子法}
\index{L!Lagrange Multiplier}
拉格朗日乘子法用于求解{\bfseries 多元函数条件极值问题}。
给定两个一阶连续可导函数$f(X),g(X)$,$X$为变量组,在满足$g(X)=0$的条件下最优化$f(X)$。
也存在$g(X)=c$的形式,常数一般不隐藏在$g(X)$中。

拉格朗日乘子法引入了一个新的变量$\lambda$,称作拉格朗日乘子。然后构造一个新的
函数$\mathcal{L}(X,\lambda)=f(X)-\lambda g(X)$,称作拉格朗日函数。

我们在$g(X)=0$的区域上移动,如果在某处$f(X)$局部不改变的话,这个地方很有可能取得极值,
且极值点只可能在这些点取到。考虑``等势区域''$g(X)=0$与$f(X)=d$,$f(X)$的值不改变意味着
它也在等势区域上移动。那么有这两个等势区域在该处平行,意味着这个$X$在$f(X)$与$g(X)$上的
梯度平行,记为$\nabla_X f=\lambda \nabla_X g$。再加上约束$g(X)=0$,结合先前引入的
拉格朗日函数,有$\nabla_{X,\lambda} \mathcal{L}=0$。

在实际计算中,首先根据$\nabla_X \mathcal{L}=0$用$\lambda$表达出$X$,然后代入$g(X)=0$
求出可行的$\lambda$。这时解的搜索范围被极大地缩小,直接比较每个$\lambda$对应的$f(X)$,
得到所需极值。多个$g(X)=0$的处理方法类似。

\paragraph{例题} 「NOI2012」骑行川藏

很容易将其转化为多元函数条件极值问题:
\begin{itemize}
	\item 满足约束$\displaystyle g(V)=\sum{k_i(v_i-v_i')^2s_i}=E_U$
	\item 最小化$\displaystyle f(V)=\sum{\frac{s_i}{v_i}}$
\end{itemize}

求偏导得$\nabla_{v_i} \mathcal{L}=-s_iv_i^{-2}-2\lambda s_ik_i(v_i-v_i')$,
令其为0,化简得$2\lambda k_i(v_i-v_i')v_i^2+1=0$。注意到$\lambda$为负数,
$\lambda$越大,每个$v_i$越大,耗费的能量$g(V)$越大。

那么可以二分$\lambda$,根据式子使用牛顿迭代法求出$v_i$,回代求出$g(V)$,与$E_U$比较。

一个比较trick的方法是以上一次迭代的解为初始解,当$\delta<\varepsilon$时退出迭代,
得到不错的性能提升。\CJKsout{对着数据调参数}

参考代码:
\lstinputlisting{Source/Source/Optmize/LOJ2671.cpp}

\subsubsection{KKT条件}
\index{K!Karush–Kuhn–Tucker\\ Conditions}
KKT条件是拉格朗日乘子法的扩展,用于求解不等式约束下的最优化问题。

看上去不太好求解,留坑待补。
\index{*TODO!KKT条件}

\section{分数规划}
\index{F!Fractional Programming}

\section{线性规划}\label{LP}
\index{L!Linear Programming}
\subsection{定义与规范描述形式}
\subsubsection{定义}
\paragraph{线性函数} 给定$n$个实数$a_1,a_2,\cdots,a_n$与$n$个变量
$x_1,x_2,\cdots,x_n$,线性函数$f$是这些变量的线性加权和,即
\begin{displaymath}
    f(x_1,x_2,\cdots,x_n)=\sum_{i=1}^n{a_ix_i}
\end{displaymath}
\paragraph{线性约束}
给定一个实数$b$,线性约束是满足$f(x_1,x_2,\cdots,x_n)=b,\leq b$或$\geq b$的
线性等式/不等式。
\paragraph{线性规划}
一个线性规划问题是在使一组变量满足一组有限个线性约束的前提下,
最大(小)化某个线性函数值。
\paragraph{可行解}
满足所有线性约束的解。
\paragraph{目标函数与目标值}
目标函数是我们希望最大(小)化其值的线性函数,目标值是特定变量组合对应的目标函数值。
\paragraph{线性规划的解}
一个线性规划称为不可行的当且仅当它没有可行解。一个可行线性规划称为
无界的当且仅当它没有最优值。
\paragraph{矩阵表示}
下面的内容中,大小为$n$的向量$x$表示由变量$x_1,x_2,\cdots,x_n$组成的向量,
大小为$n$的向量$c$表示目标函数$f:x\rightarrow c^Tx$的系数,大小为$m*n$的矩阵$A$
表示$m$个约束的系数,大小为$n$的向量$b$表示由这$m$个约束的常数项组成的向量。
\subsubsection{标准型}
标准型的描述如下:
\begin{itemize}
    \item 最大化 $c^Tx$
    \item 满足约束$Ax\leq b$
    \item 满足非负约束$x\geq 0$
\end{itemize}

任意线性规划都可以按照如下方法将其转换为等价的标准型线性规划:
\begin{itemize}
    \item 要求最小化目标函数值:将目标函数的系数取反。
    \item 变量不具有非负约束:若变量$x_i$没有非负约束,则引入两个非负变量
    $x_{i1},x_{i2}$,满足$x_i=x_{i1}-x_{i2}$。然后把线性规划中的$x_i$
    替换为$x_{i1}-x_{i2}$。
    \item 约束中有等式约束:将$f(x)=b$拆为$f(x)\geq b$和$f(x)\leq b$。
    \item 约束中有$f(x)\geq b$形式的约束:约束的系数与常数均取反。
\end{itemize}
\subsubsection{松弛型}
单纯形算法需要把不等式约束转换为等式约束。
对于每一个约束$f_i(x)\leq b_i$,引入一个非负{\bfseries 松弛变量}$x_{n+i}$使得
$x_{n+i}=b_i-f_i(x)$。记目标函数值为$z$,则也可以引入等式$z=c^Tx$。在此称等式
左边的变量为{\bfseries 基本变量},等式右边的变量称为{\bfseries 非基本变量}。实际上
基本变量与目标函数值都被表示为常数+非基本变量线性加权和的形式。等式组的变量不同时出现
于等号两边,基本变量也不会出现两次。{\bfseries 注意等式左边的变量不等同于松弛变量,
因为单纯形算法的转动操作会使变量的位置改变。}
\subsection{单纯形算法}
\index{S!Simplex Algorithm}
\subsubsection{原理}
\paragraph{基本解}
将线性规划问题转换为松弛型,令非基本变量的值为0,就可以确定基本变量的值与目标函数值。
这是该线性规划的一个{\bfseries 基本解}。若它对应的基本变量的值均非负,则说明它是
一个{\bfseries 基本可行解}。

单纯形算法的步骤就是:
\begin{itemize}
    \item 找到初始基本可行解,若没有则说明该线性规划不可行。
    \item 不断迭代:计算当前基本解对应的解和目标函数值,判断是否最优。如果不是,
    根据计算结果执行``转动''操作更换基本变量得到更优解。
\end{itemize}
\subsubsection{转动pivot}
转动过程每次选取一个非基本变量$x_e$和一个基本变量$x_l$,交换它们在约束中的
位置。因此$x_e$称为替入变量,$x_l$称为替出变量。

在实现中用$A[m][n]$表示约束和目标函数,$id[1\cdots n]$表示等式右边每列的非基本
变量编号,$id[n+1\cdots n+m]$表示等式左边每列的基本变量编号。行$A[0]$用于存储
目标函数信息。$A[i][0]$满足
\begin{displaymath}
    A[i][0]+\textrm{目标函数值}z=\sum_{j=1}^n{A[i][j]x_{id[j]}}\textrm{~if~}i=0
\end{displaymath}
该式可由$z=c^Tx$推导。

或
\begin{displaymath}
    A[i][0]-x_{id[n+i]}=\sum_{j=1}^n{A[i][j]x_{id[j]}}\textrm{~otherwise~}
\end{displaymath}
该式可由$x'=b-f(x)$推导。

令非基本变量值为0后,$-A[0][0]$就是目标函数值$z$,$A[i][0](i=1\cdots m)$
就是$x_{id[n+i]}$的值。初始化矩阵时,令$A[0][0]$为0,其余参数直接填入对应位置,
然后让$id[i]=i(i=1\cdots n)$。由于最终我们只要$x_i(i=1\cdots n)$的值,不必
初始化松弛变量的编号。

执行转动时,首先交换对应位置的变量编号,然后处理替出变量所在约束的矩阵行,
最后处理其余行。

\paragraph{处理替出变量所在行}
记替入变量为$x_{id[e]}$,替出变量为$x_{id[n+l]}$,该行$L=A[l]$除
$x_{id[e]},x_{id[n+l]}$外的其余非基本变量组成的向量为$X$,系数向量为$c$,有
\begin{eqnarray*}
    L[0]-x_{id[n+l]}&=&c^TX+L[e]x_{id[e]}\\
    \Rightarrow \frac{L[0]}{L[e]}-\frac{1}{L[e]}x_{id[n+l]}&=&
    \frac{c^T}{L[e]}X+x_{id[e]}\\
    \Rightarrow \frac{L[0]}{L[e]}-x_{id[e]}&=&
    \frac{c^T}{L[e]}X+\frac{1}{L[e]}x_{id[n+l]}
\end{eqnarray*}

\paragraph{处理其余行}
类似于高斯消元法,用行$A[l]$消去其它行中的$A[i][e]$项。注意还要计算替出变量的系数,
所以$A[i][e]$要先置0。
\paragraph{性质}
如果当前基本解是可行解且pivot操作的参数$l,e$满足$A[l][0]/A[l][e]$是所有满足
$A[l][e]>0$的行中的最小值,那么pivot后得到的基本解仍然是一个可行解。

证明:
\begin{itemize}
    \item 替出变量所在行:由于$L[0]$非负且$L[e]$为正,pivot后
    $x_{id[e]}$的值$\frac{L[0]}{L[e]}$非负。
    \item 其余行:由于$A[l][0]/A[l][e]$最小,转动替出变量所在行后
    该行的常数项$A[l][0]$最小。设当前要消第$j$行:
    \begin{itemize}
        \item 若$A[j][e]>0$,则满足$A[l][0]/A[l][e]\leq A[j][0]/A[j][e]$。
        变换得$A[j][0]'=A[j][0]-A[l][0]/A[l][e]*A[j][e]\geq 0$,仍然保持非负。
        \item 若$A[j][e]\leq 0$,则$A[j][0]$消元后不减,仍然保持非负。
    \end{itemize}
\end{itemize}

\subsubsection{主过程simplex}
\paragraph{判断是否为最优解}
当对固定的基本变量组进行矩阵行变换后目标函数中所有的系数都非正,
则说明当前基本解是最优解。

证明:由于目标函数值只取决于非基本变量的值,且基本解中非基本变量均为0,只能提升
非基本变量的值来改变目标函数值。由于系数均非正,即使提升非基本变量的值仍然可行,
目标函数值也不增。
\paragraph{选取替入/替出变量}
选取目标函数值中系数非正的非基本变量进行提升是无用的,因此要选取一个满足$A[0][e]>0$
的变量当做替入变量。由pivot的性质可得只有$A[l][e]>0$且$A[l][0]/A[l][e]>0$才能在
pivot操作后保证其仍然为基本可行解。也就是说,一旦确立了替入变量,替出变量的选择就被
限制了。在使用pivot操作提升非基本变量后,不仅保证了当前基本解仍然是可行解,还能提升
目标函数值(除非pivot后替入变量的值为0)。目标函数值不提升的现象被称为{\bfseries 退化},
可能导致算法出现循环无法终止,因此需要使用Bland规则来避免循环。

\index{B!Bland's Rule}
根据Bland规则,总是选取下标最小的替入变量以及对应最优且下标最小的替出变量,可以避免
单纯形算法的循环。使用Bland规则后,可以保证算法在$\binomial{n+m}{m}$次迭代内终止,当然
实践中单纯形算法的表现很好,很少遇到如此刁钻的数据(或许随机化有助于改善算法运行时间)。

算法每次迭代后的目标函数值不降,且算法会在有限次迭代内结束(枚举完所有可能的基本变量
组后),因此算法会输出最优解。{\bfseries 这里还需要证明基本可行解集合中含有最优解。
不过基本解对应了凸集中的边界,根据经验最优解一定在边界上取得。更标准的证明留坑待补。}
\index{*TODO!单纯形算法正确性证明}
\paragraph{判断无界}
若存在系数非负的候选替入变量但不存在对应的替出变量,则该线性规划是无界的。

不存在对应的替出变量意味着所有$A[i][e]\leq 0$,那么任意提高该替入变量的值,
仍然保持其它非基本变量为0,既能保持当前基本解仍然为可行解,还能提高目标函数值。
因此该线性规划是无界的。
\subsubsection{初始化init}
找到一个初始基本可行解意味着要让松弛型中约束的常数项非负。
\paragraph{辅助线性规划}
首先检查初始松弛型是否已经对应了基本可行解。记$k$为满足$A[k][0]$最小的下标,
若基本解不可行则有$A[k][0]<0$。

然后引入一个新变量$x_0$,和原有的线性规划组成一个新的标准型线性规划:
\begin{itemize}
    \item 最大化$-x_0$
    \item 满足约束$Ax-x_0\leq b$
    \item 满足约束$x,x_0\geq 0$
\end{itemize}
原线性规划可行当且仅当该线性规划的最优值是0。

首先构造出该线性规划$B$,其中$x_0$放置于非基本变量组的末尾。然后执行\\
$pivot(k,n_B)$,这样就得到了线性规划$B$的一个基本可行解。

证明:
\begin{itemize}
    \item 替出变量所在行:$L[0]$为负且$L[e]=-1$,转动后$L[0]'=-L[0]>0$。
    \item 其余行:由于$A[i][e]=-1$,每行都要加上一倍$L$,由于$L[0]$是
    所有$A[i][0]$中的最小项,所以$A[i][0]'\geq 0$。
\end{itemize}

然后对$B$运行simplex找到最优值,最优值不为0则返回不可行。

若$x_0$此时为基本变量,在其对应行中选取一个系数非0的非基本变量作为替入变量,
把$x_0$换出去。

注意:
\begin{itemize}
    \item pivot操作并不会使最优解变小,因为转动后由基本解的定义
    可知$x_0=0$。
    \item pivot操作不会让基本解不可行,因为$L[0]=0$,其它的$A[i][0]$仍然
    保持非负。
\end{itemize}

移除$x_0$,把$B$的结果写回原线性规划中,完成初始化。
\paragraph{随机初始化}
不断迭代,每次随机选一个$l$满足$A[l][0]<0$,再随机选择一个$e$满足$A[l][e]<0$,
执行pivot后可以使$A[l][0]$为正。
\paragraph{避免初始化}
可以根据题目性质,松弛一些约束(比如将$=$松弛为$\leq$),使得初始线性规划就是合法
的松弛形。
\subsubsection{算法实现(UOJ179)}
为了过掉UOJ的Extra Test,这里使用了两种初始化方法的混合(然而还是过不去,
只能指望Mehrotra predictor–corrector method了)。

Update:事实上应该是浮点数精度不够。

\lstinputlisting{Source/Templates/SimplexSLP.cpp}

\subsubsection{稀疏矩阵优化}
单纯形算法性能的关键在于pivot操作。如果矩阵$A$为稀疏矩阵,则首先考虑使用其它算法
解决。一定要用单纯形算法解决的,可以扫描一遍行$L$记录非零元素到队列,消元时也判断
一下系数是否为0。
\subsection{对偶线性规划}
标准型的对偶线性规划为:
\begin{itemize}
    \item 最小化$b^Ty$
    \item 满足约束$A^Ty\geq c$
    \item $y\geq 0$
\end{itemize}

标准型和它的对偶线性规划最优值相等,即$c^Tx=b^Ty$。

对偶线性规划用来转化问题,但要慎用单纯形算法计算。

要注意转化问题时要不要加入选择个数限制,有时由于贪心的缘故可以去掉这个限制以
简化问题。
\subsection{全幺模矩阵}
\index{T!Totally Unimodular Matrix}

一个矩阵$A$是{\bfseries 全幺模矩阵}的充分条件如下:
\begin{itemize}
    \item $A$的元素仅有-1,0,1。
    \item 每列最多有2个非0数。
    \item 行可以分为2个集合,根据列来划分集合:
    \begin{itemize}
        \item 若列中有2个同号非0数,两行不在同一集合
        \item 若列中有2个异号非0数,两行在同一集合
    \end{itemize}
\end{itemize}

若线性规划的系数矩阵$A$为全幺模矩阵,或许可以用int代替double存储矩阵。

任何最大流以及最小费用最大流问题的线性规划矩阵都是全幺模矩阵。\CJKsout{如果发现
线性规划矩阵是个全幺模矩阵,可以考虑将其转化为简单的网络流模型来做。}

上述内容参考了算法导论\cite{ITA3}第29章与Angel\_Kitty的博客\footnote{
    线性规划之单纯形法【超详解+图解】
    \url{http://www.cnblogs.com/ECJTUACM-873284962/p/7097864.html}
},该篇文章末尾引用了Candy?的博客\footnote{
    [单纯形法与线性规划]【学习笔记】
    \url{https://www.cnblogs.com/candy99/p/lp.html}
}。网上博客的术语定义杂乱,这里使用算法导论中的定义。

\section{随机化算法}
随机化算法一般用来解决判定性问题与最优化问题。
\subsection{Monte Carlo算法}
\subsubsection{有关枚举元素的问题}
该问题需要枚举每个子集,然后对单个枚举进行计算。

对于这种问题我们有复杂度无法接受的穷举法,将穷举改为随机选取固定次数
的子集,可以在规定时间内完成计算。

使用一些贪心技巧或利用题目性质可以提高单次枚举正确率与采样数。
\subsubsection{线性加权和问题}
每次随机生成一个权重向量与向量做内积/与矩阵做乘法。
\subsubsection{复杂度与正确率}
OI中一般只需设计产生单侧错误算法。

若单次正确率为$p$,复杂度为$O(f(n))$,则测量$k$次的正确率为$1-(1-p)^k$,
复杂度为$O(kf(n))$。
\subsection{Las Vegas算法}
留坑待补。

上述内容参考了国家集训队2014论文集胡泽聪的《随机化算法在信息学竞赛中的应用》。


\chapter{理论}
\minitoc
\section{时间复杂度分析}
\subsection{主定理}
\begin{theorem}[Master Theorem]
	对于递归式
	\begin{displaymath}
		T(n)=aT(n/b)+f(n)
	\end{displaymath}
	有如下渐近界:
	\begin{itemize}
		\item 若对常数$\varepsilon>0$,有$f(n)=O(n^{\log_b{a-\varepsilon}})$,
		      则$T(n)=\Theta(n^{\log_ba})$。
		\item 若$f(n)=\Theta(n^{\log_ba})$,则$T(n)=\Theta(n^{\log_ba}\lg n)$
		\item 若对常数$\varepsilon>0$,有$f(n)=\Omega(n^{\log_b{a+\varepsilon}})$,
		      则$T(n)=\Theta(f(n))$。
	\end{itemize}
\end{theorem}
简单来说,先比较函数$f(n)$和$n^{\log_ba}$的渐近大小,若不同则选择较大的一个,
相同则再乘个$\lg n$。

证明留坑待补。
\index{*TODO!主定理证明}
\subsubsection{典例}
\begin{theorem}
	若$f(n)=\Theta(n^{\log_ba}\lg^kn)$,其中$k\geq 0$,则主递归式的解为
	$T(n)=\Theta(n^{\log_ba}\lg^{k+1}n)$。
\end{theorem}
证明:
注意此例不能使用主定理,考虑对递归式进行展开,假设$n$为$b$的幂,有
\begin{eqnarray*}
	T(n)&=&\sum_{i=0}^{\log_bn}{a^i\Theta(n^{\log_ba}\lg^kn)}\\
	&=&\Theta\left(\sum_{i=0}^{\log_bn}{\left(a^i\left(\frac{n}{b^i}\right)^
				{\log_b a}(\lg n-i)^k\right)}\right)\\
	&=&\Theta\left(\sum_{i=0}^{\log_bn}{\left(n^{\log_ba}
	(\lg n-i)^k\right)}\right)\\
	&=&\Theta\left(n^{\log_ba}\sum_{i=0}^{\log_ba}{(\lg n-i)^k}\right)
	~\textrm{因为}\lg n\leq \log_ba\\
	&=&\Theta\left(n^{\log_ba}\lg^{k+1}n\right)
\end{eqnarray*}

以上内容参考了算法导论\cite{ITA3}第4.5节。
\subsection{Akra-Bazzi法}
对于子问题规模划分不均衡的算法,不能使用主方法,
Akra-Bazzi法解决了如下递归式的渐进界计算:
\begin{displaymath}
	T(x)=\left\{\begin{array}{ll}
		\Theta(1)                     & 1\leq x \leq x_0 \\
		\sum_{i=1}^k{a_iT(b_ix)}+f(x) & x>x_0            \\
	\end{array}\right.
\end{displaymath}
其中常数$x_0\geq 1/b_i$且$x_0 \geq 1/(1-b_i)$,$0<b_i<1$,$a_i>0$,$f(x)$
满足存在正常数$c_1,c_2$使得对于$b_ix\leq u \leq x$,有$c_1f(x)\leq f(u)
	\leq c_2f(x)$(或者$|f'(x)|$的上界为多项式时也满足条件)。

首先计算满足$\sum_{i=1}^k{a_ib_i^p}=1$的$p$,然后可得
\begin{displaymath}
	T(n)=\Theta\left(x^p\left(1+\int_1^x{\frac{f(u)}{u^{p+1}} \ud u}\right)\right)
\end{displaymath}

上述方法参考了算法导论\cite{ITA3}第4章的本章注记。

\section{数值编码}
\subsection{整数编码}
以下三种编码的首位都是符号位,正0负1。
\paragraph{原码}数值位为真值的绝对值。
\paragraph{反码}正数不变,负数数值位为原码数值位取反。
\paragraph{补码}正数不变,负数数值位为原码数值位取反+1。补码转反码
时-1取反或者取反+1。
\subsection{浮点数编码}
留坑待补。
\index{*TODO!浮点数编码}

\section{常见排序算法}
\subsection{比较排序算法}
\begin{tabular}{|c|c|c|c|c|}
	\hline
	算法     & 复杂度(最好/平均/最坏) & 稳定性  & 原址排序 \\ \hline
	快速排序 & $n\lg n/n\lg n/n^2$      &         & $\surd$  \\	\hline
	归并排序 & $n\lg n$                 & $\surd$ &          \\ \hline
	堆排序   & $n\lg n$                 &         & $\surd$  \\ \hline
	插入排序 & $n/n^2/n^2$              & $\surd$ & $\surd$  \\ \hline
	选择排序 & $n^2$                    &         & $\surd$  \\ \hline
	希尔排序 & $n\lg n/-/n^{4/3}$       &         & $\surd$  \\ \hline
	冒泡排序 & $n/n^2/n^2$              & $\surd$ & $\surd$  \\ \hline
\end{tabular}
\subsection{非比较排序算法}
\begin{itemize}
	\item 计数排序
	\item 基数排序
	\item 桶排序
\end{itemize}
上述内容参考了Wikipedia-EN\footnote{Sorting algorithm - Wikipedia
	\url{https://en.wikipedia.org/wiki/Sorting\_algorithm
		\#Comparison\_of\_algorithms}
}

\section{NP完全性}
\subsection{定义}
\paragraph{P} P类问题是在多项式时间内可被解决的问题。
\paragraph{NP} NP类问题是在多项式时间内解可被检验的问题。
\paragraph{NP-Complete} NP完全问题是所有NP类问题在多项式时间内
可约化的NP问题,若存在NPC问题有多项式算法,则P=NP。
\paragraph{NP-Hard}NP-Hard问题是所有NP类问题在多项式时间内可约化的非NP问题。
\subsection{常见NPC问题}
\begin{itemize}
    \item $k-SAT,k>2$
    \item 哈密尔顿回路
    \item 最大团
    \item 最小点覆盖
    \item 旅行商问题
    \item 子集和问题
    \item 子图同构问题
    \item 整数线性规划问题
    \item 集合划分问题
    \item 最长简单回路问题
\end{itemize}


\section{杂讲}
\subsection{竞赛图}
\index{T!Tournament}
竞赛图是一个无向完全图被定向后得到的图。
\begin{theorem}
	竞赛图缩点后是一条链。
\end{theorem}
\subsubsection{竞赛图判定}
竞赛图可以用来指示两两选手比赛的胜负,判定比分是否
合法即判定是否存在合法的竞赛图。
\index{L!Landau's Theorem}
\begin{theorem}[Landau's Theorem]
	对于一个有序的竞赛图度数序列/得分序列$0\leq s_1 \leq s_2 \leq \cdots \leq s_n$,
	有$\displaystyle \forall 1\leq k\leq n,\sum_{i=1}^k{s_i}\geq \binomial{k}{2}$
	,当$k=n$时等号必须成立。
\end{theorem}
对度数/胜利场数排序后逐个判断即可。
\subsection{最小平均值环}
对于一个有向图,找出平均值最小的环。

类似于分数规划的思想,对平均值进行二分,将所有边权减去二分值,
若存在负环则说明存在环的平均值小于该二分值,不断二分即可。
\subsection{平面图性质}
\index{P!Planar Graph}
以下仅讨论$V\geq 3$的情况:
\begin{property}
	$E\leq 3V-6$
\end{property}
\begin{property}
	$F\leq 2V-4$
\end{property}
使用这些性质可以限制边数以加速平面图判定。

上述内容参考了Wikipedia-EN\footnote{Planar graph - Wikipedia
	\url{https://en.wikipedia.org/wiki/Planar\_graph}
}
\subsection{拓扑排序判环}
若拓扑排序无法使得所有点都入队则说明存在环。
可以通过枚举点,将其度数-1取得删掉一条边的效果。
例如CF915D Almost Acyclic Graph:
\lstinputlisting{Source/Source/TopSort/CF915D.cpp}
\subsection{Lindström–Gessel–Viennot Lemma}
\index{L!Lindström–Gessel\\–Viennot Lemma}
给定一个DAG,以及$n$个起点$a_1,a_2,\cdots,a_n$和对应终点$b_1,b_2,\cdots,b_n$,
求这$n$条点不相交(包括终点)路径的方案数。

根据Lindström–Gessel–Viennot Lemma,记$e(a_i,b_j)$为$a_i\rightarrow b_j$的
路径方案数,答案为\begin{displaymath}
	det\left(\left[\begin{array}{cccc}
			e(a_1,b_1) & e(a_1,b_2) & \cdots & e(a_1,b_n) \\
			e(a_2,b_1) & e(a_2,b_2) & \cdots & e(a_2,b_n) \\
			\vdots     & \vdots     & \ddots & \vdots     \\
			e(a_n,b_1) & e(a_n,b_2) & \cdots & e(a_2,b_n) \\
		\end{array}\right]\right)
\end{displaymath}
\subsubsection{推广}
实际上$e(a_i,b_j)$为$a_i\rightarrow b_j$的所有路径上边权积之和,
类似Matrix-Tree定理的讨论可扩展到边权相关问题。

上述内容参考了Wikipedia-EN\footnote{
	Lindström–Gessel–Viennot lemma - Wikipedia
	\url{https://en.wikipedia.org/wiki/Lindstr\%C3\%B6m\%E2\%80\%93Gessel\%E2\%80\%93Viennot\_lemma}
}。

\appendix
\chapter{资料推荐}
\printindex
\begin{thebibliography}{999}
	\bibitem{NFTGC} Even, Shimon; Tarjan, R. Endre (1975).
	"Network Flow and Testing Graph Connectivity".
    \emph{SIAM Journal on Computing}.
    \bibitem{DSNA} Tarjan, R. E. (1983).
    \emph{Data structures and network algorithms.}
    \bibitem{MCIOI}胡伯涛(2007). \emph{最小割模型在信息学竞赛中的应用}
    \bibitem{ITA3}  Thomas H.Cormen / Charles E.Leiserson /
     Ronald L.Rivest / Clifford Stein.
     \emph{Introduction to Algorithms Third Edition}
    \bibitem{kdTree}n+e. \emph{K-D Tree 在信息学竞赛中的应用}
    \bibitem{huffman}M. J. Golin and G. Rote.
    \emph{A dynamic programming algorithm for constructing optimal
    prefix-free codes with unequal letter costs}
    \bibitem{LLH}L. L. Larmore and D. S. Hirschberg.
    \emph{A fast algorithm for optimal length-limited Huffman codes}
\end{thebibliography}

\backmatter
\chapter{后记}
\section{初稿}
2018年10月23日

写这本笔记大概花了我两个月的时间,此时这本笔记大概有200千字(50千字为中文),我
为自己的成果感到骄傲。在此期间我系统地复习了NOIP复赛以上的知识点,学到了许多新的方法/技巧/
思路,重新理解了之前看不懂的算法,收获颇多。我将自己的想法记录于此,以便之后重新翻阅。
该笔记在内容上还是较为完整的,在接下来的时间内我会在巩固的过程中继续学习并补充新的高级算法
/数据结构,以应战省选和NOI2019。

在我作为D类队员参加了NOI2018现场赛后,我发现了自己的严重问题------知识点掌握不扎实,
导致我没能拿到该得的分数。所以我决定自己写一写复习笔记来建立出一套自己的思维体系。

同时作为目前校内在OI道路上走的最远的人(其他人初赛都过不了。。。孤单),我没有
信息学强校学生所拥有的资源——学长。有学长的指导和遗留下的资料,能够少走许多弯路。
既然自己没有学长,只好自己做学长了。这本笔记算是是我的一份心意吧,希望学校的OI事业
能够有所发展,学弟们(???我好像根本不知道有学弟)如果需要我帮助的地方,随时联系我。
我的表达能力不太好,而且有些简单的地方没有细讲,你们可以打开页底下的参考资料链接细看
(这些资料是我自己能够接受并理解的),虽然这些东西在百度和维基上都能轻易找到,但我之前
根本就不知道这些东西的存在(直到考完后才知道,已经晚了)。。。所以你们就把它当做一种索引吧。

在写这本笔记的过程中,我发现自己在使用\LaTeX{}和排版方面还存在许多问题:
\begin{itemize}
    \item 目录太长,分类过细
    \item 参考资料链接过多
    \item 不恰当的知识点分类
    \item 过长的TODO List
    \item 滥用lstlisting
    \item 书籍排版密度过低
    \item 滥用Index
    \item 未熟练使用BibTex
    \item 数学公式排版掌握不熟
    \item 缺少图片
    \item 语言表达不当或过于口语化(才发现``即可''之类的词是从其它blog上学来的)
\end{itemize}

上述问题将会在接下来几轮的review中得到改善。

在此我要感谢学校领导、老师和同学的理解与支持,感谢父母(还有我妹)对我的
大力支持。感谢写出优秀博客的OIers和Wikipedia-EN的维护者,他们为我提供了大量的
参考资料与经验总结,使我受益匪浅,我所做的不过是将它们聚集在一块而已。
\CJKsout{差点忘了感谢CCF}。

刚开始写这本笔记时,我遇到了一件让我感到最幸福的事:我喜欢的女生向我表白了。
在此之前我已经预先认真考虑了几个月,所以当时我毫不犹豫就接受了。那个月是我最幸福的日子,
我开始改变自己,更加努力地学习,以应对将来需要承担的责任。她给了我写这本笔记的动力,她能够
理解许多别人所不能理解我的事情。我当时认为自己是这个世上最幸运的人。但好景不长,在本笔记
初稿即将完成时,我最担心的事情还是发生了——双方突然无话可说。我是预先知道这个问题会发生,
而且已经预先提醒过她了——遇到问题要多沟通,商量解决方案。但在短短几天内,她对我的态度越来
越差。最后她以一堆自相矛盾的理由离开了我,根本不给我与之沟通的机会。从前的誓言,以及我对
她的好,她似乎已经完全忘记了。我不知道她的真实意图,她或许是为了让我安心学习。我仍然相
信自己的单元测试是可靠的,她最终会回来的。我决定履行自己之前的诺言:我等她两年。我好想在
她身边给她讲解数学题;好想在她生理期时给她端杯热水,陪在她身边;好想和她一起去公园散步,
在绿道慢跑,一起看动画电影;好想在她心情不好时,自己在旁边给她疏导,给她一个拥抱;好想
静静地坐在一旁,听她弹琴唱歌;好想再让她笑着敲我的头;好想高中毕业后用自己的Renderer给她
渲一个鸽了两年的生日礼物。唉,这些愿望怕是无法实现了。现在自己还是先努力学习,认真搞OI,
考上理想的大学。\CJKsout{得之我幸,不得我命。未来会怎么样,得看自己的造化了。}
{\bfseries I can do it!}

$\frac{\sin 4\theta}{\sin \theta}$,我等你回来。

\section{二轮复习}
2019年1月3日

不得不吐槽Review花的时间居然比编写还要久。。。

二轮复习后我对该笔记做了大量修改,主要内容如下:

\begin{itemize}
    \item 补充了大量高级内容
    \item 补充了刷题过程中的一些解题方法、坑点
    \item 修改了一些不恰当、不详细的文字表述
    \item 更正文中错误
    \item 规范符号与术语的表达
    \item 解决并同时新增了一些TODO
    \item \CJKsout{消灭了``即可''}
\end{itemize}

NOIP2018已经过去了,我初赛成绩86,复赛成绩496,成绩还算令人满意(这里的人指老师、
校领导和局领导们)。虽然我能凭这个成绩去WC2019和THUWC2019,但是我知道自己的实力仍然不够。
NOIP2018暴露了我的另外一个问题:粗心。初赛因两题水题而不能AK,复赛因没有从理论上论证算法
正确性而留下成绩不能上500的遗憾(感谢CCF负责任的数据)。根本原因仍然是我对知识点的掌握不够
熟练。

我接下来的规划:

\begin{itemize}
    \item 针对薄弱专题继续复习与刷题。
    \item 巩固新学的知识。
    \item 在WC2019后继续学习新知识。
    \item 参加线上比赛以提高应试能力和分析思考能力
    \item 学会使用Vim,提高码速
\end{itemize}

在此我也总结一下我的2018吧。
\begin{itemize}
    \item 2018年,我结识了一些OIer,见识了各类型比赛,意识到自己的水平很菜。
    \item 2018年,我借助CUDA写了一个玩具级光栅化渲染器/基于物理的渲染器,感受到了
    CG的魅力,以及其幕后需要的大量数学/物理知识储备。
    \item 2018年,我经历了从年段第2一路跌到年段第九十几名,再一路从60,30,15回到
    年段第2的``过山车''式的惊险。在此感谢我的老师对我的鼓励,段长对我的教导,以及她
    在那段时间对我的支持与鼓励。
    \item 2018年,我经历了从喜欢到恋爱,再到失恋的标准早恋结局(事实上我的遭遇比
    早恋要惨得多)。我仍然等着她,因为我对她有承诺,对自己也有承诺。
    \item 2018年,我发现自己的计算机/OI/数学知识可以应用到许多方面,包括但不限于:
    \begin{itemize}
        \item 物理探究课上给使用传感器的物理实验设备安装驱动\CJKsout{,然后被物理老师
        当做苦力给每台机子都安装一遍}
        \item 给班里的英语剧做音频剪辑
        \item 被地理老师请去给机器人编程
        \item 用扒注册表的手段清除了班班通的病毒(顺便安利了一波火绒杀毒)
        \item 给数学老师和同学们安利了一波GeoGebra(垃圾几何画板)
        \item 在语文课上放映用\LaTeX{}做的\CJKsout{ppt}beamer,惊艳全场
        \item 在英语课上高举《线性代数及其应用》给老师讲``列''的单词应该是column
    \end{itemize}
    \item 到了年底,感觉到自己的身体快撑不住了(或许是因失恋而过度伤心)。
    于是我提早了运动计划的实施(本来想在退役后才开始的),坚持每天晨跑。一个星期后,
    困扰我多年的过敏性鼻炎有了好转。至此以后我上课更加精神了,因此被英语老师表扬。
    上个星期我还被路上的一位老人表扬了,继续坚持!!!
\end{itemize}

没有她的这段时间里,起初我是十分痛苦的。我的人生计划第二次被她打断,原来幻想的幸福美满
的家庭成了泡影。那段时间我一直都在听Right Here Waiting。后来我意识到再这样颓废下去,
不仅这个梦想破灭了,另一个梦想也将受到影响。如果这两个梦想能够实现一个,我就很满足了。
偶然有一次我听到《中国男儿》后,更加坚定了改变自己的决心。我第一次听这首歌是在电视剧
《五星红旗迎风飘扬》中。片尾氢弹爆炸的场景,以及这首歌的旋律和歌词,给我的印象很深。

歌词一起放在这里:
\begin{center}
中国男儿,中国男儿,要将只手撑天空。\\
睡狮千年,睡狮千年,一夫振臂万夫雄。\\
长江大河,亚洲之东,峨峨昆仑,翼翼长城,\\
天府之国,取多用宏,黄帝之胄神明种。\\
风虎云龙,万国来同,天之骄子吾纵横。\\
中国男儿,中国男儿,要将只手撑天空。\\
睡狮千年,睡狮千年,一夫振臂万夫雄。\\
我有宝刀,慷慨从戎,击楫中流,泱泱大风,\\
决胜疆场,气贯长虹,古今多少奇丈夫。\\
碎首黄尘,燕然勒功,至今热血犹殷红。\\
中国男儿,中国男儿,要将只手撑天空。\\
睡狮千年,睡狮千年,一夫振臂万夫雄。\\
长江大河,亚洲之东,峨峨昆仑,翼翼长城,\\
天府之国,取多用宏,黄帝之胄神明种。\\
风虎云龙,万国来同,天之骄子吾纵横。\\
\end{center}

前几天我又看了《中国面壁者》,感触很深。或许这才是我的归宿吧。

她提出分手前,我正在对她的感情稳定性进行测试(我是个奇怪的\CJKsout{人}计算机)。
当她说出我们已经无话可聊时,我的内心是欣慰的。我努力了这么久,终于构造出了一个
极端条件,这是一个很好的锻炼机会。但是她的下一句``分吗''如同晴天霹雳。既然她不喜欢我
也不再理我了,我的测试也变得既没有意义也不能实行。我只能努力改变自己,做好充分准备,
抓住一切机会。

出了这档子事,我的职业选择也不再局限于NVIDIA那种自由的工作了,本来想多陪陪家人的。
现在我感兴趣的职业方向如下:
\begin{itemize}
    \item CG:Disney的动画电影对我影响很深,所希望自己的技术能够协助
    电影艺术家创作出优秀的动画电影,向世人宣传正能量
    \item HPC:指挥一堆CPU协作和指挥千军万马一样浪漫
\end{itemize}

一言难尽啊。。。。适可而止吧,睡觉时间快到了。

比赛在即,我要更加努力。

THUWC2019\&WC2019\&FJOI2019 Round 1加油!!!

\section{三轮复习}
2019年4月4日

一轮更比一轮咕。

这三个月来我对该笔记做了大量修改,主要内容如下:

\begin{itemize}
    \item 补充了一些学习计划外的技巧(\CJKsout{需求刺激进步})
    \item \CJKsout{升级了Checker}
    \item \CJKsout{将某些连自己都看不懂的文字表述修改为
    将来的我看不懂的文字表述}
    \item 准备对解题思路程序化,添加了几类问题的系统解题思路。
    未来可能会做思维导图?
\end{itemize}

再记一下上一轮复习牵挂的事:

\begin{itemize}
    \item THUWC2019:进了面试,口语测试体验极差(\CJKsout{老师你刚才只叫我读啊,
    我就一个单词一个单词地念啊。老师:好了你可以出去了。}),拿到了神奇的三等约:
    再来一次。
    \item WC2019:冬眠营果然名不虚传,我从去广州开始一直睡到了FJWC2019。考试体验
    极差,交互题暴零又让我拿了一次Cu(要是CTSC2018没有Day3我就可以达成集齐NOI系列
    赛事Cu的成就了,可惜今年为了不浪费时间没报CTSC和APIO)。考试前还被去年CTSC/APIO
    的室友嘲笑了一番:你怎么看起来这么落寞啊。在考试前知道了他去年PKUWC已经签了无条件
    本一,进考场时心里很不是滋味。
    \item FJOI2019 Round 1:鸽到和FJOJ2019 Round 2一起进行,还有一个星期左右。
    不过从FJWC2019的模拟成绩来看,我连D类都够不着。
    \item 她:与我不再有任何联系,不过看她的学习和生活没有受到任何影响我也就放心了。
\end{itemize}

接下来的计划:
\begin{itemize}
    \item 完成系统解题思路梳理,把会的东西全部挂上去,不熟的东西舍弃掉。
    \item 继续按照专题刷题。
    \item 从头开始Review(下一次后记的时间会不会咕到退役?)。
    \item 省选后开始看集训队论文。
    \item 刷各大赛事的题目。
\end{itemize}

她离开后我感觉心里空落落的,有开心的事没有人分享,受到了打击也没有人安慰。再加上这一个
月多都在家里,每天除了吃饭睡觉散步,其它时间都待在电脑面前,一天没说几句话。我感觉自己
快要疯了,白天有时会摸鱼看新闻写Checker,晚上有时候写题写到一点多(有时在写Checker),
在自己房间里会颓废看视频,会一个人和自己用记事本聊天,会一个人缩在被子里泣\CJKsout{
(《水浒传》~第二十五回~王婆计啜西门庆~淫妇药鸩武大郎:看官听说:原来但凡世上妇人,哭有
三样:有泪有声谓之哭,有泪无声谓之泣,无泪有声谓之号。)}。我不知道再这样下去自己的生理
和心理会不会崩溃,写初稿的第一个月自己可不是这样的。只能强制自己死撑了,崩溃了也算是一种
解脱,只是感觉对不起身边对自己寄予厚望的人们。好怀念那一个月的时光啊,那可能是我最幸福
也是最痛苦的回忆了。

对于她的行为,我不能理解,无法接受,但是必须尊重。一年后的生日礼物按照计划准备,但自己
不想再打扰她,还是自己保存着吧。对于自己的感情,认识她之前我就已经跟自己讲得很清楚,只能
按照自己的承诺继续等。虽然复合的希望渺茫(我也不知道即使她愿意复合,我会不会再次信任她),
但这种来自直觉的决策应该是正确的,我等她两年。

感觉自己离开课堂好久了,有点厌学,根本不知道老师到底在讲什么,我已经没有退路了。
moe又在今年开始缩减自主招生规模,提高要求,更是把我逼到绝境。要是像室友那样早一年
进入面试,或许自己就不会遭受这么多打击了。看着他签了无条件本一后一脸轻松的样子,自己
很是羡慕,这两年省排名在自己左右的人基本都签了,就我只有一张废纸。

目前还有两次机会,我希望自己也能有一个True Ending。
\section{FCS NOI2019随笔}
\subsection{day0}
看着自己的笔记发现自己学了好多,这几天也不知道自己要做什么了。希望有备而来的今年能比
懵逼的去年考得更好。

真的好害怕自己在考场上又想起她。。。只要题目不会做我就会想她,然后崩盘(从NOI2018到
WC2019都有过)。而且现在已经分开了,我再也没办法找她要安慰了。今天晚上再小小地哭一会吧,
明早考试应该情绪会好些。

我感觉自己好奇怪,过着和正常学生不一样的生活。每天在不同计算机的面前切换,一整天很少说话,
在学校也很少参加集体活动,常年失踪,在同学眼里我就是个呆子。

但我想说,OIer/程序猿不是这样的!!!他们也可以多才多艺,也可以有丰富的生活,而不是大家公认的
``人傻钱多死的快''。只不过,我也不是正常的程序猿,我让身边人对程序猿的认知出现了偏差。我到现在
还记得初中毕业班的英语老师说的``不要像那些程序猿一样''

可能这里还有一些心理问题吧。不知道为什么,自托儿所开始就不合群,但是我一直都希望自己能够融入
集体。但我努力学习了,成为了老师眼里的乖孩子,但是换来的只不过是同学们的尊敬(仍然记得一年级
被段霸放过的场景),而不是关注。为了刷点存在感,我经常用超出课程的数学来给同学们讲题,但是我
看到的不是同学对未知的渴望,而是在讲台下各做各的。有时甚至连老师都没听,一个好的idea就这样被
埋没了,变成了我的个人表演。现在我在想出一个idea后,也不大愿意与大家分享了,自己晚上回去哭一
哭吧,祭奠一下它。分手前想给她做的题,在分手后18天我终于想出了一个绝妙的纯几何方法,可惜连这个
唯一的听众也离我而去。

所以我的心理才很脆弱啊。如果外校OIer对我防备,对我不理不睬,或者是父母说我这不会那不会,我会哭
一晚上。但如果有人对我很好,关心我(比如她),我可以给出一生的承诺。那是我第一次收到生日礼物,
那是我第一次与女生有长时间的对话,那是我第一次能够把自己的糗事分享出来,那是我第一次去关心一个人
的爱好。这就是喜欢的感觉?但是现在不存在了,那些是第一次,也可能是最后一次。

我实在是不理解,为什么她会把生活过的精彩说得这么容易。在高一,唯一可以调剂我的生活的,只有与
她聊天和捣鼓CG,在高一末被段长劝退CG后,她就是我的唯一。我也考虑过,在自己没有自己的爱好后,就会
对她特别依赖,这会导致不正常的结果。在她表白前也以影响学习为由询问她要不要暂停接触。可最后还是没有
狠下心,造成了现在这个尴尬的局面。我不后悔自己喜欢她,但很疑惑当初为什么她会喜欢我。

有的时候我在想,自己为什么会呆在计算机面前,而不是像正常高中生一样过着精彩的生活。最让人伤心的,
莫过于别人在科技文体艺术节、在元旦晚会上唱歌,而我却在机房里刷题。就算我去参与这些活动也做不好,
现在的我,离开了计算机,什么都不是。\CJKsout{给我一台计算机,我什么都能干!}

可能身边人都觉得我状态很好(没人发现我失恋了),天天乐呵呵的,但他们不知道我在被子里因为一些人
的不理睬、防备、嫌弃哭了多久才假装自己很正常。

好羡慕debug18,能和女朋友一起上THU\url{http://debug18.com/posts/my-2018/}。

好羡慕zhblue,能够引导自己的孩子探寻未知\url{http://www.hustoj.com/?p=1331}。

但对于我,她离开后,这些梦可能都无法实现了。但我希望自己的QQ头像永远不换。

希望这些东西在deadline之前不要被她知道,不希望因为这个而影响她对自己幸福的选择,不过或许这些
东西只会让大家觉得我更幼稚罢了。

\CJKsout{似乎跑题了。。。这是我的书,只要不是政治敏感的东西,爱写啥写啥。好像只有第一行与标题有关。}

\end{document}
