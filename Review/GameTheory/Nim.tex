\section{Nim系列游戏}

\subsection{Nim游戏}

普通Nim游戏的定义:
有两个玩家轮流从许多堆中移除对象。在每个回合中,玩家选择一个非空的堆,可以移除任何数量
的对象,但至少移除一个对象。无法操作的玩家为败者。

此类游戏可看做是Bash游戏的特殊化。

\begin{Theorem}
	$SG_{Nim}(x)=x$
\end{Theorem}

证明略。

\subsection{Bash游戏}

Bash游戏与普通Nim游戏的区别是增加了每次最多移除k个对象的限制。

\begin{Theorem}
	$SG_{Bash}(x)=x~mod~(k+1)$
\end{Theorem}

证明略。

\subsection{NimK游戏}

NimK游戏与普通Nim游戏的区别是每次可以从不超过k个堆中移除任意数目对象。

\begin{Theorem}\label{NimK}
	将每堆对象的数目拆位,若每位上1的个数mod(k+1)均为0,则必败,反之必胜。
\end{Theorem}

记忆:普通Nim游戏可理解为mod 2的情况。

算法正确性证明:



定理~\ref{NimK}得证。

\subsection{Anti Nim}

不能操作的玩家胜利。

\begin{Theorem}\label{AntiNim}
	先手必胜当且仅当满足以下条件之一:
	\begin{enumerate}
		\item $SG(x)=0$ 且所有堆的对象数都为1
		\item $SG(x)\not=0$ 且至少有一堆对象数大于1
	\end{enumerate}

\end{Theorem}

证明:
定义对象数为1的叫A堆,大于1的叫B堆。

\begin{enumerate}
	\item 若所有堆均为A堆,则奇数堆先手必败,反之必胜。
	\item 若B堆数等于1,显然$SG(x)\not=0$,则可根据堆的总数确定取该堆的数目,
	      使下一状态为情况1的奇数堆,所以先手必胜。
	\item 若B堆数大于1,则
	      \begin{enumerate}
		      \item 若$SG(x)=0$,则必须留下超过2个B堆并使$SG(x')\not=0$,否则会使
		            对方进入情况2的必胜态。
		      \item 若$SG(x)\not=0$,则根据Nim游戏的理论(必胜态->必败态),存在一种方法转移至情况3的子情况1。
	      \end{enumerate}
	      若玩家处于情况3的子情况2中,则可以在有限次回合内使对方无法转移至子情况2,
	      因此该状态为必胜态。
\end{enumerate}

定理~\ref{AntiNim}得证。

\subsection{阶梯博弈(Staircase Nim)}


\subsubsection{例题}

Luogu P3480 [POI2009]KAM-Pebbles\footnote{\url{https://www.luogu.org/problemnew/show/P3480}}

对于这题可将原条件通过差分转换为阶梯博弈模型($A_i \geq A_i-1 \Leftrightarrow
	A_i-A_i-1 \geq 0$)。

\lstinputlisting[title=Luogu P3480]{Source/'Game Theory'/3480.cpp}

出题灵感:Anti BashK游戏

以上内容参考了forezxl\footnote{anti-Nim游戏(反Nim游戏)简介
	\url{https://blog.csdn.net/a1799342217/article/details/78274410}}和
hehedad\footnote{关于nimk类型博弈的详细理解与解释
	\url{https://blog.csdn.net/chenshibo17/article/details/79783523}}的博客。
