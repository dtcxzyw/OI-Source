\section{平面最近点对}
平面最近点对的经典算法采用分治思想,即每次将点集分成等规模的两组,
递归求两组点内部的最近距离,然后根据返回的距离来求两组间的点的最近
距离。时间复杂度$O(n\lg n)$。

步骤如下:
\begin{enumerate}
    \item 若点集规模$\leq 3$,直接暴力解决,避免遇到点集规模$=1$的情况;
    \item 将点分割为左右大小相等的点集(已预先按照$x$排序)递归求解;
    \item 若子区间内的最小距离为$d$,则将整个区间内的距中轴距离$\leq d$的
    点加入候选集合,同时维持候选集合的点的纵坐标有序(可以在返回后$O(n)$归并,
    但千万不要sort,否则时间复杂度为$O(n\lg^2 n)$),对于每个点与$y$值
    差$\leq d$的点更新答案。
\end{enumerate}

\paragraph{时间复杂度分析}
关键在于证明对于每个点枚举的点数不超过某个常数。

由于左右子集都保证了其子集内部最近点对距离$\geq d$,使用计算步骤中所述的
简单剪枝,可以保证$2d*d$的矩形中需要检查的点最多为7个(每边最多4个,再扣除自己)。

代码如下(使用了全局最优值优化):
\lstinputlisting{Source/Templates/Nearest.cpp}
