\section{基础设施}
以下内容主要讨论二维空间中的计算几何。
\subsection{点,向量,直线,半平面的表示}
点与向量由2个坐标表示;半平面和直线由直线上一点$ori$与
直线的方向向量$dir$表示;直线上的一点可表示为$ori+dir*t$,通过控制$t$的取值还可以
表示射线或线段;
可以人为规定半平面的顺时针/逆时针180$\circ$为半平面所在点集,
通过叉积来判断点在半平面的哪一边。
\begin{lstlisting}
typedef double FT;
struct Vec {
    FT x,y;
    //constructor
    //operator+-*
};
struct Line {
    Vec ori,dir;
    Vec operator()(FT t) const {
        return ori+dir*t;
    }
};
\end{lstlisting}
\subsection{点乘与叉乘}
向量点乘$dot(a,b)=a.x*b.x+a.y*b.y=|a||b|cos<a,b>$,一般用来
判断与法向量的夹角,以及辅助叉乘计算两向量夹角。点乘满足加法分配律和交换律。

向量叉乘$cross(a,b)=a.x*b.y-b.x*a.y=|a||b|sin<a,b>$,这是两向量
构成的平行四边形的有向面积。一般用来判断向量的相对方向以及计算多边形的面积。
叉乘满足$cross(a,b)=-cross(b,a)$和加法分配律。

\begin{theorem}[拉格朗日公式]
cross(a,cross(b,c))=b*dot(a,c)-c*dot(a,b)
\end{theorem}

三维向量的叉乘计算了垂直于这两个向量的向量(两向量组成平面的法向量),即
\begin{displaymath}
    cross(a,b)=\left(\begin{array}{c}
        a.y*b.z-b.y*a.z\\
        a.z*b.x-b.z*a.x\\
        a.x*b.y-b.x*a.y
    \end{array}\right)
\end{displaymath}
其方向满足右手定则(右手四指与大拇指垂直,食指指向向量$a$,其余三指指向向量$b$,
大拇指方向即为叉乘方向),长度满足
$|cross(a,b)|=|a||b|sin<a,b>$。

结合点乘和叉乘可得点$A$与点$B,C,D$所组成的三棱锥$A-BCD$的有向体积为
\begin{displaymath}
    V=|\frac{1}{6}dot(\overrightarrow{BA},cross(\overrightarrow{BC},
    \overrightarrow{BD}))|
\end{displaymath}
\subsection{点到直线的距离}
设偏移向量$delta=p-ori$:
\begin{itemize}
    \item 计算$dot(delta,dir)/|dir|$可得到投影长度$d'$,根据
    勾股定理得到$d^2=|delta|^2-d'^2$。
    \item 计算$|cross(delta,dir)|$可得偏移向量与方向向量构成的平行四边形的面积,
    根据面积公式得到$d=|\frac{cross(delta,dir)}{|dir|}|$。
\end{itemize}
叉乘法的运算量少且精度较高,总是被选用。
\subsection{直线、线段的交点}
先上代码:
\begin{lstlisting}
Vec intersect(const Line& a,const Line& b) {
    Vec delta=a.ori-b.ori;
    FT t=cross(b.dir,delta)/cross(a.dir,b.dir);
    return a.ori+a.dir*t;
}
\end{lstlisting}
\paragraph{证明} 首先可以发现两直线的方向向量都已经被单位化了。
然后通过两个直角三角形中边与$sin$的关系可证。

线段相交也是如此,但首先要判断两线段是否相交。将该问题转换为
两线段是否互相平分。设线段为$a-b,c-d$,首先判断$a-b$平分$c-d$,
即$c,d$分别位于$a-b$两边,有$cross(c-a,b-a)*cross(d-a,b-a)\leq 0$,
同理对$c-d$做一遍即可。
\subsection{判定点是否在多边形内}
\subsubsection{随机射线法}
从点P开始随机引出一条射线,计算其与多边形的边的交点个数,若为奇数次则
在多边形内。注意射线恰好经过点时要重新选择方向。
\subsubsection{旋转角法}
从点P与多边形的一个点开始不断旋转到下一个点,直至转完一圈为止。
此时若点P旋转了0$\circ$,则在多边形外;若点$P$旋转了360$\circ$,
则在多边形内。旋转角可以使用$atan(cross(a,b),dot(a,b))$计算,以
$\pi(180\circ)$为界进行比较。
\subsection{向量的旋转}
\subsection{坐标系的切换}
\subsection{反射与折射}
\subsection{pick定理}
\subsection{切比雪夫距离}
\subsection{精度处理}
