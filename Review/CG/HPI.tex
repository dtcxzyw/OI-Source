\section{半平面交}
\index{H!Half-Plane Intersection}
\subsection{基本算法}
基本思路是对半平面按照极角排序,然后按照极角序加入半平面集合,
将无效的半平面(交点在其它半平面外)删除,最后删除首尾的半平面。
注意处理两半平面平行的情况,此时保留较近的半平面。

代码如下:
\lstinputlisting{Source/Templates/HPI.cpp}

一般可将线性不等式转换为半平面,然后使用半平面交来
判断是否有解/计算最优解,除去排序后时间复杂度$O(n)$。
\subsection{线性判空集}
半平面交的时间复杂度为$O(n\lg n)$,但当只要求
半平面交是否为空集时,可以使用期望时间复杂度为$O(n)$的随机化算法。

该算法维护当前半平面交的纵坐标最高点$P$。每次{\bfseries 随机}加入新半平面$L$时,
若$P$在$L$内,则直接跳过;否则新的$P'$必然在$L$与之前的半平面的交点上,用
之前的半平面在当前直线上截出可行线段,保留最高点作为新的点$P$。

该方法源自WC2012上钟诚的讲稿《概率与随机化算法》与解轶伦的文章《随机增量算法》。
