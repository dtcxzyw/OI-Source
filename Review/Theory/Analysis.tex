\section{时间复杂度分析}
\subsection{主定理}
\begin{theorem}[Master Theorem]
	对于递归式
	\begin{displaymath}
		T(n)=aT(n/b)+f(n)
	\end{displaymath}
	有如下渐近界:
	\begin{itemize}
		\item 若对常数$\varepsilon>0$,有$f(n)=O(n^{\log_b{a-\varepsilon}})$,
		      则$T(n)=\Theta(n^{\log_ba})$。
		\item 若$f(n)=\Theta(n^{\log_ba})$,则$T(n)=\Theta(n^{\log_ba}\lg n)$
		\item 若对常数$\varepsilon>0$,有$f(n)=\Omega(n^{\log_b{a+\varepsilon}})$,
		      则$T(n)=\Theta(f(n))$。
	\end{itemize}
\end{theorem}
简单来说,先比较函数$f(n)$和$n^{\log_ba}$的渐近大小,若不同则选择较大的一个,
相同则再乘个$\lg n$。

证明留坑待补。
\index{*TODO!主定理证明}
\subsubsection{典例}
\begin{theorem}
	若$f(n)=\Theta(n^{\log_ba}\lg^kn)$,其中$k\geq 0$,则主递归式的解为
	$T(n)=\Theta(n^{\log_ba}\lg^{k+1}n)$。
\end{theorem}
证明:
注意此例不能使用主定理,考虑对递归式进行展开,假设$n$为$b$的幂,有
\begin{eqnarray*}
	T(n)&=&\sum_{i=0}^{\log_bn}{a^i\Theta(n^{\log_ba}\lg^kn)}\\
	&=&\Theta\left(\sum_{i=0}^{\log_bn}{\left(a^i\left(\frac{n}{b^i}\right)^
				{\log_b a}(\lg n-i)^k\right)}\right)\\
	&=&\Theta\left(\sum_{i=0}^{\log_bn}{\left(n^{\log_ba}
	(\lg n-i)^k\right)}\right)\\
	&=&\Theta\left(n^{\log_ba}\sum_{i=0}^{\log_ba}{(\lg n-i)^k}\right)
	~\textrm{因为}\lg n\leq \log_ba\\
	&=&\Theta\left(n^{\log_ba}\lg^{k+1}n\right)
\end{eqnarray*}

以上内容参考了算法导论\cite{ITA3}第4.5节。
\subsection{Akra-Bazzi法}
对于子问题规模划分不均衡的算法,不能使用主方法,
Akra-Bazzi法解决了如下递归式的渐进界计算:
\begin{displaymath}
	T(x)=\left\{\begin{array}{ll}
		\Theta(1)                     & 1\leq x \leq x_0 \\
		\sum_{i=1}^k{a_iT(b_ix)}+f(x) & x>x_0            \\
	\end{array}\right.
\end{displaymath}
其中常数$x_0\geq 1/b_i$且$x_0 \geq 1/(1-b_i)$,$0<b_i<1$,$a_i>0$,$f(x)$
满足存在正常数$c_1,c_2$使得对于$b_ix\leq u \leq x$,有$c_1f(x)\leq f(u)
	\leq c_2f(x)$(或者$|f'(x)|$的上界为多项式时也满足条件)。

首先计算满足$\sum_{i=1}^k{a_ib_i^p}=1$的$p$,然后可得
\begin{displaymath}
	T(n)=\Theta\left(x^p\left(1+\int_1^x{\frac{f(u)}{u^{p+1}} \ud u}\right)\right)
\end{displaymath}

上述方法参考了算法导论\cite{ITA3}第4章的本章注记。
