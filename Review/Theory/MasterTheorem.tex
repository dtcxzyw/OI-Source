\section{时间复杂度分析}
\subsection{主定理}
\begin{theorem}[Master Theorem]
    对于递归式
    \begin{displaymath}
        T(n)=aT(n/b)+f(n)
    \end{displaymath}
    有如下渐近界:
    \begin{itemize}
        \item 对常数$\varepsilon>0$,若$f(n)=O(n^{\lg_b{a-\varepsilon}})$,
        则$T(n)=\Theta(n^{\lg_ba})$。
        \item 若$f(n)=\Theta(n^{\lg_ba})$,则$T(n)=\Theta(n^{\lg_ba}\lg n)$
        \item 若$f(n)=\Omega(n^{\lg_ba})$,且对于常数$c<1$和足够大的$n$满足
        $af(n/b)\leq cf(n)$,则$T(n)=\Theta(f(n))$。
    \end{itemize}
\end{theorem}
简单来说,先比较函数$f(n)$和$n^{\lg_ba}$的渐近大小,若不同则选择较大的一个,
相同则再乘个$\lg n$。注意第三种情况的特殊要求。

证明留坑待补。
\index{*TODO!主定理证明}
以上内容参考了算法导论\cite{ITA3}第4.5节。
\subsection{Akra-Bazzi法}
留坑待补。
\index{*TODO!Akra-Bazzi法}
