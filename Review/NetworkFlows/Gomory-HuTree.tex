\section{最小割树}
\index{G!Gomory–Hu Tree}
\subsection{构造}
考虑单次求最小割的过程,最小割将顶点集合一分为二,设求$u-v$的最小割$cut(u,v)$,
顶点被分割为集合$U,V$。
\begin{lemma}
	$\forall x\in U,y\in V,cut(x,y)\leq cut(u,v)$
\end{lemma}
\paragraph{证明}
若存在$x,y$使得$cut(x,y)>cut(u,v)$,则$cut(u,v)$无法把$x,y$分开,也就意味着无法把
$u,v$分开,$cut(u,v)$不是割。

然后对每个点集再次选择两个点求最小割,将其切分为两个集合,直到所有集合都只有一个点为止。
每次求完最小割后给$u-v$连一条权重为$cut(u,v)$的边,这样做最后能得到一棵树。
\begin{theorem}
	$u-v$在树上的链上最小边权等于$cut(u,v)$。
\end{theorem}
\subsection{询问}
建出树后可以使用倍增法或线段树+树链剖分$O(\lg n)$查询点对答案。

板子:
\lstinputlisting{Source/Templates/GHT.cpp}
上述内容参考了UranusITS的博客\footnote{
	[学习笔记]最小割树(Gomory-Hu Tree)
	\url{http://www.cnblogs.com/coder-Uranus/p/9771919.html}
}。

\subsection{Gusfield算法}
\index{G!Gusfield Algorithm}

\subsection{应用}
\subsubsection{k小割}
\subsubsection{最小割计数}
