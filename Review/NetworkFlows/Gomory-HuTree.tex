\section{最小割树}
\index{G!Gomory–Hu Tree}
\subsection{Gusfield算法构造}
\index{G!Gusfield Algorithm}

考虑单次求最小割的过程,最小割将顶点集合一分为二,设求$u-v$的最小割$cut(u,v)$,
顶点被分割为集合$U,V$。
\begin{lemma}
	$\forall x\in U,y\in V,cut(x,y)\leq cut(u,v)$
\end{lemma}
\paragraph{证明}
若存在$x,y$使得$cut(x,y)>cut(u,v)$,则$cut(u,v)$无法把$x,y$分开,与
$U\cap V=\emptyset$矛盾。

然后对每个点集再次选择两个点求最小割,将其切分为两个集合,直到所有集合都只有一个点为止。
每次求完最小割后给$u-v$连一条权重为$cut(u,v)$的边,这样做最后能得到一棵树。
\begin{theorem}
	$u-v$在最小割树中的链上最小边权等于$cut(u,v)$。
\end{theorem}
时间复杂度$O(V^3E)$。
\subsubsection{实现细节}
Dinic求出最小割后,根据最后一次BFS找增广路时存储的层次标号是否有效对两个集合重新划分。
若使用ISAP求最小割,仍然需要使用Dinic的BFS划分集合。
\subsection{询问}
建出树后可以使用倍增法或线段树+树链剖分$O(\lg n)$查询点对答案。

板子:
\lstinputlisting{Source/Templates/GHT.cpp}
上述内容参考了UranusITS的博客\footnote{
	[学习笔记]最小割树(Gomory-Hu Tree)
	\url{http://www.cnblogs.com/coder-Uranus/p/9771919.html}
}。
