\section{最大流}
\subsection{Dinic算法}
\index{D!Dinic}
个人比较喜欢使用Dinic算法(因为我只会这个)。

Dinic的计算流程如下:
\begin{enumerate}
	\item BFS建分层图;
	\item DFS找增广路。
\end{enumerate}

附上板子:
\lstinputlisting[title=Dinic]{Source/Source/'Network Flows'/2764.cpp}

算法正确性证明:

算法时间复杂度$O(n^2m)$,证明:

\subsubsection{优化}
\begin{itemize}
	\item 当前弧优化:
	\item 记录无法增广的点,避免重复计算。
	\item (玄学)BFS找到一条增广路就退出,无法解释。
\end{itemize}

\subsection{ISAP算法}
\index{I!ISAP}
留坑待补。
\index{TODO!ISAP算法}
\subsection{HLPP算法}
\index{H!HLPP}
留坑待补。
\index{TODO!HLPP算法}
\subsection{最大流与最小割}
