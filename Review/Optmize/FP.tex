\section{01分数规划}
\index{F!Fractional Programming}
该问题可表述为有一堆物品,每个物品有价值$a_i$与代价$b_i$,
询问选择的物品集合所能得到的$\displaystyle \frac{\sum_{i\in S}{a_i}}
	{\sum_{i\in S}{b_i}}$的最大值。易知目标函数为凸。

{\bfseries 注意当$\displaystyle \sum_{i\in S}{a_i}$或
$\displaystyle \sum_{i\in S}{b_i}$表示为$|S|$时,这是一个隐式的分数规划问题。
血泪史:[SCOI2014]方伯伯运椰子}

\subsection{二分答案法}

设答案为$x$,存在点集$S$满足$\displaystyle \frac{\sum_{i\in S}{a_i}}
	{\sum_{i\in S}{b_i}}\geq x$,将该式转化为$\displaystyle
	\sum_{i\in S}{a_i}-x\cdot \sum{i\in S}{b_i}\geq 0$。因此可二分答案$x$,
对原数据进行一些修改,便可将原问题转化为解决一个判定问题(或是新的最优化问题,即检查
$\displaystyle f(x)=\sum_{i\in S}{a_i}-x\cdot \sum{i\in S}{b_i}$的最大值
是否不小于0)。
一般使用网络流/SPFA判负环/MST辅助判定。

例如对于最小平均值环问题,二分环平均长度$x$,将每条边的长度减去$x$,然后
判断负环,得到下一次迭代的二分范围。

\subsection{Dinkelbach法}
\index{D!Dinkelbach Method}
易知每一个选择方案都对应着一条自变量为二分答案$x$的直线,其中横截距为该方案的目标函数值。
若$f_S(x)\geq 0$则说明该直线所对应的横截距$\geq x$。

在二分过程中检查二分点$x$的合法性时,通常求$max\{f(x)\}$来判定,
满足检查条件时仅仅令$L=x$就显得浪费了,因为检查下一二分点时得到的最优解可能不变。
Dinkelbach法的思路是维护答案的下界,使用最优解对应的直线横截距来当做下一次
迭代的起点,这种做法比二分法更快。不过需要保存解这一要求可能限制了它的应用范围(编码更麻烦)。

此法参考了tianxiang971016的博客\footnote{
	01分数规划问题相关算法与题目讲解(二分法与Dinkelbach算法) - ztx
	\url{https://blog.csdn.net/hzoi\_ztx/article/details/54898323}
}。
