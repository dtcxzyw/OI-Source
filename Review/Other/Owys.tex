\section{卡常}
\subsection{取模}\label{mod}
\begin{itemize}
    \item 对于多个两整数乘积之和的取模(比如模意义下矩阵乘法),
    可以设置一个阈值,(绝对值)超过该阈值才取模,最后再做一次取模。这种方法
    在保证加法不溢出的情况下大幅减少取模次数,同时将值存储在寄存器内访问速度更快,
    并且if分支的命中概率小,分支预测效率高。
    \begin{lstlisting}
    const Int64 end=std::numeric_limits<Int64>
        ::max()-asInt64(mod-1)*(mod-1);
    ...
        for(int i=0;i<n;++i)
            for(int j=0;j<n;++j) {
                Int64 sum=0;
                for(int k=0;k<n;++k) {
                    sum+=asInt64(A[i][k])*B[k][j];
                    if(sum>=end)
                        sum%=mod;
                }
                C[i][j]=sum%mod;
            }
    \end{lstlisting}
    \item 若可以肯定最终答案在整型范围内且只有加减运算,可以允许暂时的加法溢出;
    \item 若模意义加减操作多,则保证在所有计算过程中的数$\in[0,mod)$。
    \begin{lstlisting}
    int add(int a,int b) {
        a+=b;
        return a<mod?a:a-mod;
    }
    int sub(int a,int b) {
        a-=b;
        return a>=0?a:a+mod;
    }
    \end{lstlisting}
    \item 若模意义乘法操作多,则仅保证中间数$\in (-mod,mod)$,没有必要
    $\cdots =clamp(\cdots)$。在最后输出时$clamp$。
\end{itemize}
\subsection{矩阵乘法}
不同优化下的矩阵乘法性能差异巨大,下面记录一些常用的优化。
\begin{itemize}
    \item 见~\ref{mod}第一点;
    \item 考虑访问矩阵时cache的连续性,
    发现按照$i,j,k$访问时$B[k][j]$的访问位置跳跃较大,cache性能较低;
    但是如果按照$i,k,j$的顺序计算,就可以使$C[i][j]$与$B[k][j]$的访问位置连续,
    提高访问速度;
    \item 在第3层循环内进行循环展开。
    \item 遇到稀疏矩阵时使用$ikj$枚举顺序,提前判断$A[i][k]$是否为0。用这个方法
    能够获得跑过规模为$400$的矩阵快速幂的信仰。
\end{itemize}
性能测试代码:
\lstinputlisting{Other/Mat.cpp}

2000*2000矩阵乘法性能测试结果如下(i7-4790K):

mulStandard 105793.233 ms AC

mulOptimizedMod 27042.733 ms AC

mulOptimizedCache 12499.686 ms AC

mulOptimizedUnfold 11096.677 ms AC

每个算法由上一个算法修改而来,mulOptimizedUnfold使用了所有优化,
比原始算法快了接近10倍,可见对矩阵乘法进行优化还是很划算的。
\subsection{基于硬件的优化}
\begin{itemize}
    \item 循环展开:指定一个步长,满步长区间硬编码,剩余部分暴力。
    \item 多路并行:在循环展开的同时避免修改同一变量,即保持循环间的写独立,
    这样避免CPU流水线被打断。
    \item Cache优化:尽可能保证循环时访问位置连续。
    \item 尽可能地使用临时变量,这样可以保持数据在寄存器中,最后再写回数组。
    不仅减少了寻址时间,还能简化代码编写。
    \item 结构体对齐:安排结构体元素时按照对齐大小从大到小定义
    \item Cache优化:大数组的每一维大小都不要是2的幂,尤其是状压dp和FFT。
    因为Cache会存储额外信息。
    \item 寻址优化:高维数组寻址时使用指针代替。
    \item Cache优化:在性能敏感的地方,使用指针存储儿子(当然还要考虑if判断
    nullptr的开销)。
    \item 尽可能使用引用而不是指针:当派生类有多个基类,在继承体系中向上转型时,
    由于空引用是非法的,程序只要计算指针的偏移;由于允许存在空指针,根据规定,空指针
    向上转型后仍然是空指针,需要一次特判。
\end{itemize}
\subsection{位运算}\label{Bitwise}
\subsubsection{符号判断}
\begin{displaymath}
    sign(x)=\left\{\begin{array}{cc}
        1&x>0\\
        0&x=0\\
        -1&x<0
    \end{array}\right.
\end{displaymath}
\begin{lstlisting}
int sign(int x) {
    return (x>0)-(x<0);
}
\end{lstlisting}

此法同样适用于浮点数符号的判断。
\subsubsection{判断异号}
\begin{lstlisting}
bool flag=((x^y)<0);
\end{lstlisting}
\subsubsection{绝对值}
\begin{lstlisting}
int iabs(int x) {
    int mask=x>>31;
    return (x+mask)^mask;
}
\end{lstlisting}

若$x$为非负则为$(x+0)\oplus 0=x$,若$x$为负则为
$(x-1)\oplus \textrm{0xffffffff}=-x$(注意有符号右移时高位补符号位)。
\subsubsection{去末尾1}
\index{B!Brian Kernigan's Bit\\ Counting}
$v\&=v-1$,即使$v$末尾的$100\cdots$部分与$011\cdots$按位与。
\subsubsection{取末尾0的数量}
首先使用$w=v\&-v$取得最低位1,然后无符号乘以De Bruijn Sequences
\index{D!De Bruijn Sequences}常数0x077CB531U,最后前五位与每种$w$
一一对应。LUT可预处理。
\begin{lstlisting}
int countTZ(int x) {
    unsigned int bit=x&-x;
    return LUT[(bit*0x077CB531U)>>27];
}
\end{lstlisting}

64位整数下用的常数是0x07EDD5E59A4E28C2ULL。
\subsubsection{子集枚举}
一般使用如下写法:
\begin{lstlisting}
for(int i=S;i;i=(i-1)&S)
\end{lstlisting}

可以理解为每次都做一次忽略S非零位的减法,从全集开始枚举自然能够
枚举所有子集。
\paragraph{父集枚举}
将子集枚举对应操作取反,注意要指定全集end。
\begin{lstlisting}
for(int i=S;i<=end;i=(i+1)|S)
\end{lstlisting}
\subsubsection{右起连续的0/1取反}
0->1:x|(x-1),1->0:x\&(x+1)
\subsubsection{取右边连续的1}
$(x \oplus (x+1))>>1$
\subsubsection{取最高位1}
\begin{lstlisting}
int getHighest(int x) {
    for(int i=1;i<=16;i<<=1)
        x|=x>>i;
    return x^(x>>1);
}
\end{lstlisting}

其原理为倍增长度使最高位右边全置1,最后清空除最高位外的位。
\subsubsection{大小写转换}
由于ASCII码中大小写字母的比特位之间的特殊关系,异或一个空格\\(0b100000)可以切换大小写。

\subsubsection{整数log2}
一种写法是取最高位1+取末尾0的数量。不过这种写法比较麻烦。
$O(n)$预处理+查表虽然好写但又太浪费。考虑最大位数为$b$,满足
$2^b>n$,可以预处理$c=\lceil \frac{b}{2}\rceil$的表,查询时先
比较判断其位数是否小于$2^c$,若小于则直接查表,否则查$>>c$的值,再加回$c$。
$O(\sqrt{n})-O(1)$的复杂度绝对不会成为性能瓶颈。

以上算法基本参考了Sean Eron Anderson的文章\footnote{
    Bit Twiddling Hacks
    \url{http://graphics.stanford.edu/\~seander/bithacks.html}
}。
\subsection{搜索优化}
\begin{itemize}
    \item 维护全局最优值,尽可能剪枝;
    \item 对于一些计算几何题,通过随机旋转坐标系+不正确的贪心来提高
    寻找最优解的速度(也可以作为预处理指导剪枝)。
    \item 在求权值最优的点对时,使用随机算法骗分;
    \item 对于多次修改+全局询问的问题,其最优解不会移动太远,
    考虑从上一个最优解开始移动搜索。例如[ZJOI2015]幻想乡战略游戏。
\end{itemize}
\subsection{数组清零}
\begin{itemize}
    \item 整个数组的清零可以使用memset,因为它的实现可能有循环展开/SIMD优化。
    \item 若仅修改整个数组的部分数据,可以重新扫一遍修改时的数据,撤销修改操作/
    直接将对应位置置0(这会影响到算法时间复杂度,尤其是对于Dsu On Tree/cdq分治);
    \item 对于树状数组,在模拟树状数组修改算法置零时,若当前值为0,则直接退出,因为
    接下来的值肯定都为0(肯定在清除其他链上数时被清零了)。
    \begin{lstlisting}
    void clear(int x) {
        while(x<=siz && A[x]) {
            A[x]=0;
            x+=x&-x;
        }
    }
    \end{lstlisting}
    \item 对于bool数组无需清零,使用int记录其最后一次被标记为true的时间戳
    $timeStamp$,若时间戳与当前时间戳相等则为true,将当前时间戳+1可实现$O(1)$清零。
    对于其它值的存储也可以如此清零(std::pair<时间戳,实际数据>)。
    不过对其它值的存储的实际性能不如直接记录修改位置清零快(每次访问都要检查一次时间戳)。
\end{itemize}
\subsection{读入优化}
被别人的时间优势刺激到了才入了fread/fwrite的坑。。。

下面的测试使用同一组整数与字符串转换的算法:
\lstinputlisting[title=IOTest.h]{Other/IOTest.hpp}

数据生成器如下:
\lstinputlisting[title=IOGen.cpp]{Other/IOGen.cpp}

常规getchar/putchar:31058 ms
\lstinputlisting[title=Test A]{Other/IOTestA.cpp}

使用8MB自带缓冲区,显式fread/fwrite,关闭内置缓冲区:9724 ms
\lstinputlisting[title=Test B]{Other/IOTestB.cpp}

重置内置缓冲区为8MB,使用全缓冲模式:23461 ms
\lstinputlisting[title=Test C]{Other/IOTestC.cpp}

{\bfseries 注意\_IOFBF,\_IONBF这些东西在NOI系列赛事不能使用,不过它们都是常数,
可以预先取得它们的值。}

结果很明显,显式fread/fwrite速度最快。
\subsection{快速乘法取模}
当模数的平方超过long long的表示范围时,可以使用类似快速幂的方式计算快速乘法。

还有一个无法严格证明正确性的trick:将$a*b\%mod$表示为$a*b-\lfloor a/mod*b\rfloor*mod$,
其中$floor$内部使用long double计算。$floor$操作直接使用强制转型,因为$a/mod*b$就在
long long的表示范围内。由于最终结果在范围内,两个乘法事实上计算的是模$2^64$意义下的乘法,
暂时溢出并没有关系。

实现代码:
\begin{lstlisting}
typedef long long Int64;
Int64 mulmB(Int64 a, Int64 b) {
    Int64 res =
        (a * b -
         static_cast<Int64>(
             static_cast<long double>(a) / mod * b) *
             mod) %
        mod;
    return res < 0 ? res + mod : res;
}
\end{lstlisting}

无法证明正确性的原因在于long double的精度是否足够,而且在MSVC中long double只有64位
精度而不是80位。不过让人放心的是无论是long double还是double,在$\geq 10^8$次随机测试
中没有出现错误。
