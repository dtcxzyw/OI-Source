\section{卡常}
\subsection{取模}
\begin{itemize}
    \item 对于多个两整数乘积之和的取模(比如模意义下矩阵乘法),
    可以设置一个阈值,(绝对值)超过该阈值才取模,最后再取一次取模。这种方法
    在保证加法不溢出的情况下大幅减少取模次数,同时将值存储在寄存器内访问速度更快,
    并且if的命中概率小,分支预测效率高。
    \begin{lstlisting}
    const Int64 end=std::numeric_limits<Int64>
        ::max()-asInt64(mod-1)*(mod-1);
    ...
        for(int i=0;i<n;++i)
            for(int j=0;j<n;++j) {
                Int64 sum=0;
                for(int k=0;k<n;++k) {
                    sum+=asInt64(A[i][k])*B[k][j];
                    if(sum>=end)
                        sum%=mod;
                }
                C[i][j]=sum%mod;
            }
    \end{lstlisting}
    \item 若可以肯定最终答案在整型范围内,可以允许暂时的加法溢出;
    \item 若模意义加减操作多,则保证在所有计算过程中的数$\in[0,mod)$。
    \begin{lstlisting}
    int add(int a,int b) {
        a+=b;
        return a<mod?a:a-mod;
    }
    int sub(int a,int b) {
        a-=b;
        return a>=0?a:a+mod;
    }
    \end{lstlisting}
    \item 若模意义乘法操作多,则仅保证中间数$\in (-mod,mod)$即可,没有必要
    $\cdot =clamp(\cdot)$。在最后输出时$clamp$。
\end{itemize}
\subsection{矩阵乘法}

\subsection{基于硬件的优化}
\subsection{位运算}
\subsection{搜索优化}
\subsection{数组清零}
