\section{线性基}
在此仅描述异或线性基。

记数组$b$存储了基中的向量,这是个01上三角矩阵。

\begin{property}
    {\bfseries 线性基内的向量是线性无关的。}
\end{property}
通常利用这个性质+贪心来解题。
\subsection{插入}
\begin{lstlisting}
int b[bitSize+1];
bool insert(int x) {
    for(int i=bitSize;i>=0;--i)
        if(x&(1<<i)){
            if(b[i])x^=b[i];
            else {
                b[i]=x;
                return true;
            }
        }
    return false;
}
\end{lstlisting}
按位遍历进行高斯消元,如果被消为0则可被原基构造。
\subsection{合并}
暴力将一个线性基内的所有数插到另一个线性基进去。
\subsection{存在性查询}
按位扫描进行高斯消元,如果被消为0则在基的张成中。
\subsection{最大值}
\begin{lstlisting}
int maxv() {
    int res=0;
    for(int i=bitSize;i>=0;++i)
        res=std::max(res,res^b[i]);
    return res;
}
\end{lstlisting}
由于线性基可以张成出整个空间,因此从高位到低位贪心就能得到最大值。

此外还有一种等价写法,不使用比较操作,可以使用于bitset上。
\begin{lstlisting}
typedef std::bitset<bitSize> Bit;
Bit maxv() {
    Bit res;
    for(int i=bitSize-1;i>=0;++i)
        if(!res[i])
            res^=b[i];
    return res;
}
\end{lstlisting}
\subsection{最小非0值}
\begin{lstlisting}
int minv() {
    for(int i=0;i<=bitSize;++i)
        if(b[i])
            return b[i];
    return -1;
}
\end{lstlisting}
最小值就是最小的基向量。
\subsection{第k小值}
首先变换线性基,使位与位之间独立(即仅$b[i]$含有$1<<i$),然后挑出非0向量。
\begin{lstlisting}
int vb[bitSize+1],vcnt=0;
void cook() {
    for(int i=bitSize;i>=0;--i)
        for(int j=i-1;j>=0;--j)
            if(b[i]&(1<<j))b[i]^=b[j];
    for(int i=0;i<=bitSize;++i)
        if(b[i])
            vb[vcnt++]=b[i];
}
\end{lstlisting}
计算第k小时扫描一遍k,按k的比特位异或对应的基向量。
\begin{lstlisting}
int kth(int k) {
    if(k>(1<<vcnt))return -1;
    int res=0;
    for(int i=0;i<vcnt;++i)
        if(k&(1<<i))
            res^=vb[i];
    return res;
}
\end{lstlisting}

\paragraph{注意事项}
\begin{itemize}
    \item 要求第$k$大异或和,输入的$k$要-1。
    \item 如果要求异或的子集非空,则需考虑$0$是否能被构造。
    当且仅当输入各向量{\bfseries 线性无关}时(即输入向量数=B中非0向量数),
    $0$不可被构造(仅有平凡解)。因此若$0$可被构造则$k$要-1,否则$k$不变。
\end{itemize}
