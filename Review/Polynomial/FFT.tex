\section{快速傅里叶变换FFT}
\subsection{FFT原理}
FFT求多项式卷积的过程为:$\Theta(n\lg n)$求值->$\Theta(n)$点值乘法->
$\Theta(n\lg n)$插值。

$\Theta(n\lg n)$求值/插值的复杂度是在单位复数根上计算得到的。

\subsubsection{单位复数根}

定义{\bfseries $n$次单位复数根}是满足$\omega^n=1$的复数$\omega$,恰好有$n$个,即
$\omega_n^k=e^{2\pi ik/n},k=0,1,\cdots,n-1$。

定义{\bfseries 主$n$次单位根}$\omega_n=e^{2\pi i/n}$。

下面是关于$n$次单位复数根的性质:

\begin{lemma}[消去引理]\label{FFTL1}
	对于任意整数$n\geq 0,k \geq 0,d>0$,
	\begin{displaymath}
		\omega_{dn}^{dk}=\omega_n^k
	\end{displaymath}
\end{lemma}
证明:
\begin{displaymath}
	\omega_{dn}^{dk}=e^{2\pi i dk/dn}=e^{2\pi i k/n}=\omega_n^k
\end{displaymath}

\begin{inference}\label{FFTI2}
	对于任意偶数$n>0$,有
	\begin{displaymath}
		\omega_n^{n/2}=\omega_2=-1
	\end{displaymath}
\end{inference}

\begin{lemma}[折半引理]
	对于偶数$n>0$,$n$个$n$次单位复数根的平方组成的集合为$n/2$个$n/2$
	次单位复数根的集合。
\end{lemma}
证明:根据引理~\ref{FFTL1}可得$(\omega_n^k)^2=(\omega_n^{k+n/2})^2=
	\omega_{n/2}^k$,每个$n/2$次单位复数根恰好被得到2次。

\begin{lemma}[求和引理]
	对于任意整数$n\geq 1$与不能被$n$整除的非负整数$k$,有
	\begin{displaymath}
		\sum_{i=0}^{n-1}{(w_n^k)^i}=0
	\end{displaymath}
\end{lemma}
证明:
\begin{displaymath}
	\sum_{i=0}^{n-1}{(w_n^k)^i}=\frac{(w_n^k)^n-1}{w_n^k-1}=0
\end{displaymath}
$n$不整除$k$保证了分母不为0。

\subsubsection{DFT}
对于次数界为$n$的多项式
\begin{displaymath}
	A(x)=\sum_{i=0}^{n-1}{a_ix^i}
\end{displaymath}
其DFT为
\begin{displaymath}
	DFT_n(a)=(y_0,y_1,\cdots,y_{n-1})=
	(A(\omega_n^0),A(\omega_n^1),\cdots,A(\omega_n^{n-1}))
\end{displaymath}

\subsubsection{FFT}
FFT采用分治策略,假设$n$是2的幂(不足补0),其步骤如下:
\begin{enumerate}
	\item 若次数界为1,则返回$a_0$。
	\item 定义新的次数界为$n/2$多项式
	      \begin{eqnarray*}
		      A^{[0]}(x)&=&a_0+a_2x+\cdots+a_{n-2}x^{n/2-1}\\
		      A^{[1]}(x)&=&a_1+a_3x+\cdots+a_{n-1}x^{n/2-1}
	      \end{eqnarray*}
	      递归计算其在点$(\omega_n^0)^2,(\omega_n^1)^2,\cdots,(\omega_n^{n-1})^2$
	      的值(实际上递归只求了前一半)。
	\item 该多项式满足等式\begin{equation}\label{RFFTE}
		      A(x)=A^{[0]}(x^2)+xA^{[1]}(x^2)
	      \end{equation}
	      可利用递归计算的值合并。
	      对于$k=0,1,\cdots,n/2-1$,
	      \begin{eqnarray*}
		      y_k&=&y_k^{[0]}+\omega_n^ky_k^{[1]}\\
		      y_{k+n/2}&=&y_k^{[0]}-\omega_n^ky_k^{[1]}
	      \end{eqnarray*}
	      正确性证明:
	      \begin{eqnarray*}
		      y_k&=&y_k^{[0]}+\omega_n^ky_k^{[1]}\\
		      &=&A^{[0]}(\omega_{n/2}^k)+\omega_n^kA^{[1]}(\omega_{n/2}^k)\\
		      &=&A^{[0]}(\omega_n^{2k})+\omega_n^kA^{[1]}(\omega_n^{2k})
		      \textrm{~(根据引理~\ref{FFTL1})}\\
		      &=&A(\omega_n^k) \textrm{~(根据式~\ref{RFFTE})}\\
		      y_{k+n/2}&=&y_k^{[0]}-\omega_n^ky_k^{[1]}\\
		      &=&A^{[0]}(\omega_{n/2}^k)+\omega_n^{k+n/2}A^{[1]}(\omega_{n/2}^k)
		      \textrm{~(根据推论~\ref{FFTI2})}\\
		      &=&A^{[0]}(\omega_n^{2k+n})+\omega_n^{k+n/2}A^{[1]}(\omega_n^{2k+n})
		      \textrm{~(根据引理~\ref{FFTL1}与$\omega_n^n=1$)}\\
		      &=&A(\omega_n^{k+n/2}) \textrm{~(根据式~\ref{RFFTE})}\\
	      \end{eqnarray*}
\end{enumerate}
\subsubsection{逆DFT}
\begin{theorem}
	\begin{displaymath}
		DFT_n^{-1}(a)=\frac{1}{n}DFT_n(DFT_n(a))
	\end{displaymath}
\end{theorem}
证明:

以上内容来自算法导论\cite{ITA3}第30章 多项式与快速傅里叶变换。
\subsection{迭代FFT实现}
\subsubsection{单位复数根预处理}
\subsubsection{离线位逆序}
\subsubsection{在线位逆序}
\subsection{实序列DFT}
\subsection{MTT之拆系数FFT}
