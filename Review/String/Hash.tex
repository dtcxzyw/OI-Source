\section{Hash}
\subsection{BKDRHash}
BKDRHash在一般情况下表现良好,速度快,Hash质量高。
代码如下:
\begin{lstlisting}
typedef unsigned long long HashT;
HashT BKDRHash(const char*  str) {
    HashT res=0;
    while(*str) {
        res=res*131+*str;
        ++str;
    }
    return res;
}
\end{lstlisting}
\subsection{混合Hash}
若要快速求得子串Hash,预处理前缀Hash值,查询时化到同一幂次再差分即可。

若要判断两个串是否为同一子串或对称子串,可以预处理后缀Hash值,
然后将正向Hash值与逆向Hash值相乘得到新的Hash值。

若要忽略整个字符串的某一连续区间,可以将其前缀的前缀Hash与后缀的后缀Hash
加权混合。
\subsection{子串Hash}
注意一个字符串的子串有$O(n\lg n)$个,预处理整个字符串的所有子串的
复杂度为$O(n\lg n)$。

一般来说二分长度后Hash是个不错的选择。
