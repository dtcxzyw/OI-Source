\section{回文自动机}
\index{P!Palindromic Tree}
\subsection{构造}
回文自动机需要维护每个节点所对应的回文串的长度$len$,两端加入字符$c$
后的后继节点$nxt[c]$,自身的最长后缀回文串所对应的节点$fail$,以及
自身代表的回文串数(需要最后在fail树上上传才是完整的)。
还要维护当前已加入自动机的字符$buf$,以及最后一次加入字符后的状态$last$。

构造PAM的复杂度为$O(n\lg |\Sigma|)$。
\subsubsection{初始化}
首先在空PAM中加入两个根:偶数长度的根0和奇数长度的根1。其中节点0的fail
指向1,len为0,节点1的len为-1(避免特判)。同时令$buf[0]=-1$(避免特判)。
\subsubsection{状态转移}
构造PAM时,按顺序向PAM加入字符。
首先向$buf$加入该字符,然后在fail树上跳最长后缀回文串,直至找到对称点
与自身相同为止。注意如果找不到这样的对称点,就会到达奇数长度的根,它的len
为-1,其对称点就是自己,所以迭代必定会结束。

接下来查看是否有该节点的后继节点,如果没有就新建一个节点,len比父亲多2,
fail指针重新在父亲的fail链上找。然后把该节点挂在父亲下。

最后重置last,++T[last].cnt结束加入。
\subsubsection{后处理}
注意自身的fail肯定是自己的子回文串,最后做一次后处理,按照节点编号的逆序
往上累加cnt即可。
\subsubsection{代码}
\lstinputlisting{Source/Templates/PAM.cpp}
\subsection{应用}
\subsubsection{统计串S的前缀本质不同的回文串个数}
extend该前缀的所有字符后,PAM的$siz-1$就是本质不同的回文串个数。
\subsubsection{统计串S的每个本质不同的回文串的出现次数}
由于PAM中每个节点代表一个回文串,后处理后每一个节点的
cnt就是它的出现次数。
\subsubsection{统计串S的回文串的个数}
答案即为后处理后的cnt之和。
\subsubsection{统计以节点$i$结尾的回文串个数}
对每个节点记录其在fail树上的深度,fail链上的节点所代表的回文串都是
自己的后缀回文串,所以答案即为深度。

以上内容参考了poursoul的博客\footnote{
    Palindromic Tree——回文树【处理一类回文串问题的强力工具】
    \url{https://blog.csdn.net/u013368721/article/details/42100363}
}。
