\section{后缀仙人掌}
\index{S!Suffix Cactus}
\subsection{概述}
后缀仙人掌将后缀Trie的每个非叶子结点与它的一个儿子合并,
形成``树枝''。每个树枝表示的是从根节点到自身叶子节点的后缀。
树枝的深度为其顶端节点的父节点的深度,根树枝的深度为0。记后缀排名为$s$
的后缀为$s$,深度为$depth(s)$。一个树枝的父亲树枝为包含其顶端节点左兄弟
的树枝。

后缀仙人掌有如下性质:
\begin{property}
	$depth(s)=LCP(SA[s],SA[s-1])$
\end{property}
显然$depth(s)$是它和它前一个后缀的最小公共祖先的深度。
$depth(s)$其实就是$height[s]$。
\begin{property}
	树枝$r(r>1)$的父亲树枝是满足$depth(s)\leq depth(r),s<r$的
	编号最大的树枝$s$。
\end{property}
\subsection{构造}
根据$depth(s)$与$height[s]$的联系,首先预处理后缀数组与$height$数组。

然后按照字典序从小到大考虑,维护一个单调栈,栈内存储可能的父亲树枝编号,
找到需要的父亲树枝$k$后将边$(k,depth(s))=s$插入HashTable。

代码如下:
\begin{lstlisting}
int st[size];
void buildSC(int n) {
    int top=1;
    st[top]=1;
    for(int i=2;i<=n;++i) {
        while(height[st[top]]>height[i])
            --top;
        addEdge(st[top],height[i],i);
        st[++top]=i;
    }
}
\end{lstlisting}
\subsection{应用}
多次询问串T的前缀为串S的子串的最大长度。

维护当前所在树枝$id$,与匹配长度$d$。每字符匹配时尽可能沿树枝走,
走不到就跳到子树枝,若没有子树枝则返回。
\begin{lstlisting}
int match(const char* str,int n) {
    int id=1,d=0,i;
    for(i=0;str[i];++i) {
        char c=str[i];
        bool flag=false;
        while(sa[id]+d>n || buf[sa[id]+d]!=c) {
            int nxt=find(id,d);
            if(!nxt) {
                flag=true;
                break;
            }
            id=nxt;
        }
        if(flag)
            break;
        ++d;
    }
    return i;
}
\end{lstlisting}
上述内容参考了WC2014营员交流课件《Suffix Cactus》。
