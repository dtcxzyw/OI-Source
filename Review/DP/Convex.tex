\section{WQS二分凸优化}
需求:给定$n$个物品,需要从中选取$C$个物品,最大/小化权值和。设$g(x)$为选取$x$
个物品的最优解,题目还满足$g(x)$是一个凸函数,可以快速计算出$g(x)$的最值和取到最值的$x$。

因为$g(x)$是凸函数,所以$g'(x)$单调,取到最值的$x$就是$g'(x)$的根。如果让根移动到$C$,
就可以取得$g(C)$的值。那么可以根据当前的$x$与$C$的大小,二分$g'(x)$的上下偏移,使得
$f'(x)=g'(x)+k$的根移动到$C$。此时对应的代价函数$f(x)=g(x)+kx$,最后的答案
$g(C)=f(C)-kC$。

二分需要求出$f(x)$的极值点,$k$的实际意义是每多选一个物品需要额外增加$k$的权值,可以根据
实际意义理解二分时的区间缩小。引入$k$后忽略了个数限制,可以做更低维的DP/贪心。这种优化方法
不仅在DP中使用,比如[国家集训队2]Tree I。

注意这里只需要整数二分。由于DP的特殊性,$g(x)$的值域为整数,其图像由许多横向长度为1的线段
组成。那么$g'(x)$的函数图像就是许多取值为整数的水平线,因此只需二分整数偏移。

{\bfseries BZOJ5311 贞鱼:注意可能存在连续极值点的情况,此时要求极左或极右的极值点,
然后根据极值点是极左还是极右判断二分中更新答案的位置。在本题中求的是极左极值点,因此在
极左极值点$\leq$目标点时更新答案。如果在极左极值点$\geq$目标点时更新答案,则无法取到极左
极值点到极右极值点覆盖目标点的情况。}

{\bfseries CF739E Gosha is hunting:当有两个个数约束时使用二分套二分解决。虽然使用
浮点二分,但仍然要注意极值点取极左/极右,控制不等号来控制区间缩小方向。}

优化:如果发现极值点恰好是目标点,直接break。

上述内容参考了FlashHu的博客\footnote{
    DP的各种优化(动态规划,决策单调性,斜率优化,带权二分,单调栈,单调队列)\\
    \url{https://www.cnblogs.com/flashhu/p/9480669.html}
}。

至于二分的范围,考虑两类极端情况,即选取所有与选取一个的情况,调整范围就是它们的极值,
可以根据输入数据计算而不是预置。

除了选取恰好$k$个物品外,恰好将序列分成$k$段也是典型应用。
