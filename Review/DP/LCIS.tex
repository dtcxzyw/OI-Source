\section{LCIS}
LCIS即Longest Common Increasing Subsequence,最长公共上升子序列。
注意这个问题不能用LCS+LIS解决。

输入长度为$n$的数组A和长度为$m$的数组B。
\subsection{常规dp}
记$dp[i][j]$为数组A的前$i$个元素与数组B的前$j$个元素中,以$B[j]$为结尾的最长
公共上升子序列长度。

接下来分类讨论状态转移方程:
\begin{itemize}
    \item $A[i]=B[j]$,则考虑将$A[i]$与$B[j]$匹配。显然转移位置的第一维为
    $i-1$,第二维只能枚举$B[k]<B[j]$转移。
    \item 若不匹配则直接继承$dp[i-1][j]$的值。
\end{itemize}

综上所述,状态转移方程为:
\begin{displaymath}
    dp[i][j]=\left\{\begin{array}{lr}
        dp[i-1][j]&\textrm{if~}A[i]\neq B[j]\\
        max\left\{dp[i-1][k]\right\}+1 (0\leq k <j \wedge B[k]<B[j])&
        \textrm{if~}A[i]=B[j]
    \end{array}\right.
\end{displaymath}
时间复杂度$O(nm^2)$。
\subsection{优化}
注意到第一层循环后$A[i]$是固定的,而枚举取$max$的情况仅在$A[i]=B[j]$时出现。
因此可以在第一层循环内维护一个最优转移值,当$A[i]>B[j]$时更新该值,当$A[i]=B[j]$时
用该值更新,时间复杂度$O(nm)$。如果不需要输出方案还可以滚动数组节省空间。

上述内容参考了ojnQ的博客\footnote{
    LCIS 最长公共上升子序列问题DP算法及优化
    \url{https://www.cnblogs.com/WArobot/p/7479431.html}
}。
