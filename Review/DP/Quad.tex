\section{四边形不等式优化}\label{Quad}
在解区间动规题时通常会推出以下状态转移方程:
\begin{displaymath}
    dp[i][j]=min(dp[i][k]+dp[k+1][j])+w[i][j],i\leq k <j
\end{displaymath}
此时可考虑使用四边形不等式将$O(n^3)$优化到$O(n^2)$。

接下来定义两个性质:
\paragraph{区间包含的单调性}
对于$a\leq b\leq c\leq d$,有$f(b,c)\leq f(a,d)$。
\paragraph{四边形不等式}
对于$a\leq b\leq c\leq d$,有$f(a,d)+f(b,c)\geq f(a,c)+f(b,d)$。

\begin{theorem}
    若$w$函数满足上述两个性质,则$dp$函数满足四边形不等式性质。
\end{theorem}

\begin{theorem}
    若$dp$函数满足四边形不等式,设$s(i,j)$为$dp[i][j]$的转移点$k$,有
    $s(i,j)\leq s(i,j+1)\leq s(i+1,j+1)$。
\end{theorem}

根据该定理可以缩小转移点的搜索范围:
\begin{displaymath}
    dp[i][j]=min(dp[i][k]+dp[k+1][j])+w[i][j],s[i][j-1]\leq k \leq s[i+1][j]
\end{displaymath}

\paragraph{复杂度分析} 对于每一个左端点,最多扫一遍右边所有点,
因此时间复杂度降为$O(n^2)$。

\paragraph{优化} 计算$w[i][j]$时要考虑区间统计方面的优化(如前缀和)。

以上内容参考了XDU\_Skyline的博客\footnote{动态规划专题小结:四边形不等式优化
    \url{https://blog.csdn.net/u014800748/article/details/45750737}
}。
