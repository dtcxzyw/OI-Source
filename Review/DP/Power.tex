\section{矩阵快速幂优化}
\subsection{常规矩阵快速幂}
若dp状态转移方程满足如下形式:
\begin{displaymath}
    dp[i]=\sum_{j=1}^k{c_idp[i-j]}
\end{displaymath}
或对于图满足如下形式:
\begin{displaymath}
    dp[d][i][j]=\sum_{(i,k)\in E,(k,j)\in E}{dp[d-1][i][k]*dp[d-1][k][j]}
\end{displaymath}
则可以使用矩阵快速幂优化。

建一个$k*k$的矩阵$A$,dp初始值为$1*k$的向量$v_0$。
要构造$A$,使其每乘一次$A$,向量表示的区间后移一格,那么
$A[i][j]$表示其在做乘法后将第$i$点的值贡献到第$j$点去。

矩阵满足乘法结合律,可以使用快速幂进行计算。

\subsection{特殊矩阵快速幂}

若大小为$n*n$的矩阵$A$可表示为大小为$n*k$的矩阵$B$与大小为
$k*n$的矩阵$C$的乘积,其中$k\ll n$。
那么可以将$A$的幂拆开,错位结合,计算$k*k$的矩阵$D=CB$,对$D$快速幂后
计算需要的值即可{\bfseries (尽量不要计算实际的转移矩阵)}。

\paragraph{例题} bzoj3583 杰杰的女性朋友

使用上述方法优化快速幂。这里还有一个问题,矩阵$A$的$k$次幂求的是走$k$
步的方案数的转移矩阵,但是答案要的矩阵为矩阵的等比数列求和。因此我们
可以对于每次询问再加一个累加计数器,自己向自己连边,终点向自己连边,
最后单独求出起点到自己的方案即可。
