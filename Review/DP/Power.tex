\section{矩阵快速幂优化}
\subsection{常规矩阵快速幂}
若dp状态转移方程满足如下形式:
\begin{displaymath}
    dp[i]=\sum_{j=1}^k{c_idp[i-j]}
\end{displaymath}
或对于图满足如下形式:
\begin{displaymath}
    dp[d][i][j]=\sum_{(i,k)\in E,(k,j)\in E}{dp[d-1][i][k]\cdot dp[d-1][k][j]}
\end{displaymath}
则可以使用矩阵快速幂优化。

计算$k*k$的转移矩阵$A$,dp初始值为$1*k$的向量$v_0$。
构造$A$,使其每乘一次$A$,向量表示的区间后移一格,那么
$A[i][j]$表示其在做一次乘法后将第$i$点的值贡献到第$j$点中的权值。

矩阵乘法满足结合律,因此可以使用矩阵快速幂进行计算。

\subsection{矩阵链乘因式分解}

若大小为$n*n$的矩阵$A$可表示为大小为$n*k$的矩阵$B$与大小为
$k*n$的矩阵$C$的乘积,其中$k\ll n$。
那么可以将$A$的幂拆开,错位结合,计算$k*k$的矩阵$D=CB$,对$D$快速幂后
计算需要的值{\bfseries (答案向量为$v_0AD^{c-1}B$,尽量按需计算结果)}。

\paragraph{例题} bzoj3583 杰杰的女性朋友

使用上述方法优化矩阵快速幂的效率。此外还存在一个问题,矩阵$A$的$k$次幂求的是走$k$
步的方案数的转移矩阵,但是答案要的矩阵为矩阵幂求和。因此我们可以对于每次询问再加一个
累加计数器,自己向自己连边,对应点向自己连边,最后单独求出起点到自己的方案数。

\subsection{dp步伐不一致时的解决方案}
例题~LOJ\#2180. 「BJOI2017」魔法咒语

如果单次转移最多需要跳跃$k$步($k$为小常数),可以给每个状态$S_0$引入$k-1$个``延迟状态''
$S_i$,若有状态$S$跳跃$k$步到达状态$T$,则实际转移为
$S\rightarrow T_{k-1} \rightarrow T_{k-2} \rightarrow \cdots \rightarrow T_0$。
注意后面的转移链是固定的,可以单独预处理。而第一个转移与原来的处理方式相同,只要根据跳跃
步数计算需要延迟转移的时间,然后连到链上对应的节点。统计时仍然只统计$S_0$,因为延迟状态
的贡献是不完整的。
