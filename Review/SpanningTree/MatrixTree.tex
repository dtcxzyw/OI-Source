\section{Matrix-Tree定理}\label{MatrixTree}
\subsection{基本定义}
\index{K!Kirchhoff's Matrix Tree Theorem}
\begin{theorem}[Kirchhoff's Matrix Tree Theorem]
	一个无向图的生成树个数为度数矩阵(第$u$行第$u$列为点$u$的度数)减
	邻接矩阵(第$u$行第$v$列为$u,v$之间的边数)去掉第$i$行第$i$列后
	的行列式值。
\end{theorem}
根据这个定理,$O(n^3)$便可以求解无向图的生成树计数问题。{\bfseries 注意
高斯消元时交换两行会使行列式值取反,处理时记录符号或者直接返回其绝对值
(仅限于非模意义下求值)。}
\subsection{扩展}
\subsubsection{完全图生成树}
\index{C!Cayley's Formula}
\begin{theorem}[Cayley's Formula]
	大小为$n$的完全图的生成树个数为$n^{n-2}$。
\end{theorem}
套用Matrix-Tree定理或者使用Purfer序列可证明。若点与点之间的边数为$m$,
方案再乘上$m^{n-1}$。
\subsubsection{有向图生成树计数}
邻接矩阵只记录有向边,度数矩阵只记录入度,以$s$为根时删去第$s$行第$s$列后求行列式。
\subsubsection{边权乘积和}
把度数矩阵改为与某点相连的边的边权和,把邻接矩阵的边数改为该边的边权和。
若边权为整数则可以将其理解为将权值为$w$的边拆成$w$条边后求生成树数。

在「长乐集训 2017 Day10」生成树求和 加强版 这道题中,按位拆分后发现需要做不进位
加法,但是矩阵树定理只能做乘法。考虑使用生成函数表示和为0,1,2的种类数,最后做高斯消元。
但是多项式高斯消元并不好做,可以代入多个点值求高斯消元,然后插值。注意生成函数的乘法需要
做循环卷积,可以使用3个单位根作为代入点值。最后计算以3为基的IDFT插值(暴力计算以点值为
多项式系数,在3个单位根的逆上的值,最后将结果除以3)。注意到由于模数
1e9+7模3余2,在整数域上只有1个单位根,因此只能使用模意义下的复数根作为单位根。

\subsubsection{概率扩展}
Luogu P3317 [SDOI2014]重建\footnote{【P3317】[SDOI2014]重建 - 洛谷
\url{https://www.luogu.org/problemnew/show/P3317}
}

图中的每条边都有出现的概率,求图恰好连成一棵生成树的概率。

答案为每种方案的树边出现的概率和非树边不出现的概率之积的和。
将答案除以所有边都不出现的概率,转化为每种方案的树边出现的概率除以
树边不出现的概率的和。由此可以将其转化为边权乘积问题,即令出现概率为$p$的边的边权为
$\frac{p}{1-p}$,求完行列式后乘以$\displaystyle \prod_{i=1}^m{(1-p)}$。
注意使用偏移$\varepsilon$来防止除零。

代码如下:
\lstinputlisting{Source/Unclassified/Done/3317.cpp}

\subsubsection{限制边数}
图上的边有两种颜色,限制生成树中一种颜色的边的数量,求方案数。

令该颜色的边对应$x$,另一种颜色对应$1$,构造多项式,最后求出的行列式多项式的
$x^k$项的系数就对应使用$k$条边的方案数。多项式高斯消元不太方便,
可以先预处理$|V|$个$x$所对应的行列式值,然后插值出多项式。

上述内容参考了MoebiusMeow的博客\footnote{
	康复计划\#5 Matrix-Tree定理(生成树计数)的另类证明和简单拓展
	\url{https://www.cnblogs.com/meowww/p/6485422.html}
}和Wikipedia-EN\footnote{
	Kirchhoff's theorem - Wikipedia\\
	\url{https://en.wikipedia.org/wiki/Kirchhoff\%27s\_theorem}
}。
