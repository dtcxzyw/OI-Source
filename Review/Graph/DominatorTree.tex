\section{支配树}
\index{D!DominatorTree}
\subsection{定义}
支配树基于某个原点$s$,若$s$到某个点$v$的节点必定经过$u$,则称$u$支配$v$。
起点$s$与自己$v$称为平凡支配点。

若$v$的某个支配点$w$满足其被$v$的其余非平凡支配点支配,则$w$为$v$的最近支配点,
记作$idom(v)=w$。每个点到自己的$idom$连边,则构造出了一棵树,称为支配树。
\subsection{DAG的支配树}
DAG的支配树构造较为简单。考虑按照top序加入节点,当前节点的$idom$就是其前驱节点在支配树上的LCA。
由于树的形态是固定的,使用倍增可以$O((n+m)\lg n)$实现。
\subsection{一般图的支配树}
该内容留坑待补。
\index{*TODO!一般图支配树}

上述内容参考了MoebiusMeow的博客\footnote{
    康复计划\#4 快速构造支配树的Lengauer-Tarjan算法\\
    \url{https://www.cnblogs.com/meowww/archive/2017/02/27/6475952.html}
}。
