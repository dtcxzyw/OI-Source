\section{强连通分量}
\index{S!Strongly Connected\\ Component}
\subsection{定义}
\paragraph{强连通} 如果一对点存在路径互相可达,就称这对点强连通。
\paragraph{强连通子图} 强连通子图的点两两互相可达。
\paragraph{强连通分量} 有向图的极大强连通子图。

一般可以将强连通分量缩成一个点,然后对缩点后的图进行dp。
\subsection{Tarjan算法}
Tarjan算法求SCC的步骤如下:

\begin{enumerate}
	\item DFS遍历每个未遍历的点。
	\item 对于每个点,维护其访问时间$dfn$,可访问到的栈上的
	      最早的点的访问时间$low$。DFS处理该点时,将该点加入栈中。
	\item 若$dfn[u]==low[u]$,则说明栈上从栈顶到自己的点构成了强连通分量,
	      新建一个强连通分量,记录每个点所属的强连通分量。
\end{enumerate}

正确性证明留坑待补。
\index{*TODO!Tarjan算法正确性证明}

代码如下:
\begin{lstlisting}
int dfn[size],low[size],st[size],top=0,col[size],ccnt=0;
bool flag[size];
void DFS(int u) {
    static int icnt=0;
    dfn[u]=low[u]=++icnt;
    flag[u]=true;
    st[++top]=u;
    for(int i=last[u];i;i=E[i].nxt) {
        int v=E[i].to;
        if(dfn[v]) {
            if(flag[v])
                low[u]=std::min(low[u],dfn[v]);
        }
        else {
            DFS(v);
            low[u]=std::min(low[u],low[v]);
        }
    }
    if(dfn[u]==low[u]) {
        int c=++ccnt,v;
        do {
            v=st[top--];
            flag[v]=false;
            col[v]=c;
        } while(u!=v);
    }
}
for(int i=1;i<=n;++i)
    if(!dfn[i])
        DFS(i);
\end{lstlisting}
