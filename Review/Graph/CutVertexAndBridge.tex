\section{割点与桥}
\subsection{割点}
若删除{\bfseries 无向连通图}中的一个点及与它相连的边,使得整个图不连通,
那么称这个点为割点。

查找割点的步骤如下:
\begin{enumerate}
    \item 从一个点开始DFS遍历未被遍历的点;
    \item 对于每个点维护其访问时间$dfn$和不经过与父亲相连的边所能访问到的点的访问时间
    最小值$low$;
    \item 如果自己不是DFS树的根,若DFS树中儿子的$low$不超过自己的$dfn$,则说明删掉
        自己后DFS树上自己的父亲与自己的儿子不连通,自己为割点;
    \item 如果自己为DFS树的根,并且自己在DFS树上有两个及以上的儿子,说明自己也是割点。
\end{enumerate}

代码如下(求的是每个连通图的割点):
\lstinputlisting{Source/Review/Graph/CutVertex.cpp}

\subsection{桥}
若删除{\bfseries 无向连通图}中的一条边,使得整个图不连通,那么称这条边为桥。

同样维护$dfn$与$low$,在DFS树上处理$u->v$的过程中,
$(u,v)$为桥当且仅当$(u,v)$无重边且$dfn[u]<low[v]$。

为了判断无重边的情况,在DFS过程中让其儿子返回是否存在重边。

\begin{lstlisting}
struct EdgeT {
    int u,v;
    EdgeT(int u,int v):u(u),v(v) {}
} E[maxm];
int dfn[size], low[size], ccnt = 0, ecnt = 0;
bool DFS(int u, int p, int e) {
    static int icnt = 0;
    dfn[u] = low[u] = ++icnt;
    int pcnt = 0;
    for(int i = last[u]; i; i = E[i].nxt) {
        int v = E[i].to;
        if(v != p) {
            if(dfn[v])
                low[u] = std::min(low[u], dfn[v]);
            else {
                bool flag = DFS(v, u, i);
                low[u] = std::min(low[u], low[v]);
                if(flag && dfn[u] < low[v])
                    E[++ecnt] = EdgeT(u, v);
            }
        }
        else ++pcnt;
    }
    return pcnt==1;
}
\end{lstlisting}
