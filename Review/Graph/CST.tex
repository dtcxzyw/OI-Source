\section{仙人掌与圆方树}
\subsection{仙人掌}
\paragraph{仙人掌} 任意一条边最多只存在于一个环中的无向连通图叫做仙人掌。
\subsubsection{仙人掌的判定}
假设整个图已经连通,可以先对仙人掌进行DFS,记录DFS序与点的深度。然后
树上差分求出经过这条边的环数。判定的同时还可以把仙人掌取环变成一棵森林。
\begin{lstlisting}
bool flag[size];
int p[size], d[size], id[size], icnt;
void DFS(int u) {
    id[++icnt] = u;
    flag[u] = true;
    for(int i = last[u]; i; i = E[i].nxt) {
        int v = E[i].to;
        if(!flag[v]) {
            d[v] = d[u] + 1;
            p[v] = u;
            DFS(v);
        }
    }
}
int tag[size];
bool graph2Forest(int n) {
    icnt = 0;
    memset(flag, n);
    DFS(1);
    memset(tag, n);
    for(int i = 1; i <= cnt; i += 2) {
        int u = E[i].to, v = E[i ^ 1].to;
        if(d[u] < d[v])
            std::swap(u, v);
        if(p[u] != v)
            ++tag[u], --tag[v];
    }
    initGraph(n);
    for(int i = n; i >= 1; --i) {
        int u = id[i];
        tag[p[u]] += tag[u];
        if(tag[u] == 0) {
            if(p[u]) {
                addEdge(u, p[u]);
                addEdge(p[u], u);
            }
        } else if(tag[u] > 1)
            return false;
    }
    return true;
}
\end{lstlisting}
如果graph2Forest返回false则说明这个图不是仙人掌。若返回true
则建出一个去掉环的森林。
\subsubsection{DFS树dp法}
对于简单的仙人掌问题可以使用DFS树做法:

首先可以使用类似Tarjan的算法判断是否出现了环,然后对于环和桥分别dp,
将环的信息记录在环上节点在DFS树上最浅的节点上(即在回到环上最浅点时
另外dp)。
\subsection{圆方树}
圆方树是解决仙人掌问题的利器,主要思想是把仙人掌构造为一棵树,然后使用
熟悉的树上操作来处理。
\subsubsection{构造}
在Tarjan时连树边(圆圆边),在环上深度最浅的点上处理环,把环上的节点(圆点)
都连到新建的点(方点)上(圆方边)。

\begin{lstlisting}

\end{lstlisting}

圆方树具有以下性质:
\begin{property}
    圆方树的子树对应仙人掌的子仙人掌。
\end{property}
\begin{property}
    圆方树不存在方点与方点的连边。
\end{property}
但是上述版本甚至不存在圆点与圆点的连边,所以只要考虑圆方边即可(其实上面建的是
广义圆方树)。
\subsubsection{应用}
\paragraph{最短路}

\paragraph{点分治}
注意处理方点时的复杂度,一般将方点设为环的大小,点分治时找带权重心。

\subsection{广义圆方树}
对于每个点双,将点双内的点(称为圆点)连到新点(称为方点)即可。
\begin{lstlisting}
int dfn[size], low[size], st[size], timeStamp = 0,
    top = 0, nsiz;
void tarjan(int u) {
    dfn[u] = low[u] = ++timeStamp;
    st[++top] = u;
    for(int i = g1.last[u]; i; i = g1.E[i].nxt) {
        int v = g1.E[i].to;
        if(dfn[v])
            low[u] = std::min(low[u], dfn[v]);
        else {
            tarjan(v);
            low[u] = std::min(low[u], low[v]);
            if(dfn[u] <= low[v]) {
                int s = ++nsiz, p;
                g2.addEdge(u, s);
                do {
                    p = st[top--];
                    g2.addEdge(s, p);
                } while(p != v);
            }
        }
    }
}
nsiz = n;
\end{lstlisting}
广义圆方树的点与边数的最大规模都是$2V$级别的(原点与方点相间,
并且连成一棵树)。

关键在于考虑圆点和方点的权值和边权,注意更新方点时不要更新父亲,
在另外的计算过程中考虑父亲,以保证更新的复杂度。

上述内容参考了小蒟蒻yyb\footnote{仙人掌\&圆方树学习笔记
    \url{https://www.cnblogs.com/cjyyb/p/9098400.html}
}和immortalCO\footnote{圆方树——处理仙人掌的利器
    \url{http://immortalco.blog.uoj.ac/blog/1955}
}的博客。
