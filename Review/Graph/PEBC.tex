\section{双连通分量}
\index{B!Biconnected Component}
\subsection{点双连通分量}
如果连通图中任意两点之间至少存在两条点不重复路径,则称该图为点双连通的。
无向图中点双连通的极大子图称为点双连通分量。

可以发现当找到一个割点时,就已经访问了一个点双连通分量,所以在Tarjan算法求割点
的基础上记录栈中元素就能得到点双。注意一个割点可能属于多个点双,因此割点的编号是无意义的,
并且需要记录的元素是边而不是点。

\begin{lstlisting}
int dfn[size], low[size], st[size], top = 0, col[size],
    ccnt = 0;
std::vector<int> bcc[size];
void DFS(int u, int p) {
    static int icnt = 0;
    dfn[u] = low[u] = ++icnt;
    for(int i = last[u]; i; i = E[i].nxt) {
        int v = E[i].to;
        if(dfn[v]) {
            if(dfn[v] < dfn[u] && v != p) {
                st[++top] = i;
                low[u] = std::min(low[u], dfn[v]);
            }
        }
        else {
            st[++top] = i;
            DFS(v, u);
            low[u] = std::min(low[u], low[v]);
            if(dfn[u] <= low[v]) {
                int c = ++ccnt,eid;
                do {
                    eid = st[top--];
                    int eu = E[eid ^ 1].to;
                    int ev = E[eid].to;
                    if(col[eu] != c)
                        bcc[c].push_back(eu);
                    if(col[ev]!=c)
                        bcc[c].push_back(ev);
                    col[eu] = col[ev] = c;
                } while(eid != i);
            }
        }
    }
}
\end{lstlisting}

\subsection{边双连通分量}
如果连通图中任意两点之间至少存在两条边不重复路径,则称该图为边双连通的。
无向图中边双连通的极大子图称为边双连通分量。

把桥删去后分出的连通块就是边双连通分量,可以一遍DFS求出桥,再一遍DFS染色。

上述内容参考了vufw\_795的博客\footnote{
    Tarjan三大算法之双连通分量(双连通分量)
\url{https://blog.csdn.net/fuyukai/article/details/51303292}
}。
